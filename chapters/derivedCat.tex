\chapter{Derived Categories}

\section{Idempotent Completeness and Krull-Schmidt Categories}

    This section will introduce the concepts of idempotent complete categories, weakly idempotent complete categories, and Krull-Schmidt categories. These are extra conditions that may be put onto an exact category. The conditions to be introduced are three different levels of strengthening, where weakly idempotent completeness is the weakest and Krull-Schmidt is the strongest condition. The concepts introduced in this section is based on the ideas from \cite{buhler}, \cite{Kra12} and \cite{Rei95}

    \begin{definition}
        An idempotent complete category is an additive category where every idempotent split. That is, if there is an idempotent $p:A\rightarrow A$ $(p^2=p)$, and there is an isomorphism $A\simeq I\oplus K$ such that $p\simeq \begin{pmatrix} 0 & 0 \\ 0 & 1 \end{pmatrix}$. 
    \end{definition}

    Every idempotent in an idempotent complete category admits an analysis. That is the idempotent $p:A\rightarrow A$ has a kernel, cokernel, image, and coimage. In fact, the kernel is isomorphic to the cokernel, and the image is canonically isomorphic to the coimage. As $p$ is isomorphic to the matrix $\begin{pmatrix} 0 & 0 \\ 0 & 1 \end{pmatrix}$ one may observe that the inclusion $\iota_1:I\rightarrow A$ is the kernel of $p$, while the projection $\pi_1:A\rightarrow I$ is the cokernel. Similarly the maps $\iota_2:K\rightarrow A$ and $\pi_2:A\rightarrow K$ are the kernel and cokernel of the map $1-p$ respectively. Using the fact that $p$ splits one can construct the following analysis.

    \begin{center}
        \begin{tikzcd}
            & A \ar{rr}{p} \ar[two heads]{rd}{\pi_2}[marking]{\circ} & & A \ar[two heads]{rd}{\pi_1}[marking]{\circ} \\
            I \ar[tail]{ru}{\iota_1}[marking]{\circ} & & K \ar[tail]{ru}{\iota_2}[marking]{\circ} & & I 
        \end{tikzcd}
    \end{center}

    \begin{remark}
        Assuming that every idempotent in an additive category $\mathcal{A}$ has a kernel is sufficient for $\mathcal{A}$ to be idempotent complete. The limits and colimits as described above may be found with the idempotents $p$ and $1-p$.
    \end{remark}

    Every additive category $\mathcal{A}$ has a fully faithful embedding $i_{\mathcal{A}}:\mathcal{A}\rightarrow \widehat{\mathcal{A}}$ into an idempotent complete category $\widehat{\mathcal{A}}$. This completion satisfies the universal property in which if there is a functor $F:\mathcal{A}\rightarrow\mathcal{B}$ which sends every idempotent $p$ in $\mathcal{A}$ to a splitting idempotent, then the functor factors through the idempotent complete category $\widehat{\mathcal{A}}$.

    \begin{center}
        \begin{tikzcd}
            \mathcal{A} \ar{rd}{i_{\mathcal{A}}} \ar{rr}{F} & & \mathcal{B} \\
            & \widehat{\mathcal{A}} \ar{ru}{\widehat{F}}
        \end{tikzcd}
    \end{center}

    One may define this completion $\widehat{\mathcal{A}}$ to be the category with objects $(A,p)$, where $A$ is an object of $\mathcal{A}$ and $p:A\rightarrow A$ is an idempotent. A morphism $\widehat{f}:(A,p)\rightarrow (B,q)$ is defined as the morphism $\widehat{f} = q \circ f \circ p$ for some morphism $f:A\rightarrow B$. The injection functor is defined as $i_{\mathcal{A}}(A)=(A,id_A)$. More on this injection can be found in \cite{buhler}.
    
    Many of the useful theorems needed to describe the triangulated subcategory needed for the construction of the derived category will arise from the weaker condition of weakly split idempotents.

    \begin{lemma}
        The following are equivalent in an additive category:
        \begin{enumerate}
            \item Every split-epi has a kernel
            \item Every split-mono has a cokernel
        \end{enumerate}
    \end{lemma}

    \begin{proof}
        It suffices to prove that (1.) $\implies$ (2.), as the other claim is dual. Suppose that $g:B\rightarrow A$ is split-epi with $f:A\rightarrow B$ as the corresponding split-mono such that $gf=id_A$. Since $g$ is split-epi it has a kernel $h:C\rightarrow B$.
        
        \begin{center}
            \begin{tikzcd}
                A \ar[bend left]{r}{f} & \ar[bend left]{l}{g} B \ar[dashed, bend left]{r}{i} & \ar[bend left]{l}{h} C
            \end{tikzcd}
        \end{center}

        Looking at the map $id_B-fg$, one may see that $g(id_B-fg)=g-gfg=g-g=0$, thus $id_B-fg$ factors over the kernel of h as indicated by the dashed arrow.

        h is split-mono as $hih = (id_B-fg)h=h-fgh=h$. As h is mono from being a kernel, it follows that $ih=id_C$. $B$ is the biproduct $B\simeq A\oplus C$ as $id_B -fg = hi \iff id_B = fg + hi$. This in turn implies that $i$ is the cokernel of $f$.
    \end{proof}

    This lemma is at the core of weakly idempotent complete categories.

    \begin{definition}
        An additive category $\mathcal{A}$ is weakly idempotent complete if it satisfies either of the conditions of Lemma 4.1.
    \end{definition}

    \begin{corollary}
        Let $(\mathcal{A},\mathcal{E})$ be an exact category, then the following are equivalent:
        \begin{enumerate}
            \item The category $\mathcal{A}$ is weakly idempotent complete
            \item Every split-mono is an inflation
            \item Every split-epi is a deflation
        \end{enumerate}
    \end{corollary}

    With the notion of a weakly idempotent complete category, the Obscure axiom can be strengthened into Heller's cancellation axiom.

    \begin{prop} \textbf{Heller's cancellation axiom}
        For an exact category $(\mathcal{A},\mathcal{E})$ the following are equivalent:
        \begin{enumerate}
            \item $\mathcal{A}$ is weakly idempotent complete
            \item Let $f: A\rightarrow B$ and $g: B\rightarrow C$ be two morphisms in $\mathcal{A}$. Then if $gf:A\rightarrow C$ is a deflation, then $g$ is.
        \end{enumerate}
    \end{prop}

    \begin{proof}
        Suppose (1.). Let $f:A\rightarrow B$ and $g:B\rightarrow C$ be morphisms such that their composition $gf:A\rightarrow C$ is a deflation. Since $gf$ is a deflation, the pullback square exists.

        \begin{center}
            \begin{tikzcd}
                A \ar[bend right]{ddr}{id_A} \ar[bend left]{rrd}{f} \ar[dashed]{rd}{f'} \\
                & B' \ar[phantom]{rd}[very near start]{\ulcorner} \ar{d}{h} \ar[two heads]{r}{f'}[marking]{\circ} & B \ar{d}{g} \\
                & A \ar[two heads]{r}{gf}[marking]{\circ} & C
            \end{tikzcd}
        \end{center}

        By using the universal property, one may see that $g'$ is split-mono, hence it admits an inflation $h':A'\rightarrow B'$. The claim is that $hh':A'\rightarrow B$ is the kernel of $g$. If the claim is true, the Obscure axiom yields that $g$ is a deflation.

        To show that $hh'$ is the kernel one must show the universal property. Let $t:T\rightarrow B$ be a test object, such that $gt=0$.

        \begin{center}
            \begin{tikzcd}
                T \ar[bend left]{rrd}{t} \ar[dashed]{rd}{t'} \\
                & A' \ar{r}{hh'} \ar{rd}{0} & B \ar{d}{g} \\
                & & C
            \end{tikzcd}
        \end{center}

       It is known that $t'$ exists as $t$ factors through $B'$ with $t''$, by the pull-back property. As $g't''=0$, $t''$ factors through $A'$ using the fact that $h'$ is the kernel of $g'$, this proves the claim.
        
        For the other direction, suppose (2.) instead and let $gf=id_A$,  $gf$ is a deflation and $g$ is split-epi. By the assumption, $g$ is a deflation, so it has a kernel.
    \end{proof}

    Lastly, suppose that $\mathcal{A}$ is an idempotent complete category and that there are some idempotents over an object $A$. These idempotents admit a description of A as a direct sum of kernels and cokernels. There is, however, no guarantee that these decompositions are unique. To fix this, define the following category.

    \begin{definition}
        Let $\mathcal{A}$ be an additive category. An object $A$ is called indecomposable if the endomorphism ring of $A$ is local.

        An object is called decomposable if it is not indecomposable.
    \end{definition}

    \begin{definition}
        An additive category $\mathcal{A}$ is called Krull-Schmidt if any object $A$ decomposes into a finite direct sum of indecomposable objects.
    \end{definition}

    Having that each indecomposable object is local is enough for the following proposition to hold.

    \begin{prop}
        Every decomposition in a Krull-Schmidt category is unique up to isomorphism
    \end{prop}

    As being Krull-Schmidt admits decomposition whenever an endomorphism ring is not local implies a connection to idempotent completeness. That is whenever there is an idempotent over an object, this idempotent gives rise to two comaximal ideals for the endomorphism ring. This gives us the decomposition which is required for the idempotent to split. Moreover, there is a deeper connection between being Krull-Schmidt and idempotent complete.

    \begin{definition}
        Let R be a ring. We say that R is semiperfect if $R$ as a module over itself admits a decomposition $_RR\simeq P_1\oplus P_2\oplus ... \oplus P_n$ such that each $P_i$ has a local endomorphism ring.
    \end{definition}

    \begin{remark}
        For a ring R the following conditions are equivalent:
        \begin{itemize}
            \item The category $mod_R$ is a Krull-Schmidt category
            \item R is semiperfect
            \item Every simple R-module has a projective cover
            \item Every finitely generated R-module has a projective cover
        \end{itemize}
        Thus any of these conditions can be taken to be the definition of semiperfect.
    \end{remark}

    With this definition, one can state the following proposition, which says whenever an idempotent complete category is Krull-Schmidt.

    \begin{prop}
        Let $\mathcal{A}$ be an additive category, then the following are equivalent:
        \begin{enumerate}
            \item $\mathcal{A}$ is Krull-Schmidt
            \item $\mathcal{A}$ is idempotent complete and every endomorphism ring is semiperfect.
        \end{enumerate}
    \end{prop}

    %\begin{proof}
    %    Suppose that $\mathcal{A}$ is Krull-Schmidt. We need to show that every idempotent splits, and that every endomorphism ring are semiperfect.

    %    Let $p:A\rightarrow A$ be an idempotent. Then $End(A)$ is not local, as the ideals $(p)$ and $(1-p)$ are comaximal. Thus $A\simeq A_1\oplus A_2$ and $p\simeq \begin{pmatrix} 0 & \\ 0 & 1 \end{pmatrix}$. Thus every idempotent split.

    %    If $A$ is indecomposable, then by assumption $End(A)$ is local. Conversely, assume that $A$ is decomposable, then $A$ admits a finite decomposition $\bigoplus_{i=1}^{n}A_i$. 
    %\end{proof}

    \begin{example}
        Let $\Lambda$ be any artin R-algebra, then $mod_{\Lambda}$ is a Krull-Schmidt category. As an example, the category of finitely generated real vector spaces is Krull-Schmidt. Every vector space is a finite direct summand of the only indecomposable vector space $\mathbb{R}$.
    \end{example}

    More details and examples of Krull-Schmidt categories may be found in Henning Krause's notes (\cite{Kra12}).

\section{Normal Morphisms and Long Exact Sequences}
    
    \begin{definition}
        Let $(\mathcal{A},\mathcal{E})$ be an exact category. A morphism $f:A\rightarrow B$ is called normal if it has a deflation-inflation factorization. They will be drawn as in the following diagram.
        \begin{center}
            \begin{tikzcd}
                A \ar[two heads]{rd}[marking]{\circ} \ar{rr}{f}[marking]{\circ} & & B \\
                & I \ar[tail]{ru}[marking]{\circ}
            \end{tikzcd}
        \end{center}
    \end{definition}

    \begin{remark}
        A monomorphism is normal if and only if it is an inflation. Dually, an epimorphism is normal if and only if it is a deflation.
    \end{remark}

    \begin{remark}
        In general, the composition $gf$ of two normal morphisms $f$ and $g$ are not normal. However, if g is a deflation, the composition can be seen to normal, as deflations are closed under composition. One may also observe that an exact category is abelian if and only if normal morphisms are closed under composition.
    \end{remark}

    \begin{lemma}
        \textbf{Hellers factorization lemma}. The factorization of normal morphisms is unique up to unique isomorphisms.
    \end{lemma}

    \begin{proof}
        Suppose that a normal morphism admits two different factorizations. That means there exists a commutative diagram as follows.
        \begin{center}
            \begin{tikzcd}
                A \ar[two heads]{r}{p}[marking]{\circ} \ar[two heads]{d}{q}[marking]{\circ} & I \ar[dashed, shift left]{ld}{\phi} \ar[tail]{d}{i}[marking]{\circ} \\
                I' \ar[dashed, shift left]{ru}{\phi '} \ar[tail]{r}{j}[marking]{\circ} & B
            \end{tikzcd}
        \end{center}
        By assumption $ip=jq$, thus $jq\circ Ker(p)=0$. $q\circ Ker(p)=0$ as $j$ is mono, thus there exists a morphism $\phi:I\rightarrow I'$ uniquely such that $q=\phi p$. Now $ip=jq=j\phi p$, and as $p$ is epi it follows that $i=j\phi$. Reiterating the argument, but with $Ker(q)$ instead, there exists a $\phi '$ uniquely such that $p = \phi 'q$ and $j=i\phi '$. Thus $i=j\phi = i\phi '\phi$, and since $i$ is mono it follows that $id_I=\phi '\phi$; dually $Id_{I'}=\phi\phi '$.
    \end{proof}

    \begin{remark}
        Due to Heller's factorization axiom, one may see that normal morphisms admit an analysis.

        \begin{center}
            \begin{tikzcd}
                & A \ar{rr}{f}[marking]{\circ} \ar[two heads]{rd}{p}[marking]{\circ} & & B \ar[two heads]{rd}{Cok(i)}[marking]{\circ} \\
                K \ar[tail]{ru}{Ker(p)}[marking]{\circ} & & I \ar[tail]{ru}{i}[marking]{\circ} & & C 
            \end{tikzcd}
        \end{center}

        Observe that the object $I$ coincides with the image and coimage of $f$. This object will then be referred to as the image of $f$. As a consequence of this unique factorization, a normal morphism is iso if and only if it is mono and epi.
    \end{remark}

    \begin{definition}
        A sequence of normal morphisms is exact if the inflation of the factorization together with the consecutive deflation forms a conflation. That is there are conflations between morphisms as in the following diagram. The conflation pairs are highlighted with different colors.

        \begin{center}
            \begin{tikzcd}
                \dots \ar[two heads, orange]{rd}[black]{p_{-2}}[marking]{\circ} \ar{rr}{f_{-2}}[marking]{\circ} & & A_{-1} \ar[two heads, green]{rd}[black]{p_{-1}}[marking]{\circ} \ar{rr}{f_{-1}}[marking]{\circ} & & A_0 \ar[two heads, teal]{rd}[black]{p_0}[marking]{\circ} \ar{rr}{f_0}[marking]{\circ} & & A_1 \ar{rr}{f_1}[marking]{\circ} \ar[two heads, magenta]{rd}[black]{p_1}[marking]{\circ} & & \dots \\
                & I_{-2} \ar[tail, green]{ru}[black]{i_{-2}}[marking]{\circ} & & I_{-1} \ar[tail, teal]{ru}[black]{i_{-1}}[marking]{\circ} & & I_0 \ar[tail, magenta]{ru}[black]{i_0}[marking]{\circ} & & I_1 \ar[tail, purple]{ru}[black]{i_1}[marking]{\circ}
            \end{tikzcd}
        \end{center}

        A morphism of exact sequences is the same as a morphism of sequences. That is a collection of morphisms $(...,\phi_{-1},\phi_0,\phi_1,...)$ such that the squares in the diagram commute.

        \begin{center}
            \begin{tikzcd}
                \dots \ar{r}{a_{-2}}[marking]{\circ} & A_{-1} \ar{d}{\phi_{-1}} \ar{r}{a_{-1}}[marking]{\circ} & A_0 \ar{d}{\phi_0} \ar{r}{a_0}[marking]{\circ} & A_1 \ar{d}{\phi_1} \ar{r}{a_1}[marking]{\circ} & \dots \\
                \dots \ar{r}{b_{-2}}[marking]{\circ} & B_{-1} \ar{r}{b_{-1}}[marking]{\circ} & B_0 \ar{r}{b_0}[marking]{\circ} & B_1 \ar{r}{b_1}[marking]{\circ} & \dots
            \end{tikzcd}
        \end{center}
    \end{definition}

    \begin{remark}
        An exact sequence of normal morphisms is called short exact if it consists of morphisms on the form $(,0,i,p,0,)$, i.e. as in the following diagram.

        \begin{center}
            \begin{tikzcd}[cramped, row sep=small]
                0 \ar[equal, orange]{rd}{} \ar{rr}{0}[marking]{\circ} & & A \ar[equal, green]{rd}{} \ar{rr}{i}[marking]{\circ} & & B \ar[two heads, teal]{rd}[black, below]{p}[marking]{\circ} \ar{rr}{p}[marking]{\circ} & & C \ar{rr}{0}[marking]{\circ} \ar[two heads, magenta]{rd}[black]{0}[marking]{\circ} & & 0 \\
                & 0 \ar[tail, green]{ru}[black]{0}[marking]{\circ} & & A \ar[tail, teal]{ru}[black, below]{i}[marking]{\circ} & & C \ar[equal, magenta]{ru}{} & & 0 \ar[equal, purple]{ru}{}
            \end{tikzcd}
        \end{center}

        Observe how conflations are exactly the class of short exact sequences.
    \end{remark}

    This definition admits properties that mimic properties from homological algebra.

    \begin{lemma}
        \textbf{5 Lemma}. Given two 5 term exact sequences and a morphism between them as in the diagram. Then $\phi$ is an isomorphism as well.
        \begin{center}
            \begin{tikzcd}
                A_0 \ar{r}{a_0}[marking]{\circ} \ar{d}{\simeq} & A_1 \ar{r}{a_1}[marking]{\circ} \ar{d}{\simeq} & A_2 \ar{r}{a_2}[marking]{\circ} \ar{d}{\phi} & A_3 \ar{r}{a_3}[marking]{\circ} \ar{d}{\simeq} & A_4 \ar{d}{\simeq} \\
                B_0 \ar{r}{b_0}[marking]{\circ} & B_1 \ar{r}{b_1}[marking]{\circ} & B_2 \ar{r}{b_2}[marking]{\circ} & B_3 \ar{r}{b_3}[marking]{\circ} & B_4
            \end{tikzcd}
        \end{center}
    \end{lemma}

    % \begin{proof}
    %     Later, or maybe never
    % \end{proof}

    \begin{lemma}
        \textbf{Kernel-Cokernel sequence}.
        Let $(\mathcal{A},\mathcal{E})$ be an exact category which is weakly idempotent complete. Suppose that there are composable normal morphisms $f$ and $g$ such that $gf$ is normal as well. Then there exists an exact sequence.
        \begin{center}
            \begin{tikzcd}
                Ker(f) \ar{r}[marking]{\circ} & Ker(gf) \ar{r}[marking]{\circ} & Ker(h) \ar{r}[marking]{\circ} & Cok(f) \ar{r}[marking]{\circ} & Cok(gf) \ar{r}[marking]{\circ} & Cok(g)
            \end{tikzcd}
        \end{center} 
    \end{lemma}

    % \begin{proof}
    %     Probably not
    % \end{proof}

    \begin{remark}
        If $(\mathcal{A},\mathcal{E})$ is an exact category, then one may show that the category $\mathcal{A}$ admits Kernel-Cokernel sequences if and only if it is weakly idempotent complete.
    \end{remark}

    The Kernel-Cokernel sequence enables one to prove that the snake lemma holds in weakly idempotent complete categories.

    \begin{corollary}
        \textbf{Snake Lemma.}
        Let $(\mathcal{A},\mathcal{E})$ be a weakly idempotent complete category. Suppose there is a diagram in $\mathcal{A}$ having exact rows.
        \begin{center}
            \begin{tikzcd}
                & A \ar{d}{f}[marking]{\circ} \ar{r}[marking]{\circ} & B \ar{d}{g}[marking]{\circ} \ar{r}[marking]{\circ} & C \ar{d}{h}[marking]{\circ} \ar{r} & 0 \\
                0 \ar{r} & A' \ar{r}[marking]{\circ} & B' \ar{r}[marking]{\circ} & C'
            \end{tikzcd}
        \end{center}

        Then there is an exact sequence.
        \begin{center}
            \begin{tikzcd}
                Ker(f) \ar{r}[marking]{\circ} & Ker(g) \ar{r}[marking]{\circ} & Ker(h) \ar[dashed]{r}{\delta}[marking]{\circ} & Cok(f) \ar{r}[marking]{\circ} & Cok(g) \ar{r}[marking]{\circ} & Cok(h)
            \end{tikzcd}
        \end{center}
    \end{corollary}

\section{Homology and Derived Categories}

    This section aims to provide a construction for derived categories and discuss the underlying assumptions. In the case where $\mathcal{A}$ is abelian, the derived category is constructed by localizing at quasi-isomorphisms. However, in the context of exact categories, how much of the theory transfers? For abelian groups, homology is defined to be the quotient of the kernel of a map by the image of the preceding map. For exact categories, the discussion is a bit more complex. Consider the usual construction of homology, when does the homology exist?.

    \begin{center}
        \begin{tikzcd}
            \dots \ar{r}{d_{\chain{A}}^{-2}}[marking]{\circ} & A_{-1} \ar[two heads]{d}{p}[marking]{\circ} \ar{r}{d_{\chain{A}}^{-1}}[marking]{\circ} & A_0 \ar{r}{d_{\chain{A}}^0}[marking]{\circ} & A_1 \ar{r}{d_{\chain{A}}^1}[marking]{\circ} & \dots \\
            & Im(d_{\chain{A}}^{-1}) \ar[tail]{ru}{\iota}[marking]{\circ} \ar[dashed]{r}{h} & Ker(d_{\chain{A}}^0) \ar[tail]{u}{\kappa}[marking]{\circ} \ar[dotted]{r}{?} & H^0(\chain{A})
        \end{tikzcd}
    \end{center}

    The complex must admit an analysis at each differential, so assume that the complex only contains normal morphisms. By looking at the 0-th homology one can find a condition for when the homology exists. Using the fact that $d^0_{\chain{A}}\iota = 0$, there is an unique morphism $h$, such that $\iota = \kappa h$ by the universal property. The 0-th homology exists whenever the morphism $h$ has a cokernel, and then $h$ satisfies the assumption of the Obscure axiom, making $h$ an inflation. One way to not break this condition is to assume that $\mathcal{A}$ is weakly idempotent complete. By Heller's cancellation axiom it is known that $h$ is an inflation, which then proves the existence of the cokernel. However, this only allows the construction of quasi-isomorphisms at weakly idempotent complete categories. \\
    
    Recall that a quasi-isomorphism is a chain map $\chain{f}:\chain{A}\rightarrow\chain{B}$ such that $H^*(\chain{f}):H^*(\chain{A})\rightarrow H^*(\chain{B})$ is an isomorphism in homology. In the abelian case, suppose that $\chain{f}:\chain{A}\rightarrow\chain{B}$ is a quasi-isomorphism. Consider the standard triangle \\ \begin{tikzcd}\chain{A} \ar{r}{\chain{f}} & \chain{B} \ar{r}{} & cone(\chain{f}) \ar{r}{} & \chain{A}[1]\end{tikzcd} in the homotopy category $K(\mathcal{A})$, then $\chain{f}$ becomes an isomorphism in the derived category $D(\mathcal{A})$. This shows that there is a quasi-isomorphism between $\chain{0}$ and $cone(\chain{f})$ by corollary 1.1.4.1. It follows that $cone(\chain{f})$ is an exact sequence, and this motivate to the following definition.

    \begin{definition}
        Let $(\mathcal{A},\mathcal{E})$ be an exact category. Define the category $Ac(\mathcal{A})\subset K(\mathcal{A})$ to be the full category whose objects are exact sequences.
    \end{definition}

    The exact complexes are also referred to as acyclic complexes. Note that this subcategory is not in general either thick or closed under isomorphisms. To be able to show that it is a triangulated subcategory, it suffices to show that the mapping cone of two acyclic complexes is again acyclic.

    \begin{lemma}
        Let $\chain{f}:\chain{A}\rightarrow\chain{B}$ be a chain map between acyclic chain complexes, then $cone(\chain{f})$ is acyclic as well.
    \end{lemma}

    \begin{proof}
        Suppose that $\chain{A}$ and $\chain{B}$ are acyclic chain complexes and that $\chain{f}:\chain{A}\rightarrow\chain{B}$ is a chain map. The chain map factorizes through the images/kernels, as in the diagram below.
        \begin{center}
            \begin{tikzcd}[column sep = small]
                A^{-1} \ar{rr}{d_{\chain{A}}^{-1}} \ar[two heads]{rd}[marking]{\circ} \ar{ddd}{f^-1} & & A^0 \ar{rr}{d_{\chain{A}}^0} \ar[two heads]{rd}[marking]{\circ} \ar{ddd}{f^0}& & A^1 \ar{ddd}{f^1} \\
                & Im(d_{\chain{A}}^{-1}) \ar[tail]{ru}[marking]{\circ} \ar[dashed]{d}{f^{-1}{'}} & & Im(d_{\chain{A}}^{0}) \ar[tail]{ru}[marking]{\circ} \ar[dashed]{d}{f^0{'}} \\
                & Im(d_{\chain{B}}^{-1}) \ar[tail]{rd}[marking]{\circ} & & Im(d_{\chain{B}}^{0}) \ar[tail]{rd}[marking]{\circ} \\
                B^{-1} \ar{rr}{d_{\chain{B}}^{-1}} \ar[two heads]{ru}[marking]{\circ} & & B^0 \ar{rr}{d_{\chain{B}}^0} \ar[two heads]{ru}[marking]{\circ}& & B^1
            \end{tikzcd}
        \end{center}

        This shows that the chain map may be promoted to a morphism of conflations $(f^{-1}{'},f^0,f^0{'})$. By lemma 2.1.6 this morphism factors through another conflation in the following manner.
        \begin{center}
            \begin{tikzcd}
                Im(d_{\chain{A}}^{-1}) \ar[phantom]{rd}[very near start]{\ulcorner}[very near end]{\lrcorner} \ar[tail]{r}[marking]{\circ} \ar{d}{} & A^0 \ar{d} \ar[two heads]{r}[marking]{\circ} & Im(d_{\chain{A}}^{0}) \ar[equal]{d} \\
                Im(d_{\chain{B}}^{-1}) \ar[tail]{r}[marking]{\circ} \ar[equal]{d} & C^0 \ar{d} \ar[phantom]{rd}[very near start]{\ulcorner}[very near end]{\lrcorner} \ar[two heads]{r}[marking]{\circ} & Im(d_{\chain{A}}^{0}) \ar{d}{}\\
                Im(d_{\chain{B}}^{-1}) \ar[tail]{r}[marking]{\circ} & B^0 \ar[two heads]{r}[marking]{\circ} & Im(d_{\chain{B}}^{0}) \\
            \end{tikzcd}
        \end{center}

        These factorizations exist at every index over the chain complexes. In this way, one may find the diagram below. Note that two bicartesian squares may be connected to make a bicartesian rectangle.

        \begin{center}
            \begin{tikzcd}
                A^{-1} \ar{rr}{d_{\chain{A}}^{-1}} \ar[two heads]{rd}[marking]{\circ} \ar{dd}{} & & A^0 \ar{rr}{d_{\chain{A}}^0} \ar[two heads]{rd}[marking]{\circ} \ar{dd} & & A^1 \ar{dd}{} \\
                & Im(d_{\chain{A}}^{-1}) \ar[tail]{ru}[marking]{\circ} \ar[dashed]{dd}{f^{-1}{'}} \ar[phantom]{rd}[very near start]{\ulcorner}[very near end]{\lrcorner} & & Im(d_{\chain{A}}^{0}) \ar[tail]{ru}[marking]{\circ} \ar[dashed]{dd}{f^0{'}} \ar[phantom]{rd}[very near start]{\ulcorner}[very near end]{\lrcorner} \\
                C^{-1} \ar[phantom]{rd}[very near start]{\ulcorner}[very near end]{\lrcorner} \ar[two heads]{ru}[marking]{\circ} \ar{dd}{} & & C^0 \ar[two heads]{ru}[marking]{\circ} \ar{dd}{} \ar[phantom]{rd}[very near start]{\ulcorner}[very near end]{\lrcorner} & & C^1 \ar{dd}{} \\
                & Im(d_{\chain{B}}^{-1}) \ar[tail]{rd}[marking]{\circ} \ar[tail]{ru}[marking]{\circ} & & Im(d_{\chain{B}}^{0}) \ar[tail]{rd}[marking]{\circ} \ar[tail]{ru}[marking]{\circ} \\
                B^{-1} \ar{rr}{d_{\chain{B}}^{-1}} \ar[two heads]{ru}[marking]{\circ} & & B^0 \ar{rr}{d_{\chain{B}}^0} \ar[two heads]{ru}[marking]{\circ}& & B^1
            \end{tikzcd}
        \end{center}

        By proposition 2.1.3 there are conflations, along the cone of $\chain{f}$.
        \begin{center}
            \begin{tikzcd}
                C^0 \ar{d}{} \ar[two heads]{r}[marking]{\circ} \ar[phantom]{rd}[very  near start]{\ulcorner}[very near end]{\lrcorner} & Im(d_{\chain{A}}^{0}) \ar[tail]{r}[marking]{\circ} \ar[dotted]{d}{f^0{'}} \ar[phantom]{rd}[very near start]{\ulcorner}[very near end]{\lrcorner} & A^1 \ar{d}{} \\
                B^0 \ar[two heads]{r}[marking]{\circ} & Im(d_{\chain{B}}^0) \ar[tail]{r}[marking]{\circ} & C^1
            \end{tikzcd} $\implies$
            \begin{tikzcd}
                C^0 \ar[tail]{r}{} & A^1 \oplus B^0 \ar[two heads]{r}{} & C^1
            \end{tikzcd}
        \end{center}

        By Heller's cancellation axiom, combining these sequences creates a long exact sequence. This shows that $cone(\chain{f})$ is acyclic.
        \begin{center}
            \begin{tikzcd}[column sep = small]
                ... \ar{r}{d_{cone(\chain{f})}^{-2}} & A^{0}\oplus B^{-1} \ar{rr}{d_{cone(\chain{f})}^{-1}} \ar[two heads]{rd}[marking]{\circ} & & A^{1}\oplus B^0 \ar{rr}{d_{cone(\chain{f})}^{0}} \ar[two heads]{rd}[marking]{\circ} & & A^2\oplus B^1 \ar{r}{d_{cone(\chain{f})}^{1}} & ... \\
                & & C^{-1} \ar[tail]{ru}[marking]{\circ} & & C^0 \ar[tail]{ru}[marking]{\circ}
            \end{tikzcd}
        \end{center}
    \end{proof}

    \begin{corollary}
        $Ac(\mathcal{A})$ is a triangulated subcategory of $K(\mathcal{A})$.
    \end{corollary}

    Since $Ac(\mathcal{A})$ is triangulated, it makes sense to talk about the class of morphisms $Mor_{Ac(\mathcal{A})}$. By definition, a morphism $\chain{f}$ is in $Mor_{Ac(\mathcal{A})}$ if and only if $cone(\chain{f})$ is in $Ac(\mathcal{A})$. Therefore it makes sense to say that the class of morphisms $Mor_{Ac(\mathcal{A})}$ may be regarded as quasi-isomorphisms. 

    \begin{corollary}
        The derived category is defined by the Verdier quotient $D(\mathcal{A})=K(\mathcal{A})/Ac(\mathcal{A})$ whenever it exists, and it is triangulated.
    \end{corollary}

    \begin{remark}
        The derived category exists whenever $Mor_{Ac(\mathcal{A})}$ is a locally small multiplicative system.
    \end{remark}

    As stated, it is not true a priori that $Ac(\mathcal{A})$ is either thick or closed under isomorphisms. When this is not true, it might happen that $\chain{C}\simeq\chain{A}\oplus\chain{B}$ where $\chain{C}$ is acyclic, but neither $\chain{A}$ or $\chain{B}$ need not be acyclic. However, the kernel of localization $\widehat{Ac(\mathcal{A})}$ contains all of these objects. In this way $\chain{A}$ and $\chain{B}$ will be related in the derived category, even though they are not quasi-isomorphic. The following lemma and corollary say whenever this is not a problem.

    \begin{lemma}
        The following are equivalent:
        \begin{enumerate}
            \item Every null-homotopic chain complex is acyclic
            \item The category $\mathcal{A}$ is idempotent complete
            \item The subcategory $Ac(\mathcal{A})$ is closed under isomorphisms
        \end{enumerate}
    \end{lemma}

    \begin{corollary}
        The subcategory $Ac(\mathcal{A})$ is thick if and only if $\mathcal{A}$ is idempotent complete.
    \end{corollary}

    To weaken the conditions above, one can set boundedness conditions on the chain complexes. A chain complex is called left bounded if there is some $m:\mathbb{N}$ such that for any $n:\mathbb{N}$ and $n\leq m<0$ it is true that $A^n = 0$. Likewise, right bounded complexes are the defined for $n\geq m>0$ such that $A^n=0$. A chain complex is called bounded if it is both left bounded and right bounded.

    \begin{definition}
        The category $K(\mathcal{A})^+$,$K(\mathcal{A})^-$ and $K(\mathcal{A})^{\flat}$ are the homotopy categories of left bounded, right bounded and bounded respectively. $Ac(\mathcal{A})^* \subset K(\mathcal{A})^*$ for $*:\{+,-,\flat\}$ will be the subcategory of acyclic chain complexes satisfying the correct boundedness condition.
        
        Similarly one defines the (left/right) bounded derived category as $D^*(\mathcal{A}) = K^*(\mathcal{A})/Ac(\mathcal{A})^*$.
    \end{definition}
        
    \begin{lemma}
        The following are equivalent:
        \begin{enumerate}
            \item The subcategories $Ac(\mathcal{A})^* \subset K(\mathcal{A})^*$ for $*:\{+,-\}$ are thick
            \item The subcategory $Ac(\mathcal{A})^{\flat}$ is thick
            \item The category $\mathcal{A}$ is weakly idempotent complete
        \end{enumerate}
    \end{lemma}

\section{The Way Forward}

    At the end of this thesis, some topics involving derived categories will be looked. This is a sneak peek into the theory of derived functors and Auslander-Reiten triangles. This section is based on \cite{happel} and \cite{Kel96}. 

    \subsection{Derived Functors}

        % Motivate through abelian categories
        Suppose that $\mathcal{A}$ and $\mathcal{B}$ are abelian categories, and that there is an additive functor $F:\mathcal{A}\rightarrow\mathcal{B}$ between the categories. This functor can be lifted to an additive functor between the chain complex categories and a triangulated functor between the homotopy categories by applying the functor level-wise. Unfortunately, when localizing down to the derived categories there is no simple lift. As there is no way of assuring that a quasi-isomorphism gets mapped to another quasi-isomorphism, the lift will not be well-defined. This is however true whenever $F(Ac(\mathcal{A}))\subseteq Ac(\mathcal{B})$, which is the same as $F$ being an exact functor. As known in the classical sense of derived functors, is that they admit a left and right construction using projective and injective resolutions respectively.

        \begin{center}
            \begin{tikzcd}
                K(\mathcal{A}) \ar{r}{F} \ar{d}{q_{\mathcal{A}}} & K(\mathcal{B}) \ar{d}{q_{\mathcal{B}}} \\
                D(\mathcal{A}) \ar[dotted]{r}{?F} & D(\mathcal{B})
            \end{tikzcd}
        \end{center}

        % Give the construction of derived functors for any Verdier Quotient
        For a more general discussion, suppose that $\mathcal{T}$ and $\mathcal{S}$ are triangulated categories and that there is a triangulated subcategories $\mathcal{M}\subseteq\mathcal{T}$ such that the Verdier quotient exists. When does a triangulated functor $F:\mathcal{T}\rightarrow\mathcal{S}$ induce a functor $?F : \mathcal{T}/\mathcal{M}\rightarrow \mathcal{S}$? Similar to above, F admits a lift if every morphism in $Mor_{\mathcal{M}}$ gets inverted. This functor is induced by the universal property of $\mathcal{T}/\mathcal{M}$

        \begin{center}
            \begin{tikzcd}[column sep=small]
                \mathcal{T} \ar{rd}{q} \ar{rr}{F} & & \mathcal{S} \\
                & \mathcal{T}/\mathcal{M} \ar[dotted]{ru}{?F}
            \end{tikzcd}
        \end{center}

        To answer the question about derived functors, this thesis will follow Deligne's approach on how to construct right derived functors. Suppose there is an object $A:\mathcal{T}/\mathcal{M}$ and a functor $F:\mathcal{T}\rightarrow\mathcal{S}$, then let $\textbf{r}$$F(\_,A):\mathcal{S}^{op}\rightarrow Ab$ be a contravariant functor sending objects $B:\mathcal{S}$ to groups of semi left fractions $(f:\mathcal{S}(B,FC)|g:\mathcal{T}(A,C))$ such that $g:Mor_{\mathcal{M}}$. The caveat is that the functor sends objects of $\mathcal{T}/\mathcal{M}$ and $\mathcal{S}$ to arrows in $\mathcal{T}$ and $\mathcal{S}$. This is well-defined since $q(A)=A$ by definition and no choice of arrows need to be specified by the functor. Thus the localization is hidden from the functor at the element level. To illustrate how these elements look like the semi left fraction is to the left, and the equivalence relation on this set is given by the diagram on the right. In the same manner, as before, blue arrows denote arrows located in $Mor_{S}$.

        \begin{center}
            \begin{tikzcd}
                B \ar{r}{f} & FC & \\
                C & \ar[blue]{l}{g} A
            \end{tikzcd}
            \begin{tikzcd}
                & FC' \ar{d}{Fu} & C' \ar[blue]{d}{u} \\
                B \ar{ru}{f'} \ar{r}{f'''} \ar{rd}{f''} & FC''' & C''' & \ar[blue]{lu}{g'} \ar[blue]{l}{g'''} \ar[blue]{ld}{g''} A \\
                & FC'' \ar{u}{Fv} & C'' \ar[blue]{u}{v}
            \end{tikzcd}
        \end{center}
        
        The aim is to construct a functor $\mathbb{R}F:\mathcal{U}\rightarrow \mathcal{S}$ on a subcategory $\mathcal{U}$ of $\mathcal{T}/\mathcal{M}$. The functor $\mathbb{R}F$ is said to be defined at $A$ if the functor $\textbf{r} F(\_,A)$ is represented, i.e. $\mathcal{S}(\_,\mathbb{R}FA)\simeq$$\textbf{r}$$F(\_,A)$. This isomorphism also defines a canonical morphism $can_r:F\implies \mathbb{R}F\circ q$ given by the identity semi left fraction on every object $A:\mathcal{U}$.
        
        \begin{center}
            \begin{tikzcd}
                FA \ar[equal]{r} & FA & A & A \ar[blue, equal]{l}
            \end{tikzcd} $\iff$
            \begin{tikzcd}
                FA \ar{r}{can} & \mathbb{R}FA
            \end{tikzcd}
        \end{center}

        In order to show that $\mathbb{R}F$ is a functor, one must show that $\textbf{r}$$F(\_,A)$ is functorial at the index $A$. That is, suppose there is a left fraction $t^{-1}s:A\rightarrow A'$ in $\mathcal{T}/\mathcal{M}$, then for $(f|g)$ as described above one may define $\textbf{r}$$F(t^{-1}s)(f|g)=(Fs'\circ f|g't)$. $g'$ and $s'$ are defined with the right Ore condition on $Mor_{\mathcal{M}}$, this may be seen with the diagram below.

        \begin{center}
            \begin{tikzcd}
                FC'' & & C''\\
                FC' \ar{u}{Fs'} & C' \ar{ru}{s'} & & A'' \ar[blue]{lu}{g'} \\
                B \ar{u}{f} & & A \ar[blue]{lu}{g} \ar{ru}{s} & & A' \ar[blue]{lu}{t}
            \end{tikzcd}
        \end{center}

        Functoriality of $\textbf{r}$$F(\_,\_)$ shows that there is a functor $\mathbb{R}F$, where the action on maps is defined with the representative isomorphism.
        
        \begin{center}
            \begin{tikzcd}
                \mathcal{S}(\_,\mathbb{R}FA) \ar{r}{\simeq} \ar{d}{\mathbb{R}F(t^{-1}s)} & \textbf{r}F(\_,A) \ar{d}{\textbf{r}F(t^{-1}s)} \\
                \mathcal{S}(\_,\mathbb{R}FA') \ar{r}{\simeq} & \textbf{r}F(\_,A')
            \end{tikzcd}
        \end{center}
        
        If $\mathcal{U}$ is the full subcategory of objects which are defined for $\mathbb{R}F$, then there is a triangulated functor $\mathbb{R}F:\mathcal{U}\rightarrow \mathcal{S}$. $\mathcal{U}$ is in fact triangulated as $F$ is a triangulated functor, so it commutes with the autoequivalence $\Sigma_{\mathcal{T}}$. Moreover, if $I:q^{-1}\mathcal{U}\rightarrow \mathcal{T}$ is the inclusion from the preimage of $\mathcal{U}$ into $\mathcal{T}$, then the canonical morphism induces a morphism of triangulated functors $can:FI\implies\mathbb{R}FI$. \\

        Dually, one may define left derived functors using semi right fractions. As the set $Mor_{\mathcal{M}}$ is multiplicative, this doesn't make any problems. If the functor $F$ is in fact exact in the sense that $F$ sends morphisms in $Mor_{\mathcal{M}}$ to isomorphisms, then $\mathbb{R}F \simeq \mathbb{L}F$. If one would define derived functors between abelian categories $\mathcal{A}$ and $\mathcal{B}$, then the classical versions may be retrieved by taking homology, i.e. $\mathbb{R}^nF = H^n(\mathbb{R}F)$ and $\mathbb{L}_nF = H^{-n}(\mathbb{L}F)$. \\
        
        % Go back to the abelian case
        Suppose that the categories $\mathcal{A}$ and $\mathcal{B}$ have exact structures and that there is an additive functor $F:\mathcal{A}\rightarrow \mathcal{B}$, then there is a connection to the classical derived functors. Let $\mathcal{A}^I$ be the full subcategory of injective objects, since every conflation in $\mathcal{A}^I$ splits, $F|_{\mathcal{A}^{I}}:\mathcal{A}^I\rightarrow \mathcal{B}$ is an exact functor. Thus $F$ induced on $K(\mathcal{A})$ preserves quasi-isomorphisms of left bounded injective complexes. In this manner there is a functor $\mathbb{R}F:D^+(\mathcal{A}^I)\rightarrow D^+(\mathcal{B})$, and moreover this functor extends to the subcategory $\mathcal{U}\subseteq D^+(\mathcal{A})$ where every object in $\mathcal{U}$ admits an injective resolution. \\

    \subsection{Auslander-Reiten Triangles}

        For the rest of this thesis it will be assumed that $\mathcal{A}$ is an additive $\mathbb{K}$-linear category, that is it is enriched over $Mod \mathbb{K}$, and it is also Krull-Schmidt. With these assumptions, one may be able to define a special set of triangles whenever $\mathcal{A}$ has a triangulation. As $\mathcal{A}$ is Krull-Schmidt, the notion of indecomposable objects exists, which motivates the definition of triangles which acts as indecomposable triangles.

        \begin{definition}[Auslander-Reiten triangles]
            A triangle $(X,Y,Z,u,v,w)$ is called an Auslander-Reiten triangle if it satisfies the following conditions:
            \begin{enumerate}
                \item (AR1) $X$ and $Z$ are indecomposable
                \item (AR2) $w\neq 0$
                \item (AR3) If a morphism $f:X\rightarrow W$ is not split mono, then there is a morphism $f':Y\rightarrow W$, such that $f=f'u$
            \end{enumerate}
            Auslander-Reiten triangles will be abbreviated to AR-triangles
        \end{definition}

        A category $\mathcal{A}$ has enough AR-triangles if, for every indecomposable object, it is a part of a triangle satisfying the conditions above. One may show that the axioms for AR-triangles are self dual, that is $(X,Y,Z,u,v,w)$ is an AR-triangle if and only if $(X^{op},Y^{op},Z^{op},u^{op},v^{op},w^{op})$ is a triangle. Note that (AR2) forces the morphisms $u$ and $v$ to not be split. To see that these triangles represent irreducibility, the following definition is needed.

        \begin{definition}[Irreducible morphisms]
            A morphism $f:X\rightarrow Y$ is called irreducible if f is neither split mono nor split epi, and for any factorization $f=f_2f_1$ either $f_1$ is split mono or $f_2$ is split epi.
        \end{definition}

        \begin{prop}
            Let $(X,Y,Z,u,v,w)$ be an AR-triangle, then the following hold:
            \begin{enumerate}
                \item Z is unique up to isomorphism of triangles
                \item $u$ and $v$ are irreducible morphisms
                \item If $f:X\rightarrow X_1$ is irreducible, then there is a split epi $f':Y\rightarrow X_1$ such that $f=f'u$.
            \end{enumerate}
        \end{prop}

        The motivation for studying AR-triangles comes from their similarity with Auslander-Reiten sequences. An Auslander-Reiten sequence is a short exact sequence over a finite-dimensional $\mathbb{K}$-algebra, which is not split and satisfying (AR1) and (AR3). Thus one may observe that the definition of AR-triangles is exactly the stable Auslander-Reiten sequences. If $A$ is a finite-dimensional $\mathbb{K}$-algebra, then one has the following results.

        \begin{theorem}
            If $A$ is a finite-dimensional $\mathbb{K}$-algebra of finite global dimension, then the derived category $D^{\flat}(A)$ has enough AR-triangles.
        \end{theorem}

        \begin{prop}
            The following are equivalent:
            \begin{enumerate}
                \item $(X,Y,Z,u,v,w)$ is an AR-triangle in $D^{\flat}(A)$
                \item $inj.dim(X)\leq 1$ and $proj.dim Z \leq 1$
                \item $Hom_A(I,X)=0$ for any injective $I$ and $Hom_A(Z,P)=0$ for any projective $P$
            \end{enumerate}
        \end{prop}
    
% \section{Description of Derived Categories}

    % Add something to the introduction about derived categories. The goal is to describe the derived category of a fairly simple algebra.

%\clearpage

% \nocite{*}
% \bibliographystyle{unsrt}
% \bibliography{bib}