\chapter{Triangulated Categories}

In this chapter the notion of a triangulated will be introduced. In litterature one are able to find two different types of triangulated categories, they are called algebraic and topological. How these categories differ is however quite subtle. Different properties of triangulated categories will be discussed. Localization categories will be explained as well as how triangulated subcategories gives rise to such localizations. Lastly an embedding of a triangulated category in an abelian will be shown to exist.

\section{Definition and First Properties}
    \todo[noline]{Skriv om hva som skal være med i dette delkapittelet}

    In this section $\mathcal{T}$ denotes an additive category and $T:\mathcal{T}\rightarrow\mathcal{T}$ is an additive autoequivalence of $\mathcal{T}$, which is often called translation or suspension functor.
    % Definition candidate triangle
    \begin{definition}
        A candidate triangle is a collection $(A,B,C,a,b,c)$ of objects \\ $A,B,C\in T$ and morphisms $a:A\rightarrow B$, $b:B\rightarrow C$, $c:C\rightarrow TA$. These candidate triangles can be drawn as diagrams in the following way:

        \begin{center}
            \begin{tikzcd}
                A \arrow{r}{a} & B \arrow{r}{b} & C \arrow{r}{c} & TA
            \end{tikzcd}
        \end{center}

        A morphism between candidate triangles is a triple of morphism $(\alpha, \beta, \gamma)$, where $\alpha : A \rightarrow A'$, $\beta : B \rightarrow B'$ and $\gamma : C \rightarrow C'$ such that the following diagram commutes.

    \begin{center}
        \begin{tikzcd}
            A \arrow{r}{a} \arrow{d}{\alpha} & B \arrow{r}{b} \arrow{d}{\beta} & C \arrow{r}{c} \arrow{d}{\gamma} & TA \arrow{d}{T\alpha} \\
            A' \arrow{r}{a'} & B' \arrow{r}{b'} & C \arrow{r}{c'} & TA'
        \end{tikzcd}
    \end{center}

    \end{definition}

    \todo[noline]{Dette avsnittet trengs å skrives om, det gir ikke lenger mening :(}

    The naming convention of the candidate triangles isn't standarized, some literatures calls the candidate triangles for triangles instead; \todo[color=red]{Fotnote?} see \cite{keller}. This name arises from an alternate description of the diagrams given above. To remove confusion about the domain or codomain of the arrows, one arrow of the triangle is decorated with \todo[color=green]{få fram tydeligere hvordan pilene ser ut} "$_T$|". This decorator means that the functor T has to be applied to the corresponding edge of the arrow. With this notation the c arrow points to TA, not A.

    \begin{center}
        \begin{tikzcd}[row sep=tiny]
            A \arrow{rd}{a} & \\
            & B \arrow{dl}{b} & & \\
            C \arrow[very near end, "|" marking]{uu}[near start]{c}[near end]{T}
        \end{tikzcd}
        \begin{tikzcd}[row sep=tiny]
            A \arrow{rd}[description]{a} \arrow{rrr}{\phi_a} & & & A' \arrow[ld, "a'" description] \\
            & B \arrow{dl}[description]{b} \arrow{r}{\phi_b} & B' \arrow{rd}[description]{b'}\\
            C \arrow{uu}[very near end, marking]{|}[near start, description]{c}[near end]{T} \arrow{rrr}{\phi_c} & & & C' \arrow{uu}[very near end, marking]{|}[near start, description]{c'}[near end]{T}
        \end{tikzcd}
    \end{center}

    \todo[color=pink]{Kommer definisjonen av triangulerte kategorier for brått?} A triangulated category is an additive category together with a translation functor $T$ and a triangulation class $\Delta$ consisting of candidate triangles. When a candidate triangle is an element of $\Delta$ it is usually called a triangle, an exact triangle or a distuingished triangle. Note that if the candidate triangles are referred to as triangles it is common to either call the elements of $\Delta$ for exact triangles or distuingished triangles. In this thesis the elements of $\Delta$ will be called for triangles.

    % Definition of Triangulation
    \begin{definition}
        A triangulation of an additive category $\mathcal{T}$ with translation $T$ is a collection $\Delta$ of triangles consisting of candidate triangles in $\mathcal{T}$ satisfying the following axioms: 

        \begin{enumerate}
            \item (TR1) Bookeeping axiom

                \begin{enumerate}
                    \item A candidate triangle isomorphic to a triangle is a triangle.
                    \item Every morphism $a : A \rightarrow B$ can be embedded into a triangle $(A,B,C,a,b,c)$.
                    \begin{center}
                        \begin{tikzcd}
                            A \arrow{r}{a} & B \arrow{r}{b} & C \arrow{r}{c} & TA
                        \end{tikzcd}
                    \end{center}
                    \item For every object A there is a triangle $(A,A,0,id_A,0,0)$.
                    \begin{center}
                        \begin{tikzcd}
                            A \arrow{r}{id_A} & A \arrow{r}{0} & 0 \arrow{r}{0} & TA
                        \end{tikzcd}
                    \end{center}
                \end{enumerate}
            \item (TR2) Rotation axiom

                For every triangle $(A,B,C,a,b,c)$ there is a triangle $(B,C,TA,b,c,-Ta)$.
                \begin{center}
                    \begin{tikzcd}
                        A \arrow{r}{a} & B \arrow{r}{b} & C \arrow{r}{c} & TA
                    \end{tikzcd} $\implies$
                    \begin{tikzcd}
                        B \arrow{r}{b} & C \arrow{r}{c} & TA \arrow{r}{-Ta} & TB
                    \end{tikzcd}
                    
                \end{center}
            \item (TR3) Morphism axiom
            
                Given the two triangles $(A,B,C,a,b,c)$ (1) and $(A',B',C',a',b',c')$ (2)
                \begin{center}
                    (1)
                    \begin{tikzcd}[column sep=small]
                        A \ar{r}{a} & B \ar{r}{b} & C \ar{r}{c} & TA
                    \end{tikzcd}
                    (2) 
                    \begin{tikzcd}[column sep=small]
                        A' \ar{r}{a'} & B' \ar{r}{b'} & C' \ar{r}{c'} & TA'
                    \end{tikzcd}
                \end{center}
                and morphisms $\phi_A : A \rightarrow A'$ and $\phi_B : B \rightarrow B'$ such that the square (1) commutes, then there is a morphism $\phi_C : C \rightarrow C'$ (not necessarily unique) such that $(\phi_A ,\phi_B ,\phi_C)$ is a morphism of triangles (2).
                
                \begin{center}
                    (1)
                    \begin{tikzcd}
                        A \ar{r}{a} \ar{d}{\phi_A} & B \ar{d}{\phi_B} & \\
                        A' \ar{r}{a'} & B'
                    \end{tikzcd}
                    (2)
                    \begin{tikzcd}
                        A \ar{r}{a} \ar{d}{\phi_A} & B \ar{r}{b} \ar{d}{\phi_B} & C \ar{r}{c} \ar[dashed]{d}{\phi_C} & TA \ar{d}{T\phi_A} \\
                        A' \ar{r}{a'} & B' \ar{r}{b'} & C' \ar{r}{c'} & TA'
                    \end{tikzcd}
                \end{center}
            \item (TR4) Octahedron axiom
            
                Given the triangles $(A,B,C',a,x,x')$ (1), $(B,C,A',b,y,y')$ (2) \\ and $(A,C,B',b\circ a,z,z')$ (3)
                \begin{center}
                    (1)
                    \begin{tikzcd}[column sep=small]
                        A \ar{r}{a} & B \ar{r}{x} & C' \ar{r}{x'} & TA
                    \end{tikzcd}

                    (2)
                    \begin{tikzcd}[column sep=small]
                        B \ar{r}{b} & C \ar{r}{y} & A' \ar{r}{y'} & TB
                    \end{tikzcd}

                    (3)
                    \begin{tikzcd}[column sep=small]
                        A \ar{r}{b\circ a} & C \ar{r}{z} & B' \ar{r}{z'} & TA
                    \end{tikzcd}                         
                \end{center}
                then there exist morphisms $f : C' \rightarrow B'$ and $g : B' \rightarrow A'$, the following diagram commutes and the third row is a triangle.

                \begin{center}
                    \begin{tikzcd}
                        T^{-1}B' \ar{r}{T^{-1}z'} \ar{d}{T^{-1}g} & A \ar[equal]{r}{id_A} \ar{d}{a} & A \ar{d}{b\circ a} \\
                        T^{-1}A' \ar{r}{T^{-1}y'} & B \ar{r}{b} \ar{d}{x} & C \ar{r}{y} \ar{d}{z} & A' \ar{r}{y'} \ar[equal]{d}{id_{A'}} & TB \ar{d}{Tx'} \\
                        & C' \ar{r}{f} \ar{d}{x'} & B' \ar{r}{g} \ar{d}{z'} & A' \ar{r}{Ti \circ y'} & TC' \\
                        & TA \ar[equal]{r}{id_{TA}} & TA
                    \end{tikzcd}
                \end{center}
        \end{enumerate}
    \end{definition}

    A triangulated category is denoted as $(\mathcal{T}, T, \Delta)$, where $\mathcal{T}$ is the additive category, $T$ is the translation and $\Delta$ is the triangulation. When $\mathcal{T}$ is called a triangulated category, it should be understood as the triple given above.
    
    \begin{remark}
        The third object in a triangle is usually called cone, fiber or cofiber. These names are in use due to historic reasons, rather than portraying their functionality. The names weak kernel or weak cokernel would be better in the sense that it tells what the function of this object is. In this thesis it will either be referred to as cone, weak kernel or weak cokernel.
    \end{remark}
    % Rotation axiom dual
    \begin{remark}
        The rotation axiom has a dual, and it can be thought of as a shift in the opposite direction. The dual roation axiom goes as:
        
        \begin{quote}
            Given a triangle \begin{tikzcd}[column sep=small]
                A \ar{r}{a} & B \ar{r}{b} & C \ar{r}{c} & TA
            \end{tikzcd},\\
            there is a triangle \begin{tikzcd}[column sep=small]
                T^{-1}C \ar{r}{-T^{-1}c} & A \ar{r}{a} & B \ar{r}{b} & C
            \end{tikzcd}
        \end{quote} %Hvordan fikser jeg linebreaks for at det blir pent??? Mener at denne skal kun brukes innad i en setning, idk man.
        
        To be able to prove this, some more lemmata are needed.
    \end{remark}

    \begin{remark}
        By the previous remark one may see that the definition of a triangulated category is self dual. That is a category $\mathcal{T}$ is triangulated if and only if $\mathcal{T}^{op}$ is triangulated.
    \end{remark}

    % Octahedron axiom alternate
    \begin{remark}
        The final axiom is referred to as the octahedron axiom. By using the alternative description of the triangle diagram, it is possible to rewrite the diagram as an octahedron. The axiom can be restated as the following.

        \begin{quote}
            Given the triangles $(A,B,C',a,x,x')$ (1), $(B,C,A',b,y,y')$ (2) \\ and $(A,C,B',b\circ a,z,z')$ (3)
            \begin{center}
                (1)
                \begin{tikzcd}[row sep=tiny]
                    A \arrow[red]{rd}[black]{a} & \\
                    & B \arrow[red]{dl}[black]{x} & & \\
                    C' \arrow[red, very near end, "|" marking]{uu}[near start, black]{x'}[near end]{T}
                \end{tikzcd}
                (2)
                \begin{tikzcd}[row sep=tiny]
                    B \arrow[orange]{rd}[black]{b} & \\
                    & C \arrow[orange]{dl}[black]{y} & & \\
                    A' \arrow[orange, very near end, "|" marking]{uu}[near start, black]{y'}[near end]{T}
                \end{tikzcd}\\                
                (3)
                \begin{tikzcd}[row sep=tiny]
                    A \arrow[violet]{rd}[black]{b\circ a} & \\
                    & C \arrow[violet]{dl}[black]{z} & & \\
                    B' \arrow[violet, very near end, "|" marking]{uu}[near start, black]{z'}[near end]{T}
                \end{tikzcd}
            \end{center}
            then there exists morphisms $f: C' \rightarrow B'$ and $g: B' \rightarrow A'$, the following diagram commutes and the squiggly teal back face is a triangle.
            \begin{center}
                \begin{tikzcd}[row sep=tiny]
                    \color{white}.\color{black} & & B' \ar[teal, dashed, squiggly]{ddddr}[black, description]{g} \ar[violet]{dddddl}[black, description]{z'}[pos=0.9, marking]{|}[pos=0.91]{T} & & \\
                    \textcolor{white}{.} \\
                    \textcolor{white}{.} \\
                    \textcolor{white}{.} \\
                    C' \ar[teal, squiggly]{uuuurr}[black, description]{f} \ar[red]{dr}[black, description]{x'}[very near end, marking]{|}[near end]{T} & & & A' \ar[teal, dashed, squiggly]{lll}[pos=0.45, black, description]{Tx\circ y'}[pos=0.91, marking]{|}[very near end, above]{T} \ar[orange, dashed]{dddddl}[black, description]{y'}[pos=0.9, marking]{|}[pos=0.89, above]{T} \\
                    \color{white}.\color{black} & A \ar[red]{ddddr}[black, description]{a} \ar[violet]{rrr}[black, description]{b\circ a} & & & C \ar[orange, dashed]{ul}[black, description]{y} \ar[violet]{uuuuull}[black, description]{z}\\
                    \textcolor{white}{.} \\
                    \textcolor{white}{.} \\
                    \textcolor{white}{.} \\
                    & & B \ar[orange]{uuuurr}[black, description]{b} \ar[red]{uuuuull}[black, description]{x} & &
                    
                \end{tikzcd}
            \end{center}
        \end{quote}
    \end{remark}


    \begin{prop}
        The axiom TR3 can be proven from TR1 and TR4.
    \end{prop}


    \begin{proof}
        Suppose that there are two triangles and a commutative square as follows.
        \begin{center}
            \begin{tikzcd}
                A \ar{r}{a} \ar{d}{\phi_A} \ar{rd}{\eta} & B \ar{d}{\phi_B} & \\
                A' \ar{r}{a'} & B'
            \end{tikzcd}
            \begin{tikzcd}
                A \ar{r}{a} \ar{d}{\phi_A} & B \ar{r}{b} \ar{d}{\phi_B} & C \ar{r}{c} & TA \ar{d}{T\phi_A} \\
                A' \ar{r}{a'} & B' \ar{r}{b'} & C' \ar{r}{c'} & TA'
            \end{tikzcd}
        \end{center}
        The upper and lower simplex of the square may be completed to two sets of triangles satisfying the condition of TR4. Applying the Octahedron axiom twice yields the diagrams as below.
        \begin{center}
            (1)
            \begin{tikzcd}[row sep=tiny]
                A \ar[red]{rd}[black]{a} \\
                & B \ar[red]{ld}[black]{b} \\
                C \ar[red, very near end, "|" marking]{uu}[black, near start]{c}[near end]{T}
            \end{tikzcd}
            \begin{tikzcd}[row sep=tiny]
                B \ar[orange]{rd}[black]{\phi_B} \\
                & B' \ar[orange]{ld}[black]{\phi_B'} \\
                B'' \ar[orange, very near end, "|" marking]{uu}[black, near start]{\phi_B''}[near end]{T}
            \end{tikzcd}
            \begin{tikzcd}[row sep=tiny]
                A \ar[violet]{rd}[black]{\eta} \\
                & B' \ar[violet]{ld}[black]{\eta'} \\
                E \ar[violet, very near end, "|" marking]{uu}[black, near start]{{\eta}''}[near end]{T}
            \end{tikzcd}\\
            (2)
            \begin{tikzcd}[row sep=tiny]
                A \ar[red]{rd}[black]{\phi_A} \\
                & A' \ar[red]{ld}[black]{\phi_A'} \\
                A'' \ar[red, very near end, "|" marking]{uu}[black, near start]{\phi_A''}[near end]{T}
            \end{tikzcd}
            \begin{tikzcd}[row sep=tiny]
                A' \ar[orange]{rd}[black]{a'} \\
                & B' \ar[orange]{ld}[black]{b'} \\
                C' \ar[orange, very near end, "|" marking]{uu}[black, near start]{c'}[near end]{T}
            \end{tikzcd}
            \begin{tikzcd}[row sep=tiny]
                A \ar[violet]{rd}[black]{\eta} \\
                & B' \ar[violet]{ld}[black]{\eta'} \\
                E \ar[violet, very near end, "|" marking]{uu}[black, near start]{{\eta}''}[near end]{T}
            \end{tikzcd}
        \end{center}
        \begin{minipage}[t]{0.47\textwidth}
            \begin{center}
                (1) \\
                \begin{tikzcd}[row sep=tiny]
                    \textcolor{white}{.} & & E \ar[teal, dashed]{ddddr}[black, description]{g} \ar[violet]{dddddl}[black, description]{{\eta}''}[pos=0.9, marking]{|}[pos=0.91]{T} & & \\
                    \textcolor{white}{.} \\
                    \textcolor{white}{.} \\
                    \textcolor{white}{.} \\
                    C \ar[teal, squiggly]{uuuurr}[black, description]{f} \ar[red]{dr}[black, description]{c}[very near end, marking]{|}[near end]{T} & & & B'' \ar[teal, dashed]{lll}[pos=0.45, black, description]{Tc\circ \phi_B''}[pos=0.91, marking]{|}[very near end, above]{T} \ar[orange, dashed]{dddddl}[black, description]{\phi_B''}[pos=0.9, marking]{|}[pos=0.89, above]{T} \\
                    \textcolor{white}{.} & A \ar[red]{ddddr}[black, description]{a} \ar[violet]{rrr}[black, description]{\eta} & & & B' \ar[orange, dashed]{ul}[black, description]{\phi_B'} \ar[violet]{uuuuull}[black, description]{{\eta}'}\\
                    \textcolor{white}{.} \\
                    \textcolor{white}{.} \\
                    \textcolor{white}{.} \\
                    & & B \ar[orange]{uuuurr}[black, description]{\phi_B} \ar[red]{uuuuull}[black, description]{b} & &
                \end{tikzcd}
            \end{center}
        \end{minipage}
        \begin{minipage}[t]{0.48\textwidth}
            \begin{center}
                (2) \\
                \begin{tikzcd}[row sep=tiny]
                    \textcolor{white}{.} & & E \ar[teal, dashed, squiggly]{ddddr}[black, description]{g'} \ar[violet]{dddddl}[black, description]{{\eta}''}[pos=0.9, marking]{|}[pos=0.91]{T} & & \\
                    \textcolor{white}{.} \\
                    \textcolor{white}{.} \\
                    \textcolor{white}{.} \\
                    A'' \ar[teal]{uuuurr}[black, description]{f'} \ar[red]{dr}[black, description]{\phi_A''}[very near end, marking]{|}[near end]{T} & & & C' \ar[teal, dashed]{lll}[pos=0.45, black, description]{T\phi_A''\circ c'}[pos=0.91, marking]{|}[very near end, above]{T} \ar[orange, dashed]{dddddl}[black, description]{c'}[pos=0.9, marking]{|}[pos=0.89, above]{T} \\
                    \textcolor{white}{.} & A \ar[red]{ddddr}[black, description]{\phi_A} \ar[violet]{rrr}[black, description]{\eta} & & & B' \ar[orange, dashed]{ul}[black, description]{b'} \ar[violet]{uuuuull}[black, description]{{\eta}'}\\
                    \textcolor{white}{.} \\
                    \textcolor{white}{.} \\
                    \textcolor{white}{.} \\
                    & & A' \ar[orange]{uuuurr}[black, description]{a'} \ar[red]{uuuuull}[black, description]{\phi_A'} & &
                \end{tikzcd}
            \end{center}
        \end{minipage} \\
        The teal squiggly lines at the back faces of each octahedra forms a morphism $g'f:C\rightarrow C'$. It remains to see that the morphism is a triangle morphism. Diagram chasing reveals that the following diagram is commutative, which is exactly the requirement for the collection $(\phi_A,\phi_B,g'f)$ to be a morphism of triangles.
        \begin{center}
            \begin{tikzcd}
                B \ar[red]{r}[black]{b} \ar[orange]{d}[black]{\phi_B'}& C \ar[red]{rd}[black]{c} \ar[teal]{d}[black]{f} \\
                B' \ar[orange]{rd}[black]{b'} \ar[violet]{r}[black]{\eta '} & E \ar[violet]{r}[black]{\eta ''} \ar[teal]{d}[black]{g'} & TA \ar[red]{d}[black]{T\phi_A} \\
                & C' \ar[orange]{r}[black]{c'} & TA'
            \end{tikzcd}
        \end{center}
    \end{proof}

    \begin{lemma}
        Let $(A,B,C,a,b,c)$ be a triangle, then $b\circ a=0$
    \end{lemma}

    \begin{proof}
        By TR2 the triangle $(A,B,C,a,b,c)$ can be rotated to $(B,C,TA,b,c,-Ta)$.
        \begin{center}
            \begin{tikzcd}[row sep=tiny]
                A \arrow{rd}{a} & \\
                & B \arrow{dl}{b}\\
                C \arrow[very near end, "|" marking]{uu}[near start]{c}[near end]{T}
            \end{tikzcd} $\implies$
            \begin{tikzcd}[row sep=tiny]
                B \arrow{rd}{b} \\
                & C \arrow{dl}{c} \\
                TA \arrow{uu}[near start]{-Ta}[very near end, marking]{|}[near end]{T}
            \end{tikzcd}
        \end{center}
        The triangle $(C,C,0,id_C,0,0)$ exists by TR1 and TR3 states that there exists a morphism from TA to 0 making the diagram below commute.
        \begin{center}
            \begin{tikzcd}
                B \ar{r}{b} \ar{d}{b} & C \ar{r}{c} \ar{d}{id_C} & TA \ar{r}{-Ta} \ar[dashed]{d}{0} & TB \ar{d}{Tb} \\
                C \ar{r}{id_C} & C \ar{r}{0} & 0 \ar{r}{0} & TC
            \end{tikzcd}
        \end{center}
        Thus $0 = Tb\circ -Ta = T(-ba) \implies b\circ a = 0$ as T is a translation.
    \end{proof}
    % Triangulatd functor
    \todo{Burde jeg introdusere triangulerte funktorer, eller går det greit å bare definere de brått?}
    \begin{definition}
        An additive functor between triangulated categories $F: (\mathcal{T}, T, \Delta) \rightarrow (\mathcal{R}, R, \Gamma)$ is called triangulated if there exist a natural isomorphisms $\alpha : FT \rightarrow RF$ such that $F(\Delta) \subseteq \Gamma$.

        A functor $F : \mathcal{T} \rightarrow \mathcal{R}$ is called a triangle-equivalence if it is triangulated and an equivalence of categories. In this case $\mathcal{T}$ and $\mathcal{R}$ are called triangle-equivalent.
    \end{definition}

    \begin{remark}
        Some literatures refer to triangulated functors as exact functor. \todo[color = red]{Trenger kilder her}
    \end{remark}
    % Homological functor
    \begin{definition}
        Let $\mathcal{T}$ be a triangulated category and $\mathcal{A}$ be an abelian category. A covariant functor $H:\mathcal{T} \rightarrow \mathcal{A}$ is called homological if $\forall (A,B,C,a,b,c):\Delta$ there is a long exact sequence in $\mathcal{A}$.
        \begin{center}
            \begin{tikzcd}[row sep=tiny]
                A \arrow{rd}{a} & \\
                & B \arrow{dl}{b}\\
                C \arrow[very near end, "|" marking]{uu}[near start]{c}[near end]{T}
            \end{tikzcd} $\implies$
            \begin{tikzcd}[column sep=small]
                ... \ar{r} & H(T^{i}A) \arrow{r}{H(T^ia)} & H(T^iB)\arrow{r}{H(T^ib)} \arrow[d,phantom, ""{coordinate, name=Z}]& H(T^iC) \arrow[dll, "H(T^ic)" description, rounded corners,to path={ --([xshift=2ex]\tikztostart.east)|- (Z)[near end]\tikztonodes-| ([xshift=-2ex]\tikztotarget.west)-- (\tikztotarget)}] \\
                & H(T^{i+1}A) \arrow{r}{H(T^{i+1}a)} & H(T^{i+1}B) \arrow{r}{H(T^{i+1}b)} & H(T^{i+1}C) \ar{r} & ...
            \end{tikzcd}
        \end{center}

        Dually, a contravariant functor $H:\mathcal{T} \rightarrow \mathcal{A}$ is called cohomological if $\forall (A,B,C,a,b,c):\Delta$ there is a long exact sequence in $\mathcal{A}$.
        \begin{center}
            \begin{tikzcd}[row sep=tiny]
                A \arrow{rd}{a} & \\
                & B \arrow{dl}{b}\\
                C \arrow[very near end, "|" marking]{uu}[near start]{c}[near end]{T}
            \end{tikzcd} $\implies$
            \begin{tikzcd}[column sep=small]
                ... & H(T^{i-1}A) \arrow{l} & H(T^{i-1}B) \arrow{l}{H(T^{i-1}a)} \arrow[d,phantom, ""{coordinate, name=Z}]& H(T^{i-1}C) \ar{l}{H(T^{i-1}b)} \\
                & H(T^{i}A) \arrow[urr, "H(T^ic)" description, rounded corners,to path={ --([xshift=-2ex]\tikztostart.west)|- (Z)[near end]\tikztonodes-| ([xshift=2ex]\tikztotarget.east)-- (\tikztotarget)}] & H(T^{i}B) \ar{l}{H(T^ia)} & H(T^{i}C) \ar{l}{H(T^ib)} & ... \ar{l}
            \end{tikzcd}
        \end{center}
    \end{definition}
    % Long exact sequence of representations
    \begin{lemma}
        Let $M:\mathcal{T}$ be any object of $\mathcal{T}$, then the represented functor $\mathcal{T}(M,\_)$ is homological and $\mathcal{T}(\_,M)$ is cohomological.
    \end{lemma}

    \begin{proof}
        Only the covariant case needs to be proved, as the contravariant case is dual. For $\mathcal{T}(M,\_)$ to be homological, it has to create long exact sequences for every triangle in $\Delta$. Let $(A,B,C,a,b,c):\Delta$ be a triangle, then sequences in Ab can be extracted for any $i:\mathbb{Z}$.

        \begin{center}
            \begin{tikzcd}[row sep=tiny]
                A \ar{dr}{a} \\
                & B \ar{dl}{b} \\
                C \ar{uu}{c}[near end]{T}[very near end, marking]{|}
            \end{tikzcd} $\implies$
            \begin{tikzcd}
                \mathcal{T}(M,T^iA) \ar{r}{T^ia_*} & \mathcal{T}(M,T^iB) \ar{r}{T^ib_*} & \mathcal{T}(M,T^iC)
            \end{tikzcd}
        \end{center}

        It is enough to prove that these types of diagrams are exact at $B$, as the other diagrams are obtained by the rotation axiom. Thus it remains to prove that $Im(T^ia_*)=Ker(T^ib_*)$. Since $ba=0$ it follows that $Im(T^ia_*) \subseteq Ker(T^ib_*)$. Assume that $f:Ker(T^ib_*)$, that is $f:M\rightarrow T^iB$ such that $b_*(f)=0$. Showing that $f$ factors through $T^iA$ proves exactness, as this means that $Ker(T^ib_*)\subseteq Im(T^ia_*)$. Note that since $T$ is a translation, it is necessarily a right adjoint to the inverse translation; $\mathcal{T}(M,T^iB) \simeq\mathcal{T}(T^{-i}M,B)$ and by this assertion it suffices to assume that $f:T^{-i}M\rightarrow B$ such that $b\circ f = 0$. By TR1 and TR2 there exists triangles $(T^{-i}M,0,T^{-i+1}M,0,0,-T^{-i+1}id)$ and $(B,C,TA,b,c,-Ta)$. 
        \begin{center}
            \begin{tikzcd}
                T^{-i}M \ar{r}{0} \ar{d}{f} & 0 \ar{r}{0} \ar{d}{0} & T^{-i+1}M \ar{r}{-T^{-i+1}id} \ar[dashed]{d}{g} & T^{-i+1}M \ar{d}{Tf} \\
                B \ar{r}{b} & C \ar{r}{c} & TA \ar{r}{-Ta} & TB
            \end{tikzcd}
        \end{center}
        The left square commutes by the assumption, thus the morphism g exist by TR3 such that $-Ta\circ h = -Tf\circ T^{-i+1}id = -Tf \implies Ta\circ h = Tf$. This shows that $f = a\circ T^{-1}h$, asserting that $f$ factors through A.
    \end{proof}
    % 2 out of 3 property
    \todo{Her kan det også være veldig naturlig å skrive noe. Sammenligne med 5-lemma er vel det jeg tenker på.}
    \begin{lemma}
        Let $(\phi_A, \phi_B, \phi_C):(A,B,C,a,b,c) \rightarrow (A',B',C',a',b',c')$ be a morphism of triangles. If 2 of the maps are isomorphisms, then the last one is an isomorphism as well.
        \begin{center}
            \begin{tikzcd}
                A \ar{r}{a} \ar{d}{\phi_A}[rotate=90, above]{\simeq} & B \ar{r}{b} \ar{d}{\phi_B}[rotate=90, above]{\simeq} & C \ar{r}{c} \ar[dashed]{d}{\phi_C}[rotate=90, above]{\simeq} & TA \ar{d}{T\phi_A}[rotate=90, above]{\simeq} \\
                A' \ar{r}{a'} & B' \ar{r}{b'} & C' \ar{r}{c'} & TA'
            \end{tikzcd}
        \end{center}
    \end{lemma}

    \begin{proof}
        Without loss of generality, assume that $\phi_A$ and $\phi_B$ are the isomorphisms. This can be done as the rotation axiom reduce the other cases to this case. Then the diagram depicted below exists.
        \begin{center}
            \begin{tikzcd}
                A \ar{r}{a} \ar{d}{\phi_A}[rotate=90, above]{\simeq} & B \ar{r}{b} \ar{d}{\phi_B}[rotate=90, above]{\simeq} & C \ar{r}{c} \ar{d}{\phi_C} & TA \ar{d}{T\phi_A}[rotate=90, above]{\simeq} \\
                A' \ar{r}{a'} & B' \ar{r}{b'} & C' \ar{r}{c'} & TA'
            \end{tikzcd}
        \end{center}
        Applying the functor $\mathcal{T}(C',\_)$ to the diagram yields the following diagram in Ab:
        \begin{center}
            \begin{tikzcd}
                \mathcal{T}(C',A) \ar{r}{a_*} \ar{d}{(\phi_A)_*}[rotate=90, above]{\simeq} & \mathcal{T}(C',B) \ar{r}{b_*} \ar{d}{(phi_B)_*}[rotate=90, above]{\simeq} & \mathcal{T}(C',C) \ar{r}{c_*} \ar{d}{(\phi_C)_*} & \mathcal{T}(C',TA) \ar{r}{Ta_*} \ar{d}{(\phi_TA)_*}[rotate=90, above]{\simeq} & \mathcal{T}(C',TB) \ar{d}{(T\phi_B)_*}[rotate=90, above]{\simeq} \\
                \mathcal{T}(C',A') \ar{r}{a'_*} & \mathcal{T}(C',B') \ar{r}{b'_*} & \mathcal{T}(C',C') \ar{r}{c'_*} & \mathcal{T}(C',TA') \ar{r}{Ta_*} & \mathcal{T}(C',TB)
            \end{tikzcd}
        \end{center}
        By the 5-lemma, we get that $(\phi_C)_*$ is an isomorphisms, i.e. $(\phi_C)_*$ is both mono and epi. Thus for some unique $s$ in $\mathcal{T}(C',C)$, ${\phi_C}_*(s)=id_{C'}$. 

        By applying the functor $\mathcal{T}(\_,C)$ we get the diagram:
        \begin{center}
            \begin{tikzcd}
                \mathcal{T}(A,C) & \mathcal{T}(B,C) \ar{l}{a^*} & \mathcal{T}(C,C) \ar{l}{b^*} & \mathcal{T}(TA,C) \ar{l}{c^*} & \mathcal{T}(TB,C) \ar{l}{Ta^*} \\
                \mathcal{T}(A',C) \ar{u}{(\phi_A)^*}[rotate=90, below]{\simeq} & \mathcal{T}(B,C) \ar{l}{a'^*} \ar{u}{(\phi_B)^*}[rotate=90, below]{\simeq} & \mathcal{T}(C',C) \ar{l}{b'^*} \ar{u}{(\phi_C)^*} & \mathcal{T}(TA',C) \ar{l}{c'^*} \ar{u}{(\phi_TA)^*}[rotate=90, below]{\simeq} & \mathcal{T}(TB',C) \ar{l}{Ta'^*} \ar{u}{(\phi_TB)^*}[rotate=90, below]{\simeq}
            \end{tikzcd}
        \end{center}
        Again, the 5-lemma asserts that $(\phi_C)^*$ is an isomorphisms, and by the same argument $id_{C} = s'\circ\phi_C$ for some unique $s'$. $\phi_C$ is both split mono and split epi, which means it is an isomorphism.
    \end{proof}

    \begin{corollary}
        $(A,B,0,a,0,0)$ is a triangle if and only if a is an isomorphism.
    \end{corollary}

    \begin{proof}
        Assume that a is an isomorphism. Then it is seen that $(a,id_B,0)$ is an isomorphism of triangles.
        \begin{center}
            \begin{tikzcd}
                A \ar{r}{a} \ar{d}{a}[rotate=90, above]{\simeq} & B \ar{r}{0} \ar{d}{id_B}[rotate=90, above]{\simeq} & 0 \ar{r}{0} \ar{d}{0}[rotate=90, above]{\simeq} & TA \ar{d}{Ta}[rotate=90, above]{\simeq} \\
                B \ar{r}{id_B} & B \ar{r}{0} & 0 \ar{r}{0} & TB
            \end{tikzcd}
        \end{center}
        Converesly, assume that $(A,B,0,a,0,0)$ is a triangle. Then the same diagram as above can be constructed, and by the 2 out of 3 property, a has to be an isomorphism.
    \end{proof}

    \begin{lemma}
        For a triangle $(A,B,C,a,b,c)$ the following are equivalent:

        \begin{center}
            \begin{minipage}[c]{0.3\textwidth}
                \begin{tikzcd}[row sep=tiny]
                    A \arrow{rd}{a} & \\
                    & B \arrow{dl}{b} & & \\
                    C \arrow[very near end, "|" marking]{uu}[near start]{c}[near end]{T}
                \end{tikzcd}
            \end{minipage}
            \begin{minipage}[c]{0.3\textwidth}
                \begin{itemize}
                    \item $a$ is split mono
                    \item $b$ is split epi
                    \item $c = 0$
                \end{itemize}
            \end{minipage}
        \end{center}
    \end{lemma}

    \begin{proof}
        The proof has two parts. First assume that $a$ is split mono, then prove that $b$ is split epi and $c = 0$. By duality, it is then known that $b$ being split epi implies that $a$ is split mono and $c = 0$. The final part is to assume that $c = 0$, and prove either $a$ is split mono or $b$ is split epi.

        Assume that $a$ is split mono, then there exist an $a^{-1}$ such that $id_A = a^{-1}a$. Let $M:\mathcal{T}$ be any object, then there is a long exact sequence.
        \begin{center}
            \begin{tikzcd}
                \mathcal{T}(M,T^{-1}C) \ar{r}{T^{-1}c_*} & \mathcal{T}(M,A) \ar[bend left]{r}{a_*} & \mathcal{T}(M,B) \ar{r}{b_*} \ar[dashed, bend left]{l}{a^{-1}_*} & \mathcal{T}(M,C) \ar{r}{c_*} & \mathcal{T}(M,TA)
            \end{tikzcd}
        \end{center}
        By assumption $a_*$ is split mono, thus $T^{-1}c_* = 0$ and in particular $c = 0$. This implies that $b_*$ is epi, making a split short exact sequence.
        \begin{center}
            \begin{tikzcd}
                0 \ar{r}{0} & \mathcal{T}(M,A) \ar[bend left]{r}{a_*} & \mathcal{T}(M,B) \ar[bend left]{r}{b_*} \ar[dashed, bend left]{l}{a^{-1}_*} & \mathcal{T}(M,C) \ar{r}{0} \ar[dashed, bend left]{l}{b^{-1}_*} & 0
            \end{tikzcd}
        \end{center}
        This shows that b is split epi, completing the first part of the proof.

        For the final part, assume that $c = 0$; construct the following triangles.
        \begin{center}
            (1)
            \begin{tikzcd}[row sep=tiny]
                A \arrow{rd}{a} \\
                & B \arrow{dl}{b} \\
                C \arrow[very near end, "|" marking]{uu}[near start]{0}[near end]{T}
            \end{tikzcd} $\implies$
            \begin{tikzcd}[row sep=tiny]
                C \arrow{rd}{0} \\
                & TA \arrow{dl}{-Ta} \\
                TB \arrow[very near end, "|" marking]{uu}[near start]{-Tb}[near end]{T}
            \end{tikzcd}
        \end{center}
        \begin{center}
            (2)
            \begin{tikzcd}[row sep=tiny]
                A \arrow{rd}{id_A} \\
                & A \arrow{dl}{0} \\
                0 \arrow[very near end, "|" marking]{uu}[near start]{0}[near end]{T}
            \end{tikzcd} $\implies$
            \begin{tikzcd}[row sep=tiny]
                0 \arrow{rd}{0} & \\
                & TA \arrow{dl}{-id_{TA}}\\
                TA \arrow[very near end, "|" marking]{uu}[near start]{0}[near end]{T}
            \end{tikzcd}
        \end{center}
        \todo{Dette kan jeg nok utdype mer, slik at det ikke er så forvirrende} (1) is constructed by applying TR2 twice, while (2) is constructed with TR1 and TR2 twice. Observe that there is a commutative square between the triangles, allowing for TR3 to make a morphism of triangles.
        \begin{center}
            \begin{tikzcd}
                C \ar{r}{0} \ar{d}{0} & TA \ar{r}{-Ta} \ar[equal]{d}{id_{TA}} & TB \ar{r}{-Tb} \ar[dashed]{d}{Ta^{-1}} & TC \ar{d}{0} \\
                0 \ar{r}{0} & TA \ar[equal]{r}{-id_{TA}} & TA \ar{r}{0} & 0
            \end{tikzcd}
        \end{center}
        Thus $T(a^{-1}a)=id_{TA}=T(id_A) \implies id_A = a^{-1}a$, making a split mono.
    \end{proof}

    \begin{lemma}
        Given two triangles $(A,B,C,a,b,c)$ and $(A',B',C',a',b',c')$ the following are equivalent:
        \begin{center}
            \begin{minipage}[c]{0.4\textwidth}
                \begin{tikzcd}
                    A \ar{r}{a} \ar{d}{f} & B \ar{r}{b} \ar{d}{g} & C \ar{r}{c} \ar{d}{h} & TA \ar{d}{Tf} \\
                    A' \ar{r}{a'} & B' \ar{r}{b'} & C' \ar{r}{c'} & TA'
                \end{tikzcd}
            \end{minipage}
            \begin{minipage}[c]{0.5\textwidth}
                \begin{enumerate}
                    \item $(f,g,h)$ is a morphism of triangles
                    \item $\exists g:B\rightarrow B'$ such that $b'ga = 0$
                \end{enumerate}
            \end{minipage}
        \end{center}
        Moreover, if $\mathcal{T}(A,T^{-1}C')\simeq 0$, then f and h are unique.
    \end{lemma}

    \begin{proof}
        $1. \implies 2.$ The composition $b'ga = ba = 0$ shows the claim. 
        
        $2. \implies 1.$ The existence of $f$ and $h$ are \todo[color = pink]{er det virkelig så evident?} evident from the long exact sequence of the bottom triangle at the covariant functor represented by $A$. 
        \begin{center}
            \begin{tikzcd}
                \mathcal{T}(A,T^{-1}C') \ar{r}{T^{-1}c'_*} & \mathcal{T}(A,A') \ar{r}{a'_*} & \mathcal{T}(A,B') \ar{r}{b'_*} & \mathcal{T}(A,C')
            \end{tikzcd}
        \end{center}
        \todo{Hva som foregår her burde komme fram tydeligere} The morphism $ga:\mathcal{T}(A,B')$ such that $b'ga=b'_*(ga)=0$, thus $ga:Ker(b'_*)$. By exactness $\exists f:\mathcal{T}(A,A')$ such that $a'f = ga$, and by TR3 $\exists h: C \rightarrow C'$ such that $(f,g,h)$ is a morphism of triangles.
        Now assume that $\mathcal{T}(A,T^{-1}C')\simeq 0$. Exactness determines that $a'_*$ is a monomorphism, and $f$ is then unique. Since $T$ is a translation, we have that $\mathcal{T}(A,T^{-1}C')\simeq\mathcal{T}(TA,C')$. By using the functor $\mathcal{T}(\_,C')$ at the top triangle, we get that $b^*$ is a monomorphism, and thus $h$ is chosen uniquely.
    \end{proof}

    \begin{lemma} \textbf{Opposite Rotation Axiom; $TR2^{op}$.}
        If $(A,B,C,a,b,c)$ is a triangle, then $(T^{-1}C,A,B,-T^{-1}c,a,b)$ is a triangle.
    \end{lemma}

    \begin{proof}
        Apply TR2 twice to construct the triangle below.
        \begin{center}
            \begin{tikzcd}[row sep=tiny]
                A \ar{rd}{a} \\
                & B \ar{ld}{b} \\
                C \ar{uu}[near start]{c}[very near end, marking]{|}[near end]{T}
            \end{tikzcd}
            $\implies$
            \begin{tikzcd}[row sep=tiny]
                C \ar{rd}{c} \\
                & TA \ar{ld}{-Ta} \\
                TB \ar{uu}[near start]{-Tb}[very near end, marking]{|}[near end]{T}
            \end{tikzcd}
        \end{center}
        The morphism $T^{-1}c$ has a triangle $(T^{-1}C,A,B',T^{-1}c,a',b')$ by TR1. Use TR3 to find a morphism between these associated candidate triangles.
        \begin{center}
            \begin{tikzcd}
                C \ar{r}{c} \ar[equal]{d}{id_C} & TA \ar{r}{Ta'} \ar[equal]{d}{id_{TA}} & TB' \ar{r}{Tb'} \ar[dashed]{d}{h} & TC \ar[equal]{d}{id_{TC}} \\
                C \ar{r}{c} & TA \ar{r}{-Ta} & TB \ar{r}{-Tb} & TC
            \end{tikzcd}
        \end{center}
        By the 2 out of 3 property it is seen that h is an isomorphism, so the triple $(id_{T^{-1}C}, id_A, T^{-1}h)$ is an isomorphism of candidate triangles, and by TR1, is an isomorphism of triangles, asserting that $(T^{-1}C,A,B,-T^{-1}c,a,b)$ is in fact a triangle.
    \end{proof}

    \begin{lemma}
        Let $(A,B,C,a,b,c)$ and $(A',B',C',a',b',c')$ be two triangles, then the direct sum of these triangles is a triangle.
    \end{lemma}

    \begin{proof}
        \todo{Hva er det jeg prøvde å si her???} Observe that for any functor $\mathcal{T}(K,\_)$ there is still a long exact sequence of Hom(ology)since $\mathcal{T}(K,A\oplus A')\simeq\mathcal{T}(K,A)\oplus\mathcal{T}(K,A')$. Thus for the direct sum of the triangles we have the following.
        \begin{center}
            \begin{tikzcd}[ampersand replacement=\&]
                A\oplus A' \ar{r}{\begin{pmatrix}a & 0 \\ 0 & a'\end{pmatrix}} \& B\oplus B' \ar{r}{\begin{pmatrix}b & 0 \\ 0 & b'\end{pmatrix}} \& C\oplus C' \ar{r}{\begin{pmatrix}c & 0 \\ 0 & c'\end{pmatrix}} \& TA\oplus TC
            \end{tikzcd} \\
            $\Downarrow$ \\
            \begin{tikzcd}[column sep=small]
                ... \ar{r} & \mathcal{T}(K,A)\oplus\mathcal{T}(K,A') \ar{r} & \mathcal{T}(K,B)\oplus\mathcal{T}(K,B') \ar[dll, rounded corners,to path={ --([xshift=2ex]\tikztostart.east)|- (Z)[near end]\tikztonodes-| ([xshift=-2ex]\tikztotarget.west)-- (\tikztotarget)}] \\
                \mathcal{T}(K,C)\oplus\mathcal{T}(K,C') \ar{r} & \mathcal{T}(K,TA)\oplus\mathcal{T}(K,TA') \ar{r} & ...    
            \end{tikzcd}
        \end{center}
        Thus the 2 out of 3 property holds for the direct sum. By TR1 there is a triangle 
        \begin{center}
            \begin{tikzcd}
                A\oplus A' \ar{r} & B\oplus B' \ar{r} & D \ar{r} & TA\oplus TA'
            \end{tikzcd}
        \end{center}
        By TR3 there are morphisms from this triangle to to the direct summands. Adding these maps together there is a map from this triangle to direct sum, and by using the 2 out of 3 property this is an isomorphism of candidate triangles. Thus the direct sum is a triangle.
        \begin{center}
            \begin{tikzcd}
                A\oplus A' \ar{r} \ar{d} & B\oplus B' \ar{r} \ar{d} & D \ar{r} \ar[dashed]{d} & TA\oplus TA' \ar{d} \\
                A \ar{r} & B \ar{r} & C \ar{r} & TA
            \end{tikzcd} \\
            \& \\
            \begin{tikzcd}
                A\oplus A' \ar{r} \ar{d} & B\oplus B' \ar{r} \ar{d} & D \ar{r} \ar[dashed]{d} & TA\oplus TA' \ar{d} \\
                A' \ar{r} & B' \ar{r} & C' \ar{r} & TA'
            \end{tikzcd} \\
            $\Downarrow$ \\
            \begin{tikzcd}
                A\oplus A' \ar{r} \ar[equal]{d} & B\oplus B' \ar{r} \ar[equal]{d}& D \ar{r} \ar[dashed]{d}[below, rotate=90]{\simeq} & TA\oplus TA' \ar[equal]{d} \\
                A\oplus A' \ar{r} & B\oplus B' \ar{r} & A''\oplus B'' \ar{r} & TA\oplus TA'
            \end{tikzcd}
        \end{center}
    \end{proof}

    \begin{lemma}
        The direct summands of a triangle is a triangle.
    \end{lemma}

    \begin{proof}
        The proof can be found in \cite{neeman}
    \end{proof}
    

\section{Mapping Cones, Homotopies and Contractibility}

    The observant reader might have seen that the Octahedron axiom have not yet been used once, other than for proving TR3. A lot of the theory proven for triangulated categories works without this axiom, and this motivates the definition of a pre-triangulated category.

    \begin{definition}
        A pre-triangulation of an additive category $\mathcal{T}$ with translation $T$ is a collection $\Delta '$ of triangles consisting of candidate triangles in $\mathcal{T}$ satisfying TR1, TR2 and TR3.

        The category $\mathcal{T}$ with the pre-triangulation $\Delta '$ is called a pre-triangulated category, and the candidate triangles in $\Delta '$ are called  triangles. This notion of triangles will only be used in this subsection.
    \end{definition}

    \todo{Kanskje få denne biten tidligere?} The main goal of this subsection is to see how we can find  triangles, and when these are triangles. We will also look at triangulated functors and triangulated subcategories. For the rest of this subsection it is assumed that we work in a pre-triangulated category $\mathcal{T}$.

    \begin{definition}
        Let $\phi : (A,B,C,a,b,c) \rightarrow (A',B',C',a',b',c')$ be a morphism of candidate triangles.
        \begin{center}
            \begin{tikzcd}
                A \arrow{r}{a} \arrow{d}{\phi_A} & B \arrow{r}{b} \arrow{d}{\phi_B} & C \arrow{r}{c} \arrow{d}{\phi_C} & TA \arrow{d}{T\phi_A} \\
                A' \arrow{r}{a'} & B' \arrow{r}{b'} & C \arrow{r}{c'} & TA'
            \end{tikzcd}
        \end{center}
        The mapping cone of $\phi$ is defined to be the candidate triangle below.
        \begin{center}
            \begin{tikzcd}[ampersand replacement=\&]
                A' \oplus B \ar{r}{\begin{pmatrix}
                    -b & \phi_B \\ 0 & a'
                \end{pmatrix}} \& B'\oplus C \ar{r}{\begin{pmatrix}
                    -c & \phi_C \\ 0 & b'
            \end{pmatrix}} \& C'\oplus TA \ar{r}{\begin{pmatrix}
                -Ta & T\phi_A \\ 0 & c'
            \end{pmatrix}} \& TA'\oplus TB
            \end{tikzcd}
        \end{center}
    \end{definition}

    \begin{definition}
        A morphism $\alpha : A,B,C,a,b,c) \rightarrow (A',B',C',a',b',c')$ between candidate triangles is called null-homotopic if it factors through a homotopy. A homotopy is defined to be a triple of maps $\Theta, \Phi, \Psi$ in the following sense.
        \begin{center}
            \begin{tikzcd}
                A \arrow{r}{a} \arrow{d}{\alpha_A} & B \arrow{r}{b} \arrow{ld}{\Theta} \ar{d}{\alpha_B} & C \arrow{r}{c} \arrow{ld}{\Phi} \ar{d}{\alpha_C} & TA \arrow{ld}{\Psi} \ar{d}{T\alpha_A} \\
                A' \arrow{r}{a'} & B' \arrow{r}{b'} & C \arrow{r}{c'} & TA'
            \end{tikzcd}
        \end{center}
        It is required that $\alpha_A  = \Theta a + T^{-1}(c'\Psi)$, $\alpha_B = \Phi b + a'\Theta$ and $\alpha_C = \Psi c + b'\Phi$ for the triple to be a homotopy.
        Two maps are called homotopic if their difference is null-homotopic
    \end{definition}

    \begin{lemma}
        The mapping cone only depends on morphisms up to homotopy. I.e. if two maps are homotopic, their mapping cones are isomorphic.
    \end{lemma}

    \begin{proof}
        Suppose that $(f,g,h)$ and $(f',g',h')$ are two homotopic morphisms of triangles:
        \begin{center}
            \begin{tikzcd}
                A \ar{r}{a} \ar{d} & B \ar{r}{b} \ar{d} & C \ar{r}{c} \ar{d} & TA \ar{d} \\
                A' \ar{r}{a'} & B' \ar{r}{b'} & C' \ar{r}{c'} & TA'
            \end{tikzcd}
        \end{center}
        Let $(\Theta,\Phi,\Psi)$ be the homotopy between the triangle morphisms. Then there is an isomorphism of triangles.
        \begin{center}
            \begin{tikzcd}[ampersand replacement=\&]
                A' \oplus B \ar{r}{\begin{pmatrix}
                    -b & g \\ 0 & a'
                \end{pmatrix}} \ar{d}{\begin{pmatrix} 1 & \Theta \\ 0 & 1 \end{pmatrix}} \& B'\oplus C \ar{r}{\begin{pmatrix}
                    -c & h \\ 0 & b'
            \end{pmatrix}} \ar{d}{\begin{pmatrix} 1 & \Phi \\ 0 & 1\end{pmatrix}} \& C'\oplus TA \ar{r}{\begin{pmatrix}
                -Ta & Tf \\ 0 & c'
            \end{pmatrix}} \ar{d}{\begin{pmatrix}1 & \Psi \\ 0 & 1\end{pmatrix}} \& TA'\oplus TB \ar{d}{\begin{pmatrix}1 & T\Theta \\ 0 & 1 \end{pmatrix}}\\
            A' \oplus B \ar{r}[below]{\begin{pmatrix}
                -b & g' \\ 0 & a'
            \end{pmatrix}} \& B'\oplus C \ar{r}[below]{\begin{pmatrix}
                -c & h' \\ 0 & b'
        \end{pmatrix}} \& C'\oplus TA \ar{r}[below]{\begin{pmatrix}
            -Ta & Tf' \\ 0 & c'
        \end{pmatrix}} \& TA'\oplus TB
            \end{tikzcd}
        \end{center}
    \end{proof}

    \begin{lemma}
        Let $A$ denote the candidate triangle $(A,A',A'')$ and $B$ denote $(B,B',B'')$. Suppose $\alpha, \beta : A \rightarrow B$ are two homotopic morphisms of candidate triangles. Then for any map $\gamma : \widetilde{A} \rightarrow A$ and any map $\delta : B \rightarrow \widetilde{B}$ the maps $\delta\alpha\gamma$ and $\delta\beta\gamma$ are homotopic as well.
    \end{lemma}

    \begin{proof}
        To prove this statement it is enough to prove that $\alpha\gamma$ is homotopic to $\beta\gamma$ du to the symmetry of the statement. The goal is then to show that $(\Theta\gamma ',\Phi\gamma '',\Psi T\gamma)$ is the homotopy between these maps. This can be seen as
        \begin{multline*}
            {\alpha}'{\gamma}'-{\beta}'{\gamma}' = ({\alpha}'-{\beta}'){\gamma}' = (b\Theta +\Phi a'){\gamma}' = b\Theta {\gamma}' + \Phi a'{\gamma}' = b({\Theta}{\gamma}') + ({\Phi}{\gamma}'')\widetilde{a}'
        \end{multline*}.
    \end{proof}

    \begin{definition}
        A candidate triangle $A$ is called a contractible triangle if $id_A$ is null-homotopic.
    \end{definition}

    \begin{remark}
        If $A$ is a contractible triangle and $F:\mathcal{T}\rightarrow \mathcal{A}$ is an additive functor to an abelian category, then the identity of the cochain is null-homotopic as well.
        \begin{center}
            \begin{tikzcd}
                ... \ar{r} & F(A) \ar{r} & F(A') \ar{r} & F(A'') \ar{r} & F(TA) \ar{r} & ...
            \end{tikzcd}
        \end{center}
        The homology of this sequence is therefore $0$ everywhere, asserting that it is an exact sequence.
        The exactness of such sequences allow us to use the 2 out of 3 property on morphisms between contractible triangles.
    \end{remark}

    \begin{corollary}
        If A is a contractible triangle, then any map in $\mathcal{T}(A,\_)$ or $\mathcal{T}(\_,A)$ is null-homotopic.
    \end{corollary}

    \begin{proof}
        By definition, being contractible is the same as the existence of a homotopy between the map and the zero map. If $id_A\sim 0 \implies f\circ id_A = f \sim f\circ 0 = 0$. So any map $f$ is null-homotopic.
    \end{proof}

    \begin{lemma}
        A contractible triangle is a triangle.
    \end{lemma}

    \begin{proof}
        Let $A$ be the contractible triangle $(A,A',A'')$. Writing everything out, there is a homotopy between candidate triangles.
        \begin{center}
            \begin{tikzcd}
                A \ar{r}{a} \ar{d}{id_A} & A' \ar{r}{a'} \ar{d}{id_{A'}} \ar{ld}{\Theta} & A'' \ar{r}{a''} \ar{d}{id_{A''}} \ar{ld}{\Phi} & TA \ar{d}{id_{TA}} \ar{ld}{\Psi} \\
                A \ar{r}{a} & A' \ar{r}{a'} & A'' \ar{r}{a''} & TA
            \end{tikzcd}
        \end{center}
        By using TR1 there is a triangle, and consequently, a long exact sequence.
        \begin{center}
            \begin{tikzcd}
                A \ar{r}{a} & A' \ar{r}{e} & E \ar{r}{e'} & TA
            \end{tikzcd} \\
            $\Downarrow$ \\
            \begin{tikzcd}
                ... \ar{r} & \mathcal{T}(TA,A) \ar{r} & \mathcal{T}(TA,A') \ar{r} & \mathcal{T}(TA,E) \ar{r}{e'_*} & \mathcal{T}(TA,TA) \ar{r}{Ta_*} & ...    
            \end{tikzcd}
        \end{center}
        Since the map $Ta\circ a''\Psi = 0$ and by exactness at $\mathcal{T}(TA,TA)$, the kernel $KerTa_*=Ime'_*\neq 0$. This shows that there is a map ${\Psi}':\mathcal{T}(TA,E)$ such that $e'{\Psi}'=a''\Psi$, and the map $(id_A,id_{A'},e\Theta+{\Psi}'a'')$ is a well defined map of candidate triangles. By the remark, we can use the 2 out of 3 properties to assert that the map found is an isomorphism, giving an isomorphism of triangles, showing that the contractible triangle is a triangle by Bookeeping. 
    \end{proof}

    \begin{corollary}
        The mapping cone of the zero map between  triangles is a triangle. 
    \end{corollary}

    \begin{proof}
        The mapping cone of the zero map can be seen to be the direct sum of two triangles. Thus it is a triangle.
    \end{proof}

    \begin{corollary}
        The mapping cone of a null-homotopic map between triangles is a triangle.
    \end{corollary}

    \begin{remark}
        Suppose we have a morphism of triangles where one of the triangles are contractible, then the mapping cone is a triangle as well.
    \end{remark}

    \todo{Skriv noe om hvordan dette har bygget opp til en annen variant av TR4.}

    \begin{definition}
        A morphism of triangles will be called good if the mapping cone of the morphism is a triangle.
    \end{definition}

    \begin{theorem}
        A pre-triangulated category $\mathcal{T}$ is triangulated if given two triangles $(A,B,C,a,b,c)$ and $(A',B',C',a',b',c')$ and diagram (1) commutes, then diagram (1) can be completed to diagram (2) such that $\phi$ is good.
        \begin{center}
            (1)
            \begin{tikzcd}
                A \ar{r}{a} \ar{d}{\phi_A} & B \ar{d}{\phi_B} & \\
                A' \ar{r}{a'} & B'
            \end{tikzcd}
            (2)
            \begin{tikzcd}
                A \ar{r}{a} \ar{d}{\phi_A} & B \ar{r}{b} \ar{d}{\phi_B} & C \ar{r}{c} \ar[dashed]{d}{\phi_C} & TA \ar{d}{T\phi_A} \\
                A' \ar{r}{a'} & B' \ar{r}{b'} & C' \ar{r}{c'} & TA'
            \end{tikzcd}
        \end{center}
    \end{theorem}

    \begin{remark}
        \todo{Utdyp hva som står her} This condition is equivalent to the Octahedron axiom. \todo[color = red]{Legg til kilder}
    \end{remark}

    \todo[color = pink]{Skal jeg skrive også om Verdier sitt aksiom her? Jeg synes det kan være en gøy med en exposition av alt kaoset man kan finne :)}

    \begin{definition}
        \todo[color = pink]{Denne definisjonen er rar og er ikke presis, men jeg synes den er litt søt} A commutative square (1) is called homotopy cartesian if it arises from a triangle. That is, (2) is a triangle.
        \begin{center}
            (1)
            \begin{tikzcd}
                D \ar{r} \ar{d} \ar[phantom]{rd}{HO}[very near start]{\ulcorner}[very near end]{\lrcorner}& A \ar{d} \\
                B \ar{r} & C
            \end{tikzcd}
            $\implies$
            (2)
            \begin{tikzcd}[row sep=small]
                D \ar{rd} \\
                & A\oplus B \ar{ld} \\ 
                C \ar[very near end, "|" marking]{uu}[near end]{T} 
            \end{tikzcd}
        \end{center}
    \end{definition}
        
    \begin{remark}
        A way to construct homotopy cartesian squares is with homotopy pullbacks. This is done by using TR1 on the following map to get a triangle.
        \begin{center}
            \begin{tikzcd}
                & A \ar{d}{a} \\
                B \ar{r}{b} & C
            \end{tikzcd}
            $\implies$
            \begin{tikzcd}[ampersand replacement=\&]
                A\oplus B \ar{r}{\begin{pmatrix}a & b\end{pmatrix}} \& C
            \end{tikzcd} \\
            $\implies$
            \begin{tikzcd}[ampersand replacement=\&]
                A\oplus B \ar{r}{\begin{pmatrix}a & b\end{pmatrix}} \& C \ar{r} \& TD \ar{r} \& TA\oplus TB
            \end{tikzcd}
            $\implies$
            \begin{tikzcd}
                D \ar{r} \ar{d} \ar[phantom]{rd}{HO}[very near start]{\ulcorner}[very near end]{\lrcorner}& A \ar{d} \\
                B \ar{r} & C
            \end{tikzcd}
        \end{center}

        Dually, one may use homotopy push-outs to get homotopy cartesian squares.
    \end{remark}

    \begin{remark}
        A remark about good maps and homotopy cartesian squares???
    \end{remark}

    \begin{lemma}
        Suppose that there is a homotopy cartesian square (1). Then there are triangles and a triangle isomorphism as in (2).
        \begin{center}
            (1)
            \begin{tikzcd}
                D \ar{r}{g'} \ar{d}{f'} \ar[phantom]{rd}{HO}[very near start]{\ulcorner}[very near end]{\lrcorner}& A \ar{d}{f} \\
                B \ar{r}{g} & C
            \end{tikzcd}
            (2)
            \begin{tikzcd}
                D \ar{r} \ar{d}{f'} & A \ar{d}{f} \ar{r} & E \ar{r} \ar[equal]{d} & TD \ar{d}{Tf'} \\
                B \ar{r} & C \ar{r} & E \ar{r} & TB
            \end{tikzcd}
        \end{center}
    \end{lemma}

    \begin{proof}
        There is a commutative square (1) which satisfies the requirements of the octahedron axiom (2), yielding a triangle (3).
        \begin{center}
            (1)
            \begin{tikzcd}[ampersand replacement=\&]
                D \ar{r}{\begin{pmatrix}g' \\ f'\end{pmatrix}} \ar[equal]{d} \& A\oplus B \ar{d}{\begin{pmatrix}1 & 0\end{pmatrix}} \\
                D \ar{r}{g'} \& A
            \end{tikzcd} \\
            (2)
            \begin{tikzcd}[row sep=small, ampersand replacement=\&]
                D \ar[red]{rd}{\begin{pmatrix}g' \\ f'\end{pmatrix}} \\
                \& A\oplus B \ar[red]{ld}{\begin{pmatrix}f & g\end{pmatrix}} \\
                C \ar[red, very near end, "|" marking]{uu}[near end]{T}[pos=0.5]{0}
            \end{tikzcd}
            \begin{tikzcd}[row sep=small, ampersand replacement=\&]
                A\oplus B \ar[orange]{rd}{\begin{pmatrix} 1 & 0\end{pmatrix}} \\
                \& A \ar[orange]{ld}{0} \\
                TB \ar[orange, very near end, "|" marking]{uu}[near end]{T}[near start]{\begin{pmatrix}0 \\ 1\end{pmatrix}}
            \end{tikzcd}
            \begin{tikzcd}[row sep=small]
                D \ar[violet]{rd}{f'} \\
                & A \ar[violet]{ld}{} \\
                E \ar[violet, very near end, "|" marking]{uu}[near end]{T}
            \end{tikzcd} \\
            (3)
            \begin{tikzcd}
                \textcolor{white}{.} \\
                C \ar[teal]{r} & E \ar[teal]{r}{} & TB \ar[teal]{r}{} & TC \\
                \textcolor{white}{.}
            \end{tikzcd}
        \end{center}
        This setup shows that there is a triangle with a commutative square as below.
        \begin{center}
            \begin{tikzcd}[ampersand replacement=\&]
                A\oplus B \ar{r}{\begin{pmatrix}1 & 0\end{pmatrix}} \ar{d}{\begin{pmatrix}f & g\end{pmatrix}} \& A \ar{d}{} \\
                C \ar{r}{} \& E
            \end{tikzcd}
        \end{center}.
        Since $\begin{pmatrix}1 & 0\end{pmatrix}$ is a splitmono, there is a morphism of triangles proving the statement.
        \begin{center}
            \begin{tikzcd}
                D \ar{r}{} \ar{d}{} & A \ar{r}{} \ar{d}{} & E \ar{r}{} \ar[equal]{d}{} & TD \ar{d}{} \\
                B \ar{r}{} & C \ar{r}{} & E \ar{r}{} & TB
            \end{tikzcd}
        \end{center}
    \end{proof}

\section{Calculus of Fractions and the Verdier Quotient}
    \todo[noline]{Dette avsnittet er skrevet litt rart. Flyten kan bli mye bedre.} Localization is a method for adding formal inverses to a category. It is most notably known in commutative algebra where we can invert elements with respect to some ideal of the ring. The rational numbers can be shown to be a localization of the integers at every number except 0. The category gained from localizing at some set $S$ of morphisms is the universal category where these morphisms are isomorphisms.
    \begin{definition}
        Let $S$ be a collection of morphisms in the category $\mathcal{C}$. The Localization of $\mathcal{C}$ on $\mathcal{S}$ is the category $\mathcal{C}[S^{-1}]$ together with a functor $q:\mathcal{C}\rightarrow \mathcal{C}[S^{-1}]$ such that:
        \begin{itemize}
            \item $\forall s:S|q(s)$ is an isomorphism
            \item Any functor $F:\mathcal{C}\rightarrow\mathcal{D}$ such that $\forall s:S$ $F(s)$ is an isomorphism, then $F$ factors through $q$. That is to say that there is a natural isomorphism $\eta : F\rightarrow F'\circ q$ so that $\mathcal{C}[S^{-1}]$ is the universal category where morphisms in $S$ are isomorphisms.
        \end{itemize}
        \begin{center}
            \begin{tikzcd}[row sep = tiny]
                \mathcal{C} \ar{rr}{F} \ar{rd}[below]{q} & & \mathcal{D} \\
                & S^{-1}\mathcal{C} \ar[dashed]{ru}[below]{F'}
            \end{tikzcd}
        \end{center}
    \end{definition}

    \begin{remark}
        Even though it is known that $\mathcal{C}$ is locally small, it is not clear a priori that the category $\mathcal{C}[S^{-1}]$ is again locally small.
    \end{remark}

    \todo[color = pink]{Burde jeg endre på noe her. Kan det være greit å forklare hvorfor de er vanskelig å beskrive?} These categories are in generel pretty hard to describe. When the set of morphisms is a multiplicative system, there is a calculus of fractions description of these localization in the same style as for localizations of rings.

    \begin{definition}
        A set $S$ of morphisms in a category $\mathcal{C}$ is called right multiplicative if it satisfies the following conditions:
        \begin{itemize}
            \item $S$ is closed under composition, i.e. if $f,g : S$ are composable then $gf : S$. Every identity morphism in $\mathcal{C}$ is in $S$.
            \item (Right Ore condition) If $t : X \rightarrow Y$ is a morphism in $S$, then $\forall g:Z\rightarrow Y$ there is a commutative square (1) such that $f:W\rightarrow X$ and $s:W\rightarrow Z$ exists and $s:S$ as well.
            \begin{center}
                (1)
                \begin{tikzcd}
                    W \ar[dashed]{r}{f} \ar[dashed]{d}{s} & X \ar{d}{t} \\
                    Z \ar{r}{g} & Y
                \end{tikzcd}
            \end{center}
            \item (Left cancellation) Suppose $f,g:X\rightarrow Y$ are parallell morphisms in $\mathcal{C}$, then 1. $\implies$ 2.:
            \begin{enumerate}
                \item $sf = sg$ for som $s:S$ starting at $Y$
                \item $ft = gt$ for som $t:S$ ending at $X$
            \end{enumerate}
        \end{itemize}
    \end{definition}

    \begin{remark}
        The previous definition has a dual statement. A set $S$ of morphisms is left multiplicative if it satisfies:
        \begin{itemize}
            \item $S$ is closed under composition, i.e. if $f,g : S$ are composable then $gf : S$. Every identity morphism in $\mathcal{C}$ is in $S$.
            \item (Left Ore condition) If $s : Y \rightarrow Z$ is a morphism in $S$, then $\forall f:Y\rightarrow X$ there is a commutative square (1) such that $g:Z\rightarrow W$ and $t:X\rightarrow W$ exists and $t:S$ as well.
            \begin{center}
                (1)\begin{tikzcd}
                    Y \ar{r}{f} \ar{d}{s} & X \ar[dashed]{d}{t} \\
                    Z \ar[dashed]{r}{g} & W
                \end{tikzcd}
            \end{center}
            \item (Right cancellation) Suppose $f,g:X\rightarrow Y$ are parallell morphisms in $\mathcal{C}$, then 1. $\implies$ 2.:
            \begin{enumerate}
                \item $ft = gt$ for som $t:S$ ending at $X$
                \item $sf = sg$ for som $s:S$ starting at $Y$
            \end{enumerate}
        \end{itemize}
        If $S$ is both right multiplicative and left multiplicative then it is called multiplicative.
    \end{remark}

    \begin{prototype}
        \todo[color = pink]{Fjerne dette?} Let $R$ be a commutative integral domain, ... (look at Bacharaya and how they define the field of fractions, or ask Andreas if he have any good literature on this topic)
    \end{prototype}

    As with the definition of localization of rings, localization of a category $\mathcal{C}$ at a multiplicative system will be defined with fractions. That is the morphisms will be "fractions" of morphisms. These morphisms will be described as diagrams over spans for right multiplicative systems (or dually cospans for left multiplicative systems), together with an equivalence relation.

    \begin{definition}
        A span is a diagram of the form:
        \begin{center}
            \begin{tikzcd}
                \cdot & \cdot \ar{l} \ar{r} & \cdot
            \end{tikzcd}
        \end{center}
    \end{definition}

    \begin{definition}
        Let $S$ be a right multiplicative system of morphisms in a category $\mathcal{C}$. Given a morphism $s : Y\rightarrow X$ in $S$ and a morphism $t:Y\rightarrow Z$, define the right fraction of $s$ and $t$ to be the span of the morphisms. That is $s$ and $t$ fit in the diagram below.
        \begin{center}
            \begin{tikzcd}
                X & Y \ar{l}{s} \ar{r}{t} & Z
            \end{tikzcd}
        \end{center}
        Right fractions are denoted as $ts^{-1}$.
        Let $\sim$ be the equivalence relation of right fractions given by the diagram (1) such that $ts^{-1}\sim t's'^{-1}$ if and only if $\exists w,w':\mathcal{C}$ making the diagram commute and that the middle row is a right fraction.
        \begin{center}
            \begin{tikzcd}
                & Y \ar{ld}[above]{s} \ar{rd}{t} \\
                X & W \ar{r} \ar{l} \ar{u}{w} \ar{d}{w'} & Z \\
                & Y' \ar{lu}{s'} \ar{ru}[below]{t'}
            \end{tikzcd}
        \end{center}
    \end{definition}

    Dually, define left fractions as diagrams over cospans such that if $t:S$, then there is a left fraction $t^{-1}s$ as the diagram below.
    \begin{center}
        \begin{tikzcd}
            X \ar{r}{s} & Y & Z \ar{l}{t}
        \end{tikzcd}
    \end{center}

    The equivalence relation $\sim$ is given by the diagram in the same manner as above.
    \begin{center}
        \begin{tikzcd}
            & Y \ar{d}{w} \\
            X \ar{ru}{s} \ar{r} \ar{rd}{s'} & W & Z \ar{lu}{t} \ar{l} \ar{ld}{t'} \\
            & Y' \ar{u}{w'}
        \end{tikzcd}
    \end{center}

    \begin{prop}
        Suppose that $S$ is a right multiplicative system, then the relation stated above is in fact an equivalence relation.
    \end{prop}

    \begin{proof}
        An equivalence relation is proven by showing that $\sim$ is reflexive, symmetric and transitive.
        \begin{itemize}
            \item (Reflexive) Let $fs^{-1}$ be a right fraction. Then diagram (1) shows that $fs^{-1}\sim fs^{-1}$.
            \begin{center} (1)
                \begin{tikzcd}
                    & W \ar{ld}{s} \ar{rd}{f} \ar[equal]{d} \\
                    X & W \ar{r}{f} \ar{l}{s} & Y \\
                    & W \ar{lu}{s} \ar[equal]{u} \ar{ru}{f}
                \end{tikzcd}
            \end{center}
            \item (Symmetric) Let $fs^{-1}$ and $gt^{-1}$ be two right fractions such that $fs^{-1}\sim gt^{-1}$, that is diagram (2) commute. Due to inherent symmetric nature of the diagram it follows that $gt^{-1}\sim fs^{-1}$.
            \begin{center} (2)
                \begin{tikzcd}
                    & W \ar{ld}{s} \ar{rd}{f} \\
                    X & \widetilde{W} \ar{r} \ar{l} \ar{u}{w} \ar{d}{w'} & Y \\
                    & W' \ar{lu}{t} \ar{ru}{g}
                \end{tikzcd}
                $\implies$
                \begin{tikzcd}
                    & W' \ar{ld}{t} \ar{rd}{g} \\
                    X & \widetilde{W} \ar{l} \ar{r} \ar{u}{w'} \ar{d}{w} & Y \\
                    & W \ar{lu}{s} \ar{ru}{f}
                \end{tikzcd}
            \end{center}
            \item (Transitive) Suppose that there are three right fractions $fs^{-1}$, $gt^{-1}$ and $hu^{-1}$ such that $fs^{-1}\sim gt^{-1}$ and $gt^{-1}\sim hu^{-1}$. This may be written as diagram (3) and (4).
            \begin{center} (3)
                \begin{tikzcd}
                    & W' \ar{ld}{s} \ar{rd}{f} \\
                    X & \widetilde{W} \ar{r} \ar{l} \ar{u}{w'} \ar{d}{\widetilde{w'}} & Y \\
                    & W \ar{lu}{t} \ar{ru}{g}
                \end{tikzcd} 
                (4)
                \begin{tikzcd}
                    & W \ar{ld}{t} \ar{rd}{g} \\
                    X & \widetilde{\widetilde{W}} \ar{r} \ar{l} \ar{u}{\widetilde{w''}} \ar{d}{w''} & Y \\
                    & W'' \ar{lu}{u} \ar{ru}{h}
                \end{tikzcd}
            \end{center}
            Diagram (5) may be created by using the Ore condition on the maps $\widetilde{w'}$ and $\widetilde{w''}$. Since both morphisms are assumed to be in $S$, it follows that both $\widetilde{w'}$ and $\widetilde{w''}$ are in $S$ as well. Diagram (6) then shows that $fs^{-1}\sim hu^{-1}$.
            \begin{center} (5)
                \begin{tikzcd}
                    \widetilde{\widetilde{\widetilde{W}}} \ar{d}{\widetilde{\widetilde{w'}}} \ar{r}{\widetilde{\widetilde{w''}}} & \widetilde{\widetilde{W}} \ar{d}{\widetilde{w''}} \\
                    \widetilde{W} \ar{r}{\widetilde{w'}} & W
                \end{tikzcd}
                (6)
                \begin{tikzcd}
                    & W' \ar{ldd}{s} \ar{rdd}{f} \\
                    & \widetilde{W} \ar{ld} \ar{rd} \ar{u}{\widetilde{w'}} \\
                    X & \widetilde{\widetilde{\widetilde{W}}} \ar{l} \ar{r} \ar{u}{\widetilde{\widetilde{w'}}} \ar{d}{\widetilde{\widetilde{w''}}}& Y \\
                    & \widetilde{\widetilde{W}} \ar{lu} \ar{ru} \ar{d}{\widetilde{w''}} \\
                    & W'' \ar{luu}{u} \ar{ruu}{h}
                \end{tikzcd}
            \end{center}
        \end{itemize}
    \end{proof}

    \begin{definition}
        Let $S$ be a multiplicate system in a category $\mathcal{C}$. Given two right fractions $fs^{-1}$ and $gt^{-1}$
        \begin{center}
            \begin{tikzcd}
                X & W \ar{l}{s} \ar{r}{f} & Y
            \end{tikzcd}
            \&
            \begin{tikzcd}
                Y & W' \ar{l}{t} \ar{r}{g} & Z
            \end{tikzcd}
        \end{center}
        the composition of the fractions are defined to be $gt^{-1}\circ fs^{-1}$. The Ore condition describes how this composition should be defined,
        \begin{center}
            \begin{tikzcd}
                & \widetilde{W} \ar{d}{u} \ar{r}{h} & W' \ar{d}{t} \ar{r}{g} & Z \\
                X & W \ar{l}{s} \ar{r}{f} & Y
            \end{tikzcd}
        \end{center}
        the composite is the right fraction $gt^{-1}\circ fs^{-1} = gh(su)^{-1}$.
    \end{definition}

    \begin{prop}
        The composition of right fractions is well-defined up to equivalence.
    \end{prop}

    \begin{proof}
        In order to prove that the composite is well-defined one must prove that the composite is independent from the different options of morphisms provided by the right Ore condition, and that it is therefore independent from the choice of right fraction. There will only be presented a proof for that the choice of Ore maps is independent, as the other other case is analogous.

        Suppose there are two right fractions $fs^{-1}$ and $gt^{-1}$ as indicated by the diagrams.

        \begin{center}
            \begin{tikzcd}
                X & W_1 \ar{l}{s} \ar{r}{f} & Y
            \end{tikzcd}
            \&
            \begin{tikzcd}
                Y & W_2 \ar{l}{t} \ar{r}{g} & Z
            \end{tikzcd}
        \end{center}
        Further suppose that there are at least two different choices for the morphisms provided by the right Ore condition, for example \todo[color = pink]{Dette er et incosist format}$(\widetilde{W},\widetilde{s},\widetilde{f})$ and $(\widehat{W},\widehat{s}, \widehat{g})$. The two compositions may be drawn as the diagrams below.
        \begin{center}
            \begin{tikzcd}
                & \widetilde{W} \ar{r}{\widetilde{g}} \ar{d}{\widetilde{s}} & W_2 \ar{r}{g} \ar{d}{t} & Z \\
                X & W_1 \ar{r}{f} \ar{l}{s} & Y
            \end{tikzcd}
            \begin{tikzcd}
                & \widehat{W} \ar{r}{\widehat{f}} \ar{d}{\widehat{s}} & W_2 \ar{r}{g} \ar{d}{t} & Z \\
                X & W_1 \ar{l}{s} \ar{r}{f} & Y
            \end{tikzcd}
        \end{center}
        Combining the diagrams at $W_1$ by using the right Ore condition, the objects and maps \todo[color = magenta]{Er dette inkonsekvent?} $(W, \widetilde{w}, \widehat{w})$ exists as in the diagram below. 
        \begin{center}
            \begin{tikzcd}
                \bar{W} \ar[dashed]{rd}{\xi} \\
                & W \ar{r}{\widehat{w}} \ar{d}{\widetilde{w}} & \widehat{W} \ar{d}[near start, below]{\widehat{s}} \ar{r}{\widehat{f}} & W_2 \ar{d}{t} \ar{r}{g} & Z \\
                & \widetilde{W} \ar{r}{\widetilde{s}} \ar{rru}[near start]{\widetilde{g}} & W_1 \ar{r}{f} \ar{d}{s} & Y \\
                & & X
            \end{tikzcd}
        \end{center}
        Observe that the three squares commute, as by the definition of right Ore condition. Thus it follows that $s\widetilde{s}\widetilde{w} = s\widehat{s}\widehat{w}$, and that $t\widehat{f}\widehat{w}=t\widetilde{g}\widetilde{w}$. As $t:S$ one may use right cancellation to find a $\xi:\bar{W}\rightarrow W$ such that $\widehat{f}\widehat{w}\xi = \widetilde{g}\widetilde{w}\xi \implies g\widehat{f}\widehat{w}\xi = g\widetilde{g}\widetilde{w}\xi$. Thus the equivalence relation diagram commutes.
        \begin{center}
            \begin{tikzcd}
                & \widehat{W} \ar{ld}[above]{s\widehat{s}} \ar{rd}{g\widehat{f}} \\
                X & \bar{W} \ar{u}{\widehat{w}\xi} \ar{d}{\widetilde{w}\xi} \ar{l} \ar{r} & Z \\
                & \widetilde{W} \ar{lu}{s\widetilde{s}} \ar{ru}[below]{g\widetilde{g}}
            \end{tikzcd}
        \end{center}
    \end{proof}

    \begin{prop}
        The composition of right fractions is associative.
    \end{prop}

    \begin{proof}
        \todo[color = pink]{Kommer det fram at dette er en proof sketch?} Let $fs^{-1}$, $gt^{-1}$ and $hu^{-1}$ be right fractions as in the diagrams below.
        \begin{center}
            \begin{tikzcd}
                A & X \ar{l}{s} \ar{r}{f} & B
            \end{tikzcd}
            ,
            \begin{tikzcd}
                B & Y \ar{l}{t} \ar{r}{g} & C
            \end{tikzcd}
            \&
            \begin{tikzcd}
                C & Z \ar{l}{u} \ar{r}{h} & D
            \end{tikzcd}
        \end{center}
        There are two different ways of calculating the compostion. Every morphism in $S$ will be marked blue.
        \begin{center}
            \begin{minipage}[c]{0.4\textwidth}
                \underline{$hu^{-1}\circ (gt^{-1}\circ fs^{-1})$}\\
                \begin{tikzcd}
                    W \ar{rr} \ar[blue]{d}{} & & Z \ar{r}{} \ar[blue]{d}{} & D \\
                    V \ar[blue]{d}{} \ar{r}{} & Y \ar{r}{} \ar[blue]{d}{} & C \\
                    X \ar[blue]{d}{} \ar{r}{} & B \\
                    A
                \end{tikzcd}
            \end{minipage}
            \begin{minipage}[c]{0.4\textwidth}
                \underline{$(hu^{-1}\circ gt^{-1})\circ fs^{-1}$}\\
                \begin{tikzcd}
                    V' \ar{r}{} \ar[blue]{dd}{} & W' \ar{r}{} \ar[blue]{d}{} & Z \ar{r}{} \ar[blue]{d}{} & D \\
                    & Y \ar{r}{} \ar[blue]{d}{} & C \\
                    X \ar{r}{} \ar[blue]{d}{} & B \\
                    A
                \end{tikzcd}
            \end{minipage}
        \end{center}
        To be able to find a relation between these diagrams create another diagram with the right Ore condition.
        \begin{center}
            \begin{minipage}[c]{0.3\textwidth}
                \begin{tikzcd}
                    T \ar[dashed, blue]{r}{} \ar[dashed, blue]{d}{} & V' \ar[blue]{d} \\
                    W \ar[blue]{r}{} & X
                \end{tikzcd}
            \end{minipage}
            \begin{minipage}[c]{0.5\textwidth}
                To finish the proof, one would need to show that the maps to $A$ and $D$ commute. The maps to A commute right out of the bat, by the right Ore condition. To prove that the maps to D commute, first apply right cancellation on the maps to B, then on the maps to C.
            \end{minipage}
        \end{center}
    \end{proof}

    \begin{definition}
        Let $S$ be a right multiplicative system in a category $\mathcal{C}$. Define a category $\mathfrak{r}S^{-1}\mathcal{C}$ to have objects $\mathfrak{Obr}S^{-1}\mathcal{C}=\mathfrak{Ob}\mathcal{C}$ and morphisms $\mathfrak{Arr}S^{-1}\mathcal{C} = \{$right fractions of $S\}/\sim$. This means that the morphisms $\mathfrak{r}S^{-1}\mathcal{C}(X,Y)$ are spans in $\mathcal{C}$ where one of the maps are in $S$ up to equivalence.
        \begin{center}
            \begin{tikzcd}
                X & A \ar[blue]{l} \ar{r} & Y
            \end{tikzcd}
        \end{center}
        This is well-defined by the previous results and the identity morphisms are the right fractions of the form:
        \begin{center}
            \begin{tikzcd}
                X & X \ar[equal]{l} \ar[equal]{r} & X
            \end{tikzcd}
        \end{center}
    \end{definition}

    \begin{remark}
        Dually there is a category $\mathfrak{l}S^{-1}\mathcal{C}$ for a left multiplicative system $S$ in a category $\mathcal{C}$. It is defined in the same manner as $\mathfrak{r}S^{-1}\mathcal{C}$, but with left fractions instead.
    \end{remark}

    \begin{remark}
        \todo{Det virker som at det som foregår her er galt. Jeg må først skaffe oversikt over hva som skjer} Given that $S$ is right multiplicative, A right fraction from the object $A$ to the object $B$ can be described with a special kind of diagram. Let $A\downarrow S$ be the comma category of arrows from $S$ starting at $A$ and let $\delta : A\downarrow S\rightarrow\mathcal{C}$ be the forgetful functor, sending each arrow to its codomain. A morphism in $A\downarrow S$ from the objects $(b,B')$ to $(c,C')$ is a morphism $t : B'\rightarrow C'$ such that $b=ct$. We see that there is a correspondance between right fractions and elements in components of diagrams over $A\downarrow S$ such as $\mathcal{C}(\delta b, B)=\mathcal{C}(B',B)$ and right fractions. That is, let $f:\delta(b, B)\rightarrow B'$, then $f$ can be regarded as $fb^{-1}$. A morphism from $\mathcal{C}(\delta c, B)$ to $\mathcal{C}(\delta b, B)$ is a morphism induced by a morphism from $(b,B')$ to $(c,C')$ in $A\downarrow S$. By the equivalence relation above we want to fractions $fb^{-1}$ and $gc^{-1}$ to be identified if there exists morphisms from $\mathcal{C}(\delta d, B)$ with maps $b' : (d,D')\rightarrow (b,B')$ and $c' : (d,D')\rightarrow (c,C')$ in $A\downarrow S$ such that $b'*f = c'*g$. This would be the same as saying that the right fractions are the coequalizer of the diagram $\mathcal{C}(\delta b, B)\coprod \mathcal{C}(\delta c, B)\rightrightarrows \mathcal{C}(\delta d, B)$. This observation motivates that the right fractions from $A$ to $B$ is described as the colimit of the functor $\mathcal{C}(\delta\_, B):A\downarrow S\rightarrow SET$. Dually, if $S$ is left multiplicative we get that the left fractions from $A$ to $B$ can be described as the colimit of the functor $\mathcal{C}(A, \rho\_):S\downarrow B\rightarrow SET$. More details can be found in \cite{zisman} and \cite{weibel}.
    \end{remark}

    To ensure us that these categories $\mathfrak{r}S^{-1}\mathcal{C}$ does indeed exist there are many different criteria which we can place upon our assumptions. A natural restriction is to ensure that the colimits above exists as sets

    \begin{definition}
        \todo{Dette må også fikses etter at det over er fikset!!!} A multiplicative system $S$ in a locally small category $\mathcal{C}$ is called locally small on the right if for every object $X:\mathcal{C}$ there is a set $S_X$ of morphisms from $S$ such that for every morphism $f : X_1 \rightarrow X$ in $S$ there is a morphism $f' : X'\rightarrow X$ in $S_X$ factoring thorugh $f$.

        The dual of this definition will be called a locally small multiplicative system on the left. If it is both locally small on the left and the right, we will simply call it locally small. 
    \end{definition}

    \begin{remark}
        If $S$ is a left multiplicative system, then $S\downarrow A$ is a filtered category for every object $A$. Dually, if $S$ is right multiplicative then $A\downarrow S$ is cofiltered.
    \end{remark}

    \begin{remark}
        Equipped with this notion we are now able to prove that the localizations exists as locally small categories. Locally small right multiplicative systems allows us to prove that the classes $\mathfrak{r}S^{-1}\mathcal{C}(X,Y)$ are sets. This can be seen as we can regard $S_X$ as a small category. By using the right Ore condition and left cancellation we can extend $S_X$ such that it is again cofiltered and admits the same colimit, i.e.\\
         $\varinjlim\mathcal{C}(\delta\_,B):A\downarrow S\rightarrow Set\simeq\varinjlim\mathcal{C}(\delta\_,B):S_A\rightarrow Set$. 
    \end{remark}

    \begin{theorem}
        \textbf{Gabriel-Zisman}. Let $S$ be a locally small right multiplicative system of morphisms in a category $\mathcal{C}$. Then the category $\mathfrak{r}S^{-1}\mathcal{C}$ exists and it is the localization of $\mathcal{C}$ on $S$. This mean that there is an equivalence of categories $\mathcal{C}[S^{-1}]\simeq\mathfrak{r}S^{-1}\mathcal{C}$ together with a functor $q: \mathcal{C}\rightarrow\mathfrak{r}S^{-1}\mathcal{C}$ sending a morphism $f : X\rightarrow Y$ to the right fraction $fid_X^{-1}$.
    \end{theorem}

    \begin{proof}
        To prove the theorem one must show that $q$ is a functor, and that it is universal. Suppose that $f: X\rightarrow Y$ and $g: Y\rightarrow Z$ are morphisms in $\mathcal{C}$. Then $q(gf)=(gf)id_X^{-1}$ and $q(g)q(f)=(gid_Y^{-1})\circ(fid_X^{-1})$. Choose the compostion to be defined by the diagram below.
        \begin{center}
            \begin{tikzcd}
                X \ar{r}{f} \ar[equal,blue]{d} & Y \ar{r}{g} \ar[equal,blue]{d} & Z\\
                X \ar{r}{f} \ar[equal,blue]{d} & Y \\
                X
            \end{tikzcd}
        \end{center}
        Observe that $(gid_Y^{-1})\circ(fid_X^{-1})=(gf)id_X^{-1}$, asserting the functoriality of $q$.

        To see that $q$ is universal let $\mathcal{D}$ be a category where every morphism of $S$ is an isomorphism, and suppose there is a functor $F:\mathcal{C}\rightarrow\mathcal{D}$. Define a functor $\mathfrak{r}S^{-1}F : \mathfrak{r}S^{-1}\mathcal{C}\rightarrow\mathcal{D}$ by $\mathfrak{r}S^{-1}F(fs^{-1})=F(f)F(s)^{-1}$. One may see that $F = \mathfrak{r}S^{-1}F\circ q$, it remains to show that it is well-defined. Suppose $fs^{-1}=gt^{-1}$, that means there is a diagram in $\mathcal{C}$ with the blue arrows in $S$.
        \begin{center}
            \begin{tikzcd}
                & W' \ar[blue]{ld}{s} \ar{rd}{f}  \\
                X & W \ar[blue]{u}{w'} \ar[blue]{d}{w''} & Y \\
                & W'' \ar[blue]{lu}{t} \ar{ru}{g}
            \end{tikzcd}
        \end{center}
        Thus there is a relationship in $\mathcal{D}$ such that $F(t)=F(sw')F(w'')^{-1}$ and $F(g)=F(fw')F(w'')^{-1}$. This again shows that 
        \begin{multline*}
            \mathfrak{r}S^{-1}F(gt^{-1})=F(g)F(t)^{-1}\\
            =F(fw')F(w'')^{-1}(F(fw')F(w'')^{-1})^{-1}=F(fw')F(w'')^{-1}F(w'')F(sw')^{-1}\\
            =F(f)F(w')F(w')^{-1}F(s)^{-1}=F(f)F(s)^{-1}=\mathfrak{r}S^{-1}F(fs^{-1})
        \end{multline*}
        It follows that $\mathfrak{r}S^{-1}F$ is well-defined and is unique by construction.
    \end{proof}

    \begin{corollary}
        If $S$ is a locally small left multiplicative system instead then $\mathfrak{l}S^{-1}\mathcal{C}$ is the localization of $\mathcal{C}$ on $S$.

        If moreover $S$ is a locally small multiplicative system, then there is an equivalence of categories $\mathfrak{r}S^{-1}\mathcal{C}\simeq\mathfrak{l}S^{-1}\mathcal{C}$.
    \end{corollary}

    \begin{proof}
        The first statement is dual to the theorem.

        To see the other statement, note that both $\mathfrak{r}S^{-1}\mathcal{C}$ and $\mathfrak{l}S^{-1}\mathcal{C}$ are the universal categories where the morphisms of $S$ are isomorphisms. Thus it follows that these categories have to be equivalent.
    \end{proof}

    \begin{remark}
        Since righthandedness of lefthandedness of the multiplicative system $S$ doesn't affect the localization, one simply call the localization of a (left/right) multiplicative system for $S^{-1}\mathcal{C}$.
    \end{remark}

    \begin{remark}
        A morphisms $f:\mathcal{C}(X,Y)$ will be invertible in the localized category if it is in the same equivalence class as the identity, both $id_X$ and $id_Y$. This forces a morphism $f$ to be invertible in $S^{-1}\mathcal{C}$ if and only if there is $g,h:S$ such that $fg,hf:S$.
    \end{remark}

    \begin{prop}
        Let $\mathcal{C}$ be a category, and $S$ a right multiplicative set of morphisms. The cannonical functor $q:\mathcal{C}\rightarrow S^{-1}\mathcal{C}$ commutes with finite limits.
    \end{prop}

    \begin{proof}
        Let $T:\mathcal{D}\rightarrow\mathcal{C}$ be a diagram over a finite category $\mathcal{D}$. Then for any object $A:S^{-1}\mathcal{C}$ one may find the following equation.
        \begin{multline*}
            S^{-1}\mathcal{C}(qA,q(\varprojlim T\_)\simeq \varinjlim\mathcal{C}(\delta\_,\varprojlim T\_))\\
            \simeq \varinjlim\varprojlim\mathcal{C}(\delta\_,T\_)\simeq \varprojlim\varinjlim\mathcal{C}(\delta\_,T\_)\simeq \varprojlim S^{-1}\mathcal{C}(qA,q(T\_))
        \end{multline*}
        The first isomorphism is given by the remark, the second is given by the representative nature of finite limits and the third isomorphism is given by that filtered colimits commute with finite limits. The colimits are filtered by the remark that $S_A$ is cofiltered and that the functor $\mathcal{C}(_,A)$ is contravariant.
    \end{proof}

    \todo[color=red]{I am not quite sure yet how this argument proves the statement I want to prove, but that can be figure out later. I also don't know the proof for why filtered colimits commute with finite colimits, but it is in Riehls book.}

    \begin{prop}
        Let $\mathcal{C}$ be a category with a zero. That is an object which is both initial and terminal. Suppose that $S$ is a right multiplicative system, then $q0$ is a zero object in $S^{-1}\mathcal{C}$.
    \end{prop}

    \begin{proof}
        The claim that $q0$ is initial follows from that initial is a limit of a diagram over the empty category. To see that $q0$ is terminal one have to prove that every right fraction of the form $0f^{-1}$ is equivalent to $0id_A^{-1}$, where $A$ is the codomain of $f$. This fact can be seen with the diagram below.
        \begin{center}
            \begin{tikzcd}
                & X \ar{ld}[above]{f} \ar{d}{f} \ar{rd}{0}\\
                A & \ar[equal]{l} A \ar{r}{0} & 0
            \end{tikzcd}
        \end{center}
    \end{proof}

    \begin{prop}
        If $\mathcal{A}$ is an additive category and $S$ is a right multiplicative system, then $S^{-1}\mathcal{A}$ is additive as well.
    \end{prop}

    \begin{proof}
        From the previous propositions it is known that $q0$ is the zero object and that $q(A\times B)\simeq qA\times qB$. By proving that there is an addition induced by $\mathcal{A}$ and that $q$ preserves this addition one obtains that the product is the biproduct induced by the maps in $\mathcal{A}$.

        Suppose that there are fractions $fs^{-1}, gt^{-1}:S^{-1}\mathcal{C}(A,B)$. Define their addition by using the right Ore condition to find new morphisms $f'$, $g'$ and $u$ such that $fs^{-1} = f'u^{-1}$ and $gt^{-1} = g'u^{-1}$.
        \begin{equation*}
            fs^{-1}+gt^{-1} = (f'+g')u^{-1}
        \end{equation*}
        To prove that this is an addition one must prove that it is well defined; associativity, inverses and commutativity will be inherited from $\mathcal{A}$.
        Let $\bar{f}$, $\bar{g}$ and $v$ be another choice provided by the right Ore condition. To summarize, the equations $\bar{f}v^{-1}=fs^{-1}=f'u^{-1}$ and $\bar{g}v^{-1}=gt^{-1}=g'u^{-1}$ have been established. In order to prove well-definedness, one must show that $(\bar{f}+\bar{g})v^{-1}-(f'+g')u^{-1}=0$. By definition $(\bar{f}+\bar{g})v^{-1}-(f'+g')u^{-1}=\bar{f}v^{-1}-f'u^{-1}+\bar{g}v^{-1}-g'u^{-1}$. Proving that the whole sum is $0$, is the same as proving that $\bar{f}v^{-1}+(-f')u^{-1}=(\bar{\bar{f}}-f'')w^{-1}=0$. This can be done by writing out the diagrams after repeatedly applying the right Ore condition.
        \begin{center}
            \begin{tikzcd}
                \cdot \ar[bend right, dashed, blue]{rddd}{p} \ar[bend left, dashed, blue]{rrrd}{p} \ar[dashed, blue]{rd} \\
                & \cdot \ar[blue]{r} \ar[blue]{d} \ar[blue]{rd}{w} & \cdot \ar[blue]{r} \ar[blue]{d}{u} & \cdot \ar{r}{f} \ar[blue]{ld}{s} & B \\
                & \cdot \ar[blue]{d} \ar[blue]{r}{v} & A \\
                & \cdot \ar[blue]{ru}{s} \ar{d}{f} \\
                & B
            \end{tikzcd}
        \end{center}
        The line to the bottom represents $\bar{\bar{f}}$ and the line to the right represents $f''$. Using left cancellation on the common morphism $s$ into $A$ one obtains the morphism $p$, which relates the two fractions and make the sum go to zero.

        It remains to show that $q:\mathcal{C}\rightarrow S^{-1}\mathcal{C}$ respects addition. Assume that $f,g:\mathcal{C}(X,Y)$, then
        \begin{equation*}
            q(f+g)=(f+g)id_X^{-1}=fid_X^{-1}+gid_X^{-1}=qf+qg.
        \end{equation*}
    \end{proof}

    \begin{corollary}
        If $\mathcal{A}$ is abelian and $S$ is a multiplicative system, then $S^{-1}\mathcal{A}$ is abelian as well.
    \end{corollary}

    \todo{Kan skrive noe her for å løsne overgangen fra lokalisering til lokalisering av triangulerte kategorier}

    \todo[color = pink]{Jeg må fikse funktor notasjonen min. Den er vanskelig å lese...}

    \begin{definition}
        A triangulated functor $F: \mathcal{T} \rightarrow \mathcal{S}$ between two triangulated categories $(\mathcal{T}, T, \Delta_\mathcal{T}$ and $(\mathcal{S}, S, \Delta_\mathcal{S})$, is an additive functor along with a natural isomorphism $\phi_X : F(T(X)) \rightarrow S(F(X))$ such that $F(\Delta_{\mathcal{T}}) \subseteq \Delta_{\mathcal{S}}$. This means that for every triangle in $\mathcal{T}$ there is a triangle in $\mathcal{S}$.
        \begin{center}
            \begin{tikzcd}[row sep=tiny]
                A \arrow{rd}{a} & \\
                & B \arrow{dl}{b} & & \\
                C \arrow[very near end, "|" marking]{uu}[near start]{c}[near end]{T}
            \end{tikzcd}
            $\implies$
            \begin{tikzcd}[row sep=tiny]
                F(A) \arrow{rd}{F(a)} & \\
                & F(B) \ar{dl}{F(b)} & & \\
                F(C) \arrow[very near end, "|" marking]{uu}[near start]{F(c)}[near end]{T}
            \end{tikzcd}
        \end{center}
    \end{definition}

    \begin{definition}
        A triangulated subcategory $\mathcal{S}$ of a triangulated category $\mathcal{T}$ is a full additive subcategory such that the inclusion functor is triangulated.
    \end{definition}

    \begin{definition}
        Let $F : \mathcal{S} \rightarrow \mathcal{T}$ be a triangulated functor. The kernel of $F$ is defined to be the full subcategory $Ker(F)$ of $\mathcal{S}$ such that every object in $Ker(F)$ gets mapped to $0$ by $F$. That is, $Ker(F)$ is the class of objects $\{K : \mathcal{S} | F(K)\simeq 0\}$.
    \end{definition}

    \begin{lemma}
        The kernel of a triangulated functor $F:\mathcal{C}\rightarrow{D}$ is a triangulated subcategory.
    \end{lemma}

    \begin{proof}
        Let $X:KerF$, since $F$ is a triangulated functor $CX:KerF$ as $F(CX)=D(FX)=D0=0$. As $F$ is triangulated, one has that every triangle maps to a triangle. Let $X,Y:KerF$, then:
        \begin{center}
            \begin{tikzcd}[row sep=small]
                X \ar{rd} \\
                & Y \ar{ld} \\
                Z \ar[very near end, "|" marking]{uu}[near end]{T}
            \end{tikzcd}
            $\implies$
            \begin{tikzcd}[row sep=small]
                0 \ar{rd} \\
                & 0 \ar{ld}\\
                F(Z) \ar[very near end, "|" marking]{uu}[near end]{T}
            \end{tikzcd}
        \end{center}
        By TR3 and the 2 out of 3 property $F(Z)\simeq 0 \implies Z:KerF$. Thus $KerF$ is a triangulated subcategory of $\mathcal{C}$.
    \end{proof}

    \begin{definition}
        A subcategory $\mathcal{S}$ of a triangulated category $\mathcal{T}$ is called thick if it contains all the direct summands of its objects.
    \end{definition}

    \begin{lemma}
        The kernel of a triangulated functor $F:\mathcal{C}\rightarrow\mathcal{D}$ is thick.
    \end{lemma}

    \begin{proof}
        Let $X\oplus Y:KerF$, since $F$ is additive one may see that $0\simeq F(X\oplus Y)\simeq F(X)\oplus F(Y)$, but then there is a splitmono $F(X)\rightarrow 0 \implies F(X)\simeq 0 \simeq F(Y)$.
    \end{proof}

    \begin{lemma}
        Let $F:\mathcal{C}\rightarrow\mathcal{D}$ be a triangulated functor. Suppose that $f:X\rightarrow Y$ is a morphism such that $F(f)$ is an isomorphism. Then the cone of $f$ is in $KerF$.
    \end{lemma}

    \begin{proof}
        There is an isomorphism of triangles in $\mathcal{D}$, showing that the cone of $f$ is in $KerF$.
        \begin{center}
            \begin{tikzcd}
                FX \ar{r}{Ff} \ar[equal]{d} & FY \ar{r} \ar[equal]{d} & F(cone(f)) \ar{r} \ar[dashed]{d}[rotate=90, below]{\simeq} & FTX \ar[equal]{d} \\
                FX \ar{r}{Ff} & FY \ar{r} & 0 \ar{r} & FTX
            \end{tikzcd}
        \end{center}
    \end{proof}

    The goal for the rest of this section is to prove that there is a localization at any triangulated subcategory $\mathcal{S}\subseteq\mathcal{C}$. This localization will yield a functor $q:\mathcal{C}\rightarrow \mathcal{C}/\mathcal{S}$ such that $\mathcal{S}\subseteq Kerq$. There is a set of morphism $Mor_\mathcal{S}$ related to $\mathcal{S}$ such that this set is multiplicative.

    \begin{definition}
        Let $\mathcal{C}$ be a triangulated category and $\mathcal{S} \subseteq \mathcal{C}$ be a triangulated subcategory. Define the collection $Mor_{\mathcal{S}}$ to be a collection of morphisms in $\mathcal{C}$ such that for any $f : Mor_{\mathcal{S}}$ there is a triangle with $C : \mathcal{S}$.
        \begin{center}
            \begin{tikzcd}
                A \ar{r}{f} & B \ar{r} & C \ar{r} & TA 
            \end{tikzcd}
        \end{center}
    \end{definition}

    \begin{remark}
        Every isomorphism is in $Mor_{\mathcal{S}}$. This is because isomorphisms are found in triangles $(A,B,0,f,0,0)$ and $0 : \mathcal{S}$ for any triangulated subcategory.
    \end{remark}

    \begin{lemma}
        Let $f : X \rightarrow Y$ and $g : Y \rightarrow Z$ be two morphisms. If any two of the morphisms $f$, $g$ and $gf$ are in $Mor_{\mathcal{S}}$ then so is the third.
    \end{lemma}

    \begin{proof}
        We are able to find three triangles in $\mathcal{C}$.
        \begin{center}
            (1)
            \begin{tikzcd}[row sep=tiny]
                X \arrow[red]{rd}[black]{f} & \\
                & Y \arrow[red]{dl} & & \\
                Z' \arrow[red, very near end, "|" marking]{uu}[black, near end]{T}
            \end{tikzcd}
            (2)
            \begin{tikzcd}[row sep=tiny]
                Y \arrow[orange]{rd}[black]{g} & \\
                & Z \arrow[orange]{dl} & & \\
                X' \arrow[orange, very near end, "|" marking]{uu}[black, near end]{T}
            \end{tikzcd}
            (3)
            \begin{tikzcd}[row sep=tiny]
                A \arrow[violet]{rd}[black]{g\circ f} & \\
                & C \arrow[violet]{dl} & & \\
                Y' \arrow[violet, very near end, "|" marking]{uu}[black, near end]{T}
            \end{tikzcd}
        \end{center}
        By the Octahedron axiom there exist another triangle in $\mathcal{C}$:
        \begin{center}
            \begin{tikzcd}
                Z' \ar[teal]{r} & X' \ar[teal]{r} & Y' \ar[teal]{r} & TZ'
            \end{tikzcd}
        \end{center}
        Note that $f$ is in $Mor_\mathcal{S}$ if and only if $Z' : S$. WLOG assume that $f$ and $g$ is in $Mor_\mathcal{S}$, this can be done by the rotation axiom. Thus one may find the triangle in $\mathcal{S}$ by TR1 $(Z',X',Y'')$ proving that $Y'\simeq Y''$.
        \begin{center}
            \begin{tikzcd}
                Z' \ar[teal]{r} \ar[equal]{d} & X' \ar[equal]{d} \ar[teal]{r} & Y' \ar[teal]{r} \ar{d}[rotate=90, below]{\simeq} & TZ' \ar[equal]{d} \\
                Z' \ar{r} & X' \ar{r} & Y'' \ar{r} & TZ'
            \end{tikzcd}
        \end{center}
        To see that $gf$ is in $Mor_\mathcal{S}$ one can construct the triangle below with the isomorphism given above.
        \begin{center}
            \begin{tikzcd}
                A & C & Y'' & TA
            \end{tikzcd}
        \end{center}
    \end{proof}

    \begin{prop}
        Let $\mathcal{S}\subseteq\mathcal{C}$ be a triangulated subcategory, then $Mor_\mathcal{S}$ satisfies the Ore condition.
    \end{prop}

    \begin{proof}
        To prove that a system satisfies the Ore condition there has to be a proof for both right and left condition. Luckily, the arguments presented here can be dualized to give a proof for the other condition. Thus there will only be presented a proof for the right Ore condition.
        Let $f:A\rightarrow C$ be in $Mor_\mathcal{S}$ and $g:B\rightarrow C$ in $\mathcal{C}$. Then one may form a homotopy pullback creating a homotopy cartesian square as below.
        \begin{center}
            \begin{tikzcd}
                & A \ar{d}{f} \\
                B \ar{r}{g} & C
            \end{tikzcd}
            $\implies$
            \begin{tikzcd}
                D \ar{r}{g'} \ar{d}{f'} \ar[phantom]{rd}[description]{HO}[very near start]{\ulcorner}[very near end]{\lrcorner} & A \ar{d}{f} \\
                B \ar{r}{g} & C
            \end{tikzcd}
        \end{center}
        By Lemma 1.2.5 there are triangles along this homotopy cartesian square identifying the cones. Since the cone of $f$ is assumed to be in $\mathcal{S}$, the cone of $f'$ is also in $\mathcal{S}$. This proves that $f':Mor_\mathcal{S}$.
    \end{proof}

    \begin{prop}
        For any parallell morphism $f,g:X\rightarrow Y$ in $\mathcal{C}$ the following are equivalent:
        \begin{enumerate}
            \item $sf=sg$ for some $s:Mor_\mathcal{S}$ starting at $Y$.
            \item $ft=gt$ for some $t:Mor_\mathcal{S}$ ending at $X$.
            \item $f-g$ factors through an object $C:\mathcal{S}$.
        \end{enumerate}
    \end{prop}

    \begin{proof}
        $(1.\iff 3.)$:
        Suppose that there exists an $s:Y\rightarrow Z$ such that $s(f-g)=0$. By TR1 there is a triangle \begin{tikzcd}Y \ar{r}{s} & Z \ar{r}{Ts'} & TC \ar{r} & TY \end{tikzcd} and a long exact sequence.
        \begin{center}
            \begin{tikzcd}
                \mathcal{C}(X,C) \ar{r}{s'_*} & \mathcal{C}(X,Y) \ar{r}{s_*} & \mathcal{T}(X,Z) \\
                p \ar[pos=0, "|" marking]{r}[pos=0.5]{s'_*}& f-g \ar[pos=0, "|" marking]{r}[pos=0.5]{s_*} & 0
            \end{tikzcd}
        \end{center}
        Since $s(f-g)=0$ there exists a $p:\mathcal{C}(X,C)$ such that $f-g = s'_*p$. By definition, $s:Mor_\mathcal{S}\iff C:\mathcal{S}$, but $s:Mor_\mathcal{S}\implies f-g$ factors through $C$, and vice versa.
        $(2.\iff 3.)$:
        This argument is dual.
    \end{proof}

    This has shown that $Mor_\mathcal{S}$ is a multiplicative system, and Theorem 1.3.4 say that the localization exists given that $Mor_\mathcal{S}$ is locally small. The category $Mor_\mathcal{S}^{-1}$ will be denoted as $\mathcal{C}/\mathcal{S}$ and it is called the Verdier quotient. As $\mathcal{C}$ is additive, it is known that $\mathcal{C}/\mathcal{S}$ is additive as well by Proposition 2.20. The remaining part is to show that $\mathcal{C}/\mathcal{S}$ is triangulated and that localization functor $q:\mathcal{C}\rightarrow \mathcal{C}/\mathcal{S}$ is a triangulated functor.

    \begin{theorem}
        Let $\mathcal{S}\subseteq\mathcal{C}$ be triangulated categories. Then the Verdier quotient $\mathcal{C}/\mathcal{S}$ together with the functor $q:\mathcal{C}\rightarrow\mathcal{C}/\mathcal{S}$ is the universal triangulated category where morphisms in $Mor_\mathcal{S}$ are isomorphisms.
    \end{theorem}

    \begin{proof}
        The triangulation on $\mathcal{C}/\mathcal{S}$ is defined as the following. Let $C/S:\mathcal{C}/\mathcal{S}\rightarrow\mathcal{C}/\mathcal{S}$ be the additive autoequivalence defined by $C/S(A)=C(A)$. Since $q:\mathcal{C}\rightarrow\mathcal{C}/\mathcal{S}$ maps every object to itself it follows that $q(C(A)) \simeq C/S(A) = C/S(q(A))$, and define $\Delta_{\mathcal{C}/\mathcal{S}}\supseteq q(\Delta_\mathcal{C})$ such that $\Delta_{\mathcal{C}/\mathcal{S}}$ has every isomorphism class of $q(\Delta_\mathcal{C})$. 
        \begin{center}
            \begin{tikzcd}[row sep=small]
                qX \ar{rd} \\
                & qY \ar{ld} \\
                qZ \ar[very near end, "|" marking]{uu}[near end]{T}
            \end{tikzcd}
            $\impliedby$
            \begin{tikzcd}[row sep=small]
                X \ar{rd} \\
                & Y \ar{ld} \\
                Z \ar[very near end, "|" marking]{uu}[near end]{T}
            \end{tikzcd}
        \end{center}
        Then by definition $q$ is triangulated if the category $\mathcal{C}/\mathcal{S}$ is triangulated.
        By definition, the triangles are closed under isomorphisms, $(X,X,0,id_X,0,0)$ is a triangle, and TR2 holds. Thus it remains to show TR1 and TR4 (TR3 is implied by the other axioms). To prove TR1, let $fs^{-1}:\mathcal{C}/\mathcal{S}(qW,qY)$. Expand $f:\mathcal{C}(X,Y)$ to a triangle in $\mathcal{C}$ with TR1, it will induce a triangle in  $\mathcal{C}/\mathcal{S}$.
        \begin{center}
            \begin{tikzcd}
                qX \ar{r}{fid_X^{-1}} & qY \ar{r}{gid_Y^{-1}} & qZ \ar{r}{hid_Z^{-1}} & qTX
            \end{tikzcd}
        \end{center}
        There is an isomorphism to the following candidate triangle from the induced triangle, proving TR1.
        \begin{center}
            \begin{tikzcd}
                qX \ar{r}{fid_X^{-1}} \ar{d}{sid_X^{-1}}[above, rotate = 90]{\simeq} & qY \ar{r}{gid_Y^{-1}} \ar[equal]{d} & qZ \ar{r}{hid_Z^{-1}} \ar[equal]{d} & qTX \ar{d}{(Ts)id_{Tx}^{-1}}[above, rotate = 90]{\simeq} \\
                qW \ar{r}{fs^{-1}} & qY \ar{r}{gid_Y^{-1}} & qZ \ar{r}{(Ts)hid_Z^{-1}} & qTW
            \end{tikzcd}
        \end{center}
        To show the Octahedron axiom, suppose that there are three triangles in $\mathcal{C}/\mathcal{S}$. By construction, these triangles can be chosen such that only the first map is a fraction up to isomorphism of triangles.
        \begin{center}
            (1)
            \begin{tikzcd}[row sep=tiny]
                Z \ar{r}{t'}& X \ar{ld}[above]{s} \ar{dd}{a} \\
                A \arrow[red]{rd}[black]{as^{-1}} \\
                & B \arrow[red]{dl}[black]{x}\\
                C' \arrow[red, very near end, "|" marking]{uu}[near start, black]{x'}[near end]{T}
            \end{tikzcd}
            (2)
            \begin{tikzcd}[row sep=tiny]
                & Y \ar{ld}[above]{t} \ar{dd}{b} \\ 
                B \arrow[orange]{rd}[black]{bt^{-1}} &  \\
                & C \arrow[orange]{dl}[black]{y}\\
                A' \arrow[orange, very near end, "|" marking]{uu}[near start, black]{y'}[near end]{T}
            \end{tikzcd}
            (3)
            \begin{tikzcd}[row sep=tiny]
                & Z \ar{ld}[above]{st'} \ar{dd}{ba'} \\
                A \arrow[violet]{rd}[black]{b\circ a} \\
                & C \arrow[violet]{dl}[black]{z} \\
                B' \arrow[violet, very near end, "|" marking]{uu}[near start, black]{z'}[near end]{T}
            \end{tikzcd}
        \end{center}
        This is possible, as when composing the fractions from $A$ to $B$ and $B$ to $C$ one may find an object $Z$ as in the diagram by using the Ore condition. To illustrate with triangle (1), there is a correspondance of triangles in $\mathcal{C}/\mathcal{S}$ and $\mathcal{C}$ by the following isomorphism.
        \begin{center}
            \begin{tikzcd}
                Z \ar{r}{at'} \ar{d}[above]{t}[below]{\simeq} & B \ar{r} \ar[equal]{d} & Z' \ar{r} \ar[dashed]{d}[below, rotate = 90]{\simeq} & TZ \ar[equal]{d} \\
                X \ar{r}{a} \ar{d}{s} & B \ar{r} & C' \ar{r} & TX \ar{d}{Ts} \\
                A & & & TA 
            \end{tikzcd}
        \end{center}
        The result of the octahedron axiom follows as one instead consider the triangles found by the composition of morphisms as below.
        \begin{center}
            \begin{tikzcd}
                Z \ar{d}{f'} \ar{rd}{bf'} \\
                Y \ar{r}{b} & C
            \end{tikzcd}
        \end{center}
    \end{proof}

    \begin{prop}
        Let $\mathcal{S}\subseteq\mathcal{C}$ be triangulated categories. If $0:X\rightarrow 0$ is an isomorphism in $\mathcal{C}/\mathcal{S}$, then there is an object $Y$ such that $X\oplus Y:\mathcal{S}$.
    \end{prop}

    \begin{proof}
        If $0:X\rightarrow 0$ is invertible, then there exist a map $0:0\rightarrow Y$, such that $0:X\rightarrow Y$ is in $Mor_S$. By definition $X\oplus Y$ is in $\mathcal{S}$.
    \end{proof}

    This proposition shows that the kernel of $q:\mathcal{C}\rightarrow\mathcal{C}/\mathcal{S}$ is the smallest thick subcategory of $\mathcal{C}$ such that $\mathcal{C}/Kerq$ is the universal category where every morphism in $Mor_\mathcal{S}$ is an isomorphism. For this reason $\widehat{\mathcal{S}}=Kerq$ is called the thick closure of $\mathcal{S}$.

\section{Universal Homological Embedding}
    Do Yoneda embedding into functor categories.

% \clearpage
