\chapter*{Abstract}
    The aim of this thesis is to give an exposition to theory of triangulated categories, and give some constructions as well. The main goal is to show that the Verdier quotient, that the homotopy category and that the derived category is triangulated.

\chapter*{Sammendrag}
    Denne bacheloroppgaven har som mål i å gi en presentasjon av teorien til triangulerte kategorier, samt gi noen konstruksjoner i tillegg. Hovedmålet er å vise at Verdier kvotientent, homotopikategorien og den deriverte kategorien er triangulerte.

\chapter*{Introduction}

%Where did triangulated come from? Algebraic vs. Topological
    Triangulated categories was defined by Puppe and Verdier independently. Puppes definition was motivated by the homotopy category of Spectra, but he missed the crucial Octahedron axiom. However, when Verdier introduced triangulated categories and derived categories in his PhD thesis published in 1967, he noticed the importance of the Octahedron axiom. As it stands, there are different ways of defining a triangulated category. For instance, Neeman showed that the octahedron axiom is equivalent to having a choice when applying the morphism axiom, such that the mapping cone becomes a triangle itself. Even though the Octahedron axiom is crucial for showing many of the important results, it is not known of a pre-triangulated category which is not triangulated. \\
    
    In practice there are two different types of triangulated categories, topological and algebraic. A triangulated category is said to be topological if it is the stable category of a model category, and likewise algebraic if it is the stable category of a Frobenius category. In definition these types of categories are not similar, but in practice their differences are quite subtle. This thesis will solely focus on algebraic triangulated categories. \\

%How is this text structured?
    This thesis is split into three parts. The first parts aim to give an exposition to the classical theory of triangulated categories. Part two aims to introduce exact categories and show that the stable Frobenius category is triangulated, among with giving some examples of triangulated categories. The third part aims to introduce the derived category of the homotopy category, and give an exposition of some of the theory. \\

    This thesis assumes that elementary category theory is known and the study of abelian categories and derived categories of abelian categories is known. It is not needed to know some representation theory of artin rings, but it is needed for the section on self-injective algebras.
        
    \section*{Notation}
        Introduce notation which will be used in text. A list of notation and description would be nice, so that the reader might scroll back up if something is unclear.

% \clearpage