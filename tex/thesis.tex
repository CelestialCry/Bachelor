\documentclass[11pt]{article}
\usepackage[utf8]{inputenc}
\usepackage[english]{babel}
\usepackage[margin=1in]{geometry}
\usepackage{graphicx}
\graphicspath{{../images/}{../../images/}}
\usepackage{xcolor}

\setlength{\parindent}{0em}
\setlength{\parskip}{1em}

\usepackage{tikz}
\usetikzlibrary{cd}
\usetikzlibrary{decorations.pathmorphing}
\usepackage{amsmath,amsthm,amssymb,amsfonts}  

\newtheorem{theorem}{Theorem}[section]
\newtheorem{corollary}{Corollary}[theorem]
\newtheorem{lemma}[theorem]{Lemma}
\newtheorem{prop}[theorem]{Proposition}

\theoremstyle{definition}
\newtheorem{definition}{Definition}[section]

\theoremstyle{remark}
\newtheorem*{remark}{Remark}
\newtheorem*{prototype}{Prototype}
\newtheorem*{example}{Example}

\newcommand{\upside}[1]{\rotatebox[origin=c]{180}{#1}}
\newcommand{\chain}[1]{#1^{\bullet}}

\usepackage{subfiles} % Best loaded last in the preamble

\title{Thesis}
\author{Thomas Wilskow Thorbjørnsen}
\date{\today}

\begin{document}
    \maketitle
    \clearpage

    \clearpage

    \tableofcontents

    \section{Introduction}
        This is an introduction, welcome! 

        
        Introduce notation which will be used in text. A list of notation and description would be nice, so that the reader might scroll back up if something is unclear.
    \clearpage
    
    \section{Triangulated Categories}
        Probably introduce this section, what is happening and what will be done etc. I can maybe say something about algebraic triangulated categories and topological triangulated categories, and explaining the name cone, fiber and cofiber.
        
        \subsection{Definition and First Properties}
            In this section $\mathcal{T}$ denotes an additive category and $T:\mathcal{T}\rightarrow\mathcal{T}$ is an additive autoequivalence of $\mathcal{T}$, which is often called translation or suspension functor.
            % Definition candidate triangle
            \begin{definition}
                A candidate triangle is a collection $(A,B,C,a,b,c)$ of objects \\ $A,B,C\in T$ and morphisms $a:A\rightarrow B$, $b:B\rightarrow C$, $c:C\rightarrow TA$. These candidate triangles can be drawn as diagrams in the following way:

                \begin{center}
                    \begin{tikzcd}
                        A \arrow{r}{a} & B \arrow{r}{b} & C \arrow{r}{c} & TA
                    \end{tikzcd}
                \end{center}

                A morphism between candidate triangles is a triple of morphism $(\alpha, \beta, \gamma)$, where $\alpha : A \rightarrow A'$, $\beta : B \rightarrow B'$ and $\gamma : C \rightarrow C'$ such that the following diagram commutes:

            \begin{center}
                \begin{tikzcd}
                    A \arrow{r}{a} \arrow{d}{\alpha} & B \arrow{r}{b} \arrow{d}{\beta} & C \arrow{r}{c} \arrow{d}{\gamma} & TA \arrow{d}{T\alpha} \\
                    A' \arrow{r}{a'} & B' \arrow{r}{b'} & C \arrow{r}{c'} & TA'
                \end{tikzcd}
            \end{center}

            \end{definition}

            The naming convention of the candidate triangles isn't standarized, some literatures calls the candidate triangles for triangles instead; see \cite{keller}. This name arises from an alternate description of the diagrams given above. To remove confusion about the domain or codomain of the arrows, one arrow of the triangle is decorated with "$_T$|". This decorator means that the functor T has to be applied to the corresponding edge of the arrow. Thus the c arrow points to TA, not A.

            \begin{center}
                \begin{tikzcd}[row sep=tiny]
                    A \arrow{rd}{a} & \\
                    & B \arrow{dl}{b} & & \\
                    C \arrow[very near end, "|" marking]{uu}[near start]{c}[near end]{T}
                \end{tikzcd}
                \begin{tikzcd}[row sep=tiny]
                    A \arrow{rd}[description]{a} \arrow{rrr}{\phi_a} & & & A' \arrow[ld, "a'" description] \\
                    & B \arrow{dl}[description]{b} \arrow{r}{\phi_b} & B' \arrow{rd}[description]{b'}\\
                    C \arrow{uu}[very near end, marking]{|}[near start, description]{c}[near end]{T} \arrow{rrr}{\phi_c} & & & C' \arrow{uu}[very near end, marking]{|}[near start, description]{c'}[near end]{T}
                \end{tikzcd}
            \end{center}

            A triangulated category is an additive category together with a translation functor $T$ and a triangulation $\Delta$ consisting of candidate triangles. When a candidate triangle is an element of $\Delta$ it is usually called a  triangle, an exact triangles or just a triangle. Note that if the candidate triangles are referred to as triangles it is common to either call the elements of $\Delta$ for  triangles or exact triangles. As this is not the case for this thesis these objects will be referred to as triangles.

            % Definition of Triangulation
            \begin{definition}
                A triangulation of an additive category $\mathcal{T}$ with translation $T$ is a collection $\Delta$ of triangles consisting of candidate triangles in $\mathcal{T}$ satisfying the following axioms: 

                \begin{enumerate}
                    \item (TR1) Formation axiom

                        \begin{enumerate}
                            \item A candidate triangle isomorphic to a triangle is a triangle.
                            \item Every morphism $a : A \rightarrow B$ can be embedded into a triangle $(A,B,C,a,b,c)$:
                            \begin{center}
                                \begin{tikzcd}
                                    A \arrow{r}{a} & B \arrow{r}{b} & C \arrow{r}{c} & TA
                                \end{tikzcd}
                            \end{center}
                            \item For every object A there is a triangle $(A,A,0,id_A,0,0)$:
                            \begin{center}
                                \begin{tikzcd}
                                    A \arrow{r}{id_A} & A \arrow{r}{0} & 0 \arrow{r}{0} & TA
                                \end{tikzcd}
                            \end{center}
                        \end{enumerate}
                    \item (TR2) Rotation axiom

                        For every triangle $(A,B,C,a,b,c)$ there is a triangle $(B,C,TA,b,c,Ta)$
                        \begin{center}
                            \begin{tikzcd}
                                A \arrow{r}{a} & B \arrow{r}{b} & C \arrow{r}{c} & TA
                            \end{tikzcd} $\implies$
                            \begin{tikzcd}
                                B \arrow{r}{b} & C \arrow{r}{c} & TA \arrow{r}{-Ta} & TB
                            \end{tikzcd}
                            
                        \end{center}
                    \item (TR3) Morphism axiom
                    
                        Given the two triangles $(A,B,C,a,b,c)$ (1) and $(A',B',C',a',b',c')$ (2)
                        \begin{center}
                            (1)
                            \begin{tikzcd}[column sep=small]
                                A \ar{r}{a} & B \ar{r}{b} & C \ar{r}{c} & TA
                            \end{tikzcd}
                            (2) 
                            \begin{tikzcd}[column sep=small]
                                A' \ar{r}{a'} & B' \ar{r}{b'} & C' \ar{r}{c'} & TA'
                            \end{tikzcd}
                        \end{center}
                        and morphisms $\phi_A : A \rightarrow A'$ and $\phi_B : B \rightarrow B'$ such that the square (1) commutes, then there is a morphism $\phi_C : C \rightarrow C'$ (not necessarily unique) such that $(\phi_A ,\phi_B ,\phi_C)$ is a morphism of triangles (2).
                        
                        \begin{center}
                            (1)
                            \begin{tikzcd}
                                A \ar{r}{a} \ar{d}{\phi_A} & B \ar{d}{\phi_B} & \\
                                A' \ar{r}{a'} & B'
                            \end{tikzcd}
                            (2)
                            \begin{tikzcd}
                                A \ar{r}{a} \ar{d}{\phi_A} & B \ar{r}{b} \ar{d}{\phi_B} & C \ar{r}{c} \ar[dashed]{d}{\phi_C} & TA \ar{d}{T\phi_A} \\
                                A' \ar{r}{a'} & B' \ar{r}{b'} & C' \ar{r}{c'} & TA'
                            \end{tikzcd}
                        \end{center}
                    \item (TR4) Octahedron axiom
                    
                        Given the triangles $(A,B,C',a,x,x')$ (1), $(B,C,A',b,y,y')$ (2) \\ and $(A,C,B',b\circ a,z,z')$ (3)
                        \begin{center}
                            (1)
                            \begin{tikzcd}[column sep=small]
                                A \ar{r}{a} & B \ar{r}{x} & C' \ar{r}{x'} & TA
                            \end{tikzcd}

                            (2)
                            \begin{tikzcd}[column sep=small]
                                B \ar{r}{b} & C \ar{r}{y} & A' \ar{r}{y'} & TB
                            \end{tikzcd}
        
                            (3)
                            \begin{tikzcd}[column sep=small]
                                A \ar{r}{b\circ a} & C \ar{r}{z} & B' \ar{r}{z'} & TA
                            \end{tikzcd}                         
                        \end{center}
                        then there exist morphisms $f : C' \rightarrow B'$ and $g : B' \rightarrow A'$, the following diagram commutes and the third row is a triangle:

                        \begin{center}
                            \begin{tikzcd}
                                T^{-1}B' \ar{r}{T^{-1}z'} \ar{d}{T^{-1}g} & A \ar[equal]{r}{id_A} \ar{d}{a} & A \ar{d}{b\circ a} \\
                                T^{-1}A' \ar{r}{T^{-1}y'} & B \ar{r}{b} \ar{d}{x} & C \ar{r}{y} \ar{d}{z} & A' \ar{r}{y'} \ar[equal]{d}{id_{A'}} & TB \ar{d}{Tx'} \\
                                & C' \ar{r}{f} \ar{d}{x'} & B' \ar{r}{g} \ar{d}{z'} & A' \ar{r}{Ti \circ y'} & TC' \\
                                & TA \ar[equal]{r}{id_{TA}} & TA
                            \end{tikzcd}
                        \end{center}
                \end{enumerate}
            \end{definition}

            A triangulated category is denoted as $(\mathcal{T}, T, \Delta)$, where $\mathcal{T}$ is the additive category, $T$ is the translation and $\Delta$ is the triangulation. When $\mathcal{T}$ is called a triangulated category, it should be understanded as the triple given above.
            \begin{remark}
                The third object in a triangle is usually called cone, fiber, cofiber. These names are more historic than describing what these objects are, as the prototypes for these categories used these names for the third object. In this thesis it will be referred to as cone.
            \end{remark}
            % Rotation axiom dual
            \begin{remark}
                The rotation axiom has a dual, and it can be thought of as a rotation in the opposite direction. This dual can be proved by the other axioms, so it is here omitted as an axiom. The dual roation axiom goes as:

                Given a triangle \begin{tikzcd}[column sep=small]
                    A \ar{r}{a} & B \ar{r}{b} & C \ar{r}{c} & TA
                \end{tikzcd}, there is a triangle \begin{tikzcd}[column sep=small]
                    T^{-1}C \ar{r}{-T^{-1}c} & A \ar{r}{a} & B \ar{r}{b} & C
                \end{tikzcd} \\ %Hvordan fikser jeg linebreaks for at det blir pent??? Mener at denne skal kun brukes innad i en setning, idk man.
                To be able to prove this, some more lemmata are needed.
            \end{remark}
            % Octahedron axiom alternate
            \begin{remark}
                The final axiom is referred to as the octahedron axiom. By using the alternative description of the triangle diagram, it is possible to rewrite the diagram as an octahedron. The axiom can be restated as the following:
                
                Given the triangles $(A,B,C',a,x,x')$ (1), $(B,C,A',b,y,y')$ (2) \\ and $(A,C,B',b\circ a,z,z')$ (3)
                \begin{center}
                    (1)
                    \begin{tikzcd}[row sep=tiny]
                        A \arrow[red]{rd}[black]{a} & \\
                        & B \arrow[red]{dl}[black]{x} & & \\
                        C' \arrow[red, very near end, "|" marking]{uu}[near start, black]{x'}[near end]{T}
                    \end{tikzcd}
                    (2)
                    \begin{tikzcd}[row sep=tiny]
                        B \arrow[orange]{rd}[black]{b} & \\
                        & C \arrow[orange]{dl}[black]{y} & & \\
                        A' \arrow[orange, very near end, "|" marking]{uu}[near start, black]{y'}[near end]{T}
                    \end{tikzcd}
                    (3)
                    \begin{tikzcd}[row sep=tiny]
                        A \arrow[violet]{rd}[black]{b\circ a} & \\
                        & C \arrow[violet]{dl}[black]{z} & & \\
                        B' \arrow[violet, very near end, "|" marking]{uu}[near start, black]{z'}[near end]{T}
                    \end{tikzcd}
                \end{center}
                then there exists morphisms $f: C' \rightarrow B'$ and $g: B' \rightarrow A'$, the following diagram commutes and the squiggly teal back face is a triangle.
                \begin{center}
                    \begin{tikzcd}[row sep=tiny]
                        \color{white}.\color{black} & & B' \ar[teal, dashed, squiggly]{ddddr}[black, description]{g} \ar[violet]{dddddl}[black, description]{z'}[pos=0.9, marking]{|}[pos=0.91]{T} & & \\
                        \textcolor{white}{.} \\
                        \textcolor{white}{.} \\
                        \textcolor{white}{.} \\
                        C' \ar[teal, squiggly]{uuuurr}[black, description]{f} \ar[red]{dr}[black, description]{x'}[very near end, marking]{|}[near end]{T} & & & A' \ar[teal, dashed, squiggly]{lll}[pos=0.45, black, description]{Tx\circ y'}[pos=0.91, marking]{|}[very near end, above]{T} \ar[orange, dashed]{dddddl}[black, description]{y'}[pos=0.9, marking]{|}[pos=0.89, above]{T} \\
                        \color{white}.\color{black} & A \ar[red]{ddddr}[black, description]{a} \ar[violet]{rrr}[black, description]{b\circ a} & & & C \ar[orange, dashed]{ul}[black, description]{y} \ar[violet]{uuuuull}[black, description]{z}\\
                        \textcolor{white}{.} \\
                        \textcolor{white}{.} \\
                        \textcolor{white}{.} \\
                        & & B \ar[orange]{uuuurr}[black, description]{b} \ar[red]{uuuuull}[black, description]{x} & &

                    \end{tikzcd}
                \end{center}
            \end{remark}

            \begin{prop}
                The axiom TR3 can be proven from TR1 and TR4.
            \end{prop}

            \begin{proof}
                Suppose that there are two triangles and a commutative square as follows.
                \begin{center}
                    \begin{tikzcd}
                        A \ar{r}{a} \ar{d}{\phi_A} \ar{rd}{\eta} & B \ar{d}{\phi_B} & \\
                        A' \ar{r}{a'} & B'
                    \end{tikzcd}
                    \begin{tikzcd}
                        A \ar{r}{a} \ar{d}{\phi_A} & B \ar{r}{b} \ar{d}{\phi_B} & C \ar{r}{c} & TA \ar{d}{T\phi_A} \\
                        A' \ar{r}{a'} & B' \ar{r}{b'} & C' \ar{r}{c'} & TA'
                    \end{tikzcd}
                \end{center}
                Now we are able to use the Octahedron axiom twice after the square has been completed to triangles in the following sense. The upper commutative simplex and the lower commutative simplex form the following diagrams, and thus the following octahedra:
                \begin{center}
                    (1)
                    \begin{tikzcd}[row sep=tiny]
                        A \ar[red]{rd}[black]{a} \\
                        & B \ar[red]{ld}[black]{b} \\
                        C \ar[red, very near end, "|" marking]{uu}[black, near start]{c}[near end]{T}
                    \end{tikzcd}
                    \begin{tikzcd}[row sep=tiny]
                        B \ar[orange]{rd}[black]{\phi_B} \\
                        & B' \ar[orange]{ld}[black]{\phi_B'} \\
                        B'' \ar[orange, very near end, "|" marking]{uu}[black, near start]{\phi_B''}[near end]{T}
                    \end{tikzcd}
                    \begin{tikzcd}[row sep=tiny]
                        A \ar[violet]{rd}[black]{\eta} \\
                        & B' \ar[violet]{ld}[black]{\eta'} \\
                        E \ar[violet, very near end, "|" marking]{uu}[black, near start]{{\eta}''}[near end]{T}
                    \end{tikzcd}\\
                    (2)
                    \begin{tikzcd}[row sep=tiny]
                        A \ar[red]{rd}[black]{\phi_A} \\
                        & A' \ar[red]{ld}[black]{\phi_A'} \\
                        A'' \ar[red, very near end, "|" marking]{uu}[black, near start]{\phi_A''}[near end]{T}
                    \end{tikzcd}
                    \begin{tikzcd}[row sep=tiny]
                        A' \ar[orange]{rd}[black]{a'} \\
                        & B' \ar[orange]{ld}[black]{b'} \\
                        C' \ar[orange, very near end, "|" marking]{uu}[black, near start]{c'}[near end]{T}
                    \end{tikzcd}
                    \begin{tikzcd}[row sep=tiny]
                        A \ar[violet]{rd}[black]{\eta} \\
                        & B' \ar[violet]{ld}[black]{\eta'} \\
                        E \ar[violet, very near end, "|" marking]{uu}[black, near start]{{\eta}''}[near end]{T}
                    \end{tikzcd}
                \end{center}
                    \begin{minipage}[t]{0.47\textwidth}
                        \begin{center}
                            (1) \\
                            \begin{tikzcd}[row sep=tiny]
                                \textcolor{white}{.} & & E \ar[teal, dashed, squiggly]{ddddr}[black, description]{g} \ar[violet]{dddddl}[black, description]{{\eta}''}[pos=0.9, marking]{|}[pos=0.91]{T} & & \\
                                \textcolor{white}{.} \\
                                \textcolor{white}{.} \\
                                \textcolor{white}{.} \\
                                C \ar[teal, squiggly]{uuuurr}[black, description]{f} \ar[red]{dr}[black, description]{c}[very near end, marking]{|}[near end]{T} & & & B'' \ar[teal, dashed, squiggly]{lll}[pos=0.45, black, description]{Tc\circ \phi_B''}[pos=0.91, marking]{|}[very near end, above]{T} \ar[orange, dashed]{dddddl}[black, description]{\phi_B''}[pos=0.9, marking]{|}[pos=0.89, above]{T} \\
                                \textcolor{white}{.} & A \ar[red]{ddddr}[black, description]{a} \ar[violet]{rrr}[black, description]{\eta} & & & B' \ar[orange, dashed]{ul}[black, description]{\phi_B'} \ar[violet]{uuuuull}[black, description]{{\eta}'}\\
                                \textcolor{white}{.} \\
                                \textcolor{white}{.} \\
                                \textcolor{white}{.} \\
                                & & B \ar[orange]{uuuurr}[black, description]{\phi_B} \ar[red]{uuuuull}[black, description]{b} & &
                            \end{tikzcd}
                        \end{center}
                    \end{minipage}
                    \begin{minipage}[t]{0.48\textwidth}
                        \begin{center}
                            (2) \\
                            \begin{tikzcd}[row sep=tiny]
                                \textcolor{white}{.} & & E \ar[teal, dashed, squiggly]{ddddr}[black, description]{g'} \ar[violet]{dddddl}[black, description]{{\eta}''}[pos=0.9, marking]{|}[pos=0.91]{T} & & \\
                                \textcolor{white}{.} \\
                                \textcolor{white}{.} \\
                                \textcolor{white}{.} \\
                                A'' \ar[teal, squiggly]{uuuurr}[black, description]{f'} \ar[red]{dr}[black, description]{\phi_A''}[very near end, marking]{|}[near end]{T} & & & C' \ar[teal, dashed, squiggly]{lll}[pos=0.45, black, description]{T\phi_A''\circ c'}[pos=0.91, marking]{|}[very near end, above]{T} \ar[orange, dashed]{dddddl}[black, description]{c'}[pos=0.9, marking]{|}[pos=0.89, above]{T} \\
                                \textcolor{white}{.} & A \ar[red]{ddddr}[black, description]{\phi_A} \ar[violet]{rrr}[black, description]{\eta} & & & B' \ar[orange, dashed]{ul}[black, description]{b'} \ar[violet]{uuuuull}[black, description]{{\eta}'}\\
                                \textcolor{white}{.} \\
                                \textcolor{white}{.} \\
                                \textcolor{white}{.} \\
                                & & A' \ar[orange]{uuuurr}[black, description]{a'} \ar[red]{uuuuull}[black, description]{\phi_A'} & &
                            \end{tikzcd}
                        \end{center}
                    \end{minipage}
                    By following the blue squiggly lines there is a morphism $g'f:C\rightarrow C'$, it remains to see that this fits into the triangle morphism. By chasing the different paths in the octahedra and combining diagrams we get that the following diagrams are commutative, which is exactly the requirement for the collection $(\phi_A,\phi_B,g'f)$ to be a morphism of triangles.
                    \begin{center}
                        \begin{tikzcd}
                            B \ar[red]{r}[black]{b} \ar[orange]{d}[black]{\phi_B'}& C \ar[red]{rd}[black]{c} \ar[teal]{d}[black]{f} \\
                            B' \ar[orange]{rd}[black]{b'} \ar[violet]{r}[black]{\eta '} & E \ar[violet]{r}[black]{\eta ''} \ar[teal]{d}[black]{g'} & TA \ar[red]{d}[black]{T\phi_A} \\
                            & C' \ar[orange]{r}[black]{c'} & TA'
                        \end{tikzcd}
                    \end{center}

                    \emph{Har jeg glemt å bruke TR2 et sted? Er octahedra for vanskelig å bruke ettersom det skjuler hvordan ting er rotert? Eller er det dette aksiomet som er litt sterkere enn Verdier sitt axiom?}

            \end{proof}

            \begin{lemma}
                Let $(A,B,C,a,b,c)$ be a triangle, then $b\circ a=0$
            \end{lemma}

            \begin{proof}
                By TR2 the triangle $(A,B,C,a,b,c)$ can be rotated to $(B,C,TA,b,c,Ta)$.
                \begin{center}
                    \begin{tikzcd}[row sep=tiny]
                        A \arrow{rd}{a} & \\
                        & B \arrow{dl}{b}\\
                        C \arrow[very near end, "|" marking]{uu}[near start]{c}[near end]{T}
                    \end{tikzcd} $\implies$
                    \begin{tikzcd}[row sep=tiny]
                        B \arrow{rd}{b} \\
                        & C \arrow{dl}{c} \\
                        TA \arrow{uu}[near start]{-Ta}[very near end, marking]{|}[near end]{T}
                    \end{tikzcd}
                \end{center}
                The triangle exists $(C,C,0,id_C,0,0)$ by TR1 and TR3 says there exists a morphism from TA to 0 making the diagram below commute.
                \begin{center}
                    \begin{tikzcd}
                        B \ar{r}{b} \ar{d}{b} & C \ar{r}{c} \ar{d}{id_C} & TA \ar{r}{-Ta} \ar[dashed]{d}{0} & TB \ar{d}{Tb} \\
                        C \ar{r}{id_C} & C \ar{r}{0} & 0 \ar{r}{0} & TC
                    \end{tikzcd}
                \end{center}
                Thus $0 = Tb\circ -Ta = T(-ba) \implies b\circ a = 0$ as T is a translation.
            \end{proof}
            % Triangulatd functor
            \begin{definition}
                An additive functor between triangulated categories $F: (\mathcal{T}, T, \Delta) \rightarrow (\mathcal{R}, R, \Gamma)$ is called exact or triangulated if there exist a natural isomorphisms $\alpha : FT \rightarrow RF$ such that $F(\Delta) \subseteq \Gamma$.

                A functor $F : \mathcal{T} \rightarrow \mathcal{R}$ is called a triangle-equivalence if it is triangulated and an equivalence of categories. In this case $\mathcal{T}$ and $\mathcal{R}$ are called triangle-equivalent.
            \end{definition}
            % Homological functor
            \begin{definition}
                Let $\mathcal{T}$ be a triangulated category and $\mathcal{A}$ be an abelian category. A covariant functor $H:\mathcal{T} \rightarrow \mathcal{A}$ is called a homological functor if $\forall (A,B,C,a,b,c):\Delta$ there is a long exact sequence in $\mathcal{A}$.
                \begin{center}
                    \begin{tikzcd}[row sep=tiny]
                        A \arrow{rd}{a} & \\
                        & B \arrow{dl}{b}\\
                        C \arrow[very near end, "|" marking]{uu}[near start]{c}[near end]{T}
                    \end{tikzcd} $\implies$
                    \begin{tikzcd}[column sep=small]
                        ... \ar{r} & H(T^{i}A) \arrow{r}{H(T^ia)} & H(T^iB)\arrow{r}{H(T^ib)} \arrow[d,phantom, ""{coordinate, name=Z}]& H(T^iC) \arrow[dll, "H(T^ic)" description, rounded corners,to path={ --([xshift=2ex]\tikztostart.east)|- (Z)[near end]\tikztonodes-| ([xshift=-2ex]\tikztotarget.west)-- (\tikztotarget)}] \\
                        & H(T^{i+1}A) \arrow{r}{H(T^{i+1}a)} & H(T^{i+1}B) \arrow{r}{H(T^{i+1}b)} & H(T^{i+1}C) \ar{r} & ...
                    \end{tikzcd}
                \end{center}

                Dually, a contravariant functor $H:\mathcal{T} \rightarrow \mathcal{A}$ is called cohomological if $\forall (A,B,C,a,b,c):\Delta$ there is a long exact sequence in $\mathcal{A}$.
                \begin{center}
                    \begin{tikzcd}[row sep=tiny]
                        A \arrow{rd}{a} & \\
                        & B \arrow{dl}{b}\\
                        C \arrow[very near end, "|" marking]{uu}[near start]{c}[near end]{T}
                    \end{tikzcd} $\implies$
                    \begin{tikzcd}[column sep=small]
                        ... & H(T^{i-1}A) \arrow{l} & H(T^{i-1}B) \arrow{l}{H(T^{i-1}a)} \arrow[d,phantom, ""{coordinate, name=Z}]& H(T^{i-1}C) \ar{l}{H(T^{i-1}b)} \\
                        & H(T^{i}A) \arrow[urr, "H(T^ic)" description, rounded corners,to path={ --([xshift=-2ex]\tikztostart.west)|- (Z)[near end]\tikztonodes-| ([xshift=2ex]\tikztotarget.east)-- (\tikztotarget)}] & H(T^{i}B) \ar{l}{H(T^ia)} & H(T^{i}C) \ar{l}{H(T^ib)} & ... \ar{l}
                    \end{tikzcd}
                \end{center}
            \end{definition}
            % Long exact sequence of representations
            \begin{lemma}
                Let $M:\mathcal{T}$ be any object of $\mathcal{T}$, then the represented functors $\mathcal{T}(M,\_)$ is homological and $\mathcal{T}(\_,M)$ is cohomological.
            \end{lemma}

            \begin{proof}
                Only the covariant case needs to be proved, as the contravariant case is dual. For $\mathcal{T}(M,\_)$ to be homological, it has to create long exact sequences for every triangle in $\Delta$. Let $(A,B,C,a,b,c):\Delta$ be a triangle, then there can be extracted sequences in Ab for any $i:\mathbb{N}$.

                \begin{center}
                    \begin{tikzcd}[row sep=tiny]
                        A \ar{dr}{a} \\
                        & B \ar{dl}{b} \\
                        C \ar{uu}{c}[near end]{T}[very near end, marking]{|}
                    \end{tikzcd} $\implies$
                    \begin{tikzcd}
                        \mathcal{T}(M,T^iA) \ar{r}{T^ia_*} & \mathcal{T}(M,T^iB) \ar{r}{T^ib_*} & \mathcal{T}(M,T^iC)
                    \end{tikzcd}
                \end{center}
                Observe that it is enough to prove that these types of diagrams are exact, as the other diagrams can be obtained by the rotation axiom, thus reducing it to same case. 
                
                The goal is then to prove that $Im(T^ia_*)=Ker(T^ib_*)$. Since $ba=0$ it follows that $Im(T^ia_*) \subseteq Ker(T^ib_*)$. Assume that $f:Ker(T^ib_*)$, that is $f:M\rightarrow T^iB$ such that $b_*(f)=0$. The current goal is to show that $f$ factors through $T^iA$, as this means that $Ker(T^ib_*)\subseteq Im(T^ia_*)$. Note that since $T$ is a translation, it is necessarily a right adjoint to the inverse translation; thus $\mathcal{T}(M,T^iB) \simeq\mathcal{T}(T^{-i}M,B)$, and by this assertion it suffices to assume that $f:T^{-i}M\rightarrow B$ such that $b\circ f = 0$. By TR1 and TR2 there exists triangles $(T^{-i}M,0,T^{-i+1}M,0,0,-T^{-i+1}id)$ and $(B,C,TA,b,c,-Ta)$. 
                \begin{center}
                    \begin{tikzcd}
                        T^{-i}M \ar{r}{0} \ar{d}{f} & 0 \ar{r}{0} \ar{d}{0} & T^{-i+1}M \ar{r}{-T^{-i+1}id} \ar[dashed]{d}{g} & T^{-i+1}M \ar{d}{Tf} \\
                        B \ar{r}{b} & C \ar{r}{c} & TA \ar{r}{-Ta} & TB
                    \end{tikzcd}
                \end{center}
                The left square commutes by the assumption, thus the morphism g exist by TR3, such that $-Ta\circ h = -Tf\circ T^{-i+1}id = -Tf \implies Ta\circ h = Tf$, thus $f = a\circ T^{-1}h$ asserting that $f$ factors through A.
            \end{proof}
            % 2 out of 3 lemma
            \begin{lemma}
                Let $(\phi_A, \phi_B, \phi_C):(A,B,C,a,b,c) \rightarrow (A',B',C',a',b',c')$ be a morphism of triangles. If 2 of the maps are isomorphisms, then the last one is an isomorphism as well.
                \begin{center}
                    \begin{tikzcd}
                        A \ar{r}{a} \ar{d}{\phi_A}[rotate=90, above]{\simeq} & B \ar{r}{b} \ar{d}{\phi_B}[rotate=90, above]{\simeq} & C \ar{r}{c} \ar[dashed]{d}{\phi_C}[rotate=90, above]{\simeq} & TA \ar{d}{T\phi_A}[rotate=90, above]{\simeq} \\
                        A' \ar{r}{a'} & B' \ar{r}{b'} & C' \ar{r}{c'} & TA'
                    \end{tikzcd}
                \end{center}
            \end{lemma}

            \begin{proof}
                Without loss of generality, assume that $\phi_A$ and $\phi_B$ are the isomorphisms. This can be done as the rotation axiom reduce the other cases to this case. Then we have the following diagram:
                \begin{center}
                    \begin{tikzcd}
                        A \ar{r}{a} \ar{d}{\phi_A}[rotate=90, above]{\simeq} & B \ar{r}{b} \ar{d}{\phi_B}[rotate=90, above]{\simeq} & C \ar{r}{c} \ar{d}{\phi_C} & TA \ar{d}{T\phi_A}[rotate=90, above]{\simeq} \\
                        A' \ar{r}{a'} & B' \ar{r}{b'} & C' \ar{r}{c'} & TA'
                    \end{tikzcd}
                \end{center}
                By applying the functor $\mathcal{T}(C',\_)$ we get the following diagram in Ab:
                \begin{center}
                    \begin{tikzcd}
                        \mathcal{T}(C',A) \ar{r}{a_*} \ar{d}{(\phi_A)_*}[rotate=90, above]{\simeq} & \mathcal{T}(C',B) \ar{r}{b_*} \ar{d}{(phi_B)_*}[rotate=90, above]{\simeq} & \mathcal{T}(C',C) \ar{r}{c_*} \ar{d}{(\phi_C)_*} & \mathcal{T}(C',TA) \ar{r}{Ta_*} \ar{d}{(\phi_TA)_*}[rotate=90, above]{\simeq} & \mathcal{T}(C',TB) \ar{d}{(T\phi_B)_*}[rotate=90, above]{\simeq} \\
                        \mathcal{T}(C',A') \ar{r}{a'_*} & \mathcal{T}(C',B') \ar{r}{b'_*} & \mathcal{T}(C',C') \ar{r}{c'_*} & \mathcal{T}(C',TA') \ar{r}{Ta_*} & \mathcal{T}(C',TB)
                    \end{tikzcd}
                \end{center}
                By the 5-lemma, we get that $(\phi_C)_*$ is an isomorphisms, i.e. $(\phi_C)_*$ is both mono and epi. Thus for some unique $s$ in $\mathcal{T}(C',C)$, ${\phi_C}_*(s)=id_{C'}$. 

                By applying the functor $\mathcal{T}(\_,C)$ we get the diagram:
                \begin{center}
                    \begin{tikzcd}
                        \mathcal{T}(A,C) & \mathcal{T}(B,C) \ar{l}{a^*} & \mathcal{T}(C,C) \ar{l}{b^*} & \mathcal{T}(TA,C) \ar{l}{c^*} & \mathcal{T}(TB,C) \ar{l}{Ta^*} \\
                        \mathcal{T}(A',C) \ar{u}{(\phi_A)^*}[rotate=90, below]{\simeq} & \mathcal{T}(B,C) \ar{l}{a'^*} \ar{u}{(\phi_B)^*}[rotate=90, below]{\simeq} & \mathcal{T}(C',C) \ar{l}{b'^*} \ar{u}{(\phi_C)^*} & \mathcal{T}(TA',C) \ar{l}{c'^*} \ar{u}{(\phi_TA)^*}[rotate=90, below]{\simeq} & \mathcal{T}(TB',C) \ar{l}{Ta'^*} \ar{u}{(\phi_TB)^*}[rotate=90, below]{\simeq}
                    \end{tikzcd}
                \end{center}
                By the 5-lemma, we get that $(\phi_C)^*$ is an isomorphisms. By the same argument $id_{C} = s'\circ\phi_C$ for some unique $s'$. $\phi_C$ is both split mono and split epi, which means it is an isomorphism.
            \end{proof}

            \begin{corollary}
                $(A,B,0,a,0,0)$ is a triangle if and only if a is an isomorphism.
            \end{corollary}

            \begin{proof}
                Assume that a is an isomorphism. Then it is seen that $(a,id_B,0)$ is an isomorphism of triangles.
                \begin{center}
                    \begin{tikzcd}
                        A \ar{r}{a} \ar{d}{a}[rotate=90, above]{\simeq} & B \ar{r}{0} \ar{d}{id_B}[rotate=90, above]{\simeq} & 0 \ar{r}{0} \ar{d}{0}[rotate=90, above]{\simeq} & TA \ar{d}{Ta}[rotate=90, above]{\simeq} \\
                        B \ar{r}{id_B} & B \ar{r}{0} & 0 \ar{r}{0} & TB
                    \end{tikzcd}
                \end{center}
                Converesly, assume that $(A,B,0,a,0,0)$ is a triangle. Then the same diagram as above can be constructed, and by the 2 out of 3 property, a has to be an isomorphism.
            \end{proof}

            \begin{lemma}
                For a triangle $(A,B,C,a,b,c)$ the following are equivalent:

                \begin{center}
                    \begin{minipage}[c]{0.3\textwidth}
                        \begin{tikzcd}[row sep=tiny]
                            A \arrow{rd}{a} & \\
                            & B \arrow{dl}{b} & & \\
                            C \arrow[very near end, "|" marking]{uu}[near start]{c}[near end]{T}
                        \end{tikzcd}
                    \end{minipage}
                    \begin{minipage}[c]{0.3\textwidth}
                        \begin{itemize}
                            \item $a$ is split mono
                            \item $b$ is split epi
                            \item $c = 0$
                        \end{itemize}
                    \end{minipage}
                \end{center}
            \end{lemma}

            \begin{proof}
                The proof has two parts. First assume that $a$ is split mono, and prove that $b$ is split epi, and $c = 0$. By duality, it is then known that $b$ being split epi implies that $a$ is split mono and $c = 0$. The final part is to assume that $c = 0$, and prove either $a$ is split mono or $b$ is split epi.

                Assume that $a$ is split mono, then there exist an $a^{-1}$ such that $id_A = a^{-1}a$. Let $M:\mathcal{T}$ be any object, then we can make a long exact sequence:
                \begin{center}
                    \begin{tikzcd}
                        \mathcal{T}(M,T^{-1}C) \ar{r}{T^{-1}c_*} & \mathcal{T}(M,A) \ar[bend left]{r}{a_*} & \mathcal{T}(M,B) \ar{r}{b_*} \ar[dashed, bend left]{l}{a^{-1}_*} & \mathcal{T}(M,C) \ar{r}{c_*} & \mathcal{T}(M,TA)
                    \end{tikzcd}
                \end{center}
                By assumption $a_*$ is split mono, thus $T^{-1}c_* = 0$ and in particular $c = 0$. This implies that $b_*$ is epi, making a split short exact sequence.
                \begin{center}
                    \begin{tikzcd}
                        0 \ar{r}{0} & \mathcal{T}(M,A) \ar[bend left]{r}{a_*} & \mathcal{T}(M,B) \ar[bend left]{r}{b_*} \ar[dashed, bend left]{l}{a^{-1}_*} & \mathcal{T}(M,C) \ar{r}{0} \ar[dashed, bend left]{l}{b^{-1}_*} & 0
                    \end{tikzcd}
                \end{center}
                This gives that b is split epi, completing the first part of the proof.

                For the next part, assume that $c = 0$; then we can construct the following triangles.
                \begin{center}
                    (1)
                    \begin{tikzcd}[row sep=tiny]
                        A \arrow{rd}{a} \\
                        & B \arrow{dl}{b} \\
                        C \arrow[very near end, "|" marking]{uu}[near start]{0}[near end]{T}
                    \end{tikzcd} $\implies$
                    \begin{tikzcd}[row sep=tiny]
                        C \arrow{rd}{0} \\
                        & TA \arrow{dl}{-Ta} \\
                        TB \arrow[very near end, "|" marking]{uu}[near start]{-Tb}[near end]{T}
                    \end{tikzcd}
                \end{center}
                \begin{center}
                    (2)
                    \begin{tikzcd}[row sep=tiny]
                        A \arrow{rd}{id_A} \\
                        & A \arrow{dl}{0} \\
                        0 \arrow[very near end, "|" marking]{uu}[near start]{0}[near end]{T}
                    \end{tikzcd} $\implies$
                    \begin{tikzcd}[row sep=tiny]
                        0 \arrow{rd}{0} & \\
                        & TA \arrow{dl}{-id_{TA}}\\
                        TA \arrow[very near end, "|" marking]{uu}[near start]{0}[near end]{T}
                    \end{tikzcd}
                \end{center}
                (1) is constructed by applying TR2 twice, while (2) is constructed with TR1 and TR2 twice. Observe that there is a commutative square between the triangles, allowing for TR3 to make a morphism of triangles.
                \begin{center}
                    \begin{tikzcd}
                        C \ar{r}{0} \ar{d}{0} & TA \ar{r}{-Ta} \ar[equal]{d}{id_{TA}} & TB \ar{r}{-Tb} \ar[dashed]{d}{Ta^{-1}} & TC \ar{d}{0} \\
                        0 \ar{r}{0} & TA \ar[equal]{r}{-id_{TA}} & TA \ar{r}{0} & 0
                    \end{tikzcd}
                \end{center}
                Thus $T(a^{-1}a)=id_{TA}=T(id_A) \implies id_A = a^{-1}a$, making a split mono.
            \end{proof}

            \begin{lemma}
                Given two triangles $(A,B,C,a,b,c)$ and $(A',B',C',a',b',c')$ the following are equivalent:
                \begin{center}
                    \begin{minipage}[c]{0.4\textwidth}
                        \begin{tikzcd}
                            A \ar{r}{a} \ar{d}{f} & B \ar{r}{b} \ar{d}{g} & C \ar{r}{c} \ar{d}{h} & TA \ar{d}{Tf} \\
                            A' \ar{r}{a'} & B' \ar{r}{b'} & C' \ar{r}{c'} & TA'
                        \end{tikzcd}
                    \end{minipage}
                    \begin{minipage}[c]{0.5\textwidth}
                        \begin{enumerate}
                            \item $(f,g,h)$ is a morphism of triangles
                            \item $\exists g:B\rightarrow B'$ such that $b'ga = 0$
                        \end{enumerate}
                    \end{minipage}
                \end{center}
                Moreover, if $\mathcal{T}(A,T^{-1}C')\simeq 0$, then f and h are unique.
            \end{lemma}

            \begin{proof}
                $1. \implies 2.$ as the composition $ba = 0 = b'a'$, so assume 2. The existence of $f$ and $h$ is evident from the long exact sequence of the bottom triangle at the functor represented by $A$. 
                \begin{center}
                    \begin{tikzcd}
                        \mathcal{T}(A,T^{-1}C') \ar{r}{T^{-1}c'_*} & \mathcal{T}(A,A') \ar{r}{a'_*} & \mathcal{T}(A,B') \ar{r}{b'_*} & \mathcal{T}(A,C')
                    \end{tikzcd}
                \end{center}
                The morphism $ga:\mathcal{T}(A,B')$ such that $b'ga=b'_*(ga)=0$, thus $ga:Ker(b'_*)$. By exactness $\exists f:\mathcal{T}(A,A')$ such that $a'f = ga$, and by TR3 $\exists h: C \rightarrow C'$ such that $(f,g,h)$ is a morphism of triangles.
                Now assume that $\mathcal{T}(A,T^{-1}C')\simeq 0$. Exactness determines that $a'_*$ is a monomorphism, and $f$ is then unique. Since $T$ is a translation, we have that $\mathcal{T}(A,T^{-1}C')\simeq\mathcal{T}(TA,C')$. By using the functor $\mathcal{T}(\_,C')$ at the top triangle, we get that $b^*$ is a monomorphism, and thus $h$ is chosen uniquely.
            \end{proof}

            \begin{lemma}
                If $(A,B,C,a,b,c)$ is a triangle, then $(T^{-1}C,A,B,T^{-1}c,a,b)$ is a triangle.
            \end{lemma}

            \begin{proof}
                By TR2 we can construct a triangle.
                \begin{center}
                    \begin{tikzcd}[row sep=tiny]
                        A \ar{rd}{a} \\
                        & B \ar{ld}{b} \\
                        C \ar{uu}[near start]{c}[very near end, marking]{|}[near end]{T}
                    \end{tikzcd}
                    $\implies$
                    \begin{tikzcd}[row sep=tiny]
                        C \ar{rd}{c} \\
                        & TA \ar{ld}{Ta} \\
                        TB \ar{uu}[near start]{Tb}[very near end, marking]{|}[near end]{T}
                    \end{tikzcd}
                \end{center}
                By TR1 we can create a triangle $(T^{-1}C,A,B',T^{-1}c,a',b')$, and then use TR3 to find a morphism.
                \begin{center}
                    \begin{tikzcd}
                        C \ar{r}{c} \ar[equal]{d}{id_C} & TA \ar{r}{Ta'} \ar[equal]{d}{id_{TA}} & TB' \ar{r}{Tb'} \ar[dashed]{d}{h} & TC \ar[equal]{d}{id_{TC}} \\
                        C \ar{r}{c} & TA \ar{r}{Ta} & TB \ar{r}{Tb} & TC
                    \end{tikzcd}
                \end{center}
                By the 2 out of 3 property it is seen that h is an isomorphism, so the triple $(id_{T^{-1}C}, id_A, T^{-1}h)$ is an isomorphism of candidate triangles, and by TR1, is an isomorphism of triangles, asserting that $(T^{-1}C,A,B,T^{-1}c,a,b)$ is in fact a triangle.
            \end{proof}

            \begin{lemma}
                Let $(A,B,C,a,b,c)$ and $(A',B',C',a',b',c')$ be two triangles, then the direct sum of these triangles is a triangle.
            \end{lemma}

            \begin{proof}
                Observe that for any functor $\mathcal{T}(K,\_)$ there is still a long exact sequence of Hom(ology)since $\mathcal{T}(K,A\oplus A')\simeq\mathcal{T}(K,A)\oplus\mathcal{T}(K,A')$. Thus for the direct sum of the triangles we have the following.
                \begin{center}
                    \begin{tikzcd}[ampersand replacement=\&]
                        A\oplus A' \ar{r}{\begin{pmatrix}a & 0 \\ 0 & a'\end{pmatrix}} \& B\oplus B' \ar{r}{\begin{pmatrix}b & 0 \\ 0 & b'\end{pmatrix}} \& C\oplus C' \ar{r}{\begin{pmatrix}c & 0 \\ 0 & c'\end{pmatrix}} \& TA\oplus TC
                    \end{tikzcd} \\
                    $\Downarrow$ \\
                    \begin{tikzcd}[column sep=small]
                        ... \ar{r} & \mathcal{T}(K,A)\oplus\mathcal{T}(K,A') \ar{r} & \mathcal{T}(K,B)\oplus\mathcal{T}(K,B') \ar[dll, rounded corners,to path={ --([xshift=2ex]\tikztostart.east)|- (Z)[near end]\tikztonodes-| ([xshift=-2ex]\tikztotarget.west)-- (\tikztotarget)}] \\
                        \mathcal{T}(K,C)\oplus\mathcal{T}(K,C') \ar{r} & \mathcal{T}(K,TA)\oplus\mathcal{T}(K,TA') \ar{r} & ...    
                    \end{tikzcd}
                \end{center}
                Thus the 2 out of 3 property holds for the direct sum. By TR1 there is a triangle 
                \begin{center}
                    \begin{tikzcd}
                        A\oplus A' \ar{r} & B\oplus B' \ar{r} & D \ar{r} & TA\oplus TA'
                    \end{tikzcd}
                \end{center}
                By TR3 there are morphisms from this triangle to to the direct summands. Adding these maps together there is a map from this triangle to direct sum, and by using the 2 out of 3 property this is an isomorphism of candidate triangles. Thus the direct sum is a triangle.
                \begin{center}
                    \begin{tikzcd}
                        A\oplus A' \ar{r} \ar{d} & B\oplus B' \ar{r} \ar{d} & D \ar{r} \ar[dashed]{d} & TA\oplus TA' \ar{d} \\
                        A \ar{r} & B \ar{r} & C \ar{r} & TA
                    \end{tikzcd} \\
                    \& \\
                    \begin{tikzcd}
                        A\oplus A' \ar{r} \ar{d} & B\oplus B' \ar{r} \ar{d} & D \ar{r} \ar[dashed]{d} & TA\oplus TA' \ar{d} \\
                        A' \ar{r} & B' \ar{r} & C' \ar{r} & TA'
                    \end{tikzcd} \\
                    $\Downarrow$ \\
                    \begin{tikzcd}
                        A\oplus A' \ar{r} \ar[equal]{d} & B\oplus B' \ar{r} \ar[equal]{d}& D \ar{r} \ar[dashed]{d}[below, rotate=90]{\simeq} & TA\oplus TA' \ar[equal]{d} \\
                        A\oplus A' \ar{r} & B\oplus B' \ar{r} & A''\oplus B'' \ar{r} & TA\oplus TA'
                    \end{tikzcd}
                \end{center}
            \end{proof}

            \begin{lemma}
                The direct summands of a triangle is a triangle.
            \end{lemma}

            \begin{proof}
                The proof can be found in \cite{neeman}
            \end{proof}
            
            \subsection{Mapping Cones, Homotopies and Contractibility}
            \begin{remark}
                The observant reader might have seen that the Octahedron axiom have not jet been used once! A lot of the theory proven for triangulated categories work without this axiom, and this motivates the definition of a pre-triangulated category. This odd axiom might seem to be pointless, but it is one of the most fundamental axioms, as TR3 can be proven with TR1, TR2 and TR4; see [need sources here].
            \end{remark}

            \begin{definition}
                A pre-triangulation of an additive category $\mathcal{T}$ with translation $T$ is a collection $\Delta '$ of triangles consisting of candidate triangles in $\mathcal{T}$ satisfying TR1, TR2 and TR3.

                The category $\mathcal{T}$ with the pre-triangulation $\Delta '$ is called a pre-triangulated category, and the candidate triangles in $\Delta '$ are called  triangles. We will only use this notion of triangles in this subsection.
            \end{definition}

            The main goal of this subsection is to see how we can find  triangles, and when these are triangles. We will also look at triangulated functors and triangulated subcategories. For the rest of this subsection it is assumed that we work in a pre-triangulated category $\mathcal{T}$.

            \begin{definition}
                Suppose there is a morphism of candidate triangles $\phi : (A,B,C,a,b,c) \rightarrow (A',B',C',a',b',c')$.
                \begin{center}
                    \begin{tikzcd}
                        A \arrow{r}{a} \arrow{d}{\phi_A} & B \arrow{r}{b} \arrow{d}{\phi_B} & C \arrow{r}{c} \arrow{d}{\phi_C} & TA \arrow{d}{T\phi_A} \\
                        A' \arrow{r}{a'} & B' \arrow{r}{b'} & C \arrow{r}{c'} & TA'
                    \end{tikzcd}
                \end{center}
                We define the mapping cone to be the candidate triangle:
                \begin{center}
                    \begin{tikzcd}[ampersand replacement=\&]
                        B \oplus A' \ar{r}{\begin{pmatrix}
                            -b & 0 \\ \phi_B & a'
                        \end{pmatrix}} \& C\oplus B' \ar{r}{\begin{pmatrix}
                            -c & 0 \\ \phi_C & b'
                    \end{pmatrix}} \& TA\oplus C' \ar{r}{\begin{pmatrix}
                        -Ta & 0 \\ T\phi_A & c'
                    \end{pmatrix}} \& TB\oplus TA'
                    \end{tikzcd}
                \end{center}
            \end{definition}

            \begin{definition}
                A morphism $\alpha : A,B,C,a,b,c) \rightarrow (A',B',C',a',b',c')$ between candidate triangles is called null-homotopic if it factors through a homotopy. That is, there exists maps $\Theta, \Phi, \Psi$ in the following diagram:
                \begin{center}
                    \begin{tikzcd}
                        A \arrow{r}{a} \arrow{d}{\alpha_A} & B \arrow{r}{b} \arrow{ld}{\Theta} \ar{d}{\alpha_B} & C \arrow{r}{c} \arrow{ld}{\Phi} \ar{d}{\alpha_C} & TA \arrow{ld}{\Psi} \ar{d}{T\alpha_A} \\
                        A' \arrow{r}{a'} & B' \arrow{r}{b'} & C \arrow{r}{c'} & TA'
                    \end{tikzcd}
                \end{center}
                such that $\alpha_A  = \Theta a + T^{-1}(c'\Psi)$, $\alpha_B = \Phi b + a'\Theta$ and $\alpha_C = \Psi c + b'\Phi$.
                Two maps are called homotopic if their difference is null-homotopic
            \end{definition}

            \begin{lemma}
                The mapping cone only depends on morphisms up to homotopy. I.e. if two maps are homotopic, their mapping cones are isomorphic.
            \end{lemma}

            \begin{proof}
                Suppose that $(f,g,h)$ and $(f',g',h')$ are two homotopic morphisms of triangles:
                \begin{center}
                    \begin{tikzcd}
                        A \ar{r}{a} \ar{d} & B \ar{r}{b} \ar{d} & C \ar{r}{c} \ar{d} & TA \ar{d} \\
                        A' \ar{r}{a'} & B' \ar{r}{b'} & C' \ar{r}{c'} & TA'
                    \end{tikzcd}
                \end{center}
                Let $(\Theta,\Phi,\Psi)$ be the homotopy between the triangle morphisms. Then there is an isomorphism of triangles.
                \begin{center}
                    \begin{tikzcd}[ampersand replacement=\&]
                        B \oplus A' \ar{r}{\begin{pmatrix}
                            -b & 0 \\ g & a'
                        \end{pmatrix}} \ar{d}{\begin{pmatrix} 1 & 0 \\ \Theta & 1 \end{pmatrix}} \& C\oplus B' \ar{r}{\begin{pmatrix}
                            -c & 0 \\ h & b'
                    \end{pmatrix}} \ar{d}{\begin{pmatrix} 1 & 0 \\ \Phi & 1\end{pmatrix}} \& TA\oplus C' \ar{r}{\begin{pmatrix}
                        -Ta & 0 \\ Tf & c'
                    \end{pmatrix}} \ar{d}{\begin{pmatrix}1 & 0 \\ \Psi & 1\end{pmatrix}} \& TB\oplus TA' \ar{d}{\begin{pmatrix}1 & 0 \\ T\Theta & 1 \end{pmatrix}}\\
                    B \oplus A' \ar{r}[below]{\begin{pmatrix}
                        -b & 0 \\ g' & a'
                    \end{pmatrix}} \& C\oplus B' \ar{r}[below]{\begin{pmatrix}
                        -c & 0 \\ h' & b'
                \end{pmatrix}} \& TA\oplus C' \ar{r}[below]{\begin{pmatrix}
                    -Ta & 0 \\ Tf' & c'
                \end{pmatrix}} \& TB\oplus TA'
                    \end{tikzcd}
                \end{center}
            \end{proof}

            \begin{lemma}
                Let $A$ denote the triangle $(A,A',A'')$ and $B$ denote $(B,B',B'')$. Suppose $\alpha, \beta : A \rightarrow B$ are two homotopic morphisms of candidate triangles. Then for any map $\gamma : \widetilde{A} \rightarrow A$ and any map $\delta : B \rightarrow \widetilde{B}$ the maps $\delta\alpha\gamma$ and $\delta\beta\gamma$ are homotopic as well.
            \end{lemma}

            \begin{proof}
                To prove this statement it is enough to prove that $\alpha\gamma$ is homotopic to $\beta\gamma$ du to the symmetry of the statement. The goal is then to show that $(\Theta\gamma ',\Phi\gamma '',\Psi T\gamma)$ is the homotopy. This can be seen as
                \begin{multline*}
                    {\alpha}'{\gamma}'-{\beta}'{\gamma}' = ({\alpha}'-{\beta}'){\gamma}' = (b\Theta +\Phi a'){\gamma}' = b\Theta {\gamma}' + \Phi a'{\gamma}' = b({\Theta}{\gamma}') + ({\Phi}{\gamma}'')\widetilde{a}'
                \end{multline*}
            \end{proof}

            \begin{definition}
                A candidate triangle $A$ is called a contractible triangle if $id_A$ is null-homotopic.
            \end{definition}

            \begin{remark}
                If $A$ is a contractible triangle and $F:\mathcal{T}\rightarrow \mathcal{A}$ is an additive functor to an abelian category, then the identity of the cochain 
                \begin{center}
                    \begin{tikzcd}
                        ... \ar{r} & F(A) \ar{r} & F(A') \ar{r} & F(A'') \ar{r} & F(TA) \ar{r} & ...
                    \end{tikzcd}
                \end{center}
                is null-homotopic as well. Thus the homology of this sequence is $0$, asserting that it is an exact sequence.
                The exactness of such sequences allow us to use the 2 out of 3 property on morphisms between contractible triangles.
            \end{remark}

            \begin{corollary}
                If A is a contractible triangle, then any map in $\mathcal{T}(A,\_)$ or $\mathcal{T}(\_,A)$ is null-homotopic.
            \end{corollary}

            \begin{proof}
                Being null-homotopic is the same as that there is a homotopy between the map and the zero map. If $id_A\sim 0 \implies f\circ id_A = f \sim f\circ 0 = 0$. So any map $f$ is null-homotopic.
            \end{proof}

            \begin{lemma}
                A contractible triangle is a triangle.
            \end{lemma}

            \begin{proof}
                Let $A$ be the triangle $(A,A',A'')$ such that $id_A \sim 0$. This means that there is a homotopy
                \begin{center}
                    \begin{tikzcd}
                        A \ar{r}{a} \ar{d}{id_A} & A' \ar{r}{a'} \ar{d}{id_{A'}} \ar{ld}{\Theta} & A'' \ar{r}{a''} \ar{d}{id_{A''}} \ar{ld}{\Phi} & TA \ar{d}{id_{TA}} \ar{ld}{\Psi} \\
                        A \ar{r}{a} & A' \ar{r}{a'} & A'' \ar{r}{a''} & TA
                    \end{tikzcd}
                \end{center}
                By using TR1 there is a triangle, and a long exact sequence.
                \begin{center}
                    \begin{tikzcd}
                        A \ar{r}{a} & A' \ar{r}{e} & E \ar{r}{e'} & TA
                    \end{tikzcd} \\
                    $\Downarrow$ \\
                    \begin{tikzcd}
                        ... \ar{r} & \mathcal{T}(TA,A) \ar{r} & \mathcal{T}(TA,A') \ar{r} & \mathcal{T}(TA,E) \ar{r}{e'_*} & \mathcal{T}(TA,TA) \ar{r}{Ta_*} & ...    
                    \end{tikzcd}
                \end{center}
                Since the map $Ta\circ a''\Psi = 0$, the kernel $KerTa_*=Ime'_*\neq 0$. This means that there is a map ${\Psi}':\mathcal{T}(TA,E)$ such that $e'{\Psi}'=a''\Psi$. Then the map $(id_A,id_{A'},e\Theta+{\Psi}'a'')$ is a well defined map of candidate triangles. By the remark, we can use the 2 out of 3 properties to assert that the map found is an isomorphism, giving an isomorphism of triangles, asserting that the contractible triangle is a triangle. 
            \end{proof}

            \begin{corollary}
                The mapping cone of the zero map between  triangles is a triangle. 
            \end{corollary}

            \begin{proof}
                The mapping cone of the zero map can be seen to be the direct sum of two triangles. Thus it is a triangle.
            \end{proof}

            \begin{corollary}
                The mapping cone of a null-homotopic map between  triangles is a triangle.
            \end{corollary}

            \begin{remark}
                Suppose we have a morphism of triangles where one of the triangles are contractible, then the mapping cone is a triangle as well.
            \end{remark}

            \begin{definition}
                A morphism of triangles will be called good if the mapping cone of the morphism is a triangle.
            \end{definition}

            \begin{theorem}
                A pre-triangulated category $\mathcal{T}$ is triangulated if given two triangles $(A,B,C,a,b,c)$ and $(A',B',C',a',b',c')$ and diagram (1) commutes, then diagram (1) can be completed to diagram (2) such that $\phi$ is good. That is the mapping cone of $\phi$ is a triangle.
                \begin{center}
                    (1)
                    \begin{tikzcd}
                        A \ar{r}{a} \ar{d}{\phi_A} & B \ar{d}{\phi_B} & \\
                        A' \ar{r}{a'} & B'
                    \end{tikzcd}
                    (2)
                    \begin{tikzcd}
                        A \ar{r}{a} \ar{d}{\phi_A} & B \ar{r}{b} \ar{d}{\phi_B} & C \ar{r}{c} \ar[dashed]{d}{\phi_C} & TA \ar{d}{T\phi_A} \\
                        A' \ar{r}{a'} & B' \ar{r}{b'} & C' \ar{r}{c'} & TA'
                    \end{tikzcd}
                \end{center}
            \end{theorem}

            \begin{remark}
                This condition is equivalent to the Octahedron axiom.
            \end{remark}

            \begin{definition}
                A commutative square (1) is called homotopy cartesian if it arises from a triangle. That is, (2) is a triangle.
                \begin{center}
                    (1)
                    \begin{tikzcd}
                        D \ar{r} \ar{d} \ar[phantom]{rd}{HO}[very near start]{\ulcorner}[very near end]{\lrcorner}& A \ar{d} \\
                        B \ar{r} & C
                    \end{tikzcd}
                    $\implies$
                    (2)
                    \begin{tikzcd}[row sep=small]
                        D \ar{rd} \\
                        & A\oplus B \ar{ld} \\ 
                        C \ar[very near end, "|" marking]{uu}[near end]{T} 
                    \end{tikzcd}
                \end{center}
            \end{definition}
                
            \begin{remark}
                A way to construct homotopy cartesian squares is with homotopy pullbacks. That is to us TR1 on the following map to get a triangle.
                \begin{center}
                    \begin{tikzcd}
                        & A \ar{d}{a} \\
                        B \ar{r}{b} & C
                    \end{tikzcd}
                    $\implies$
                    \begin{tikzcd}[ampersand replacement=\&]
                        A\oplus B \ar{r}{\begin{pmatrix}a & b\end{pmatrix}} \& C
                    \end{tikzcd} \\
                    $\implies$
                    \begin{tikzcd}[ampersand replacement=\&]
                        A\oplus B \ar{r}{\begin{pmatrix}a & b\end{pmatrix}} \& C \ar{r} \& TD \ar{r} \& TA\oplus TB
                    \end{tikzcd}
                    $\implies$
                    \begin{tikzcd}
                        D \ar{r} \ar{d} \ar[phantom]{rd}{HO}[very near start]{\ulcorner}[very near end]{\lrcorner}& A \ar{d} \\
                        B \ar{r} & C
                    \end{tikzcd}
                \end{center}

                Dually, one can use homotopy push-outs to get homotopy cartesian squares.
            \end{remark}

            \begin{remark}
                A remark about good maps and homotopy cartesian squares???
            \end{remark}

            \begin{lemma}
                Suppose that there is a homotopy cartesian square (1). Then there are triangles and a triangle isomorphism as in (2).
                \begin{center}
                    (1)
                    \begin{tikzcd}
                        D \ar{r}{g'} \ar{d}{f'} \ar[phantom]{rd}{HO}[very near start]{\ulcorner}[very near end]{\lrcorner}& A \ar{d}{f} \\
                        B \ar{r}{g} & C
                    \end{tikzcd}
                    (2)
                    \begin{tikzcd}
                        D \ar{r} \ar{d}{f'} & A \ar{d}{f} \ar{r} & E \ar{r} \ar[equal]{d} & TD \ar{d}{Tf'} \\
                        B \ar{r} & C \ar{r} & E \ar{r} & TB
                    \end{tikzcd}
                \end{center}
            \end{lemma}

            \begin{proof}
                There is a commutative square which satisfies the requirements of the octahedron axiom.
                \begin{center}
                    \begin{tikzcd}[ampersand replacement=\&]
                        D \ar{r}{\begin{pmatrix}g' \\ f'\end{pmatrix}} \ar[equal]{d} \& A\oplus B \ar{d}{\begin{pmatrix}1 & 0\end{pmatrix}} \\
                        D \ar{r}{g'} \& A
                    \end{tikzcd}
                    \begin{tikzcd}[row sep=small, ampersand replacement=\&]
                        D \ar[red]{rd}{\begin{pmatrix}g' \\ f'\end{pmatrix}} \\
                        \& A\oplus B \ar[red]{ld}{\begin{pmatrix}f & g\end{pmatrix}} \\
                        C \ar[red, very near end, "|" marking]{uu}[near end]{T}[pos=0.5]{0}
                    \end{tikzcd}
                    \begin{tikzcd}[row sep=small, ampersand replacement=\&]
                        A\oplus B \ar[orange]{rd}{\begin{pmatrix} 0 & 1\end{pmatrix}} \\
                        \& A \ar[orange]{ld}{0} \\
                        TB \ar[orange, very near end, "|" marking]{uu}[near end]{T}[near start]{\begin{pmatrix}0 \\ 1\end{pmatrix}}
                    \end{tikzcd}
                    \begin{tikzcd}[row sep=small]
                        D \ar[violet]{rd}{f'} \\
                        & A \ar[violet]{ld}{} \\
                        E \ar[violet, very near end, "|" marking]{uu}[near end]{T}
                    \end{tikzcd}
                \end{center}
                This setup shows that there is a triangle with a commutative square:
                \begin{center}
                    \begin{tikzcd}
                        C \ar[teal]{r} & E \ar[teal]{r}{} & TB \ar[teal]{r}{} & TC
                    \end{tikzcd}
                    \begin{tikzcd}[ampersand replacement=\&]
                        A\oplus B \ar{r}{\begin{pmatrix}1 & 0\end{pmatrix}} \ar{d}{\begin{pmatrix}f & g\end{pmatrix}} \& A \ar{d}{} \\
                        C \ar{r}{} \& E
                    \end{tikzcd}
                \end{center}.
                Since $\begin{pmatrix}1 & 0\end{pmatrix}$ is a splitmono, we have that the following is a morphism of triangles.
                \begin{center}
                    \begin{tikzcd}
                        D \ar{r}{} \ar{d}{} & A \ar{r}{} \ar{d}{} & E \ar{r}{} \ar[equal]{d}{} & TD \ar{d}{} \\
                        B \ar{r}{} & C \ar{r}{} & E \ar{r}{} & TB
                    \end{tikzcd}
                \end{center}
            \end{proof}

        \subsection{Calculus of Fractions and the Verdier Quotient}
            Localization is a method for adding formal inverses to a category. It is most notably known in commutative algebra where we can invert elements with respect to some ideal of the ring. The rational numbers can be shown to be a localization of the integers at every number except 0. The category gained from localizing at some set $S$ of morphisms is the universal category where these morphisms are isomorphisms.
            \begin{definition}
                Let $S$ be a collection of morphisms in the category $\mathcal{C}$. We say that the Localization of $\mathcal{C}$ on $\mathcal{S}$ is the category $\mathcal{C}[S^{-1}]$ together with a functor $q:\mathcal{C}\rightarrow \mathcal{C}[S^{-1}]$ such that:
                \begin{itemize}
                    \item $\forall s:S|q(s)$ is an isomorphism
                    \item For any functor $F:\mathcal{C}\rightarrow\mathcal{D}$ such that $\forall s:S$ such that $F(s)$ is an isomorphism, then $F$ factors through $q$. That is to say that there is a natural isomorphism $\eta : F\rightarrow F'\circ q$ so that $\mathcal{C}[S^{-1}]$ is the universal category where morphisms in $S$ are isomorphisms.
                \end{itemize}
                \begin{center}
                    \begin{tikzcd}[row sep = tiny]
                        \mathcal{C} \ar{rr}{F} \ar{rd}[below]{q} & & \mathcal{D} \\
                        & S^{-1}\mathcal{C} \ar[dashed]{ru}[below]{F'}
                    \end{tikzcd}
                \end{center}
            \end{definition}

            \begin{remark}
                Even though if we know that $\mathcal{C}$ is locally small, then we cannot be sure that the category $\mathcal{C}[S^{-1}]$ is again locally small.
            \end{remark}

            These categories are in generel pretty hard to describe. When the set of morphisms is what we call a multiplicative system, we get the same calculus of fractions description of these localization the same style as for localizations of rings.

            \begin{definition}
                A set $S$ of morphisms in a category $\mathcal{C}$ is called right multiplicative if it satisfies the following conditions:
                \begin{itemize}
                    \item $S$ is closed under composition, i.e. if $f,g : S$ are composable then $gf : S$. Every identity morphism in $\mathcal{C}$ is in $S$.
                    \item (Right Ore condition) If $t : X \rightarrow Y$ is a morphism in $S$, then $\forall g:Z\rightarrow Y$ there is a commutative square (1) such that $f:W\rightarrow X$ and $s:W\rightarrow Z$ exists and $s:S$ as well.
                    \begin{center}
                        (1)
                        \begin{tikzcd}
                            W \ar[dashed]{r}{f} \ar[dashed]{d}{s} & X \ar{d}{t} \\
                            Z \ar{r}{g} & Y
                        \end{tikzcd}
                    \end{center}
                    \item (Left cancellation) Suppose $f,g:X\rightarrow Y$ are parallell morphisms in $\mathcal{C}$, then 1. $\implies$ 2.:
                    \begin{enumerate}
                        \item $sf = sg$ for som $s:S$ starting at $Y$
                        \item $ft = gt$ for som $t:S$ ending at $X$
                    \end{enumerate}
                \end{itemize}
            \end{definition}

            \begin{remark}
                The previous definition has a dual statement. We say that a set $S$ of morphisms is a left multiplicative system if it satisfies:
                \begin{itemize}
                    \item $S$ is closed under composition, i.e. if $f,g : S$ are composable then $gf : S$. Every identity morphism in $\mathcal{C}$ is in $S$.
                    \item (Left Ore condition) If $s : Y \rightarrow Z$ is a morphism in $S$, then $\forall f:Y\rightarrow X$ there is a commutative square (1) such that $g:Z\rightarrow W$ and $t:X\rightarrow W$ exists and $t:S$ as well.
                    \begin{center}
                        (1)\begin{tikzcd}
                            Y \ar{r}{f} \ar{d}{s} & X \ar[dashed]{d}{t} \\
                            Z \ar[dashed]{r}{g} & W
                        \end{tikzcd}
                    \end{center}
                    \item (Right cancellation) Suppose $f,g:X\rightarrow Y$ are parallell morphisms in $\mathcal{C}$, then 1. $\implies$ 2.:
                    \begin{enumerate}
                        \item $ft = gt$ for som $t:S$ ending at $X$
                        \item $sf = sg$ for som $s:S$ starting at $Y$
                    \end{enumerate}
                \end{itemize}
                If $S$ is both right multiplicative and left multiplicative then we just say that it is multiplicative.
            \end{remark}

            \begin{prototype}
                Let $R$ be a commutative integral domain, ... (look at Bacharaya and how they define the field of fractions, or ask Andreas if he have any good literature on this topic)
            \end{prototype}

            As with the definition of localization of rings, localization of a category $\mathcal{C}$ at a multiplicative system will be defined with fractions. That is the morphisms will be "fractions" of morphisms. These morphisms will be described as diagrams over spans for right multiplicative systems (or dually cospans for left multiplicative systems), together with an equivalence relation.

            \begin{definition}
                A span is a diagram of the form:
                \begin{center}
                    \begin{tikzcd}
                        \cdot & \cdot \ar{l} \ar{r} & \cdot
                    \end{tikzcd}
                \end{center}
            \end{definition}

            \begin{definition}
                Let $S$ be a right multiplicative system of morphisms in a category $\mathcal{C}$. Given a morphism $s : Y\rightarrow X$ in $S$ and a morphism $t:Y\rightarrow Z$ we define the right fraction of $s$ and $t$ to be the span of the morphisms. That is $s$ and $t$ fit in the diagram:
                \begin{center}
                    \begin{tikzcd}
                        X & Y \ar{l}{s} \ar{r}{t} & Z
                    \end{tikzcd}
                \end{center}
                We denote the right fraction as $ts^{-1}$.
                Let $\sim$ be the equivalence relation of right fractions given by the diagram (1) such that $ts^{-1}\sim t's'^{-1}$ if and only if $\exists w,w':\mathcal{C}$ making the diagram commute and that the middle row is a right fraction.
                \begin{center}
                    \begin{tikzcd}
                        & Y \ar{ld}[above]{s} \ar{rd}{t} \\
                        X & W \ar{r} \ar{l} \ar{u}{w} \ar{d}{w'} & Z \\
                        & Y' \ar{lu}{s'} \ar{ru}[below]{t'}
                    \end{tikzcd}
                \end{center}
            \end{definition}

            Dually, we define left fractions as diagrams over cospans such that if $t:S$ we get a left fraction $t^{-1}s$ as the diagram:
            \begin{center}
                \begin{tikzcd}
                    X \ar{r}{s} & Y & Z \ar{l}{t}
                \end{tikzcd}
            \end{center}

            The equivalence relation $\sim$ is given by the diagram in the same manner as above.
            \begin{center}
                \begin{tikzcd}
                    & Y \ar{d}{w} \\
                    X \ar{ru}{s} \ar{r} \ar{rd}{s'} & W & Z \ar{lu}{t} \ar{l} \ar{ld}{t'} \\
                    & Y' \ar{u}{w'}
                \end{tikzcd}
            \end{center}

            \begin{prop}
                Suppose that $S$ is a right multiplicative system, then the relation stated above is in fact an equivalence relation.
            \end{prop}

            \begin{proof}
                We will need to prove that $\sim$ is reflexive, symmetric and transitive.
                \begin{itemize}
                    \item (Reflexive) Let $fs^{-1}$ be a right fraction. Then $fs^{-1}\sim fs^{-1}$ by the diagram:
                    \begin{center}
                        \begin{tikzcd}
                            & W \ar{ld}{s} \ar{rd}{f} \ar[equal]{d} \\
                            X & W \ar{r}{f} \ar{l}{s} & Y \\
                            & W \ar{lu}{s} \ar[equal]{u} \ar{ru}{f}
                        \end{tikzcd}
                    \end{center}
                    \item (Symmetric) Let $fs^{-1}$ and $gt^{-1}$ be two right fractions such that $fs^{-1}\sim gt^{-1}$, that is the diagram commute. Due to inherent symmetric nature of the diagram it follows that $gt^{-1}\sim fs^{-1}$.
                    \begin{center}
                        \begin{tikzcd}
                            & W \ar{ld}{s} \ar{rd}{f} \\
                            X & \widetilde{W} \ar{r} \ar{l} \ar{u}{w} \ar{d}{w'} & Y \\
                            & W' \ar{lu}{t} \ar{ru}{g}
                        \end{tikzcd}
                        $\implies$
                        \begin{tikzcd}
                            & W' \ar{ld}{t} \ar{rd}{g} \\
                            X & \widetilde{W} \ar{l} \ar{r} \ar{u}{w'} \ar{d}{w} & Y \\
                            & W \ar{lu}{s} \ar{ru}{f}
                        \end{tikzcd}
                    \end{center}
                    \item (Transitive) Suppose that there are three right fractions $fs^{-1}$, $gt^{-1}$ and $hu^{-1}$ such that $fs^{-1}\sim gt^{-1}$ and $gt^{-1}\sim hu^{-1}$. That is, diagrammically speaking:
                    \begin{center}
                        \begin{tikzcd}
                            & W' \ar{ld}{s} \ar{rd}{f} \\
                            X & \widetilde{W} \ar{r} \ar{l} \ar{u}{w'} \ar{d}{\widetilde{w'}} & Y \\
                            & W \ar{lu}{t} \ar{ru}{g}
                        \end{tikzcd}
                        \begin{tikzcd}
                            & W \ar{ld}{t} \ar{rd}{g} \\
                            X & \widetilde{\widetilde{W}} \ar{r} \ar{l} \ar{u}{\widetilde{w''}} \ar{d}{w''} & Y \\
                            & W'' \ar{lu}{u} \ar{ru}{h}
                        \end{tikzcd}
                    \end{center}
                    By the Ore condition we can create two new maps from $\widetilde{w'}$ and $\widetilde{w''}$. Since both morphisms are assumed to be in $S$, then we get that both $\widetilde{w'}$ and $\widetilde{w''}$ are in $S$. The following diagram then shows that $fs^{-1}\sim hu^{-1}$. A simple diagram chase shows that it is commutative.
                    \begin{center}
                        \begin{tikzcd}
                            \widetilde{\widetilde{\widetilde{W}}} \ar{d}{\widetilde{\widetilde{w'}}} \ar{r}{\widetilde{\widetilde{w''}}} & \widetilde{\widetilde{W}} \ar{d}{\widetilde{w''}} \\
                            \widetilde{W} \ar{r}{\widetilde{w'}} & W
                        \end{tikzcd}
                        \begin{tikzcd}
                            & W' \ar{ldd}{s} \ar{rdd}{f} \\
                            & \widetilde{W} \ar{ld} \ar{rd} \ar{u}{\widetilde{w'}} \\
                            X & \widetilde{\widetilde{\widetilde{W}}} \ar{l} \ar{r} \ar{u}{\widetilde{\widetilde{w'}}} \ar{d}{\widetilde{\widetilde{w''}}}& Y \\
                            & \widetilde{\widetilde{W}} \ar{lu} \ar{ru} \ar{d}{\widetilde{w''}} \\
                            & W'' \ar{luu}{u} \ar{ruu}{h}
                        \end{tikzcd}
                    \end{center}
                \end{itemize}
            \end{proof}

            \begin{definition}
                Let $S$ be a multiplicate system in a category $\mathcal{C}$. Given two right fractions $fs^{-1}$ and $gt^{-1}$ in the diagrams:
                \begin{center}
                    \begin{tikzcd}
                        X & W \ar{l}{s} \ar{r}{f} & Y
                    \end{tikzcd}
                    \&
                    \begin{tikzcd}
                        Y & W' \ar{l}{t} \ar{r}{g} & Z
                    \end{tikzcd}
                \end{center}
                we can define the composite of these fractions $gt^{-1}\circ fs^{-1}$ by the Ore condition:
                \begin{center}
                    \begin{tikzcd}
                        & \widetilde{W} \ar{d}{u} \ar{r}{h} & W' \ar{d}{t} \ar{r}{g} & Z \\
                        X & W \ar{l}{s} \ar{r}{f} & Y
                    \end{tikzcd}
                \end{center}
                The composite is then the right fraction $gt^{-1}\circ fs^{-1} = gh(su)^{-1}$.
            \end{definition}

            \begin{prop}
                The composition of right fractions is well-defined up to equivalence.
            \end{prop}

            \begin{proof}
                In order to prove that the composite is well-defined we need to prove that the composite is independent from the different choices of the right Ore condition and that it is independent from choice of right fraction. There will only be presented a proof for that the choice of Ore maps is independent, as the other two cases are analogous.

                Suppose we have two right fractions $fs^{-1}$ and $gt^{-1}$ as indicated from the diagrams.
                \begin{center}
                    \begin{tikzcd}
                        X & W_1 \ar{l}{s} \ar{r}{f} & Y
                    \end{tikzcd}
                    \&
                    \begin{tikzcd}
                        Y & W_2 \ar{l}{t} \ar{r}{g} & Z
                    \end{tikzcd}
                \end{center}
                Further suppose that there are at least two different choices for the maps gained by the right Ore condition. That is for example $(\widetilde{W},\widetilde{s},\widetilde{f})$ and $(\widehat{W},\widehat{s}, \widehat{g})$. We can draw the two compositions as:
                \begin{center}
                    \begin{tikzcd}
                        & \widetilde{W} \ar{r}{\widetilde{g}} \ar{d}{\widetilde{s}} & W_2 \ar{r}{g} \ar{d}{t} & Z \\
                        X & W_1 \ar{r}{f} \ar{l}{s} & Y
                    \end{tikzcd}
                    \begin{tikzcd}
                        & \widehat{W} \ar{r}{\widehat{f}} \ar{d}{\widehat{s}} & W_2 \ar{r}{g} \ar{d}{t} & Z \\
                        X & W_1 \ar{l}{s} \ar{r}{f} & Y
                    \end{tikzcd}
                \end{center}
                By combining the diagrams at $W_1$ by using the right Ore condition again we can find $(W, \widetilde{w}, \widehat{w})$ as in the diagram below. 
                \begin{center}
                    \begin{tikzcd}
                        \bar{W} \ar[dashed]{rd}{\xi} \\
                        & W \ar{r}{\widehat{w}} \ar{d}{\widetilde{w}} & \widehat{W} \ar{d}[near start, below]{\widehat{s}} \ar{r}{\widehat{f}} & W_2 \ar{d}{t} \ar{r}{g} & Z \\
                        & \widetilde{W} \ar{r}{\widetilde{s}} \ar{rru}[near start]{\widetilde{g}} & W_1 \ar{r}{f} \ar{d}{s} & Y \\
                        & & X
                    \end{tikzcd}
                \end{center}
                We see that the three squares commute, as by the definition of right Ore condition. Thus we have that $s\widetilde{s}\widetilde{w} = s\widehat{s}\widehat{w}$. As the three squares commute we get that $t\widehat{f}\widehat{w}=t\widetilde{g}\widetilde{w}$. As $t:S$ we can use right cancellation to find a $\xi:\bar{W}\rightarrow W$ such that $\widehat{f}\widehat{w}\xi = \widetilde{g}\widetilde{w}\xi \implies g\widehat{f}\widehat{w}\xi = g\widetilde{g}\widetilde{w}\xi$. Thus the equivalence relation diagram commutes:
                \begin{center}
                    \begin{tikzcd}
                        & \widehat{W} \ar{ld}[above]{s\widehat{s}} \ar{rd}{g\widehat{f}} \\
                        X & \bar{W} \ar{u}{\widehat{w}\xi} \ar{d}{\widetilde{w}\xi} \ar{l} \ar{r} & Z \\
                        & \widetilde{W} \ar{lu}{s\widetilde{s}} \ar{ru}[below]{g\widetilde{g}}
                    \end{tikzcd}
                \end{center}
            \end{proof}

            \begin{prop}
                The composition of right fractions is associative.
            \end{prop}

            \begin{proof}
                Let $fs^{-1}$, $gt^{-1}$ and $hu^{-1}$ be right fractions as in the diagrams below:
                \begin{center}
                    \begin{tikzcd}
                        A & X \ar{l}{s} \ar{r}{f} & B
                    \end{tikzcd}
                    ,
                    \begin{tikzcd}
                        B & Y \ar{l}{t} \ar{r}{g} & C
                    \end{tikzcd}
                    \&
                    \begin{tikzcd}
                        C & Z \ar{l}{u} \ar{r}{h} & D
                    \end{tikzcd}
                \end{center}
                There are two different ways of calculating the compostion. Every morphism in $S$ will be marked blue.
                \begin{center}
                    \begin{minipage}[c]{0.4\textwidth}
                        \underline{$hu^{-1}\circ (gt^{-1}\circ fs^{-1})$}\\
                        \begin{tikzcd}
                            W \ar{rr} \ar[blue]{d}{} & & Z \ar{r}{} \ar[blue]{d}{} & D \\
                            V \ar[blue]{d}{} \ar{r}{} & Y \ar{r}{} \ar[blue]{d}{} & C \\
                            X \ar[blue]{d}{} \ar{r}{} & B \\
                            A
                        \end{tikzcd}
                    \end{minipage}
                    \begin{minipage}[c]{0.4\textwidth}
                        \underline{$(hu^{-1}\circ gt^{-1})\circ fs^{-1}$}\\
                        \begin{tikzcd}
                            V' \ar{r}{} \ar[blue]{dd}{} & W' \ar{r}{} \ar[blue]{d}{} & Z \ar{r}{} \ar[blue]{d}{} & D \\
                            & Y \ar{r}{} \ar[blue]{d}{} & C \\
                            X \ar{r}{} \ar[blue]{d}{} & B \\
                            A
                        \end{tikzcd}
                    \end{minipage}
                \end{center}
                To be able to find a relation between these diagrams we create another diagram with the right Ore condition.
                \begin{center}
                    \begin{minipage}[c]{0.3\textwidth}
                        \begin{tikzcd}
                            T \ar[dashed, blue]{r}{} \ar[dashed, blue]{d}{} & V' \ar[blue]{d} \\
                            W \ar[blue]{r}{} & X
                        \end{tikzcd}
                    \end{minipage}
                    \begin{minipage}[c]{0.5\textwidth}
                        To finish the proof, one would need to show that the maps to $A$ and $D$ commute. The maps to A commute right out of the bat, by the right Ore condition. To prove that the maps to D commute, first apply right cancellation on the maps to B, then on the maps to C.
                    \end{minipage}
                \end{center}
            \end{proof}

            \begin{definition}
                Let $S$ be a right multiplicative system in a category $\mathcal{C}$. We define a category $\mathfrak{r}S^{-1}\mathcal{C}$ to have objects $\mathfrak{Obr}S^{-1}\mathcal{C}=\mathfrak{Ob}\mathcal{C}$ and morphisms $\mathfrak{Arr}S^{-1}\mathcal{C} = \{$right fractions of $S\}/\sim$. This means that the morphisms $\mathfrak{r}S^{-1}\mathcal{C}(X,Y)$ are spans in $\mathcal{C}$ where one of the maps are in $S$ up to equivalence.
                \begin{center}
                    \begin{tikzcd}
                        X & A \ar[blue]{l} \ar{r} & Y
                    \end{tikzcd}
                \end{center}
                This is well-defined by the previous results and the identity morphisms are the right fractions of the form:
                \begin{center}
                    \begin{tikzcd}
                        X & X \ar[equal]{l} \ar[equal]{r} & X
                    \end{tikzcd}
                \end{center}
            \end{definition}

            \begin{remark}
                Dually there is a category $\mathfrak{l}S^{-1}\mathcal{C}$ for a left multiplicative system $S$ in a category $\mathcal{C}$. It is defined in the same manner as $\mathfrak{r}S^{-1}\mathcal{C}$, but with left fractions instead.
            \end{remark}

            \begin{remark}
                Given that $S$ is right multiplicative, A right fraction from the object $A$ to the object $B$ can be described with a special kind of diagram. Let $A\downarrow S$ be the comma category of arrows from $S$ ending in $A$ and let $\delta : A\downarrow S\rightarrow\mathcal{C}$ be the forgetful functor, sending each arrow to its domain. A morphism in $A\downarrow S$ from the objects $(b,B')$ to $(c,C')$ is a morphism $t : B'\rightarrow C'$ such that $b=ct$. We see that there is a correspondance between right fractions and elements in components of diagrams over $A\downarrow S$ such as $\mathcal{C}(\delta b, B)=\mathcal{C}(B',B)$ and right fractions. That is, let $f:\mathcal{\delta b, B}$, then $f$ can be regarded as $fb^{-1}$. A morphism from $\mathcal{C}(\delta c, B)$ to $\mathcal{C}(\delta b, B)$ is a morphism induced by a morphism from $(b,B')$ to $(c,C')$ in $A\downarrow S$. By the equivalence relation above we want to fractions $fb^{-1}$ and $gc^{-1}$ to be identified if there exists morphisms from $\mathcal{C}(\delta d, B)$ with maps $b' : (d,D')\rightarrow (b,B')$ and $c' : (d,D')\rightarrow (c,C')$ in $A\downarrow S$ such that $b'*f = c'*g$. This would be the same as saying that the right fractions are the coequalizer of the diagram $\mathcal{C}(\delta b, B)\coprod \mathcal{C}(\delta c, B)\rightrightarrows \mathcal{C}(\delta d, B)$. This observation motivates that the right fractions from $A$ to $B$ is described as the colimit of the functor $\mathcal{C}(\delta\_, B):A\downarrow S\rightarrow SET$. Dually, if $S$ is left multiplicative we get that the left fractions from $A$ to $B$ can be described as the colimit of the functor $\mathcal{C}(A, \rho\_):S\downarrow B\rightarrow SET$. More details can be found in \cite{zisman} and \cite{weibel}.
            \end{remark}

            Mitt lille utspill her må fikses. Jeg kan lage et addendum elns hvor det legges in alt som er relevant for å forstå hva som skjer elns...

            To ensure us that these categories $\mathfrak{r}S^{-1}\mathcal{C}$ does indeed exist there are many different criteria which we can place upon our assumptions. A natural restriction is to ensure that the colimits above exists as sets

            \begin{definition}
                A multiplicative system $S$ in a locally small category $\mathcal{C}$ is called locally small on the right if for every object $X:\mathcal{C}$ there is a set $S_X$ of morphisms from $S$ such that for every morphism $f : X_1 \rightarrow X$ in $S$ there is a morphism $f' : X'\rightarrow X$ in $S_X$ factoring thorugh $f$.

                The dual of this definition will be called a locally small multiplicative system on the left. If it is both locally small on the left and the right, we will simply call it locally small. 
            \end{definition}

            \begin{remark}
                If $S$ is a left multiplicative system, then $S\downarrow A$ is a filtered category for every object $A$. Dually, if $S$ is right multiplicative then $A\downarrow S$ is cofiltered.
            \end{remark}

            \begin{remark}
                Equipped with this notion we are now able to prove that the localizations exists as locally small categories. Locally small right multiplicative systems allows us to prove that the classes $\mathfrak{r}S^{-1}\mathcal{C}(X,Y)$ are sets. This can be seen as we can regard $S_X$ as a small category. By using the right Ore condition and left cancellation we can extend $S_X$ such that it is again cofiltered and admits the same colimit, i.e.\\
                 $\varinjlim\mathcal{C}(\delta\_,B):A\downarrow S\rightarrow Set\simeq\varinjlim\mathcal{C}(\delta\_,B):S_A\rightarrow Set$. 
            \end{remark}

            \begin{theorem}
                (Gabriel-Zisman) Let $S$ be a locally small right multiplicative system of morphisms in a category $\mathcal{C}$. Then the category $\mathfrak{r}S^{-1}\mathcal{C}$ exists and is the localization of $\mathcal{C}$ on $S$. That is there is an equivalence of categories $\mathcal{C}[S^{-1}]\simeq\mathfrak{r}S^{-1}\mathcal{C}$ together with a functor $q: \mathcal{C}\rightarrow\mathfrak{r}S^{-1}\mathcal{C}$ sending a morphism $f : X\rightarrow Y$ to the right fraction $fid_X^{-1}$.
            \end{theorem}

            \begin{proof}
                To prove the theorem we have to show that $q$ is a functor, and that it is universal. Suppose that $f: X\rightarrow Y$ and $g: Y\rightarrow Z$ are morphisms in $\mathcal{C}$. Then $q(gf)=(gf)id_X^{-1}$ and $q(g)q(f)=(gid_Y^{-1})\circ(fid_X^{-1})$. We can then choose the compostion to be defined by the diagram:
                \begin{center}
                    \begin{tikzcd}
                        X \ar{r}{f} \ar[equal,blue]{d} & Y \ar{r}{g} \ar[equal,blue]{d} & Z\\
                        X \ar{r}{f} \ar[equal,blue]{d} & Y \\
                        X
                    \end{tikzcd}
                \end{center}
                Thus we can see that $(gid_Y^{-1})\circ(fid_X^{-1})=(gf)id_X^{-1}$ asserting the functoriality of $q$.

                To see that $q$ is universal let $\mathcal{D}$ be a category where every morphism of $S$ is an isomorphism, and suppose there is a functor $F:\mathcal{C}\rightarrow\mathcal{D}$. We can define a functor $\mathfrak{r}S^{-1}F : \mathfrak{r}S^{-1}\mathcal{C}\rightarrow\mathcal{D}$ by $\mathfrak{r}S^{-1}F(fs^{-1})=F(f)F(s)^{-1}$. One can see that $F = \mathfrak{r}S^{-1}F\circ q$, it remains to show that it is well-defined. Suppose $fs^{-1}=gt^{-1}$, that means there is a diagram in $\mathcal{C}$ with the blue arrows in $S$:
                \begin{center}
                    \begin{tikzcd}
                        & W' \ar[blue]{ld}{s} \ar{rd}{f}  \\
                        X & W \ar[blue]{u}{w'} \ar[blue]{d}{w''} & Y \\
                        & W'' \ar[blue]{lu}{t} \ar{ru}{g}
                    \end{tikzcd}
                \end{center}
                Thus in $\mathcal{D}$ we have that $F(t)=F(sw')F(w'')^{-1}$ and $F(g)=F(fw')F(w'')^{-1}$, and this again gives us that 
                \begin{multline*}
                    \mathfrak{r}S^{-1}F(gt^{-1})=F(g)F(t)^{-1}\\
                    =F(fw')F(w'')^{-1}(F(fw')F(w'')^{-1})^{-1}=F(fw')F(w'')^{-1}F(w'')F(sw')^{-1}\\
                    =F(f)F(w')F(w')^{-1}F(s)^{-1}=F(f)F(s)^{-1}=\mathfrak{r}S^{-1}F(fs^{-1})
                \end{multline*}
                Thus $\mathfrak{r}S^{-1}F$ is well-defined and is unique by construction.
            \end{proof}

            \begin{corollary}
                If $S$ is a locally small left multiplicative system instead then $\mathfrak{l}S^{-1}\mathcal{C}$ is the localization of $\mathcal{C}$ on $S$.

                If moreover $S$ is a locally small multiplicative system, then there is an equivalence of categories $\mathfrak{r}S^{-1}\mathcal{C}\simeq\mathfrak{l}S^{-1}\mathcal{C}$.
            \end{corollary}

            \begin{proof}
                The first statement is dual to the theorem.

                To see the other statement, note that both $\mathfrak{r}S^{-1}\mathcal{C}$ and $\mathfrak{l}S^{-1}\mathcal{C}$ are the universal categories where the morphisms of $S$ are isomorphisms. Thus it follows that these categories have to be equivalent.
            \end{proof}

            \begin{remark}
                Since righthandedness of lefthandedness of the multiplicative system $S$ doesn't affect the localization we can simply call the localization of a (left/right) multiplicative system for $S^{-1}\mathcal{C}$.
            \end{remark}

            \begin{remark}
                A morphisms $f:\mathcal{C}(X,Y)$ will be invertible in the localized category if it is in the same equivalence class as the identity, both $id_X$ and $id_Y$. This forces a morphism $f$ to be invertible in $S^{-1}\mathcal{C}$ if and only if there is $g,h:S$ such that $fg,hf:S$.
            \end{remark}

            \begin{prop}
                Let $\mathcal{C}$ be a category, and $S$ a right multiplicative set of morphisms. The cannonical functor $q:\mathcal{C}\rightarrow S^{-1}\mathcal{C}$ commutes with finite limits.
            \end{prop}

            \begin{proof}
                Let $T:\mathcal{D}\rightarrow\mathcal{C}$ be a diagram over a finite category $\mathcal{D}$. Then for any object $A:S^{-1}\mathcal{C}$ we have the following equation.
                \begin{multline*}
                    S^{-1}\mathcal{C}(qA,q(\varprojlim T\_)\simeq \varinjlim\mathcal{C}(\delta\_,\varprojlim T\_))\\
                    \simeq \varinjlim\varprojlim\mathcal{C}(\delta\_,T\_)\simeq \varprojlim\varinjlim\mathcal{C}(\delta\_,T\_)\simeq \varprojlim S^{-1}\mathcal{C}(qA,q(T\_))
                \end{multline*}
                The first isomorphism is by the remark, the second is by the representative nature of limits and the third isomorphism is from the fact that filtered colimits commute with finite limits. The colimits are filtered by the remark that $S_A$ is cofiltered and that the functor $\mathcal{C}(_,A)$ is contravariant.
            \end{proof}

            \begin{remark}
                I am not quite sure yet how this argument proves the statement I want to prove, but that can be figure out later. I also don't know the proof for why filtered colimits commute with finite colimits, but it is in Riehls book.
            \end{remark}

            \begin{prop}
                Let $\mathcal{C}$ be a category with a zero. That is an object which is both initial and terminal. Suppose that $S$ is a right multiplicative system, then $q0$ is a zero object in $S^{-1}\mathcal{C}$.
            \end{prop}

            \begin{proof}
                The claim that $q0$ is initial follows from that initial is a limit of a diagram over the empty category. To see that $q0$ is terminal we only need to prove that every right fraction is equivalent with $0id_A^{-1}$. The following diagram proves this:
                \begin{center}
                    \begin{tikzcd}
                        & X \ar{ld}[above]{f} \ar{d}{f} \ar{rd}{0}\\
                        A & \ar[equal]{l} A \ar{r}{0} & 0
                    \end{tikzcd}
                \end{center}
            \end{proof}

            \begin{prop}
                If $\mathcal{A}$ is an additive category and $S$ is a right multiplicative system, then $S^{-1}\mathcal{A}$ is additive as well.
            \end{prop}

            \begin{proof}
                From the previous propositions we know that $q0$ is the zero object and that $q(A\times B)\simeq qA\times qB$. By proving that there is an addition induced by $\mathcal{A}$ and that $q$ preserves this addition we get that the product is the biproduct induced by the maps in $\mathcal{A}$.

                Suppose that we have the fractions $fs^{-1}, gt^{-1}:S^{-1}\mathcal{C}(A,B)$. We define their addition by using the Ore condition  to find new morphisms $f$, $g'$ and $u$ such that $fs^{-1} = f'u^{-1}$ and $gt^{-1} = g'u^{-1}$. Then the addition is defined as follows:
                \begin{equation*}
                    fs^{-1}+gt^{-1} = (f'+g')u^{-1}
                \end{equation*}
                To prove that this is an addition we should prove that it is well defined, associativity, inverses and commutativity will be inherited from $\mathcal{A}$.
                Let $v$ be another extension. That is $\bar{f}v^{-1}=fs^{-1}=f'u^{-1}$ and $\bar{g}v^{-1}=gt^{-1}=g'u^{-1}$, so our goal is to prove that $(\bar{f}+\bar{g})v^{-1}-(f'+g')u^{-1}=0$. By definition we have that $(\bar{f}+\bar{g})v^{-1}-(f'+g')u^{-1}=\bar{f}v^{-1}-f'u^{-1}+\bar{g}v^{-1}-g'u^{-1}$. So to prove that the whole sum is $0$ is the same to prove as $\bar{f}v^{-1}+(-f')u^{-1}=(\bar{\bar{f}}-f'')w^{-1}=0$. This can be done by writing out the diagrams after repeatedly applying right Ore condition.
                \begin{center}
                    \begin{tikzcd}
                        \cdot \ar[bend right, dashed, blue]{rddd}{p} \ar[bend left, dashed, blue]{rrrd}{p} \ar[dashed, blue]{rd} \\
                        & \cdot \ar[blue]{r} \ar[blue]{d} \ar[blue]{rd}{w} & \cdot \ar[blue]{r} \ar[blue]{d}{u} & \cdot \ar{r}{f} \ar[blue]{ld}{s} & B \\
                        & \cdot \ar[blue]{d} \ar[blue]{r}{v} & A \\
                        & \cdot \ar[blue]{ru}{s} \ar{d}{f} \\
                        & B
                    \end{tikzcd}
                \end{center}
                The line to the bottom represents $\bar{\bar{f}}$ and the line to the right represents $f''$. By using left cancellation on the common morphism $s$ into $A$ one obtains the morphism $p$ which relates the two fractions and makes the sum go to zero.

                It remains to prove that $q:\mathcal{C}\rightarrow S^{-1}\mathcal{C}$ respects addition. Assume that $f,g:\mathcal{C}(X,Y)$
                \begin{equation*}
                    q(f+g)=(f+g)id_X^{-1}=fid_X^{-1}+gid_X^{-1}=qf+qg
                \end{equation*} 
            \end{proof}

            \begin{corollary}
                If $\mathcal{A}$ is abelian and $S$ is a multiplicative system, then $S^{-1}\mathcal{A}$ is abelian as well.
            \end{corollary}

            \begin{definition}
                A triangulated functor $F: \mathcal{T} \rightarrow \mathcal{S}$ between two triangulated categories $(\mathcal{T}, T, \Delta_\mathcal{T}$ and $(\mathcal{S}, S, \Delta_\mathcal{S})$, is an additive functor along with a natural isomorphism $\phi_X : F(T(X)) \rightarrow S(F(X))$ such that $F(\Delta_{\mathcal{T}}) \subseteq \Delta_{\mathcal{S}}$. This means that for every triangle in $\mathcal{T}$ there is a triangle in $\mathcal{S}$.
                \begin{center}
                    \begin{tikzcd}[row sep=tiny]
                        A \arrow{rd}{a} & \\
                        & B \arrow{dl}{b} & & \\
                        C \arrow[very near end, "|" marking]{uu}[near start]{c}[near end]{T}
                    \end{tikzcd}
                    $\implies$
                    \begin{tikzcd}[row sep=tiny]
                        F(A) \arrow{rd}{F(a)} & \\
                        & F(B) \ar{dl}{F(b)} & & \\
                        F(C) \arrow[very near end, "|" marking]{uu}[near start]{F(c)}[near end]{T}
                    \end{tikzcd}
                \end{center}
            \end{definition}

            \begin{definition}
                A triangulated subcategory $\mathcal{S}$ of a triangulated category $\mathcal{T}$ is a full additive subcategory closed under isomorphisms such that the inclusion functor is triangulated.
            \end{definition}

            \begin{definition}
                Let $F : \mathcal{S} \rightarrow \mathcal{T}$ be a triangulated functor. The kernel of $F$ is defined to be the full subcategory $Ker(F)$ of $\mathcal{S}$ such that every object in $Ker(F)$ gets mapped to $0$ by $F$. That is, $Ker(F)$ is the class of objects $\{K : \mathcal{S} | F(K)\simeq 0\}$.
            \end{definition}

            \begin{lemma}
                The kernel of a triangulated functor $F:\mathcal{C}\rightarrow{D}$ is a triangulated subcategory.
            \end{lemma}

            \begin{proof}
                Let $X:KerF$, since $F$ is a triangulated functor $CX:KerF$ as $F(CX)=D(FX)=D0=0$. As $F$ is triangulated, we have that every triangle maps to a triangle. Let $X,Y:KerF$, then:
                \begin{center}
                    \begin{tikzcd}[row sep=small]
                        X \ar{rd} \\
                        & Y \ar{ld} \\
                        Z \ar[very near end, "|" marking]{uu}[near end]{T}
                    \end{tikzcd}
                    $\implies$
                    \begin{tikzcd}[row sep=small]
                        0 \ar{rd} \\
                        & 0 \ar{ld}\\
                        F(Z) \ar[very near end, "|" marking]{uu}[near end]{T}
                    \end{tikzcd}
                \end{center}
                By TR3 and the 2 out of 3 property $F(Z)\simeq 0 \implies Z:KerF$. Thus $KerF$ is a triangulated subcategory of $\mathcal{C}$.
            \end{proof}

            \begin{definition}
                A subcategory $\mathcal{S}$ of a triangulated category $\mathcal{T}$ is called thick if it contains all the direct summands of its objects.
            \end{definition}

            \begin{lemma}
                The kernel of a triangulated functor $F:\mathcal{C}\rightarrow\mathcal{D}$ is thick.
            \end{lemma}

            \begin{proof}
                Let $X\oplus Y:KerF$, since $F$ is additive we have that $0\simeq F(X\oplus Y)\simeq F(X)\oplus F(Y)$, but then there is a splitmono $F(X)\rightarrow 0 \implies F(X)\simeq 0 \simeq F(Y)$.
            \end{proof}

            \begin{lemma}
                Let $F:\mathcal{C}\rightarrow\mathcal{D}$ be a triangulated functor. Suppose that $f:X\rightarrow Y$ is a morphism such that $F(f)$ is an isomorphism. Then the cone of $f$ is in $KerF$.
            \end{lemma}

            \begin{proof}
                There is an isomorphism of triangles in $\mathcal{D}$, showing that the cone of $f$ is in $KerF$.
                \begin{center}
                    \begin{tikzcd}
                        FX \ar{r}{Ff} \ar[equal]{d} & FY \ar{r} \ar[equal]{d} & F(cone(f)) \ar{r} \ar[dashed]{d}[rotate=90, below]{\simeq} & FTX \ar[equal]{d} \\
                        FX \ar{r}{Ff} & FY \ar{r} & 0 \ar{r} & FTX
                    \end{tikzcd}
                \end{center}
            \end{proof}

            The goal for the rest of this section is to prove that there is a localization at any triangulated subcategory $\mathcal{S}\subseteq\mathcal{C}$. This localization will yield a functor $q:\mathcal{C}\rightarrow \mathcal{C}/\mathcal{S}$ such that $\mathcal{S}\subseteq Kerq$. We will define a set of morphism $Mor_\mathcal{S}$ related to $\mathcal{S}$ such that this set is multiplicative.

            \begin{definition}
                Let $\mathcal{C}$ be a triangulated category and $\mathcal{S} \subseteq \mathcal{C}$ be a triangulated subcategory. Define a collection $Mor_{\mathcal{S}}$ to be a collection of morphisms in $\mathcal{C}$ such that for any $f : Mor_{\mathcal{S}}$ there is a triangle with $C : \mathcal{S}$.
                \begin{center}
                    \begin{tikzcd}
                        A \ar{r}{f} & B \ar{r} & C \ar{r} & TA 
                    \end{tikzcd}
                \end{center}
            \end{definition}

            \begin{remark}
                Every isomorphism is in $Mor_{\mathcal{S}}$. That is because isomorphisms are found in triangles $(A,B,0,f,0,0)$ and $0 : \mathcal{S}$ for any triangulated subcategory.
            \end{remark}

            \begin{lemma}
                Let $f : X \rightarrow Y$ and $g : Y \rightarrow Z$ be two morphisms. If any two of the morphisms $f$, $g$ and $gf$ are in $Mor_{\mathcal{S}}$ then so is the third.
            \end{lemma}

            \begin{proof}
                We are able to find three triangles in $\mathcal{C}$.
                \begin{center}
                    (1)
                    \begin{tikzcd}[row sep=tiny]
                        X \arrow[red]{rd}[black]{f} & \\
                        & Y \arrow[red]{dl} & & \\
                        Z' \arrow[red, very near end, "|" marking]{uu}[black, near end]{T}
                    \end{tikzcd}
                    (2)
                    \begin{tikzcd}[row sep=tiny]
                        Y \arrow[orange]{rd}[black]{g} & \\
                        & Z \arrow[orange]{dl} & & \\
                        X' \arrow[orange, very near end, "|" marking]{uu}[black, near end]{T}
                    \end{tikzcd}
                    (3)
                    \begin{tikzcd}[row sep=tiny]
                        A \arrow[violet]{rd}[black]{g\circ f} & \\
                        & C \arrow[violet]{dl} & & \\
                        Y' \arrow[violet, very near end, "|" marking]{uu}[black, near end]{T}
                    \end{tikzcd}
                \end{center}
                By the Octahedron axiom there exist another triangle in $\mathcal{C}$:
                \begin{center}
                    \begin{tikzcd}
                        Z' \ar[teal]{r} & X' \ar[teal]{r} & Y' \ar[teal]{r} & TZ'
                    \end{tikzcd}
                \end{center}
                Note that $f$ is in $Mor_\mathcal{S}$ if and only if $Z' : S$. WLOG assume that $f$ and $g$ is in $Mor_\mathcal{S}$, this can be done due to the rotation axiom. Thus we can find the triangle $(Z',X',Y'')$ in $\mathcal{S}$ proving that $Y'$ is in $\mathcal{S}$ by the following diagram:
                \begin{center}
                    \begin{tikzcd}
                        Z' \ar[teal]{r} \ar[equal]{d} & X' \ar[equal]{d} \ar[teal]{r} & Y' \ar[teal]{r} \ar{d}[rotate=90, below]{\simeq} & TZ' \ar[equal]{d} \\
                        Z' \ar{r} & X' \ar{r} & Y'' \ar{r} & TZ'
                    \end{tikzcd}
                \end{center}
            \end{proof}

            \begin{prop}
                Let $\mathcal{S}\subseteq\mathcal{C}$ be a triangulated subcategory, then $Mor_\mathcal{S}$ satisfies the Ore condition.
            \end{prop}

            \begin{proof}
                To prove that a system satisfies the Ore condition there has to be a proof for both right and left condition. Luckily, the arguments presented here can be dualized to give a proof for the other condition. Thus there will only be presented a proof for the right Ore condition.
                Let $f:A\rightarrow C$ be in $Mor_\mathcal{S}$ and $g:B\rightarrow C$ in $\mathcal{C}$. Then we can form a homotopy pullback forming a homotopy cartesian square as follows:
                \begin{center}
                    \begin{tikzcd}
                        & A \ar{d}{f} \\
                        B \ar{r}{g} & C
                    \end{tikzcd}
                    $\implies$
                    \begin{tikzcd}
                        D \ar{r}{g'} \ar{d}{f'} \ar[phantom]{rd}[description]{HO}[very near start]{\ulcorner}[very near end]{\lrcorner} & A \ar{d}{f} \\
                        B \ar{r}{g} & C
                    \end{tikzcd}
                \end{center}
                By Lemma 2.12 there are triangles along this homotopy square identifying the cones. Since the cone of $f$ is assumed to be in $\mathcal{S}$, the cone of $f'$ is also in $\mathcal{S}$. This proves that $f':Mor_\mathcal{S}$.
            \end{proof}

            \begin{prop}
                For any parallell morphism $f,g:X\rightarrow Y$ in $\mathcal{C}$ the following are equivalent:
                \begin{enumerate}
                    \item $sf=sg$ for some $s:Mor_\mathcal{S}$ starting at $Y$.
                    \item $ft=gt$ for some $t:Mor_\mathcal{S}$ ending at $X$.
                    \item $f-g$ factors through an object $C:\mathcal{S}$.
                \end{enumerate}
            \end{prop}

            \begin{proof}
                $(1.\iff 3.)$:
                Suppose that there exists an $s:Y\rightarrow Z$ such that $s(f-g)=0$. By TR1 there is a triangle \begin{tikzcd}Y \ar{r}{s} & Z \ar{r}{Ts'} & TC \ar{r} & TY \end{tikzcd} and a long exact sequence of hoom(ology).
                \begin{center}
                    \begin{tikzcd}
                        \mathcal{C}(X,C) \ar{r}{s'_*} & \mathcal{C}(X,Y) \ar{r}{s_*} & \mathcal{T}(X,Z) \\
                        p \ar[pos=0, "|" marking]{r}[pos=0.5]{s'_*}& f-g \ar[pos=0, "|" marking]{r}[pos=0.5]{s_*} & 0
                    \end{tikzcd}
                \end{center}
                Since $s(f-g)=0$ there exists a $p:\mathcal{C}(X,C)$ such that $f-g = s'_*p$. By definition, $s:Mor_\mathcal{S}\iff C:\mathcal{S}$, but $s:Mor_\mathcal{S}\implies f-g$ factors through $C$, and vice versa.
                $(2.\iff 3.)$:
                This argument is dual.
            \end{proof}

            This has shown that $Mor_\mathcal{S}$ is a multiplicative system, and Theorem 2.17 say that the localization exists given that $Mor_\mathcal{S}$ is locally small. The category $Mor_\mathcal{S}^{-1}$ will be called for $\mathcal{C}/\mathcal{S}$ and it is called the Verdier quotient. As $\mathcal{C}$ is additive, it is known that $\mathcal{C}/\mathcal{S}$ is additive as well by Proposition 2.20. The remaining part is to show that $\mathcal{C}/\mathcal{S}$ is triangulated and that localization functor $q:\mathcal{C}\rightarrow \mathcal{C}/\mathcal{S}$ is a triangulated functor.

            \begin{theorem}
                Let $\mathcal{S}\subseteq\mathcal{C}$ be triangulated categories. Then the Verdier quotient $\mathcal{C}/\mathcal{S}$ and the functor $q:\mathcal{C}\rightarrow\mathcal{C}/\mathcal{S}$ is the universal triangulated category where morphisms in $Mor_\mathcal{S}$ are isomorphisms.
            \end{theorem}

            \begin{proof}
                The triangulation on $\mathcal{C}/\mathcal{S}$ is defined as the following. Let $C/S:\mathcal{C}/\mathcal{S}\rightarrow\mathcal{C}/\mathcal{S}$ be the additive autoequivalence defined by $C/S(A)=C(A)$. Since $q:\mathcal{C}\rightarrow\mathcal{C}/\mathcal{S}$ maps every object to itself it follows that $q(C(A)) \simeq C/S(A) = C/S(q(A))$, and define $\Delta_{\mathcal{C}/\mathcal{S}}\supseteq q(\Delta_\mathcal{C})$ such that $\Delta_{\mathcal{C}/\mathcal{S}}$ has every isomorphism class of $q(\Delta_\mathcal{C})$. 
                \begin{center}
                    \begin{tikzcd}[row sep=small]
                        qX \ar{rd} \\
                        & qY \ar{ld} \\
                        qZ \ar[very near end, "|" marking]{uu}[near end]{T}
                    \end{tikzcd}
                    $\impliedby$
                    \begin{tikzcd}[row sep=small]
                        X \ar{rd} \\
                        & Y \ar{ld} \\
                        Z \ar[very near end, "|" marking]{uu}[near end]{T}
                    \end{tikzcd}
                \end{center}
                Then by definition $q$ is triangulated if the category $\mathcal{C}/\mathcal{S}$ is triangulated.
                By definition, the triangles are closed under isomorphisms, $(X,X,0,id_X,0,0)$ is a triangle, and TR2 holds. Thus it remains to show TR1 and TR4 (TR3 is implied by the other axioms). To prove TR1, let $fs^{-1}:\mathcal{C}/\mathcal{S}(qW,qY)$. Then expand $f:\mathcal{C}(X,Y)$ to a triangle in $\mathcal{C}$ with TR1, it will induce a triangle in  $\mathcal{C}/\mathcal{S}$.
                \begin{center}
                    \begin{tikzcd}
                        qX \ar{r}{fid_X^{-1}} & qY \ar{r}{gid_Y^{-1}} & qZ \ar{r}{hid_Z^{-1}} & qTX
                    \end{tikzcd}
                \end{center}
                Then there is an isomorphism to the following candidate triangle from the induced triangle, proving TR1.
                \begin{center}
                    \begin{tikzcd}
                        qX \ar{r}{fid_X^{-1}} \ar{d}{sid_X^{-1}}[above, rotate = 90]{\simeq} & qY \ar{r}{gid_Y^{-1}} \ar[equal]{d} & qZ \ar{r}{hid_Z^{-1}} \ar[equal]{d} & qTX \ar{d}{(Ts)id_{Tx}^{-1}}[above, rotate = 90]{\simeq} \\
                        qW \ar{r}{fs^{-1}} & qY \ar{r}{gid_Y^{-1}} & qZ \ar{r}{(Ts)hid_Z^{-1}} & qTW
                    \end{tikzcd}
                \end{center}
                To show the Octahedron axiom, suppose that there are three triangles in $\mathcal{C}/\mathcal{S}$. By construction, these triangles can be chosen such that only the first map is a fraction up to isomorphism of triangles.
                \begin{center}
                    (1)
                    \begin{tikzcd}[row sep=tiny]
                        Z \ar{r}{t'}& X \ar{ld}[above]{s} \ar{dd}{a} \\
                        A \arrow[red]{rd}[black]{as^{-1}} \\
                        & B \arrow[red]{dl}[black]{x}\\
                        C' \arrow[red, very near end, "|" marking]{uu}[near start, black]{x'}[near end]{T}
                    \end{tikzcd}
                    (2)
                    \begin{tikzcd}[row sep=tiny]
                        & Y \ar{ld}[above]{t} \ar{dd}{b} \\ 
                        B \arrow[orange]{rd}[black]{bt^{-1}} &  \\
                        & C \arrow[orange]{dl}[black]{y}\\
                        A' \arrow[orange, very near end, "|" marking]{uu}[near start, black]{y'}[near end]{T}
                    \end{tikzcd}
                    (3)
                    \begin{tikzcd}[row sep=tiny]
                        & Z \ar{ld}[above]{st'} \ar{dd}{ba'} \\
                        A \arrow[violet]{rd}[black]{b\circ a} \\
                        & C \arrow[violet]{dl}[black]{z} \\
                        B' \arrow[violet, very near end, "|" marking]{uu}[near start, black]{z'}[near end]{T}
                    \end{tikzcd}
                \end{center}
                By composing the fractions from $A$ to $B$ and $B$ to $C$ we can find an object $Z$ as in the diagram with the Ore condition. To illustrate with triangle (1), there is a correspondance of triangles in $\mathcal{C}/\mathcal{S}$ and $\mathcal{C}$. That is by the following isomorphism:
                \begin{center}
                    \begin{tikzcd}
                        Z \ar{r}{at'} \ar{d}[above]{t}[below]{\simeq} & B \ar{r} \ar[equal]{d} & Z' \ar{r} \ar[dashed]{d}[below, rotate = 90]{\simeq} & TZ \ar[equal]{d} \\
                        X \ar{r}{a} \ar{d}{s} & B \ar{r} & C' \ar{r} & TX \ar{d}{Ts} \\
                        A & & & TA 
                    \end{tikzcd}
                \end{center}
                The result of the octahedron axiom follows as we can instead consider the triangles found by the composition of morphisms as below.
                \begin{center}
                    \begin{tikzcd}
                        Z \ar{d}{f'} \ar{rd}{bf'} \\
                        Y \ar{r}{b} & C
                    \end{tikzcd}
                \end{center}
            \end{proof}

            \begin{prop}
                Let $\mathcal{S}\subseteq\mathcal{C}$ be triangulated categories. If $0:X\rightarrow 0$ is an isomorphism in $\mathcal{C}/\mathcal{S}$, then there is an object $Y$ such that $X\oplus Y:\mathcal{S}$.
            \end{prop}

            \begin{proof}
                If $0:X\rightarrow 0$ is invertible, then there exist a map $0:0\rightarrow Y$, such that $0:X\rightarrow Y$ is in $Mor_S$. By definition $X\oplus Y$ is in $\mathcal{S}$.
            \end{proof}

            This proposition shows us that the kernel of $q:\mathcal{C}\rightarrow\mathcal{C}/\mathcal{S}$ is the smallest thick subcategory of $\mathcal{C}$ such that $\mathcal{C}/Kerq$ is the universal category where every morphism in $Mor_\mathcal{S}$ is an isomorphism. For this reason $\widehat{\mathcal{S}}=Kerq$ is called the thick closure of $\mathcal{S}$.

        \subsection{Universal Homological Embedding}
            Do Yoneda embedding into functor categories.
    
    \clearpage
    
    \section{Exact Categories}
            
        I can maybe write some of the history of the development of the idea of exact categories. 

        \subsection{Definitions and First Properties}

            In this section we will focus on defining what an exact category is and the first elementary properties. We will prove the axiom dubbed as "the obscure axiom" and motivate that it is not as obscure as its name suggest. Some "short" variants of some homological diagram lemmas will also be proved.

            To start with the exact categories we will first take a look towards the abelian ones first. Short exact sequences are of great interest, and they can be characterized with two morphisms $p:A\rightarrow B$ and $q:B\rightarrow C$ such that p is the kernel of q and q is the cokernel of p. This leads to the first definition.

            \begin{definition}
                A kernel-cokernel pair is a pair of maps $(p,q)$ such that p is the kernel of q and q is the cokernel of p. A morphism of kernel-cokernel pairs $(p,q)$ and $(p',q')$ is a triple $(f,g,h)$ such that the following diagram commutes. An isomorphisms is a triple in which each morphism is an isomorphism.
                \begin{center}
                    \begin{tikzcd}
                        A \ar[tail]{r}{p} \ar[two heads]{d}{f} & B \ar{r}{q} \ar{d}{g} & C \ar{d}{h} \\
                        A' \ar[tail]{r}{p'} & B' \ar[two heads]{r}{q'} & C'
                    \end{tikzcd}
                \end{center}
            \end{definition}

            \begin{lemma}
                Let $(p,q)$ be a kernel-cokernel pair, then the image and coimage of p exists and are isomorphic. I.e. this diagram exists, such that the left square is a push-out and the right square is a pull-back:
                \begin{center}
                    \begin{tikzcd}
                        0 \ar{r}{0} \ar{rd}{0} & A \ar[tail]{r}{p} \ar[two heads]{d} & B \ar[two heads]{r}{q} & C \\
                        & Coim(p) \ar[tail, two heads]{r}{iso} & Im(p) \ar[tail]{u} \ar{ur}{0}
                    \end{tikzcd}
                \end{center}
            \end{lemma}
            
            %Usikker på om dette beviset er riktig, jeg burde ha en måte å konstruere isomorfien på, ettersom at vi ønsker at analyse morfien skal være isomorfien, ikke at vi kan velge identiteten
            \begin{proof}
                Since $(p,q)$ is a kernel-cokernel pair we have that the first square is bicartesian and the second square is a push-out.
                \begin{center}
                    \begin{tikzcd}
                    A \ar[tail]{r}{p} \ar{rd}{0} & B \ar[two heads]{d}{q} \\ & C
                    \end{tikzcd}
                    \begin{tikzcd}
                        0 \ar{r}{0} \ar{rd}{0} & A \ar[equal]{d} \\ & A
                    \end{tikzcd}
                \end{center}
                Thus $Im(p)=Coim(p)=A$, asserting the isomorphism as the identity in the diagram.
                \begin{center}
                    \begin{tikzcd}
                        0 \ar{r}{0} \ar{rd}{0} & A \ar[tail]{r}{p} \ar[equal]{d} & B \ar[two heads]{r}{q} & C \\
                        & A \ar[equal]{r} & A \ar[tail]{u}{p} \ar{ur}{0}
                    \end{tikzcd}
                \end{center}
            \end{proof}

            \begin{corollary}
                Suppose that $(p,q)$ is a kernel-cokernel pair. If $p$ is an epimorphism, then $p$ is an isomorphism.
            \end{corollary}

            \begin{definition}
                An exact structure for an additive category $\mathcal{A}$ is a class $\mathcal{E}$ of kernel-cokernel pairs which are closed under isomorphisms. A pair $(p,q):\mathcal{E}$ is called a conflation, here $p$ is called an inflation and $q$ is called a deflation. $(\mathcal{A},\mathcal{E})$ is called exact when the following axioms holds:
                % Kan korte ned på antallet av aksiomer, jeg trenger ikke å ha op for alle, blir nok mer oversiktelig om jeg kobler noen sammen.
                \begin{itemize}
                    \item (QE0) $\forall A:\mathcal{A}$ $id_A$ is both an inflation and a deflation.
                    \item (QE1) Both inflations and deflations are closed under composition.
                    \item (QE2) The push-out of an inflation is an inflation.
                    \item (QE2$^{op}$) The pull-back of a deflation is a deflation.
                \end{itemize}

                An exact category is the additive category $\mathcal{A}$ together with an exact structure $\mathcal{E}$.
            \end{definition}


            \begin{remark}
                When writing diagrams we use decorated arrows to indicate that a morphism is either an inflation or a deflation. A tail with a circle means inflation: \begin{tikzcd}
                    A \ar[tail]{r}[marking]{\circ} & B
                \end{tikzcd}. Double heads with a circle means deflation: \begin{tikzcd}
                    A \ar[two heads]{r}[marking]{\circ} & B
                \end{tikzcd}. We can now rewrite the $(QE2^*)$ axioms as:
                \begin{center}
                    \begin{tikzcd}
                        A \ar[tail]{r}[marking]{\circ} \ar{d} \ar[phantom]{dr}[very near end]{\lrcorner} & B \ar{d} \\
                        C \ar[tail]{r}[marking]{\circ} & D
                    \end{tikzcd}
                    \begin{tikzcd}
                        A \ar[two heads]{r}[marking]{\circ} \ar{d} \ar[phantom]{dr}[very near start]{\ulcorner} & B \ar{d} \\
                        C \ar[two heads]{r}[marking]{\circ} & D
                    \end{tikzcd}
                \end{center}
            \end{remark}

            \begin{remark}
                Inflations are also called admissable monomorphisms, deflations are also called admissable epimorphisms and conflations are also called short exact sequences.
            \end{remark}

            \begin{remark}
                The axioms for an exact category is made in such a way that $\mathcal{E}$ is an exact structure for $\mathcal{A}$ if and only if $\mathcal{E}^{op}$ is an exact structure for $\mathcal{A}^{op}$.
            \end{remark}

            \begin{remark}
                There is a category of arrows and double arrows. For any category $\mathcal{C}$, there is a category $\mathcal{C}^{\rightarrow}=\mathcal{C}\downarrow\mathcal{C}$ and $\mathcal{C}^{\rightarrow\rightarrow}=\mathcal{C}\downarrow\mathcal{C}\downarrow\mathcal{C}$. If there is an additive category $\mathcal{A}$, then $\mathcal{A}^{\rightarrow}$ and $\mathcal{A}^{\rightarrow\rightarrow}$ are additive as well. It can be seen that $\mathcal{E}$ can be considered as an extension closed additive subcategory of $\mathcal{A}^{\rightarrow\rightarrow}$.
            \end{remark}

            \begin{example}
                Any abelian category is exact with every short-exact sequence as the exact structure.
            \end{example}

            \begin{example}
                Any additive category is exact with every split short-exact sequence as the exact structure.
            \end{example}

            \begin{lemma}
                The map $0:0\rightarrow A$ is an inflation. Dually, the map $0:A\rightarrow 0$ is a deflation.
            \end{lemma}

            \begin{proof}
                Consider the diagram \begin{tikzcd}
                    0 \ar[tail]{r}{0} & A \ar[two heads]{r}{id_A} & A
                \end{tikzcd}. The left morphism is the kernel of the right morphism making a kernel-cokernel pair $(0,id_A)$. The identity $id_A$ is assumed to be a deflation, implying that the pair is a conflation.
            \end{proof}

            \begin{remark}
                It can be seen that isomorphisms are deflations. Let $f:A\rightarrow B$ be an isomorphism, then there are two kernel-cokernel pairs: $(0,id_A)$ and $(0,f)$. Between these there is an isomorphism which is the triple $(0,id_A,f^{-1})$. As the conflations are closed under isomorphism, $(0,f)$ is a conflation, making f into a deflation. By dualizing this argument f is also an inflation.
                \begin{center}
                    \begin{tikzcd}
                        0 \ar{r}{0} \ar{d}{0} & A \ar[tail, two heads]{r}{f} \ar{d}{id_A} & B \ar[tail, two heads]{d}{f^{-1}} \\
                        0 \ar[tail]{r}{0}[marking]{\circ} & A \ar[two heads]{r}{id_A}[marking]{\circ} & A
                    \end{tikzcd}
                \end{center}
            \end{remark}

            \begin{corollary}
                A kernel-cokernel pair $(i,p)$ found as a split short-exact sequence (1) is a conflation. 
                
                \begin{center}
                    (1)
                    \begin{tikzcd}
                        A \ar[tail]{r}{i}[marking]{\circ} & A \oplus B \ar[two heads]{r}{p}[marking]{\circ} & B
                    \end{tikzcd}
                \end{center}
            \end{corollary}

            \begin{proof}
                In a category with an initial object the coproduct can be thought of as the push-out with the initial in the upper right corner. This can be assembled into push-out (1).
                By the lemma the zero morphisms are inflations, asserting that $i$ and $i'$ are inflations by (QE2). Thus there are conflations $(i,p)$ and $(i',p')$.

                \begin{center}
                    (1)
                    \begin{tikzcd}
                        0 \ar[tail]{r}{0}[marking]{\circ} \ar[tail]{d}{0}[marking]{\circ} & A \ar[tail]{d}{i}[marking]{\circ} \\
                        B \ar[tail]{r}{i'}[marking]{\circ} & A \oplus B
                    \end{tikzcd}
                \end{center}
            \end{proof}

            \begin{corollary}
                The direct sum of conflations is a conflation. I.e. there is a diagram:
                \begin{center}
                    \begin{tikzcd}
                        A \ar[tail]{r}{i}[marking]{\circ} & B \ar[two heads]{r}{p}{\circ} & C
                    \end{tikzcd} \&
                    \begin{tikzcd}
                        A' \ar[tail]{r}{i'}[marking]{\circ} & B' \ar[two heads]{r}{p'}[marking]{\circ} & C'
                    \end{tikzcd} \\
                    $\Downarrow$ \\
                    \begin{tikzcd}
                        A\oplus A' \ar[tail]{r}{i\oplus i'}[marking]{\circ} & B\oplus B' \ar[two heads]{r}{p\oplus p'}[marking]{\circ} & C\oplus C'
                    \end{tikzcd}
                \end{center}
            \end{corollary}

            \begin{proof}
                We start with only considering the conflation $(i,p)$. $\forall{D}$ there is a conflation $(i\oplus id_D, p\oplus 0)$, drawn as the diagram.
                \begin{center}
                    \begin{tikzcd}[ampersand replacement=\&]
                        A\oplus D \ar[tail]{r}{\begin{pmatrix}i & 0 \\ 0 & 1\end{pmatrix}}[marking]{\circ} \& B\oplus D \ar[two heads]{r}{\begin{pmatrix}p & 0\end{pmatrix}}[marking]{\circ} \& C
                    \end{tikzcd}
                \end{center}
                As kernels and cokernels are preserved by direct sums, this pair is in fact a kernel-cokernel pair. The epimorphism is a deflation as it can be factored by the deflations:
                \begin{center}
                    \begin{tikzcd}[ampersand replacement=\&]
                        B\oplus D \ar[two heads]{r}{\begin{pmatrix}1 & 0\end{pmatrix}}[marking]{\circ} \& B \ar[two heads]{r}{p}[marking]{\circ} \& C
                    \end{tikzcd}
                \end{center}
                Thus it is seen that $(i\oplus id_D, p\oplus 0)$ is a conflation, and dually $(i\oplus 0, p\oplus id_D)$ is also a conflation. To finish off the proof it is seen that the morphism $i\oplus i'$ factors as $i\oplus id_{A'}\circ id_A\oplus i'$ asserting that it is an inflation by (QE1). By the dual argument we then get that the direct sum of conflations is a conflation.
            \end{proof}

            \begin{definition}
                A square is bicartesian if it is both a pull-back and a push-out.
                \begin{tikzcd}
                    A \ar{r}{} \ar{d}{} \ar[phantom]{rd}[very near start]{\ulcorner}[very near end]{\lrcorner} & B \ar{d}{} \\
                    C \ar{r}{} & D
                \end{tikzcd}
            \end{definition}

            \begin{prop} 
                The following statements are equivalent:
                \begin{enumerate}
                    \item The square (1) is a push-out.
                    \item The sequence (2) is a conflation.
                    \item The square (1) is bicartesian.
                    \item The square (1) is a part of the commutative diagram (3)
                \end{enumerate}
                \begin{center}
                    (1)
                    \begin{tikzcd}
                        A \ar[tail]{r}{i}[marking]{\circ} \ar{d}{f} & B \ar{d}{g} \\
                        C \ar[tail]{r}{j}[marking]{\circ} & D
                    \end{tikzcd}
                    \space (2)
                    \begin{tikzcd}[ampersand replacement=\&]
                        A \ar[tail]{r}{\begin{pmatrix} 
                            i \\ -f 
                        \end{pmatrix}}[marking]{\circ} \& B\oplus C \ar[two heads]{r}{\begin{pmatrix}
                            g & j
                        \end{pmatrix}}[marking]{\circ} \& D 
                    \end{tikzcd}
                    \space (3)
                    \begin{tikzcd}
                        A \ar[tail]{r}{i}[marking]{\circ} \ar{d}{f} & B \ar[two heads]{r}{p}[marking]{\circ} \ar{d}{g} & E \ar[equal]{d} \\
                        C \ar[tail]{r}{j}[marking]{\circ} & D \ar[two heads]{r}{q}[marking]{\circ} & E
                    \end{tikzcd}
                \end{center}
            \end{prop}

            Before the proof for this proposition will be presented a useful lemma will be proved first.

            \begin{lemma}
                Assume that there is a commutative square (1) and an associatied sequence (2). (1) is a push-out square if and only if $\begin{pmatrix}
                    p & q
                \end{pmatrix}$ is the cokernel of the morphism $\begin{pmatrix}
                    i \\ -j
                \end{pmatrix}$
                \begin{center}
                    (1)
                    \begin{tikzcd}
                        A \ar{r}{i} \ar{d}{j} & B \ar{d}{p} \\
                        C \ar{r}{q} & D
                    \end{tikzcd}
                    \space (2)
                    \begin{tikzcd}[ampersand replacement=\&]
                        A \ar{r}{\begin{pmatrix}i \\ -j\end{pmatrix}} \& B\oplus C \ar{r}{\begin{pmatrix}p & q\end{pmatrix}} \& D
                    \end{tikzcd}
                \end{center}
            \end{lemma}

            \begin{proof}
                For any test object $T$ and two maps $t_1:B\rightarrow T$ and $t_2:C\rightarrow T$, we can construct the diagrams for the universal properties of both the cokernel and the push-out. It is seen that these diagrams are equivalent, proving the lemma.
                \begin{center}
                    \begin{tikzcd}[ampersand replacement=\&]
                        A \ar{rd}{0} \ar{r}{\begin{pmatrix}i \\ -f\end{pmatrix}} \& B\oplus C \ar[two heads]{d}{\begin{pmatrix}g & j\end{pmatrix}} \ar[bend left]{rd}{\begin{pmatrix}t_1 & t_2\end{pmatrix}} \\
                        \& D \ar[dashed]{r}{t'}\& T
                    \end{tikzcd}
                    $\Leftrightarrow$
                    \begin{tikzcd}[ampersand replacement=\&]
                        A \ar[phantom]{rd}[very near end]{\lrcorner} \ar{r}{i} \ar{d}{f} \& B \ar{d}{g} \ar[bend left]{rdd}{t_1} \\
                        C \ar{r}{j} \ar[bend right]{rrd}{t_2} \& D \ar[dashed]{rd}{t'} \\
                        \&\& T
                    \end{tikzcd}
                \end{center}
            \end{proof}

            \begin{corollary}
                For the same diagrams (1) and (2) as above the dual statement is also true. (1) is a pull-back square if and only if $\begin{pmatrix}
                    i \\ -j
                \end{pmatrix}$ is kernel of the morphism $\begin{pmatrix}
                    p & q
                \end{pmatrix}$. Thus we have that (1) is bicartesian (i.e. both a pull-back and a push-out) if and only if the morphisms make a kernel-cokernel pair.
            \end{corollary}

            \begin{proof}
                \textbf{of Proposition 3.3} 1. $\Rightarrow$ 2.: By the previous lemma we know that $\begin{pmatrix}
                    g & j
                \end{pmatrix}$ is the cokernel of $\begin{pmatrix}
                    i \\ -j
                \end{pmatrix}$. Thus proving that $\begin{pmatrix}
                    i \\ -j
                \end{pmatrix}$ is an inflation, will prove that the pair is a conflation. 
                
                Observe that the morphism $\begin{pmatrix}
                    i \\ -f
                \end{pmatrix}$ can be factored through the sequence. 
                \begin{center}
                    \begin{tikzcd}[ampersand replacement=\&]
                        A \ar[tail]{r}{\begin{pmatrix}
                            1 \\ 
                            0
                        \end{pmatrix}}[marking]{\circ} \& A\oplus C \ar[tail, two heads]{r}{\begin{pmatrix}
                            1 & 0 \\
                            -f & 1
                        \end{pmatrix}}[below]{\simeq}[marking]{\circ} \& A\oplus C \ar[tail]{r}{\begin{pmatrix}
                            i & 0 \\
                            0 & 1
                        \end{pmatrix}}[marking]{\circ} \& B\oplus C
                    \end{tikzcd}
                \end{center}
                By corollary 3.2.1 the first map is an inflation, as the second map is an isomorphism it is also an inflation and the last map is the direct sum of two inflations. Thus the composite of all these maps surely is an inflation by (QE1), proving the first implication.

                2. $\Rightarrow$ 3.: This follows from corollary 3.4.1.
                
                3 $\Rightarrow$ 1.: This is by definition.

                1. $\Rightarrow$ 4.: Let $p$ be the cokernel of $i$, then we can form the diagram below.
                \begin{center}
                    \begin{tikzcd}
                        A \ar[phantom]{rd}[very near end]{\lrcorner} \ar[tail]{r}{i}[marking]{\circ} \ar{d}{f} & B \ar{d}{g} \ar[two heads, bend left]{rdd}{p}[marking]{\circ} \\
                        C \ar[tail]{r}{j}[marking]{\circ} \ar[bend right]{rrd}{0} & D \ar[dashed, two heads]{rd}{p'} \\
                        & & T
                    \end{tikzcd}
                \end{center}
                $p'$ is an epimorphism as $p=p'g$ is epi. To prove that $p'$ is the cokernel of $j$ we let $T'$ be another test object with a map $t':D\rightarrow T'$ such that $0 = t'j$. By doing some diagram chases we have that $0=t'jf=t'gi$, thus by the universal property of $p$ the morphism $t'g$ factors through $T$ such that $t'g=tp$ for some unique $t$. By rearranging we have that $t'g=tp'g=tp$, and $t'j=tp'j=0$, thus since $t'$ is the unique morphism satisfying this equation we demand that $t'=tp'$. $t$ is also unique, for if there exist another map $h$ such that $tp'=hp'$, then $t=h$ as $p'$ is epic. The unique existence proves the universal property, and $p'$ is the cokernel of $j$.
                \begin{center}
                    \begin{tikzcd}
                        A \ar[phantom]{rd}[very near end]{\lrcorner} \ar[tail]{r}{i}[marking]{\circ} \ar{d}{f} & B \ar{d}{g} \ar[two heads]{r}{p}[marking]{\circ} & T \ar[equal]{d} \\
                        C \ar[tail]{r}{j}[marking]{\circ} \ar[bend right]{rrd}{0} & D \ar{rd}{t'} \ar[dashed, two heads]{r}{p'} & T \ar[dashed]{d}{t} \\
                        & & T'
                    \end{tikzcd}
                \end{center}

                4. $\Rightarrow$ 2.: We start by taking the pullback of $p$ and $p'$ using $(QE2^{op})$, and determines the diagram with the dual statement of the last implication.
                \begin{center}
                    \begin{tikzcd}
                        & A \ar[equal]{r} \ar[tail]{d}{i'}[marking]{\circ} & A \ar[tail]{d}{i}[marking]{\circ} \\
                        C \ar[equal]{d} \ar[tail]{r}{j'}[marking]{\circ} & E \ar[two heads]{r}{q'}[marking]{\circ} \ar[two heads]{d}{q}[marking]{\circ} \ar[phantom]{dr}[very near start]{\ulcorner} & B \ar[dashed]{ld}[above]{g} \ar[two heads]{d}{p}[marking]{\circ} \\
                        C \ar[tail]{r}{j}[marking]{\circ} & D \ar[two heads]{r}{p'}[marking]{\circ} & T
                    \end{tikzcd}
                    \space $\Rightarrow$
                    \begin{tikzcd}
                        B \ar[dashed]{rd}{k} \ar[bend left, equal]{rrd} \ar[bend right]{ddr}{g} \\
                        & E \ar[two heads]{r}{q'}[marking]{\circ} \ar[two heads]{d}{q}[marking]{\circ} & B \ar[two heads]{d}{p}[marking]{\circ} \\
                        & D \ar[two heads]{r}{p'}[marking]{\circ} & T
                    \end{tikzcd}
                \end{center}
                From these diagrams we can deduce that $q'$ is a split-epimorphism. The composite $q'(id_E-kq')=q'-q'kq'=q'-q'=0$ as q' is split-epi, so $(id_E-kq')$ factors over $j'$ as in the following diagram.
                \begin{center}
                    \begin{tikzcd}
                        E \ar[dashed]{rd}{l} \ar[bend left]{rrd}{id_E-kq'} \\
                        & C \ar[tail]{r}{j'}[marking]{\circ} \ar{rd}{0} & E \ar[two heads]{d}{q'}[marking]{\circ} \\
                        & & B
                    \end{tikzcd}
                \end{center}
                From these diagrams we can extract three different equations:
                \begin{itemize}
                    \item $0=k-k=k-kq'k=(id_E-kq')k=j'lk \implies lk=0$ as $j'$ is monic
                    \item $j'lj'=(id_E-kq')j'=j'\implies lj'=id_C$ as $j'$ is monic
                    \item $jli'=(qj')li'=q(id_E-kq')i'=-(qk)(q'i')=-gi=-jf \implies li'=-f$ as $j$ is monic
                \end{itemize}
                The morphisms $\begin{pmatrix}
                    k & j'
                \end{pmatrix}$ and $\begin{pmatrix}
                    q' \\ l
                \end{pmatrix}$ are inverses:
                \begin{itemize}
                    \item $\begin{pmatrix}
                        k & j'
                    \end{pmatrix}\begin{pmatrix}
                        q' \\ l
                    \end{pmatrix}=kq'+j'l=kq'+id_E-kq'=id_E$
                    \item $\begin{pmatrix}
                        q' \\ l
                    \end{pmatrix}\begin{pmatrix}
                        k & j'
                    \end{pmatrix}=\begin{pmatrix}
                        q'k & q'j' \\ lk & lj'
                    \end{pmatrix} = \begin{pmatrix}
                        id_B & 0 \\ 0 & id_C
                    \end{pmatrix}$
                \end{itemize}
                Thus we have an isomorphism of kernel-cokernel pairs $(id_A,,\begin{pmatrix}
                    q' \\ l
                \end{pmatrix}\begin{pmatrix}
                    k & j'
                \end{pmatrix})$, \\ from $(\begin{pmatrix}i \\ -f\end{pmatrix},\begin{pmatrix}f' & i'\end{pmatrix})$ to $(i',q)$. This proves 2.
            \end{proof}

            \begin{corollary}
                The pull-back of an inflation along a deflation is an inflation.
                \begin{center}
                    \begin{tikzcd}
                        A \ar[tail]{r}{i'}[marking]{\circ} \ar[two heads]{d}{e'}[marking]{\circ} \ar[phantom]{rd}[very near start]{\ulcorner} & B \ar[two heads]{d}{e}[marking]{\circ} \\
                        C \ar[tail]{r}{i}[marking]{\circ} & D
                    \end{tikzcd}
                \end{center}
            \end{corollary}

            \begin{proof}
                By (QE2) this pullback exists, as there is a deflation in the pullback. Extend the diagram by adding the deflation of the inflation in the following manner.
                \begin{center}
                    \begin{tikzcd}
                        & T \ar{d}{t} \ar[bend left]{rd}{0} \ar[bend right, dashed]{ld}{t'} \ar[bend left, dashed]{ddl}[above, near end]{t''} & \\
                        A \ar[tail]{r}{i'} \ar[two heads]{d}{e'}[marking]{\circ} \ar[phantom]{rd}[very near start]{\ulcorner} & B \ar[two heads]{d}{e}[marking]{\circ} \ar[two heads]{r}{pe}[marking]{\circ} & C \ar[equal]{d} \\
                        C \ar[tail]{r}{i}[marking]{\circ} & D \ar[two heads]{r}{p}[marking]{\circ} & C
                    \end{tikzcd}
                \end{center}
                $pe$ is a deflation by (QE1), and $i'$ is a mono as a limit of a mono is a mono. Our goal is to prove that $i'$ is the kernel of $pe$. Let T be a test object such that $pet=0$. Then we have that $te$ factorizes over $i$, such that we can apply the universal property of the pullback to factorize $te$ over $i'$. Uniqueness of $t'$ is achieved with $i'$ being monic. This proves that $(i',pe)$ is a conflation.
            \end{proof}

            \begin{theorem}
                \textbf{The obscure axiom.} Assume that $i:A\rightarrow B$ is a morphism with a cokernel. If there is a morphism $j:B\rightarrow C$ such that $ji$ is an inflation, then $i$ is an inflation.
            \end{theorem}

            \begin{proof}
                
            \end{proof}

            \begin{proof}
                Let $p:B\rightarrow D$ be the cokernel of $i$. We start the proof with forming the push-out of $i$ and $ji$.
                \begin{center}
                    \begin{tikzcd}
                        A \ar[phantom]{rd}[very near end]{\lrcorner} \ar[tail]{r}{ji}[marking]{\circ} \ar{d}{i} & C \ar{d} \\
                        B \ar[tail]{r}[marking]{\circ} & E
                    \end{tikzcd}
                \end{center}
                By proposition 3.3 we get that $\begin{pmatrix}
                    i \\ ji
                \end{pmatrix}$ is an inflation. $\begin{pmatrix}
                    i \\ 0
                \end{pmatrix}=\begin{pmatrix}
                    id_B & 0 \\ -j & id_C
                \end{pmatrix}\begin{pmatrix}
                    i \\ ji
                \end{pmatrix}$, this is an inflation by (QE1) as the 2x2 matrix is an isomorphism. Observer that the cokernel of this map is $\begin{pmatrix}
                    k & 0 \\ 0 & id_C
                \end{pmatrix}$. Our final trick will be to show that there is a pullback square, and then use (QE2) to say that k is a deflation.
                \begin{center}
                    \begin{tikzcd}[ampersand replacement=\&]
                        T \ar[bend left]{rrd}{t_1} \ar[bend right]{ddr}[below]{\begin{pmatrix}t_2 \\ 0\end{pmatrix}} \ar[dashed]{rd}{t} \\
                        \& B \ar{r}{k} \ar{d}{\begin{pmatrix}1 \\ 0\end{pmatrix}} \& D \ar{d}{\begin{pmatrix}1 \\ 0\end{pmatrix}} \\
                        \& B\oplus C \ar[two heads]{r}[below]{\begin{pmatrix}k & 0 \\ 0 & id_C\end{pmatrix}}[marking]{\circ} \& D\oplus C
                    \end{tikzcd}
                \end{center}
                Note that setting $t=t_2$ we get the universal property. This is well defined as $kt_2=t_1$ by assumption, thus $kt=t_1$. This is what we need to prove that the square is a pullback, proving the obscure axiom.
            \end{proof}

            \begin{remark}
                Write a bit about the dual of the obscure axiom.
            \end{remark}

            %Add the classical homological diagram lemmatas down here
            \begin{lemma}
                Let $(p,q)$ and $(p',q')$ be the conflations:
                \begin{itemize}
                    \item $(p,q)$: \begin{tikzcd}A \ar[tail]{r}{p} & B \ar[two heads]{r}{q} & C\end{tikzcd}
                    \item $(p',q')$: \begin{tikzcd}A' \ar[tail]{r}{p'} & B' \ar[two heads]{r}{q'} & C'\end{tikzcd}
                \end{itemize}
                A morphism of the conflations $(f,g,h):(p,q)\rightarrow (p',q')$ factors through the conflation 
                \begin{tikzcd}
                    A \ar[tail]{r} & D \ar[two heads]{r} & C'
                \end{tikzcd} such that we have the following diagram where $g = g_2g_1$.
                \begin{center}
                    \begin{tikzcd}
                        A \ar[phantom]{rd}[very near start]{\ulcorner}[very near end]{\lrcorner} \ar[tail]{r}{p}[marking]{\circ} \ar{d}{f} & B \ar[two heads]{r}{q}[marking]{\circ} \ar{d}{g_1} & C \ar[equal]{d} \\
                        A' \ar[tail]{r}[marking]{\circ}  \ar[equal]{d} & D \ar[phantom]{rd}[very near start]{\ulcorner}[very near end]{\lrcorner} \ar[two heads]{r}[marking]{\circ} \ar{d}{g_2} & C \ar{d}{h} \\
                        A' \ar[tail]{r}{p'}[marking]{\circ} & B' \ar[two heads]{r}{q'}[marking]{\circ} & C'
                    \end{tikzcd}
                \end{center}
            \end{lemma}

            \begin{proof}
                Observe that the upper part of the diagram is made by taking a push-out of $p$ and $f$, where the right part is gained from proposition 3.3. Next we will combine the upper part with the lower part using the push-out property.
                \begin{center}
                    \begin{tikzcd}
                        A \ar[tail]{r}{p}[marking]{\circ} \ar{d}{f} \ar[phantom]{rd}[very near end]{\lrcorner} & B \ar{d}{g_1} \ar[bend left]{rdd}{g} \\
                        A' \ar[tail]{r}[marking]{\circ} \ar[tail, bend right]{rrd}{p'}[marking]{\circ} & D \ar[dashed]{rd}{g_2} \\
                        & & B'
                    \end{tikzcd} $\implies$
                    \begin{tikzcd}
                        A \ar[tail]{r}{p}[marking]{\circ} \ar{d}{f} \ar[phantom]{rd}[very near start]{\urcorner}[very near end]{\lrcorner} & B \ar{d}{g_1} \ar[two heads]{r}{q}[marking]{\circ} & C \ar[equal]{d} \\
                        A' \ar[equal]{d} \ar[tail]{r}{a}[marking]{\circ} & D \ar{d}{g_2} \ar[two heads]{r}{c}[marking]{\circ} & C \ar{d}{h} \\
                        A' \ar[tail]{r}{p'}[marking]{\circ} & B' \ar[two heads]{r}{q'}[marking]{\circ} & C'
                    \end{tikzcd}
                \end{center}
                It remains to show that the lower right square is commutative, then we can apply the dual of proposition 3.3 to see that the square is bicartesian. Note that $q=cg_1$ by prop 3.3 thus $q'g_2g_1=q'g=hq=hcg_1$. By uniqueness of the push-out property we have that $hc=q'g_2$.
            \end{proof}

            \begin{corollary}
                \textbf{The short five lemma}. Suppose that there is a morphism of conflations $(f,g,h)$ as above. If $f$ and $h$ are isomorphisms, then $g$ is an isomorphism.
            \end{corollary}

            \begin{proof}
                Since $f$ is an isomorphism it is at least an inflation, thus $g_1$ is an inflation by (QE2). As colimits preserve epis, $g_1$ is also an epimorphism. by corollary 3.1.1 we know that $g_1$ is an iso, and dually that $g_2$ is an iso. Since isomorphisms are closed under composition we have that $g$ is an isomorphism. 
                \begin{center}
                    \begin{tikzcd}
                        A \ar[phantom]{rd}[very near start]{\ulcorner}[very near end]{\lrcorner} \ar[tail]{r}{p}[marking]{\circ} \ar[tail]{d}{f}[rotate=90, above]{\simeq} & B \ar[two heads]{r}{q}[marking]{\circ} \ar[tail]{d}{g_1}[rotate=90, above]{\simeq}& C \ar[equal]{d} \\
                        A' \ar[tail]{r}[marking]{\circ}  \ar[equal]{d} & D \ar[phantom]{rd}[very near start]{\ulcorner}[very near end]{\lrcorner} \ar[two heads]{r}[marking]{\circ} \ar[two heads]{d}{g_2}[rotate=90, above]{\simeq} & C \ar[two heads]{d}{h}[rotate=90, above]{\simeq} \\
                        A' \ar[tail]{r}{p'}[marking]{\circ} & B' \ar[two heads]{r}{q'}[marking]{\circ} & C'
                    \end{tikzcd}
                \end{center}
            \end{proof}

            \begin{lemma}
                \textbf{Noethers isomorphism lemma}. Suppose there is a diagram with rows as conflations and the first column as a conflation. Then the final column is also a conflation.
                \begin{center}
                    \begin{tikzcd}
                        A \ar[equal]{d} \ar[tail]{r}[marking]{\circ} & B \ar[phantom]{rd}[very near start]{\ulcorner}[very near end]{\lrcorner} \ar[two heads]{r}[marking]{\circ} \ar[tail]{d}[marking]{\circ} & X \ar[tail, dashed]{d}[marking]{\circ} \\
                        A \ar[tail]{r}[marking]{\circ} & C \ar[two heads]{d}[marking]{\circ} \ar[two heads]{r}[marking]{\circ} & Y \ar[two heads, dashed]{d}[marking]{\circ} \\
                        & Z \ar[equal]{r} & Z
                    \end{tikzcd}
                \end{center}
            \end{lemma}

            \begin{proof}
                Assume that we only have the solid part of the diagram above. By the universal property of cokernels, the upper dashed map exists, and by the dual of proposition 3.3 the square is bicartesian. We can state that the upper dashed map is an inflation, and since the square is a push-out we get that the lower dashed map exists such that the final column is a conflation by proposition 3.3.
            \end{proof}

        \subsection{The Stable Frobenis Category}

            \begin{definition}
                Let $(\mathcal{A},\mathcal{E})$ and $(\mathcal{A}',\mathcal{E}')$ be two exact categories. A functor $F:(\mathcal{A},\mathcal{E})\rightarrow (\mathcal{A}',\mathcal{E}')$ is called exact if it is additive and $F(\mathcal{E})\subseteq \mathcal{E}'$. That is to say that conflations get mapped to conflations. One speaks of a reflective exact functor for whenever the pair $(Fp,Fq)$ is a conflation, then $(p,q)$ is a conflation.
            \end{definition}

            \begin{definition}
                Let $(\mathcal{A},\mathcal{E})$ be an exact category. An object $P:\mathcal{A}$ is called projective if $\mathcal{A}(P,\_):(\mathcal{A},\mathcal{E})\rightarrow \textbf{Ab}$ is an exact functor. Objects $I:\mathcal{A}$ are called injective whenever $\mathcal{A}(\_,I):\mathcal{A}^{op}\rightarrow\textbf{Ab}$ is an exact functor.
            \end{definition}

            \begin{remark}
                In the case of exact functors $F:(\mathcal{A},\mathcal{E})\rightarrow\textbf{Ab}$, one generally speaks of a functor which maps conflations to short exact sequences.
            \end{remark}

            \begin{remark}
                The hom-functor is something that is called left-exact. That means that conflations gets mapped to sequences which is only exact in the two first terms.
            \end{remark}

            \begin{prop}
                Let $(\mathcal{A},\mathcal{E})$ be an exact ccategory. $P:\mathcal{A}$ is projective if and only if for every deflation $q:A\rightarrow B$ and morphism $f:P\rightarrow B$  there is a morphism $f':P\rightarrow A$ rendering the diagram below commutative.
                \begin{center}
                    \begin{tikzcd}
                        & P \ar[dashed]{ld}[above]{f'} \ar{d}{f} \\
                        A \ar[two heads]{r}[below]{q}[marking]{\circ} & B
                    \end{tikzcd}
                \end{center}
            \end{prop}

            \begin{proof}
                Suppose that $P$ is projective, then $\mathcal{A}(P,\_)$ is an exact functor. Let $(p:A\rightarrow B,q:B\rightarrow C)$ be a conflation, then there is a short exact sequence.
                \begin{center}
                    \begin{tikzcd}
                        0 \ar{r}{0} & \mathcal{A}(P,A) \ar[tail]{r}{p_*}& \mathcal{A}(P,B) \ar[two heads]{r}{q_*} & \mathcal{A}(P,C) \ar{r}{0} & 0
                    \end{tikzcd}
                \end{center}
                Pick $f:\mathcal{A}(P,C)$, since $q_*$ is a surjection there exists an $f':\mathcal{A}(P,B)$ such that $pf'=f$.
                Now, suppose that $P$ has the diagram property and that $(p:A\rightarrow B,q:B\rightarrow C)$ is a conflation. Then there is an exact sequence in $\textbf{Ab}$ by $\mathcal{A}(P,\_)$.
                \begin{center}
                    \begin{tikzcd}
                        0 \ar{r}{0} & \mathcal{A}(P,A) \ar[tail]{r}{p_*}& \mathcal{A}(P,B) \ar{r}{q_*} & \mathcal{A}(P,C)
                    \end{tikzcd}
                \end{center}
                To see that $q_*$ is a surjection and, let $f:P\rightarrow C$. As $q$ is a deflation there exists an $f':P\rightarrow B$ such that $q_*(f')=f$. Thus the sequence above is short exact and $P$ is projective. 
            \end{proof}

            \begin{corollary}
                Let $P$ be projective, then if $q:A\rightarrow P$ is a deflation, it is split-epi.
            \end{corollary}

            \begin{corollary}
                Two objects $P$ and $Q$ are projective if and only if $P\oplus Q$ is projective.
            \end{corollary}

            \begin{definition}
                A category $(\mathcal{A}, \mathcal{E})$ has enough projective objects if for any object $A:\mathcal{A}$ there is a projective object $P$ along with a deflation $q:P\rightarrow A$.
            \end{definition}

            \begin{definition}
                An exact category is called a Frobenius category if it has enough projective and injective objects and the class of projective objects coincide with the injective objects.
            \end{definition}

            From the Frobenius categories, one can construct the stable Frobenius category where every morphism factoring through an injective will be identified. These stable categories will give rise to a class of categories called the algebraic triangulated categories. In order to construct the stable Frobenius category we will use the construction of the quotient category.

            \begin{definition}
                A congruence relation $\sim$ on a category $\mathcal{C}$ is a relation on the homsets, such that:
                \begin{enumerate}
                    \item $\forall A,B:\mathcal{C}$ the relation $\sim_{A,B}$ is an equivalence relation.
                    \item Given that $f,f':A\rightarrow B$ is related ($f\sim f'$) and morphisms $g:A'\rightarrow A$ and $h:B\rightarrow B'$, then $hgf\sim hf'g$.
                \end{enumerate}
            \end{definition}

            \begin{prop}
                Let $\mathcal{C}$ be a category and $\sim$ be a congruence relation. Then there is a universal category $\mathcal{C}/\sim$ together with a functor $q:\mathcal{C}\rightarrow \mathcal{C}/\sim$ such that morphisms $f,g:A\rightarrow B$ are identified if $f\sim g$. Universality means that if there is a functor $H:\mathcal{C}\rightarrow \mathcal{D}$ such that $Hf=Hg$ for any $f,g$ such that $f\sim g$, then H factors uniquely through $\mathcal{C}/\sim$.
            \end{prop}

            \begin{proof}
                Define the category $\mathcal{C}/\sim$ to have the same objects as $\mathcal{C}$, and define $\mathcal{C}/\sim (A,B)=\mathcal{C}(A,B)/\sim_{a,b}$. This definition is well defined as $\sim$ is a congruence relation. A sketch of this proof can be found in \cite{Mac71}.
            \end{proof}

            \begin{remark}
                Any functor gives rise to a congruence relation. That is, if $F:\mathcal{C}\rightarrow \mathcal{D}$ is a functor, then there is a congruence relation $\sim$ defined as follows: $\forall A,B:\mathcal{C}$ and $f,g:A\rightarrow B$, we define $f\sim_{A,B}g \iff Ff=Fg$. This is a congruence as equlity within $\mathcal{D}$ gives rise to an equivalence relation, and functoriality gives the congruence.
            \end{remark}

            \begin{remark}
                For any relation $\sim$ the universal category $\mathcal{C}/\sim$ exists. As in the case for the Verdier quotient, $\mathcal{C}/\sim$ is the same as the quotient category of the smallest congruence relation having the same relations as $\sim$.
            \end{remark}

            If $\mathcal{A}$ is an additive category, the quotient categories which respects the additive structures are of interest. That is to say that the functor $q:\mathcal{A}\rightarrow \mathcal{C}/\sim$ is additive and the equivalence relation $\sim$ should respect the additive structure. Then a quotient category is additive if $f\sim f'$ and $g\sim g'$, then $f+g\sim f'+g'$. This leads to the following definition. 

            \begin{definition}
                Let $\mathcal{A}$ be an additive category. $\mathcal{I}$ is an ideal of $\mathcal{A}$ if:
                \begin{enumerate}
                    \item (subgroup) for every abelian group $\mathcal{A}(A,B)$ there is a subgroup $\mathcal{I}(A,B)\subseteq\mathcal{A}(A,B)$.
                    \item (absorption) For every $g:A'\rightarrow A$, $h:B\rightarrow B'$ and $f:\mathcal{I}(A,B)$ we have that $hfg:\mathcal{I}(A',B')$
                \end{enumerate}
                This is equivalent of saying that the equivalence relation gained from $\mathcal{I}(A,B)$ ($f\sim g \iff f-g:\mathcal{I}(A,B)$) is a congruence relation.
            \end{definition}

            \begin{corollary}
                Let $\mathcal{A}$ be an additive category and $\mathcal{I}$ be an ideal of $\mathcal{A}$, then $\mathcal{A}/\mathcal{I}$ is an additive category.
            \end{corollary}

            Let $\mathcal{A}$ be a Frobenius category. Define the ideal $\mathcal{I}$ as the subgroups of every morphism factoring through injective objects.

            \begin{prop}
                For any Frobenius category $\mathcal{A}$ the ideal $\mathcal{I}$ is well defined and $\underline{\mathcal{A}}=\mathcal{A}/\mathcal{I}$ is the stable Frobenius category.
            \end{prop}

            \begin{proof}
                To prove this we have to show that $\mathcal{I}(A,B)$ is a subgroup for any $A,B:\mathcal{A}$, and that it is absorptive.
                First observe that $0:\mathcal{I}(A,B)$. Let $f,g:\mathcal{I}(A,B)$. Since $\mathcal{A}$ has enough injectives there exists an injective object with an inflation from $A$.
                \begin{center}
                    \begin{tikzcd}
                        & J_1 \ar{rd}{f_2}\\
                        A \ar{ru}{f_1} \ar{rd}{g_1} \ar[tail]{r}{i}[marking]{\circ} & I \ar[dashed]{u}{f_1'} \ar[dashed]{d}{g_1'} & B \\
                        & J_2 \ar{ru}{g_2}
                    \end{tikzcd}
                \end{center}
                $f-g = f_2 \circ f_1 - g_2 \circ g_1 = (f_2 \circ f_1' - g_2 \circ g_1') \circ i$. Thus $f-g$ factors through an injective, and $\mathcal{I}(A,B)$ is a subgroup.
                To see that it is absorptive is to see that if $f$ factors through an injective, then $gf$ factors through an injective as well.
            \end{proof}

            Objects in the stable Frobenius category is denoted as $\underline{X}$ and morphisms are denoted as $\underline{f}$. That is the functor $q:\mathcal{A}\rightarrow\underline{\mathcal{A}}$ is defined as $q(X)=\underline{X}$ and $q(f)=\underline{f}$. One important property is that in the stable Frobenius category, taking syzygies or cozysigies is a functor.

            \begin{definition}
                A syzygy of an object $X$, if it exists, is denoted $\Omega X$. The syzygy is the defined to be the kernel object of a deflation $p:P\rightarrow X$, where $P$ is projective. A cosyzygy, denoted as \upside{$\Omega$}$X$ is defined to be the cokernel of an inflation $i:X\rightarrow I$, where $I$ is injective.
            \end{definition}

            \begin{remark}
                Note that this choice is not necessarily unique up to isomorphism. Thus syzygies and cosyzygies are not in general functors.
            \end{remark}

            \begin{lemma}
                Let $\mathcal{A}$ be a Frobenius category and suppose that there are two conflations with injectives as below. Then $\underline{X}'\simeq\underline{X}''$.
                \begin{center}
                    \begin{tikzcd}
                        X \ar[tail]{r}{i}[marking]{\circ} & I \ar[tail]{r}{i'}[marking]{\circ} & X' \\
                        X \ar[tail]{r}{j}[marking]{\circ} & J \ar[tail]{r}{j'}[marking]{\circ} & X'' 
                    \end{tikzcd}
                \end{center}
            \end{lemma}

            \begin{proof}
                Observe that with the diagram we can find morphisms as $I$ and $J$ are injective.
                \begin{center}
                    \begin{tikzcd}
                        X \ar{dr}{j} \ar[equal]{d} \ar[tail]{r}{i}[marking]{\circ} & I \ar[dashed]{d}{f} \\
                        X \ar[tail]{r}{j}[marking]{\circ} & J
                    \end{tikzcd}
                \end{center}
                By iterating this we get the following commutative diagram in $\mathcal{A}$.
                \begin{center}
                    \begin{tikzcd}
                        X \ar[equal]{d} \ar[tail]{r}{i}[marking]{\circ} & I \ar[two heads]{r}{i'}[marking]{\circ} \ar{d}{f} & X' \ar[dashed]{d}{g} \\
                        X \ar[equal]{d} \ar[tail]{r}{j}[marking]{\circ} & J \ar[two heads]{r}{j'}[marking]{\circ} \ar{d}{f'} & X'' \ar[dashed]{d}{g'} \\
                        X \ar[tail]{r}{i}[marking]{\circ} & I \ar[two heads]{r}{i'}[marking]{\circ} & X'
                    \end{tikzcd}
                \end{center}
                By chasing the diagram we get that $i-f'fi=(id_I-f'f)\circ i = 0$, this means that $(f'f-id_I)$ factors through $X'$, i.e. there exists $h:X'\rightarrow I$ and $f'f = hi'+id_I$. We also have that $g'gi' = i'f'f = i'(hi' +id_I) = i'hi' + i' = (i'h+id_{X'})i'$. As $i'$ is an epi we get that $g'g = i'h + id_{X'} \implies \underline{g'g}=id_{\underline{X}'}$ as $i'h$ factors through $I$. $\underline{gg'}=id_{\underline{X}''}$ is dual.
            \end{proof}

            \begin{corollary}
                Cosyzygy is a well defined functor \rotatebox[origin=c]{180}{$\Omega$}$:\underline{\mathcal{A}}\rightarrow\underline{\mathcal{A}}$
            \end{corollary}

            \begin{proof}
                Let $f:X\rightarrow Y$ be a morphism in $\mathcal{A}$. Then we have the following diagrams representing the different choices of syzygies.
                \begin{center}
                    \begin{tikzcd}
                        X \ar{d}{f} \ar{r}{i} & I \ar{r}{p} \ar[dashed]{d} & \rotatebox[origin=c]{180}{$\Omega$}X \ar[dashed]{d}{\rotatebox[origin=c]{180}{$\Omega$}f} \\
                        Y \ar{r}{j} & J \ar{r}{q} & \rotatebox[origin=c]{180}{$\Omega$}Y
                    \end{tikzcd}
                    \begin{tikzcd}
                        X \ar{d}{f} \ar{r}{i'} & I' \ar{r}{p'} \ar[dashed]{d} & \rotatebox[origin=c]{180}{$\Omega$}'X \ar[dashed]{d}{\rotatebox[origin=c]{180}{$\Omega$}'f} \\
                        Y \ar{r}{j'} & J' \ar{r}{q'} & \rotatebox[origin=c]{180}{$\Omega$}'Y
                    \end{tikzcd}
                \end{center}
                By the previous proof we know there are maps between the diagrams making an almost commutative diagram where all the outer squares commute.
                \begin{center}
                    \begin{tikzcd}[row sep=tiny]
                        X \ar{dddd}{f} \ar[equal]{rd}[above, pos=0.8]{\alpha (X)} \ar[tail]{rr}[marking]{\circ}[near start]{i} & & I \ar{rd}[above, pos=0.8]{\beta (X)} \ar[two heads]{rr}[marking]{\circ}[near start]{p} \ar[dashed]{dddd}{If} & &\rotatebox[origin=c]{180}{$\Omega$}X \ar{rd}[above, pos=0.8]{\gamma (X)} \ar[dashed]{dddd}{\rotatebox[origin=c]{180}{$\Omega$}f} \ar[dashed, squiggly]{lddddd}{\chi}\\
                        \textcolor{white}{.} & X \ar[tail]{rr}[marking]{\circ}[near start]{i'} \ar{dddd}{f} && I' \ar[dashed]{dddd}{I'f} \ar[two heads]{rr}[marking]{\circ}[near start]{p'} && \rotatebox[origin=c]{180}{$\Omega$}'X \ar[dashed]{dddd}{\rotatebox[origin=c]{180}{$\Omega$}'f}\\
                        \textcolor{white}{.} \\
                        \textcolor{white}{.} \\
                        Y \ar[equal]{rd}[above, pos=0.9]{\alpha (Y)} \ar[tail]{rr}[marking]{\circ}[near start]{j} & & J \ar{rd}[above, pos=0.8]{\beta (Y)} \ar[two heads]{rr}[marking]{\circ}[near start]{q} & & \rotatebox[origin=c]{180}{$\Omega$}Y \ar{rd}[above, pos=0.8]{\gamma (Y)} \\
                        & Y \ar[tail]{rr}[marking]{\circ}[near start]{j'} && J' \ar[two heads]{rr}[marking]{\circ}[near start]{q'} && \rotatebox[origin=c]{180}{$\Omega$}'Y
                    \end{tikzcd}
                \end{center}
                To see that the definition of the cosyzygy is well defined is to show that the inner squares is commutative in the quotient category.
                
                Observe that the inner square commutes by definition, and that the central inner square commutes in the quotient as every morphism gets related to $0$. Thus it remains to show that \underline{$\gamma (Y)\circ$\rotatebox[origin=c]{180}{$\Omega$}$f$}$=$\underline{\rotatebox[origin=c]{180}{$\Omega$}$'f\circ\gamma (X)$}, which is the same as to say that $\gamma (Y)\circ$\rotatebox[origin=c]{180}{$\Omega$}$f-$\rotatebox[origin=c]{180}{$\Omega$}$'f\circ\gamma (X)$ factors over an injective.

                By doing a diagram chase in the left cube one may find the following equation $(I'f\circ \beta(X)-\beta (Y)\circ If)i=0$. This means that the map $I'f\circ \beta(X)-\beta (Y)\circ If$ factors through the cokernel of $i$ as $\chi p$. By chasing the right cube we get the following equation $q'\chi p = (\gamma (Y)\circ$\rotatebox[origin=c]{180}{$\Omega$}$f-$\rotatebox[origin=c]{180}{$\Omega$}$'f\circ\gamma (X))p$, thus $q'\chi =$$\gamma (Y)\circ$\rotatebox[origin=c]{180}{$\Omega$}$f-$\rotatebox[origin=c]{180}{$\Omega$}$'f\circ\gamma (X)$.
            \end{proof}

            \begin{corollary}
                Cosyzygy \upside{$\Omega$} is an autoequivalence with syzygy $\Omega$ as inverse.
            \end{corollary}

            \begin{proof}
                The goal is to show that there is a natural isomorphisms \underline{$\Omega$\upside{$\Omega$}}$\simeq Id_{\underline{\mathcal{A}}}$ and \underline{\upside{$\Omega$}$\Omega$}$\simeq Id_{\underline{\mathcal{A}}}$. Note that since these are inverse operations we have that taking syzygy then cosyzygy is the same as taking cosyzygy then syzygy in $\mathcal{A}^{op}$.
                Let $X:\mathcal{A}$, then we want to show that the following diagram gives a natural isomorphism at the rightmost arrow in $\underline{\mathcal{A}}$.
                \begin{center}
                    \begin{tikzcd}
                        \Omega X \ar{r} \ar[equal]{d} & P \ar[dashed]{d} \ar{r} & X \ar[dashed]{d} \\
                        \Omega X \ar{r} & I \ar{r} & \upside{$\Omega$}\Omega X
                    \end{tikzcd}
                \end{center}
                Observe that this case is identical as the one proved previous. Thus there is a natural isomorphism from $X$ to \upside{$\Omega$}$\Omega$X.
            \end{proof}

            \begin{remark}
                A subtle, but important point is that the category $\mathcal{A}$ has enough projectives and injectives. This enables us to find the syzygies and cozysigies. It is also important that the projectives are the same as the injectives for this construction to give the isomorphisms as well.
            \end{remark}

            The triangulation of $\underline{\mathcal{A}}$ will be defined as the set of icandidate triangles in $\underline{\mathcal{A}}$ called standard triangles. Let $\underline{x}:\underline{X}\rightarrow\underline{Y}$ be a morphism, then by (QE2) there is a pushout in $\mathcal{A}$. Moreover, by Proposition 3.3 $(y,z)$ is a conflation.

            \begin{minipage}[c]{0.6\textwidth}
                \begin{center}
                    \begin{tikzcd}
                        X \ar{r}{x} \ar[phantom]{rd}[very near end]{\lrcorner} \ar[tail]{d}{i}[marking]{\circ} & Y \ar[tail]{d}{y}[marking]{\circ} \ar[bend left]{rdd}{0}\\
                        I(X) \ar[bend right, two heads]{rrd}{p}[marking]{\circ} \ar{r}{x'} & Z \ar[dashed, two heads]{dr}{z}[marking]{\circ} \\
                        & & \rotatebox[origin=c]{180}{$\Omega$}X
                    \end{tikzcd}
                \end{center}
            \end{minipage}
            \begin{minipage}[c]{0.4\textwidth}
                The set of standard triangles will be of the form $(\underline{X},\underline{Y},\underline{Z},\underline{x},\underline{y},\underline{z})$. Thus a triangle $(A,B,C,a,b,c):\Delta_{\underline{\mathcal{A}}}$ if and only if $(A,B,C,a,b,c)$ is isomorphic to a standard triangle.
            \end{minipage}

            \begin{prop}
                $\Delta_{\underline{\mathcal{A}}}$ is a triangulation of $\underline{\mathcal{A}}$.
            \end{prop}

            \begin{proof}
                I will give a sketch of how to prove this proposition. Most of the details will be omitted, and they can be found in \cite{Mat20} or \cite{May01}. In order to complete the proof we need to show the three defining axioms for a triangulated category, namely TR1, TR2 and TR4. Note that TR1 is satisfied by definition of $\Delta_{\underline{\mathcal{A}}}$. To see the final part of TR1 is to observe that the following diagram is a pushout.
                \begin{center}
                    \begin{tikzcd}[row sep=large]
                        X \ar[equal]{r}{} \ar[phantom]{rd}[very near end]{\lrcorner} \ar[tail]{d}{i}[marking]{\circ} & X \ar[tail]{d}{i}[marking]{\circ} \\
                        I(X) \ar[equal]{r}{} & I(X)
                    \end{tikzcd}
                \end{center}
                Thus $(\underline{X},\underline{X},\underline{0},id_{\underline{X}},\underline{0},\underline{0})$ is a standard triangle.

                % (TR3) Let $(\underline{A},\underline{B},\underline{C},\underline{a},\underline{b},\underline{c})$ and $(\underline{A}',\underline{B}',\underline{C}',\underline{a}',\underline{b}',\underline{c}')$ be two standard triangles. Suppose that there are two morphisms $\phi_A:A\rightarrow A'$ and $\phi_B:B\rightarrow B'$ suvh that $\underline{a'\phi_A}=\underline{\phi_B a}$. That means there is a morphism $\alpha:I(A)\rightarrow B'$ such that $a'\phi_A-\phi_B a = \alpha i(A)$. There is a map $I(\phi_A):I(A)\rightarrow I(A')$ induced by the map $i(A')\phi_A:A\rightarrow I(A')$, such that $I(\phi_A)i(A)=i(A')\phi_A$ and \rotatebox[origin=c]{180}{$\Omega$}$\phi_Ap(A)=p(A')I(\phi_A)$. By calculating $b'\phi_B\alpha=b'a'\phi_A-b'\alpha i(A) =\widetilde{a'}i(A')\phi_A-b'\alpha i(A)=\widetilde{a'}I(\phi_A)i(A)-b'\alpha i(A)=(\widetilde{a'}I(\phi_A)-b'\alpha)i(A)$. As $C$ is a pushout there exists a $\phi_C:C\rightarrow C'$ such that $\phi_C b = b'\phi_B$ and $\phi_C\widetilde{a}=\widetilde{a'}I(\phi_A)-b'\alpha$.
                
                % \begin{center}
                %     \begin{tikzcd}
                %         A \ar{r}{a} \ar[bend left]{rrr}{\phi_A} \ar{d}{i(A)} & B \ar{d}{b} \ar{r}{\phi_B} & B' \ar{d}{b'} & A' \ar{l}[above]{a'} \ar{d}{i(A')} \\
                %         I(A) \ar[dashed, squiggly]{rru}{\alpha} \ar[bend right]{rrr}{I(\phi_A)} \ar{r}{\widetilde{a}} & C \ar[dashed]{r}{\phi_C} & C' & I(A') \ar{l}{\widetilde{a'}} \\
                %     \end{tikzcd}
                % \end{center}
                % Now it remains to see that $c'\phi_C=$\rotatebox[origin=c]{180}{$\Omega$}$a\circ c$. By using the universal property of the pushout we can instead verify that $c'\phi_Cb=$\rotatebox[origin=c]{180}{$\Omega$}$a\circ cb$ and $c'\phi_C\widetilde{a}=$\rotatebox[origin=c]{180}{$\Omega$}$a\circ c\widetilde{a}$. We first start by seeing that
                % $c'\phi_Cb=c'b'\phi_B=0$ and \rotatebox[origin=c]{180}{$\Omega$}$a\circ cb = 0$, then $c'\phi_C\widetilde{a}=c'(\widetilde{a'}I(\phi_A)-b'\alpha)=c'\widetilde{a'}I(\phi_A)=p(A')I(\phi_A)=$\rotatebox[origin=c]{180}{$\Omega$}$a\circ p(A)=$\rotatebox[origin=c]{180}{$\Omega$}$a\circ c\widetilde{a}$.

                (TR2) Consider the standard triangle $(\underline{X},\underline{Y},\underline{Z},\underline{x},\underline{y},\underline{z})$, the goal is to show that there is a triangle $(\underline{Y},\underline{Z},$\underline{\rotatebox[origin=c]{180}{$\Omega$}$X$}$,\underline{x},\underline{y},-$\underline{\rotatebox[origin=c]{180}{$\Omega$}$x$}$)$. Let $I(X)$ and $I(Y)$ be injectives with inflations from $X$ and $Y$ respectively. Since $I(Y)$ is injective there is a unique map by the pushout property in (1).
                \begin{center}
                    (1)
                    \begin{tikzcd}
                        X \ar{r}{x} \ar[phantom]{rd}[very near end]{\lrcorner} \ar[tail]{d}{i(X)}[marking]{\circ} & Y \ar{d}{y} \ar[bend left]{rdd}{i(Y)} \\
                        I(X) \ar{r}{x'} \ar[bend right]{rrd}{f} & Z \ar[dashed]{rd}{g} \\
                        & & I(Y)
                    \end{tikzcd}
                    (2)
                    \begin{tikzcd}
                        X \ar{r}{x} \ar[phantom]{rd}[very near end]{\lrcorner} \ar{d}{i(X)} & Y \ar{d}{y} \ar{rd}{0} \\
                        I(X) \ar{rd}{f} \ar{r}{x'} & Z \ar{d}{g} \ar{r}{z} & \rotatebox[origin=c]{180}{$\Omega$}X \ar{d}{\rotatebox[origin=c]{180}{$\Omega$}x} \\
                        & I(Y) \ar{r}{p(Y)} & \rotatebox[origin=c]{180}{$\Omega$}Y
                    \end{tikzcd}
                \end{center}
                From (2) we are able to use the pushout to see that the lower right square commutes, that is the lower right square commute if $p(Y)fi(X)=$\rotatebox[origin=c]{180}{$\Omega$}$xzyz=0$. This is true as $p(Y)fi(X)=p(Y)gyx=p(Y)i(Y)x=0$ by (1). Note that since $z$ and $p(Y)$ is a deflation with equal kernels, Proposition 3.3 says that $(\begin{pmatrix}g \\ z\end{pmatrix},\begin{pmatrix}p(Y) & -\rotatebox[origin=c]{180}{$\Omega$}x\end{pmatrix})$ is a conflation. 

                We are able to find a commutative diagram and by Proposition 3.3 we are able to say that the upper left square is bicartesian.
                \begin{center}
                    \begin{tikzcd}[ampersand replacement=\&]
                        Y \ar[phantom]{rd}[very near start]{\ulcorner}[very near end]{\lrcorner} \ar[tail]{d}{i(Y)}[marking]{\circ} \ar[tail]{r}{y}[marking]{\circ} \& Z \ar[tail]{d}{\begin{pmatrix}g \\ z\end{pmatrix}}[marking]{\circ} \ar[two heads]{r}{z}[marking]{\circ} \& \rotatebox[origin=c]{180}{$\Omega$}X \ar[equal]{d} \\
                        I(Y) \ar[two heads]{d}{p(Y)}[marking]{\circ} \ar[tail]{r}[below]{\iota_1}[marking]{\circ} \& I(Y)\oplus \rotatebox[origin=c]{180}{$\Omega$}X \ar[two heads]{r}[below]{\pi_2}[marking]{\circ} \ar[two heads]{d}{\begin{pmatrix}p(Y) & -\rotatebox[origin=c]{180}{$\Omega$}x\end{pmatrix}}[marking]{\circ} \& \rotatebox[origin=c]{180}{$\Omega$}X. \\
                        \rotatebox[origin=c]{180}{$\Omega$}Y \ar[equal]{r}{} \& \rotatebox[origin=c]{180}{$\Omega$}Y
                    \end{tikzcd}
                \end{center}

                Thus $(\underline{Y},\underline{Z},$\underline{\rotatebox[origin=c]{180}{$\Omega$}$X$}$,\underline{y},\underline{z},-$\underline{\rotatebox[origin=c]{180}{$\Omega$}$x$}$)$ is a standard triangle.

                (TR4) Suppose that there are three standard triangles where $\nu\upsilon=\omega$.
                \begin{center}
                    \begin{tikzcd}
                        X \ar[phantom]{rd}[very near end]{\lrcorner} \ar{r}{\upsilon} \ar{d}{x} & Y \ar{d}{i} \\
                        I(X) \ar{r}{\bar{\upsilon}} \ar{d}{\bar{x}} & Z' \ar{d}{i'} \\
                        \upside{$\Omega$}X \ar[equal]{r} & \upside{$\Omega$}X
                    \end{tikzcd}
                    \begin{tikzcd}
                        Y \ar{r}{\nu} \ar[phantom]{rd}[very near end]{\lrcorner} \ar{d}{y} & Z \ar{d}{j} \\
                        I(Y) \ar{r}{\bar{\nu}} \ar{d}{\bar{y}} & X' \ar{d}{j'} \\
                        \upside{$\Omega$}Y \ar[equal]{r} & \upside{$\Omega$}Y
                    \end{tikzcd}
                    \begin{tikzcd}
                        X \ar[phantom]{rd}[very near end]{\lrcorner} \ar{r}{\omega} \ar{d}{x} & Z \ar{d}{k} \\
                        I(X) \ar{d}{\bar{x}} \ar{r}{\bar{\omega}} & Y' \ar{d}{k'} \\
                        \upside{$\Omega$}X \ar[equal]{r} & \upside{$\Omega$}X
                    \end{tikzcd}
                \end{center}
                By Noethers isomorphism lemma there is a conflation passing through on the right column and the middle square is bicartesian. $z'$ exists by the injectivity of $I(Y)$ and that $i$ is an inflation. $z'$ is an inflation as $y$ is an inflation, thus $\bar{z'}$ exists.
                \begin{center}
                    \begin{tikzcd}
                        Y \ar[equal]{d} \ar[tail]{r}{i}[marking]{\circ} & Z' \ar[phantom]{rd}[very near start]{\ulcorner}[very near end]{\lrcorner} \ar[tail]{d}{z'}[marking]{\circ} \ar[two heads]{r}{i'}[marking]{\circ} & \upside{$\Omega$}X \ar[tail, dashed]{d}{s}[marking]{\circ} \\
                        Y \ar[tail]{r}{y}[marking]{\circ} & I(Y) \ar[two heads]{r}{\bar{y}}[marking]{\circ} \ar[two heads]{d}{\bar{z'}}[marking]{\circ} & \upside{$\Omega$}Y \ar[two heads, dashed]{d}{r}[marking]{\circ} \\
                        & \upside{$\Omega$}Z' \ar[equal]{r} & \upside{$\Omega$}Z'
                    \end{tikzcd}
                \end{center}
                There is also a map $I_\upsilon:I(X)\rightarrow (Y)$ induced by the maps between $X$ and $I(Y)$. By using the following universal properties we can find the unique maps $f$ and $g$.
                \begin{center}
                    \begin{tikzcd}
                        X \ar[phantom]{rd}[very near end]{\lrcorner} \ar{r}{\upsilon} \ar[bend left]{rr}{\omega} \ar{d}{x} & Y \ar{d}{i} \ar{r}{\nu} & Z \ar{dd}{k} \\
                        I(X) \ar[bend right]{rrd}{\bar{\omega}} \ar{r}{\bar{\upsilon}} & Z' \ar[dashed]{rd}{f} \\
                        & & Y'
                    \end{tikzcd}
                    \begin{tikzcd}
                        X \ar{r}{\omega} \ar{d}{x} & Z \ar{d}{k} \ar[bend left]{rdd}{j} \\
                        I(X) \ar{r}{\bar{w}} \ar[bend right]{rd}{I_\upsilon} & Y' \ar[dashed]{rd}{g} \\
                        & I(Y) \ar[bend right]{r}{\bar{\nu}}& X'
                    \end{tikzcd}
                \end{center}
                These maps can be arranged in the following diagram. It can be seen that middle square is a pushout, by using the fact that the upper left square and the larger rectangles are pushouts.
                \begin{center}
                    \begin{tikzcd}
                        X \ar{r}{\upsilon} \ar{d}{x} \ar[phantom]{rd}[very near end]{\lrcorner} & Y \ar{r}{\nu} \ar{d}{i} & Z \ar{d}{k} \\
                        I(X) \ar{r}{\bar{\upsilon}} \ar{d}{\bar{x}} \ar{rd}{I_\upsilon} & Z' \ar{r}{f} \ar{d}{z'} & Y' \ar[bend left, phantom]{d}[very near start]{\lrcorner} \ar{d}{g} \\
                        \upside{$\Omega$}X \ar{rd}{s} & I(Y)\ar{d}{r} \ar{r}{\bar{\nu}} & X' \ar[bend left, phantom]{d}[very near start]{\lrcorner} \ar{d}{\upside{$\Omega$}i\circ j'} \\
                        & \upside{$\Omega$}Y \ar{r}{\upside{$\Omega$}i} & \upside{$\Omega$}Z'
                    \end{tikzcd}
                    \begin{tikzcd}
                        Z' \ar{r}{f} \ar[phantom]{rd}[very near end]{\lrcorner} \ar[tail]{d}{z'}[marking]{\circ} & Y' \ar[tail]{d}{g}[marking]{\circ} \\
                        I(Y) \ar{r}{\bar{\nu}} \ar[two heads]{d}{\bar{z'}}[marking]{\circ} & X' \ar[two heads]{d}{\upside{$\Omega$}i\circ j'}[marking]{\circ} \\
                        \upside{$\Omega$}Z' \ar[equal]{r} & \upside{$\Omega$}Z'
                    \end{tikzcd}
                \end{center}
                Thus $(\underline{Z'},\underline{Y'},\underline{X'},\underline{f},\underline{g},$\underline{\upside{$\Omega$}$i\circ j'$}$)$ is a triangle. 
            \end{proof}

            \begin{remark}
                A more detailed and different proof may be found in \cite{Hol12} or \cite{Mat20}.
            \end{remark}

            This construction of triangles givves a close relation to conflations. That is, if there is a conflation $(p:X\rightarrow Y,q:Y\rightarrow Z)$, then there is a triangle $(\underline{X},\underline{Y},\underline{Z},\underline{p},\underline{q},-\underline{r})$ constructed as follows. Let $P:\mathcal{A}$ be a projective object with a deflation $\bar{p}:P\rightarrow Y$. Then there exists a pullback (1), moreover the pullback square is bicartesian. By using TR2 one may find the triangle (2) as indicated in the diagram.
            \begin{center}
                (1)
                \begin{tikzcd}
                    \Omega Z \ar[phantom]{rd}[very near start]{\ulcorner}[very near end]{\lrcorner} \ar[two heads]{r}{\Omega r}[marking]{\circ} \ar[tail]{d}[marking]{\circ} & X \ar[tail]{d}{p}[marking]{\circ} \\
                    P \ar[two heads]{r}{\bar{p}}[marking]{\circ} \ar[two heads]{d}[marking]{\circ} & Y \ar[two heads]{d}{q}[marking]{\circ} \\
                    Z \ar[equal]{r}{} & Z 
                \end{tikzcd}
                (2)
                \begin{tikzcd}
                    \underline{X} \ar{r}{\underline{p}} & \underline{Y} \ar{r}{\underline{q}} & \underline{Z} \ar{r}{-\underline{r}} & \underline{\upside{$\Omega$}X}
                \end{tikzcd}
            \end{center}

            \begin{remark}
                For any morphism $f:A\rightarrow B$ in $\mathcal{A}$, there is an inflation $\begin{pmatrix}f \\ -i \end{pmatrix}:A\rightarrow B\oplus I$ which is in the same equivalence class as $f$. Thus $\underline{f}=\underline{\begin{pmatrix}f \\ -i \end{pmatrix}}$, and any morphism in $\underline{\mathcal{A}}$ can be obtained from an inflation in $\mathcal{A}$.
            \end{remark}

        \subsection{Self-injective Algebras}

            The first example of a triangulated category is going to be derived from finite dimensional artin algebras. More specifically, let $\Lambda$ be a self-injective finite dimensional artin $R$-algebra; that is $_{\Lambda}\Lambda$ is injective as left $\Lambda$-module, then the finitely generated projective objects coincide with the finitely generated injective objects.

            \begin{prop}
                If $\Lambda$ is a self-injective finite dimensional artin $R$-algebra, then $mod_{\Lambda}$ is a Frobenius category.
            \end{prop}

            To prove this statement we will need the following propositions.

            \begin{lemma}
                The category $mod_{\Lambda}$ has enough projectives
            \end{lemma}

            \begin{proof}
                Let $A:mod_{\Lambda}$, then $A$ is finitely generated. This means there exists an epimorphism $p:R^n\rightarrow A$, where $n$ is the number of generators of $A$.
            \end{proof}

            \begin{lemma}
                Let $R$ be an artin ring and $\mathfrak{r}$ denote the nilradical of $R$. Moreover, let $J$ be the injective envelope of $R/\mathfrak{r}$, then functor $Hom_R(\_,J):mod_{\Lambda}\rightarrow mod_{\Lambda^{op}}$ is a duality.
            \end{lemma}

            \begin{corollary}
                The category $mod_{\Lambda}$ has enough injectives
            \end{corollary}

            Detailed proofs of these statements can be found in \cite{Rei95}.

            \begin{proof}
                Suppose that $\Lambda$ is self-injective. By the lemmas above we know that $mod_{\Lambda}$ has enough projectives and enough injectives. It remains to show that the class of injectives coincide with the projectives. Since every indecomposable $\Lambda$ module is a summand of $\Lambda$ up to isomorphism, we have that they are injective. As they also are projective, we have that the class of injectives and projectives coincide.
            \end{proof}

            This shows that $mod_{\Lambda}$ is a Frobenius category, thus \underline{$mod_{\Lambda}$} is triangulated. The triangles in \underline{$mod_{\Lambda}$} are the quotients of every short exact sequence in $mod_{\Lambda}$. Note that the map $c$ can be further described by the snake lemma.
            \begin{center}
                \begin{tikzcd}
                    0 \ar{r}{} & X \ar[tail]{r}{a} & Y \ar[two heads]{r}{b} & Z \ar{r}{} & 0 
                \end{tikzcd}
                $\implies$
                \begin{tikzcd}
                    \underline{X} \ar{r}{\underline{a}} & \underline{Y} \ar{r}{\underline{b}} & \underline{Z} \ar{r}{\underline{c}} & \underline{\upside{$\Omega$}X}
                \end{tikzcd}
            \end{center}

            % \begin{prop}
            %     Let $G$ be a group, then the group ring $R[G]$ for any artinian ring $R$ is self-injective.
            % \end{prop}

            % \begin{proof}
            %     Do I even want this?
            % \end{proof}

            \begin{prop}
                Let $K$ be a field, then $K[x]/(x^n)$ is self-injective.
            \end{prop}

            \begin{proof}
                As $K[x]/(x^n)$ modules, there is only one indecomposable projective module up to isomorphism, that is $K[x]/(x^n)$. Since $K[x]/(x^n)$ is commutative, the duality functor is an automorphism of $mod_{K[x]/(x^n)}$, thus $Hom_K(K[x]/(x^n),K)$ is the indecomposable injective $K[x]/(x^n)$ module. As the duality functor preserves length the modules have equal length. By finding a monomorphism $i:K[x]/(x^n)\rightarrow Hom_K(K[x]/(x^n),K)$ we have that it is an isomorphism as the cokernel has length $0$. The socle $soc(K[x]/x^n)$ is the simple module $K$, this means that the injective envelope of $K[x]/(x^n)$ is indecomposable, thus it is in the same isomorphism class as $Hom_K(K[x]/(x^n),K)$, proving that there is a monomorphism as stated.
            \end{proof}

            In this particular case the triangles take on a somewhat special form, where repeatedly applying TR2 yields the same triangles after 6 iterations! This can be seen by calculating the triangles of the indecomposable modules. Every other triangle will be a direct sum of these.

            Observe that every submodule of $K[x]/(x^n)$ is indecomposable, these make up the class of the indecomposable modules up to isomorphism. Further observe that the cosyzygy of any submodule is \upside{$\Omega$}$(x^k)/(x^n)\simeq (x^{n-k})/(x^n)$. The repetition of the triangles can be seen as the natural isomorphism \upside{$\Omega$}$^2(x^k)/(x^n)\simeq (x^{n-(n-k)})/(x^n) = (x^k)/(x^n)$.

            To find the triangles we let $A,B:mod_{K[x]/(x^n)}$ and $T:A\rightarrow B$ be $K[x]/(x^n)$-linear. Then $T$ is in the same equivalence class as $\begin{pmatrix} T \\ -i \end{pmatrix}:A\rightarrow B\oplus I$ with $i$ as the injective envelope of $A$. Then there is a triangle as the diagram below:

            \begin{center}
                \begin{tikzcd}
                    \underline{A} \ar{r}{\underline{T}} & \underline{B} \ar{r}{} & Cok\underline{T}\oplus Ker\underline{\upside{$\Omega$}T} \ar{r}{} & \underline{\upside{$\Omega$}A}
                \end{tikzcd}
            \end{center}

            Observe that $Cok\underline{\begin{pmatrix} f \\ -i \end{pmatrix}} \simeq Cok\underline{T}\oplus Ker$\underline{\upside{$\Omega$}$T$}, so the triangle above is in fact well-defined.

            \begin{lemma}
                The category $Vect(K)$ is triangulated.
            \end{lemma}

            \begin{proof}
                This follows immediately from the discussion above. Look at $mod_{K[x]/(x^2)}$, the indecomposable objects of this category are $K[x]/(x^2)$ and $K$ up to isomorphism. As $K[x]/(x^2)$ is injective we have that $K$ is the only indecomposable object of \underline{$mod_{K[x]/(x^2)}$}, thus every object is a direct summand $K$. Also observe that the cosyzygy is naturally isomorphic to the identity functor on the quotient. In order to be precise, one would need to show that there is an equivalence of categories $Vect(K)\simeq mod_{K[x]/(x^2)}$. The triangles in $Vect(K)$ can then be seen as this three term repeating triangle.
                \begin{center}
                    \begin{tikzcd}[ampersand replacement=\&]
                        V \ar{r}{T} \& W \ar{r}{\begin{pmatrix} \pi_T \\ 0 \end{pmatrix}} \& CokT\oplus KerT \ar{r}{\begin{pmatrix} 0 & \iota_T \end{pmatrix}} \& V
                    \end{tikzcd}
                \end{center}
            \end{proof}

            Suppose now that $\Lambda$ is not a self-injective artin $R$-algebra, then there is an extension to this ring which makes it self injective.

            \begin{definition}
                Let $\Lambda$ be an artin $R$-algebra. There is an associated product called the semi-direct proudct with this ring and it's dual. That is we define $\Lambda \rtimes Hom_R(\Lambda,J)$ to have the elements $(\lambda, \phi)$, where $\lambda:\Lambda$ and $\phi:\Lambda\rightarrow J$. Let $(\lambda, \phi)$ and $(\mu, \psi)$ be two elements in this set. We define $(\lambda, \phi) + (\mu, \psi) = (\lambda + \mu, \phi + \psi)$ and $(\lambda, \phi)(\mu, \psi) = (\lambda\mu, \lambda\cdot\psi + \phi\cdot\mu)$, where $\lambda\phi (\chi)\mu = \phi(\lambda\chi\mu)$.
            \end{definition}

            \begin{prop}
                The trivial extension $\Lambda\rtimes Hom_R(\Lambda,J)$ is self-injective.
            \end{prop}

            \begin{proof}
                HELP (jeg er kaskje litt dramatisk her, for jeg har ikke prøvd å bevise det enda.)
            \end{proof}

        \subsection{The Homotopy Category}
            
            The next example of a triangulated category is something which is referred to as \emph{the} triangulated category. The homotopy category is the category which may be regarded as the prototype for every triangulated category. It can be defined with the following category.

            \begin{definition}
                Let $\mathcal{A}$ be an additive category. Define $Ch(\mathcal{A})$ to be the category of diagrams in $\mathcal{A}$ on the form
                \begin{center}
                    \begin{tikzcd}
                        ... \ar{r}{d^{-2}_{\chain{A}}} & A^{-1} \ar{r}{d^{-1}_{\chain{A}}} & A^0 \ar{r}{d^0_{\chain{A}}} & A^1 \ar{r}{d^1_{\chain{A}}} & ...
                    \end{tikzcd}
                \end{center}
                such that $d^i_{\chain{A}}\circ d^{i-1}_{\chain{A}}=0$ for every $i:\{-\infty,...,\infty\}$. These objects are reffered to as (co)chain complexes and they are denoted as $\chain{A}$, and the maps in the objects are called differential/(co)boundaries. A morphism $\chain{\phi} : \chain{A}\rightarrow \chain{B}$ between (co)chain complexes (also called chain map) is a collection of morphisms from $\mathcal{A}$, such that the morphisms commute with the differentials in the following manner
                \begin{center}
                    \begin{tikzcd}
                        ... \ar{r}{d^{-2}_{\chain{A}}} & A^{-1} \ar{d}{\phi^{-1}} \ar{r}{d^{-1}_{\chain{A}}} & A^0 \ar{r}{d^0_{\chain{A}}} \ar{d}{\phi^0} & A^1 \ar{r}{d^1_{\chain{A}}} \ar{d}{\phi^1} & ... \\
                        ... \ar{r}{d^{-2}_{\chain{B}}} & B^{-1} \ar{r}{d^{-1}_{\chain{B}}} & B^0 \ar{r}{d^0_{\chain{B}}} & B^1 \ar{r}{d^1_{\chain{B}}} & ...
                    \end{tikzcd}
                \end{center}
            \end{definition}

            \begin{remark}
                If $\mathcal{A}$ is abelian, then the category $Ch(\mathcal{A})$ is abelian. The kernels and cokernels of chain maps would be levelwise kernels and cokernels along the chain. Moreover, if $(\mathcal{A},\mathcal{E})$ is an exact category, then $(Ch(\mathcal{A}),Ch(\mathcal{E}))$ will be exact as well, by using levelwise kernels and cokernels.
            \end{remark}

            \begin{remark}
                On the category of cochain complexes there is an additive autoequivalence called the translation functor. The functor is denoted as $(\_)[1]:Ch(\mathcal{A})\rightarrow Ch(\mathcal{A})$ and it takes a complex $A^{\bullet}$ and shifts it one step to the left into $A^{\bullet + 1}$. In fact there is a family of functors $A^{\bullet}[n]=A^{\bullet + n}$. Thus $(\_)[-1]$ is the quasi inverse of $(\_)[1]$.
            \end{remark}

            \begin{definition}
                A chain map $f^{\bullet}:A^{\bullet}\rightarrow B^{\bullet}$ is called null-homotopic if there is a map $\varepsilon^{\bullet}:A^{\bullet}\rightarrow B^{\bullet}[-1]$ such that $f^{\bullet} = d_{B^{\bullet}}^{\bullet - 1}\varepsilon^{\bullet} + \varepsilon^{\bullet + 1}d_{A^{\bullet}}^{\bullet}$.
                \begin{center}
                    \begin{tikzcd}
                        ... \ar{r}{d^{-2}_{\chain{A}}} & A^{-1} \ar{d}{f^{-1}} \ar{r}{d^{-1}_{\chain{A}}} & A^0 \ar{ld}[above]{\varepsilon^{0}} \ar{r}{d^0_{\chain{A}}} \ar{d}{f^0} & A^1 \ar{r}{d^1_{\chain{A}}} \ar{d}{f^1} \ar{ld}[above]{\varepsilon^1} & ... \\
                        ... \ar{r}{d^{-2}_{\chain{B}}} & B^{-1} \ar{r}{d^{-1}_{\chain{B}}} & B^0 \ar{r}{d^0_{\chain{B}}} & B^1 \ar{r}{d^1_{\chain{B}}} & ...
                    \end{tikzcd}
                \end{center}
                $\chain{\varepsilon}$ is called the homotopy. Two chain maps $\chain{f}$ and $\chain{g}$ are said to be homotopic $\chain{f}\sim\chain{g}$ if their difference $\chain{f}-\chain{g}$ is null-homotopic.
            \end{definition}

            \begin{prop}
                There is an additive bifunctor $nullHom_{\mathcal{A}}(\_,\_):Ch(\mathcal{A})^{op}\times Ch(\mathcal{A})\rightarrow Ab$ mapping into the set of null-homotopic morphisms. The elements of $nullHom_{\mathcal{A}}(\chain{A},\chain{B})$ are pairs made of null-homotopic maps with their homotopy $(\chain{f},\chain{\varepsilon})$. This is an abelian group with the product group structure, that is $(\chain{f},\chain{\varepsilon}) + (\chain{g},\chain{\gamma}) = (\chain{f}+\chain{g},\chain{\varepsilon}+\chain{\gamma})$. The functor acts on morphisms almost the same way as the hom-functor.
            \end{prop}

            \begin{proof}
                In order to prove the proposition we must show that the assignment is in fact a functor and that it is additive as well. To show this we will show that $nullHom_{\mathcal{A}}(\chain{A},\_)$ is an additive functor. It will follow by duality that the there is an additive bifunctor as proposed.
                
                Suppose that there is a chain map $\chain{f}:\chain{B}\rightarrow\chain{C}$, then $nullHom_{\mathcal{A}}(\chain{A},\_)(\chain{f})=\chain{f}_*$. Let $(\chain{g},\chain{\varepsilon}):nullHom_{\mathcal{A}}(\chain{A},\chain{B})$ be a null-homotopic chain map. By definition $\chain{f}_*(\chain{g},\chain{\varepsilon})=(\chain{f}\chain{g},f^{\bullet-1}\chain{\varepsilon})$. We are able to see that $f^{\bullet-1}\chain{\varepsilon}$ is a homotopy by the following diagram. The commutativity of the lower left square gives us the homotopy. This it follows by functoriality from the Hom-functor that this is a functor.
                \begin{center}
                    \begin{tikzcd}
                        ... \ar{r}{d^{-2}_{\chain{A}}} & A^{-1} \ar{d}{g^{-1}} \ar{r}{d^{-1}_{\chain{A}}} & A^0 \ar{ld}[above]{\varepsilon^{0}} \ar{r}{d^0_{\chain{A}}} \ar{d}{g^0} & A^1 \ar{r}{d^1_{\chain{A}}} \ar{d}{g^1} \ar{ld}[above]{\varepsilon^1} & ... \\
                        ... \ar{r}{d^{-2}_{\chain{B}}} & B^{-1} \ar{r}{d^{-1}_{\chain{B}}} \ar{d}{f^{-1}} & B^0 \ar{r}{d^0_{\chain{B}}} \ar{d}{f^0} & B^1 \ar{r}{d^1_{\chain{B}}} \ar{d}{f^1} & ... \\
                        ... \ar{r}{d^{-2}_{\chain{C}}} & C^{-1} \ar{r}{d^{-1}_{\chain{C}}} & C^0 \ar{r}{d^0_{\chain{C}}} & C^1 \ar{r}{d^1_{\chain{C}}} & ...
                    \end{tikzcd}
                \end{center}

                The last thing to show is that it is additive. This is the same as showing that the assignment $nullHom_{\mathcal{A}}(\chain{A},\_):Hom_{Ch(\mathcal{A})}(\chain{B},\chain{C})\rightarrow Hom_{Ab}(nullHom_{\mathcal{A}}(\chain{A},\chain{B}),nullHom_{\mathcal{A}}(\chain{A},\chain{C}))$ is a group homomorphism. Let $\chain{f},\chain{g}:\chain{B}\rightarrow\chain{C}$ be two chain maps, and $(\chain{h},\chain{\varepsilon}):nullHom_{\mathcal{A}}(\chain{A},\chain{B})$. Then we have the following equation asserting the additivity:
                \begin{align*}
                    (\chain{f}+\chain{g})_*(\chain{h},\chain{\varepsilon})\\
                    =((\chain{f}+\chain{g})\chain{h},((\chain{f}+\chain{g})[-1])\chain{\varepsilon})\\
                    =(\chain{f}\chain{h}+\chain{g}\chain{h},f^{\bullet-1}\chain{\varepsilon}+g^{\bullet-1}\chain{\varepsilon})\\
                    =(\chain{f}\chain{h},f^{\bullet-1}\chain{\varepsilon})+(\chain{g}\chain{h},g^{\bullet-1}\chain{\varepsilon})\\
                    =\chain{f}_*(\chain{h},\chain{\varepsilon})+\chain{g}_*(\chain{h},\chain{\varepsilon})
                \end{align*}
            \end{proof}

            \begin{corollary}
                The equivalence relation $\sim$ stated above is an additive congruence relation. Thus we can define the category $Ch(\mathcal{A})/(\sim) = K(\mathcal{A})$. This category is called the homotopy category.
            \end{corollary}

            The goal for the rest of this section is to prove that the homotopy category is triangulated. This will be done by giving putting on an exact structure onto $Ch(\mathcal{A})$ which turns it into a Frobenius category, such that the stable Frobenius category and the Homotopy category will be equivalent. To do this we will need to find the representable nature of the $nullHom_{\mathcal{A}}(\_,\_)$ functor.

            \begin{definition}
                Let $\chain{f}:\chain{A}\rightarrow\chain{B}$ be a chain map. We define the object $cone(\chain{f})$ to be the following complex.
                \begin{center}
                    \begin{tikzcd}[ampersand replacement=\&]
                        ... \ar{r} \& B^{-1}\oplus A^0 \ar{r}[below]{\begin{pmatrix} d^{-1}_{\chain{B}} & f^0 \\ 0 & -d^0_{\chain{A}} \end{pmatrix}} \& B^0\oplus A^1 \ar{r} \& ...
                    \end{tikzcd}
                \end{center}
            \end{definition}

            \begin{remark}
                For any chain map $\chain{f}:\chain{A}\rightarrow\chain{B}$ there is a short exact sequence.
                \begin{center}
                    \begin{tikzcd}[ampersand replacement=\&]
                        \chain{B} \ar[tail]{r}{\chain{\begin{pmatrix}1 \\ 0 \end{pmatrix}}} \& cone(\chain{f}) \ar[two heads]{r}{\chain{\begin{pmatrix} 0 & \chain{-1} \end{pmatrix}}} \& \chain{A}[1]
                    \end{tikzcd}
                \end{center}
            \end{remark}

            \begin{definition}
                An object $\chain{A}$ of $Ch(\mathcal{A})$ is called contractible if $\chain{id_{\chain{A}}}$ is null-homotopic.
            \end{definition}

            \begin{example}
                Let $\chain{A}$ be a complex, then $cone(id_{\chain{A}})$ is contractible. That is \\$(\begin{pmatrix} \chain{id_{\chain{A}}} & 0 \\ 0 & \chain{id_{\chain{A}}}[1]\end{pmatrix},\begin{pmatrix} 0 & 0 \\ \chain{id_{\chain{A}}} & 0 \end{pmatrix}):nullHom_{\mathcal{A}}(cone(\chain{id_{\chain{A}}}), cone(\chain{id_{\chain{A}}}))$
            \end{example}

            \begin{prop}
                For any complex $\chain{A}$ there is a natural isomorphism $nullHom_{\mathcal{A}}(\chain{A},\_)\simeq Hom_{Ch(\mathcal{A})}(cone(\chain{id_{\chain{A}}}),\_)$. This is the same as saying that $cone(\chain{id_{\chain{A}}})$ is the universal contractible complex where null-homotopic morphisms from $\chain{A}$ factors through.
            \end{prop}

            \begin{proof}
                We will make two natural maps which we will se are inverses. This is sufficient to prove the universal property by Yoneda's lemma.

                Let $construct_{(\chain{A},\_)(\chain{B})}:nullHom_{\mathcal{A}}(\chain{A},\chain{B})\rightarrow Hom_{Ch(\mathcal{A})}(cone(\chain{id_{\chain{A}}}),\chain{B})$ and \\$destruct_{(\chain{A},\_)(\chain{B})}:Hom_{Ch(\mathcal{A})}(cone(\chain{id_{\chain{A}}}),\chain{B})\rightarrow nullHom_{\mathcal{A}}(\chain{A},\chain{B})$ be two morphisms defined the following way.
                $construct_{(\chain{A},\_)(\chain{B})}(\chain{f},\chain{\varepsilon})=\begin{pmatrix}\chain{f} & \chain{\varepsilon}\end{pmatrix}$ and \\$destruct_{(\chain{A},\_)(\chain{B})}\begin{pmatrix}\chain{f} & \chain{\varepsilon}\end{pmatrix} = (\chain{f}, \chain{\varepsilon})$. These natural transformations are constructed such that they are inverses of each other. It remains to see that these maps are well defined. This will be done by showing that there is a chain map from the cone of the identity, if and only if there is a null-homotopic map from the object.

                \begin{center}
                    \begin{tikzcd}[ampersand replacement=\&]
                        ... \ar{r}{d^{-2}_{cone(\chain{id_{\chain{A}}})}} \& A^{-1}\oplus A^0 \ar{r}{d^{-1}_{cone(\chain{id_{\chain{A}}})}} \ar{d}{\begin{pmatrix} f^{-1} & \varepsilon^{0} \end{pmatrix}} \& A^0 \oplus A^1 \ar{r}{d^{0}_{cone(\chain{id_{\chain{A}}})}} \ar{d}{\begin{pmatrix} f^0 & \varepsilon^1 \end{pmatrix}}\& ... \\
                        ... \ar{r}{d^{-2}_{\chain{B}}} \& B^{-1} \ar{r}{d^{-1}_{\chain{B}}} \& B^0 \ar{r}{d^0_{\chain{B}}} \& ... 
                    \end{tikzcd}
                \end{center}

                For $\begin{pmatrix}\chain{f} & \chain{\varepsilon}[1]\end{pmatrix}$ to be a chain map, we need the following conditions to hold, i.e. that the square commute.
                \begin{equation*}
                    \begin{pmatrix}f^0 & \varepsilon^1 \end{pmatrix}\begin{pmatrix}d^{-1}_{\chain{A}} & id^0_{\chain{A}} \\ 0 & -d^0_{\chain{A}}\end{pmatrix}=d^{-1}_{\chain{B}}\begin{pmatrix} f^{-1} & \varepsilon^0 \end{pmatrix}
                \end{equation*}

                By calculating the matrix, we see that this is the same as if the following conditions are met.

                \begin{align*}
                    f^0d^{-1}_{\chain{A}} = d^{-1}_{\chain{A}}f^{-1} \\
                    f^0 = d^{-1}_{\chain{B}}\varepsilon^0 + \varepsilon^1d^0_{\chain{A}}
                \end{align*}

                Thus we have that a morphism is a chain map from the identity cone if and only if it is a null-homotopic chain map, which proves that there is a natural isomorphism as stated.
            \end{proof}

            \begin{remark}
                The identity cone is universal with respect to homotopies. A null-homotopic chain map $\chain{f}:\chain{A}\rightarrow \chain{B}$ might admit several factorization through the identity cone. The factorizations are unique only when there is a homotopy witnessing the null-homotopy property.
                \begin{center}
                    \begin{tikzcd}[ampersand replacement=\&]
                        \chain{A} \ar{rd}[below, pos=0.3]{\begin{pmatrix}1 \\ 0\end{pmatrix}} \ar{rr}{(\chain{f},\chain{\varepsilon})} \& \& \chain{B} \\
                        \& cone(\chain{id_{\chain{A}}}) \ar[dashed]{ru}[below,pos=0.6, rotate=30]{\begin{pmatrix} \chain{f} & \chain{\varepsilon}[1]\end{pmatrix}}
                    \end{tikzcd}
                \end{center}
            \end{remark}

            \begin{corollary}
                The contravariant functor $nullHom_{\mathcal{A}}(\_,\chain{B})$ is represented by $cone(\chain{id_{\chain{B}}})[-1]$. Thus there is a factorization of null-homotopic maps which ends in $\chain{B}$ as follows.

                \begin{center}
                    \begin{tikzcd}[ampersand replacement=\&]
                        \chain{A} \ar[dashed]{rd}[below, pos=0.3]{\begin{pmatrix} \chain{\varepsilon} \\ \chain{f} \end{pmatrix}} \ar{rr}{(\chain{f},\chain{\varepsilon})} \& \& \chain{B} \\
                        \& cone(\chain{id_{\chain{B}}})[-1] \ar{ru}[below,pos=0.6, rotate=30]{\begin{pmatrix} 0 & \chain{-1} \end{pmatrix}}
                    \end{tikzcd}
                \end{center}
            \end{corollary}

            \begin{lemma}
                $\chain{f}$ is null-homotopic if and only if $\chain{f}$ factors through a contractible object.
            \end{lemma}

            \begin{proof}
                Suppose that $\chain{f}$ is null-homotopic, then by the universal property of null-homotopy, it factors through the identity cone.
                Converesly, suppose that $\chain{f}:\chain{A}\rightarrow\chain{C}$ factors thorugh a contractible object $\chain{B}$ as $\chain{g}\chain{h}$. Then $\chain{f}=\chain{g}\chain{h}=\chain{g}\chain{id_{\chain{B}}}\chain{h}$. Since homotopy equivalence is a congruence relation, we get that $\chain{f}$ is null-homotopic.
            \end{proof}

            We are now going to equip an exact structure on $Ch(\mathcal{A})$ such that the contractible objects become projective and injective objects. In this manner we get the desired Frobenius category.

            By the example in 3.1 for any additive category $\mathcal{A}$, there is an exact category $\mathcal{A},\mathcal{E}$, where $\mathcal{E}=\{$Split short-exact sequences$\}$. Then there is an exact category $(Ch(\mathcal{A}),Ch(\mathcal{E}))$, where $Ch(\mathcal{E})=\{$Levelwise split short-exact sequences$\}$. We are in fact able to further specify what this exact structure is.

            \begin{prop}
                The exact structure $(Ch(\mathcal{A}),Ch(\mathcal{E}))$ are diagrams on the form as below, where $\chain{r}:\chain{A}\rightarrow\chain{B}$ is a chain map.
                \begin{center}
                    \begin{tikzcd}[ampersand replacement=\&]
                        \chain{B} \ar[tail]{r}{\begin{pmatrix}1 \\ 0\end{pmatrix}}[marking]{\circ} \& cone(\chain{r}) \ar[two heads]{r}{\begin{pmatrix}0 & \chain{-1}\end{pmatrix}}[marking]{\circ} \& \chain{A}[1]
                    \end{tikzcd}
                \end{center}
            \end{prop}

            \begin{proof}
                Suppose that there is a conflation $(\chain{i}:\chain{Q}\rightarrow\chain{R},\chain{p}:\chain{R}\rightarrow\chain{P})$ in $Ch(\mathcal{A})$. Our goal is to realize the object $\chain{R}$ as a cone of some map. Since the conflation is levelwise split we get that in the following diagram $R^i\simeq Q^i\oplus P^i$.
                \begin{center}
                    \begin{tikzcd}[column sep=small, ampersand replacement=\&]
                        \& Q^1 \ar{rr}{i^1} \& \& R^1 \ar{rr}{p^1} \& \& P^1 \\
                        Q^0 \ar{ru}{d^0_{\chain{Q}}} \ar{rr}{i^0} \& \& R^0 \ar{ru}{d^0_{\chain{R}}} \ar{rr}{p^0} \& \& P^0 \ar{ru}{d^0_{\chain{P}}}
                    \end{tikzcd}
                \end{center}

                By the commutativity of the squares we have that
                \begin{align*}
                    d^0_{\chain{R}}i^0=i^1d^0_{\chain{Q}} \iff \begin{pmatrix} a & b \\ c & d \end{pmatrix}\begin{pmatrix} 1 \\ 0 \end{pmatrix} = \begin{pmatrix} d^0_{\chain{Q}} \\ 0 \end{pmatrix} \\
                    d^0_{\chain{P}}p^0=p^1d^0_{\chain{R}} \iff \begin{pmatrix} 0 & d^0_{\chain{P}} \end{pmatrix} = \begin{pmatrix} 0 & -1 \end{pmatrix}\begin{pmatrix} a & b \\ c & d \end{pmatrix}
                \end{align*}

                Thus $a = d^0_{\chain{R}}$, $d = -d^0_{\chain{P}}$ and $c=0$. $b:P^0\rightarrow Q^1$. Thus we can create a map $\chain{b'}:\chain{P}[1]\rightarrow\chain{Q}$. This is a chain map by the following calculation
                \begin{equation*}
                    \begin{pmatrix} d^1_{\chain{Q}} & b^1 \\ 0 & d^1_{\chain{P}} \end{pmatrix}\begin{pmatrix} d^0_{\chain{Q}} & b^0 \\ 0 & -d^0_{\chain{P}} \end{pmatrix} = \begin{pmatrix} 0 & d^1_{\chain{Q}}b^0 - b^1d^0_{\chain{P}} \\ 0 & 0 \end{pmatrix} = \begin{pmatrix} 0 & 0 \\ 0 & 0 \end{pmatrix}
                \end{equation*}
                This proves that $\chain{b}$ is a chain map and thus $\chain{R}=cone(\chain{b})$.
            \end{proof}

            To show that this is a Frobenius category we will show that every projective object is contractible. The dual case of every injective object is contractible will follow from duality, as there is a covariant and contravariant representation of null-homotopies.

            \begin{prop}
                An object $\chain{P}$ is projective if and only if it is contractible.
            \end{prop}

            \begin{proof}
                Suppose that $\chain{P}$ is projective, then it can be found in a conflation over $\chain{id_{\chain{P}}}[-1]$. Thus by the contravariant universal property of null-homotopies, the identity map is null-homotopic as described by the diagram below.
                \begin{center}
                    \begin{tikzcd}
                        \chain{P}[-1] \ar[tail]{r}[marking]{\circ} & cone(\chain{id_{\chain{P}}})[-1] \ar[two heads]{r}[marking]{\circ} & \chain{P} \ar[equal]{d} \\
                        & & \chain{P} \ar[dashed]{lu}
                    \end{tikzcd}
                \end{center}

                Converesly, suppose that $\chain{P}$ is contractible, then we can se that $\chain{P}$ is projective if and only if $cone(\chain{id_{\chain{P}}})$ is projective. This can be seen with the following diagram.

                \begin{center}
                    \begin{tikzcd}
                        \chain{P}[-1] \ar[tail]{r}[marking]{\circ} & cone(\chain{id_{\chain{P}}})[-1] \ar[two heads]{r}[marking]{\circ} & \chain{P} \ar[equal]{d} \\
                        & cone(\chain{id_{\chain{P}}}) \ar[dashed]{ru}{} \ar[dashed]{u}{} & \chain{P} \ar[tail]{l}[marking]{\circ}
                    \end{tikzcd}
                \end{center}
                
                Thus it is enough to show that every identity cone is projective, to show that every contractible is projective. In order to do this we want to show that the functor \\$Hom_{Ch(\mathcal{A})}(cone(\chain{id_{\chain{P}}}),\_):Ch(\mathcal{A})\rightarrow Ab$ is an exact functor. This is the same as saying that every conflation gets mapped to short-exact sequences.
                Suppose further that there is a morphism $\chain{p}:cone(\chain{\beta})\rightarrow\chain{B}[1]$, where $\chain{\beta}:\chain{B}\rightarrow\chain{C}$. Thus to show exactness we must show that $Hom_{Ch(\mathcal{A})}(cone(\chain{id_{\chain{P}}}),\chain{p})$ is a surjection.

                First observe that there is an isomorphism $Hom_{Ch(\mathcal{A})}(cone(\chain{id_{\chain{P}}}),\chain{p})\simeq nullHom_{\mathcal{A}}(\chain{P},\chain{p})$. Suppose that $(\chain{f},\chain{\varepsilon}):nullHom_{\mathcal{A}}(\chain{P},\chain{B})$. Then there is a null-homotopic chain map $(\chain{f'},\chain{\varepsilon '})=(\begin{pmatrix} -\beta^{\bullet - 1}\chain{\varepsilon} \\ \chain{f} \end{pmatrix}, \begin{pmatrix} 0 \\ (-1)^{\bullet + 1}\chain{\varepsilon} \end{pmatrix}):nullHom_{\mathcal{A}}(\chain{P},cone(\chain{\beta})[-1])$ such that \\$\chain{p}_*(\chain{f'},\chain{\varepsilon '})=(\chain{f},\chain{\varepsilon})$. A diagram chase suffices to check that this is a chain map and the the proposed homotopy is in fact a homotopy.
            \end{proof}

            \begin{corollary}
                The class of contractible objects is precisly the class of projectives and the class of injectives. Thus $(Ch(\mathcal{A}),Ch(\mathcal{E}))$ is a Frobenius category. The stable Frobenius category is the homotopy category, i.e. $\underline{Ch(\mathcal{A})}=K(\mathcal{A})$. 
            \end{corollary}

            Since the identity cones are injective, we can verify that the cosyzygy functor is the shift functor (\upside{$\Omega$}$\_=\_[1]$). Thus we get that the standard triangles in $K(\mathcal{A})$ are the candidate triangles on the form
            \begin{center}
                \begin{tikzcd}
                    \chain{A} \ar{r}{\underline{\chain{f}}} & \chain{B} \ar{r}{} & cone(\chain{f}) \ar{r}{} & \chain{A}[1]
                \end{tikzcd}
            \end{center}

    \clearpage
    
    \section{The Derived Category}

        \subsection{Idempotent Completeness and Krull-Schmidt Categories}

            This section is based of the ideas from \cite{buhler}, \cite{Kra12} and \cite{Rei95}. At first the ideas of idempotent complete categories will be introduced, the we will look at a weakening and a strengthening of this condition, known as weak idempotent complete and Krull-Schmidt.

            \begin{definition}
                An idempotent complete category is an additive category where every idempotent split. That is, if there is an idempotent $p:A\rightarrow A$ $(p^2=p)$, and there is an isomorphism $A\simeq I\oplus K$ such that $p\simeq \begin{pmatrix} 0 & 0 \\ 0 & 1 \end{pmatrix}$. 
            \end{definition}

            Every idempotent in an idempotent complete category admits an anlysis. That is the idempotent $p:A\rightarrow A$ has a kernel, cokernel, image and coimage. In fact, the kernel is isomorphic to the cokerenl, and the image is cannonically isomorphic to the coimage. As $p$ is isomorphic to the matrix $\begin{pmatrix} 0 & 0 \\ 0 & 1 \end{pmatrix}$ we can observe that the inclusion $\iota_1:I\rightarrow A$ is the kernel of $p$, while the projection $\pi_1:A\rightarrow I$ is the cokernel. Similarly we have that the maps $\iota_2:K\rightarrow A$ and $\pi_2:A\rightarrow K$ which are the kernel and cokernel of the map $1-p$ respectively. Using the fact that $p$ splits we are able to construct the following analysis.

            \begin{center}
                \begin{tikzcd}
                    & A \ar{rr}{p} \ar[two heads]{rd}{\pi_2}[marking]{\circ} & & A \ar[two heads]{rd}{\pi_1}[marking]{\circ} \\
                    I \ar[tail]{ru}{\iota_1}[marking]{\circ} & & K \ar[tail]{ru}{\iota_2}[marking]{\circ} & & I 
                \end{tikzcd}
            \end{center}

            \begin{remark}
                Assuming that every idempotent in an additive category $\mathcal{A}$ has a kernel is sufficient for $\mathcal{A}$ to be idempotent complete. The limits and colimits as described above may be found with the idempotents $p$ and $1-p$.
            \end{remark}

            Every additive category $\mathcal{A}$ has a fully faithful embedding $i_{\mathcal{A}}:\mathcal{A}\rightarrow \widehat{\mathcal{A}}$ into an idempotent complete category $\widehat{\mathcal{A}}$. This completion satisfies the universal property in which if there is a functor $F:\mathcal{A}\rightarrow\mathcal{B}$ which sends every idempotent $p$ in $\mathcal{A}$ to a splitting idempotent, then the functor factors through the idempotent complete category $\widehat{\mathcal{A}}$.

            \begin{center}
                \begin{tikzcd}
                    \mathcal{A} \ar{rd}{i_{\mathcal{A}}} \ar{rr}{F} & & \mathcal{B} \\
                    & \widehat{\mathcal{A}} \ar{ru}{\widehat{F}}
                \end{tikzcd}
            \end{center}

            We can define this completion $\widehat{\mathcal{A}}$ to be the category with objects $(A,p)$, where $A$ is an object of $\mathcal{A}$ and $p:A\rightarrow A$ is an idempotent. A morphism $\widehat{f}:(A,p)\rightarrow (B,q)$ is defined as the morphism $\widehat{f} = q \circ f \circ p$ for some morphism $f:A\rightarrow B$. The injection functor is then defined as $i_{\mathcal{A}}(A)=(A,id_A)$. More on this injection can be found in \cite{buhler}.
            
            Many of the useful theorems needed to describe the triangulated subcategory needed for the construction of the derived category will arise from the weaker condition of weakly split idempotents.

            \begin{lemma}
                The following are equivalent in an additive category:
                \begin{enumerate}
                    \item Every split-epi has a kernel
                    \item Every split-mono has a cokernel
                \end{enumerate}
            \end{lemma}

            \begin{proof}
                We will prove that (1.) $\implies$ (2.), the other claim is dual. Suppose that $g:B\rightarrow A$ is split-epi with $f:A\rightarrow B$ as the corresponding split-mono such that $gf=id_A$. Since $g$ is split-epi it has a kernel $h:C\rightarrow B$.
                
                \begin{center}
                    \begin{tikzcd}
                        A \ar[bend left]{r}{f} & \ar[bend left]{l}{g} B \ar[dashed, bend left]{r}{i} & \ar[bend left]{l}{h} C
                    \end{tikzcd}
                \end{center}

                If we look at the map $id_B-fg$, we see that $g(id_B-fg)=g-gfg=g-g=0$, thus $id_B-fg$ factors over the kernel of h as indicated by the dashed arrow.

                h is split-mono as $hih = (id_B-fg)h=h-fgh=h$. As h is mono from being a kernel, we have that $ih=id_C$. Thus $B\simeq A\oplus C$ as $id_B -fg = hi \iff id_B = fg + hi$. This in turn implies that $i$ is the cokernel of $f$.
            \end{proof}

            With this lemma we are able to define a weakly idempotent complete category.

            \begin{definition}
                An additive category $\mathcal{A}$ is weakly idempotent complete if it satisfies either of the conditions of Lemma 4.1.
            \end{definition}

            \begin{corollary}
                Let $(\mathcal{A},\mathcal{E})$ be an exact category, then the following are equivalent:
                \begin{enumerate}
                    \item The category $\mathcal{A}$ is weakly idempotent complete
                    \item Every split-mono is an inflation
                    \item Every split-epi is a deflation
                \end{enumerate}
            \end{corollary}

            With the notion of a weakly idempotent complete category we are able to strenghten the Obscure axiom into Heller's cancellation axiom.

            \begin{prop}
                For an exact category $(\mathcal{A},\mathcal{E})$ the following are equivalent:
                \begin{enumerate}
                    \item $\mathcal{A}$ is weakly idempotent complete
                    \item Let $f: A\rightarrow B$ and $g: B\rightarrow C$ be two morphisms in $\mathcal{A}$. Then if $gf:A\rightarrow C$ is a deflation, then $g$ is.
                \end{enumerate}
            \end{prop}

            \begin{proof}
                Suppose (1.). Let $f:A\rightarrow B$ and $g:B\rightarrow C$ be morphisms such that their composition $gf:A\rightarrow C$ is a deflation. Since $gf$ is a deflation we are able to form a pullback square.

                \begin{center}
                    \begin{tikzcd}
                        A \ar[bend right]{ddr}{id_A} \ar[bend left]{rrd}{f} \ar[dashed]{rd}{f'} \\
                        & B' \ar[phantom]{rd}[very near start]{\ulcorner} \ar{d}{h} \ar[two heads]{r}{f'}[marking]{\circ} & B \ar{d}{g} \\
                        & A \ar[two heads]{r}{gf}[marking]{\circ} & C
                    \end{tikzcd}
                \end{center}

                By using the universal property, we are able to see that $g'$ is split-mono, hence it admits an inflation $h':A'\rightarrow B'$. The claim is that $hh':A'\rightarrow B$ is the kernel of $g$. If the claim is true, the Obscure axiom yields that $g$ is a deflation.

                To show that $hh'$ is the kernel we will need to show the universal property. Let $t:T\rightarrow B$ be a test object, such that $gt=0$.

                \begin{center}
                    \begin{tikzcd}
                        T \ar[bend left]{rrd}{t} \ar[dashed]{rd}{t'} \\
                        & A' \ar{r}{hh'} \ar{rd}{0} & B \ar{d}{g} \\
                        & & C
                    \end{tikzcd}
                \end{center}

                We know that $t'$ exists as $t$ factors through $B'$ with $t''$, by the pull-back property. As $g't''=0$, we get that $t''$ factors through $A'$ using the fact that $h'$ is the kernel of $g'$, this proves the claim.
                
                For the other direction, suppose (2.) instead. Then we let $gf=id_A$, thus $gf$ is a deflation and $g$ is split-mono. By the assumption, we get that $g$ is a deflation, so it has a kernel.
            \end{proof}

            For the final part, if $\mathcal{A}$ is an idempotent complete category and there are idempotents over an object $A$, these idempotents admits a description of A as a direct sum of kernels and cokernels. There is however no guarantee that these decompositions are unique. To fix this we define the following category. These results will not be proved, and the reader is instead reffered to Henning Krause (\cite{Kra12}) and Auslander, Reiten and Smalø (\cite{Rei95}).

            \begin{definition}
                Let $\mathcal{A}$ be an additive category. An object $A$ is called indecomposable if the endomorphism ring of $A$ is local.

                An object is called decomposable if it is not indecomposable.
            \end{definition}

            \begin{definition}
                An additive category $\mathcal{A}$ is called Krull-Schmidt if any object $A$ decomposes into a finite direct sum of indecomposable objects.
            \end{definition}

            Having that each indecomposable object is local is enough for the following proposition to hold.

            \begin{prop}
                Every decomposition in a Krull-Schmidt category is unique up to isomorphism
            \end{prop}

            As being Krull-Schmidt admits decomposition whenever an endomorphism ring is not local implies a connection to idempotent completeness. That is whenever there is an idempotent over an object, this idempotent give rise to two comaximal ideals for the endomorphism ring. This gives us the decomposition which is required for the idempotent to split. Moreover, there is a deeper connection with being Krull-Schmidt and idempotent complete.

            \begin{definition}
                Let R be a ring. We say that R is semiperfect if $R$ as a module over itself admits a decomposition $_RR\simeq P_1\oplus P_2\oplus ... \oplus P_n$ such that each $P_i$ has a local endomorphism ring.
            \end{definition}

            \begin{remark}
                For a ring R the following conditions are equivalent:
                \begin{itemize}
                    \item The category $mod_R$ is a Krull-Schmidt category
                    \item R is semiperfect
                    \item Every simple R-module has a projective cover
                    \item Every finitely generated R-module has a projective cover
                \end{itemize}
                Thus any of these conditions can be taken to be the definition of semiperfect.
            \end{remark}

            With this definition we are able to state the following proposition, which says whenever an idempotent complete category is Krull-Schmidt.

            \begin{prop}
                Let $\mathcal{A}$ be an additive category, then the following are equivalent:
                \begin{enumerate}
                    \item $\mathcal{A}$ is Krull-Schmidt
                    \item $\mathcal{A}$ is idempotent complete and every endomorphism ring are semiperfect.
                \end{enumerate}
            \end{prop}

            %\begin{proof}
            %    Suppose that $\mathcal{A}$ is Krull-Schmidt. We need to show that every idempotent splits, and that every endomorphism ring are semiperfect.

            %    Let $p:A\rightarrow A$ be an idempotent. Then $End(A)$ is not local, as the ideals $(p)$ and $(1-p)$ are comaximal. Thus $A\simeq A_1\oplus A_2$ and $p\simeq \begin{pmatrix} 0 & \\ 0 & 1 \end{pmatrix}$. Thus every idempotent split.

            %    If $A$ is indecomposable, then by assumption $End(A)$ is local. Converesly, assume that $A$ is decomposable, then $A$ admits a finite decomposition $\bigoplus_{i=1}^{n}A_i$. 
            %\end{proof}

            \begin{example}
                Let $\Lambda$ be any artin R-algebra, then $mod_{\Lambda}$ is a Krull-Schmidt category. As an example, the category of finitely generated real vector spaces is Krull-Schmidt. Every vector space is a finite direct summand of the only indecomposable vector space $\mathbb{R}$.
            \end{example}

            More details and examples of Krull-Schmidt categories may be found in Henning Krause notes (\cite{Kra12}).

        \subsection{Admissable Morphisms, Homology and Long Exact Sequences}
            
            \begin{definition}
                Let $(\mathcal{A},\mathcal{E})$ be an exact category. A morphism $f:A\rightarrow B$ is called normal if it has a deflation-inflation factorization. They will be drawn as in the following diagram.
                \begin{center}
                    \begin{tikzcd}
                        A \ar[two heads]{rd}[marking]{\circ} \ar{rr}{f}[marking]{\circ} & & B \\
                        & I \ar[tail]{ru}[marking]{\circ}
                    \end{tikzcd}
                \end{center}
            \end{definition}

            \begin{remark}
                A monomorphism is normal if and only if it is an inflation. Dually, an epimorphism is normal if and only if it is a deflation.
            \end{remark}

            \begin{remark}
                In general the composition $gf$ of two normal morphisms $f$ and $g$ are not normal. However, if g is a deflation, the composition can be seen to normal, as deflations are closed under composition. One may also observe that an exact category is abelian if and only if normal morphisms are closed under composition.
            \end{remark}

            \begin{lemma}
                \textbf{Heller's factorization lemma}. The factorization of normal morphisms are unique up to unique isomorphisms.
            \end{lemma}

            \begin{proof}
                Suppose that a normal morphism admits two differnt factorization. That means there exists a commutative diagram as follows.
                \begin{center}
                    \begin{tikzcd}
                        A \ar[two heads]{r}{p}[marking]{\circ} \ar[two heads]{d}{q}[marking]{\circ} & I \ar[dashed, shift left]{ld}{\phi} \ar[tail]{d}{i}[marking]{\circ} \\
                        I' \ar[dashed, shift left]{ru}{\phi '} \ar[tail]{r}{j}[marking]{\circ} & B
                    \end{tikzcd}
                \end{center}
                By assumption $ip=jq$, thus $jq\circ Ker(p)=0$. $q\circ Ker(p)=0$ as $j$ is mono, thus there exists a morphism $\phi:I\rightarrow I'$ uniquely such that $q=\phi p$. Now $ip=jq=j\phi p$, and as $p$ is epi it follows that $i=j\phi$. Reiterating the argument, but with $Ker(q)$ instead there exists an $\phi '$ uniquely such that $p = \phi 'q$ and $j=i\phi '$. Thus we get that $i=j\phi = i\phi '\phi$, since $i$ is mono it follows that $id_I=\phi '\phi$; dually $Id_{I'}=\phi\phi '$.
            \end{proof}

            \begin{remark}
                Du to Heller's factorization axiom we are able to see that normal morphisms admits analysis.

                \begin{center}
                    \begin{tikzcd}
                        & A \ar{rr}{f}[marking]{\circ} \ar[two heads]{rd}{p}[marking]{\circ} & & B \ar[two heads]{rd}{Cok(i)}[marking]{\circ} \\
                        K \ar[tail]{ru}{Ker(p)}[marking]{\circ} & & I \ar[tail]{ru}{i}[marking]{\circ} & & C 
                    \end{tikzcd}
                \end{center}

                We may observe that the object $I$ coincide with the image and coimage of $f$. This object will then be reffered to the image of $f$. As a consequence of this unique factorization we get that a normal morphism is iso if and only if it is mono and epi.
            \end{remark}

            \begin{definition}
                A sequence of normal morphisms is exact if the inflation of the factorization together with the consecutive deflation forms a conflation. That is there are conflations between morphisms as in the following diagram. The conflations are highlighted with different colors.

                \begin{center}
                    \begin{tikzcd}
                        \dots \ar[two heads, orange]{rd}[black]{p_{-2}}[marking]{\circ} \ar{rr}{f_{-2}}[marking]{\circ} & & A_{-1} \ar[two heads, green]{rd}[black]{p_{-1}}[marking]{\circ} \ar{rr}{f_{-1}}[marking]{\circ} & & A_0 \ar[two heads, teal]{rd}[black]{p_0}[marking]{\circ} \ar{rr}{f_0}[marking]{\circ} & & A_1 \ar{rr}{f_1}[marking]{\circ} \ar[two heads, magenta]{rd}[black]{p_1}[marking]{\circ} & & \dots \\
                        & I_{-2} \ar[tail, green]{ru}[black]{i_{-2}}[marking]{\circ} & & I_{-1} \ar[tail, teal]{ru}[black]{i_{-1}}[marking]{\circ} & & I_0 \ar[tail, magenta]{ru}[black]{i_0}[marking]{\circ} & & I_1 \ar[tail, purple]{ru}[black]{i_1}[marking]{\circ}
                    \end{tikzcd}
                \end{center}

                A morphism of exact sequences is the same as a morphism of sequences. That is a collection of morphisms $(...,\phi_{-1},\phi_0,\phi_1,...)$ such that the squares in the diagram commute.

                \begin{center}
                    \begin{tikzcd}
                        \dots \ar{r}{a_{-2}}[marking]{\circ} & A_{-1} \ar{d}{\phi_{-1}} \ar{r}{a_{-1}}[marking]{\circ} & A_0 \ar{d}{\phi_0} \ar{r}{a_0}[marking]{\circ} & A_1 \ar{d}{\phi_1} \ar{r}{a_1}[marking]{\circ} & \dots \\
                        \dots \ar{r}{b_{-2}}[marking]{\circ} & B_{-1} \ar{r}{b_{-1}}[marking]{\circ} & B_0 \ar{r}{b_0}[marking]{\circ} & B_1 \ar{r}{b_1}[marking]{\circ} & \dots
                    \end{tikzcd}
                \end{center}
            \end{definition}

            \begin{remark}
                An exact sequence of normal morphisms is called short exact if it consists of morphisms on the form $(,0,i,p,0,)$, i.e. as in the following diagram.

                \begin{center}
                    \begin{tikzcd}[cramped, row sep=small]
                        0 \ar[equal, orange]{rd}{} \ar{rr}{0}[marking]{\circ} & & A \ar[equal, green]{rd}{} \ar{rr}{i}[marking]{\circ} & & B \ar[two heads, teal]{rd}[black, below]{p}[marking]{\circ} \ar{rr}{p}[marking]{\circ} & & C \ar{rr}{0}[marking]{\circ} \ar[two heads, magenta]{rd}[black]{0}[marking]{\circ} & & 0 \\
                        & 0 \ar[tail, green]{ru}[black]{0}[marking]{\circ} & & A \ar[tail, teal]{ru}[black, below]{i}[marking]{\circ} & & C \ar[equal, magenta]{ru}{} & & 0 \ar[equal, purple]{ru}{}
                    \end{tikzcd}
                \end{center}

                Observe how conflations are exactly the class of short exact sequences.
            \end{remark}

            For this definition we have the following properties which mimics properties found in homologicial algebra.

            \begin{lemma}
                \textbf{5 Lemma}. Given two 5 term exact sequences and a morphism between them as in the diagram. Then $\phi$ is an isomorphism as well.
                \begin{center}
                    \begin{tikzcd}
                        A_0 \ar{r}{a_0}[marking]{\circ} \ar{d}{\simeq} & A_1 \ar{r}{a_1}[marking]{\circ} \ar{d}{\simeq} & A_2 \ar{r}{a_2}[marking]{\circ} \ar{d}{\phi} & A_3 \ar{r}{a_3}[marking]{\circ} \ar{d}{\simeq} & A_4 \ar{d}{\simeq} \\
                        B_0 \ar{r}{b_0}[marking]{\circ} & B_1 \ar{r}{b_1}[marking]{\circ} & B_2 \ar{r}{b_2}[marking]{\circ} & B_3 \ar{r}{b_3}[marking]{\circ} & B_4
                    \end{tikzcd}
                \end{center}
            \end{lemma}

            \begin{proof}
                Later, or maybe never
            \end{proof}

            \begin{lemma}
                \textbf{Kernel-Cokernel sequence}.
                Let $(\mathcal{A},\mathcal{E})$ be an exact category which is weakly idempotent complete. Suppose that there are composable normal morphism $f$ and $g$ such that $gf$ is normal as well. Then there exists an exact sequence.
                \begin{center}
                    \begin{tikzcd}
                        Ker(f) \ar{r}[marking]{\circ} & Ker(gf) \ar{r}[marking]{\circ} & Ker(h) \ar{r}[marking]{\circ} & Cok(f) \ar{r}[marking]{\circ} & Cok(gf) \ar{r}[marking]{\circ} & Cok(g)
                    \end{tikzcd}
                \end{center} 
            \end{lemma}

            \begin{proof}
                Probably not
            \end{proof}

            \begin{remark}
                If $(\mathcal{A},\mathcal{E})$ is an exact category, then one may show that the category $\mathcal{A}$ admits Kernel-Cokernel sequences if and only if it is weakly idempotent complete.
            \end{remark}

            With the Kernel-Cokernel sequence we are able to prove that the snake lemma holds in weakly idempotent complete categories.

            \begin{corollary}
                \textbf{Snake Lemma.}
                Let $(\mathcal{A},\mathcal{E})$ be a weakly idempotent complete category. Suppose there is a diagram in $\mathcal{A}$ having exact rows.
                \begin{center}
                    \begin{tikzcd}
                        & A \ar{d}{f}[marking]{\circ} \ar{r}[marking]{\circ} & B \ar{d}{g}[marking]{\circ} \ar{r}[marking]{\circ} & C \ar{d}{h}[marking]{\circ} \ar{r} & 0 \\
                        0 \ar{r} & A' \ar{r}[marking]{\circ} & B' \ar{r}[marking]{\circ} & C'
                    \end{tikzcd}
                \end{center}

                Then there is an exact sequence.
                \begin{center}
                    \begin{tikzcd}
                        Ker(f) \ar{r}[marking]{\circ} & Ker(g) \ar{r}[marking]{\circ} & Ker(h) \ar[dashed]{r}{\delta}[marking]{\circ} & Cok(f) \ar{r}[marking]{\circ} & Cok(g) \ar{r}[marking]{\circ} & Cok(h)
                    \end{tikzcd}
                \end{center}
            \end{corollary}

            \begin{proof}
                Nooooooooooo
            \end{proof}

            The action which defines homological algebra is to find the homology of cochain complexes. For abelian groups homology is defined to be the quotient of the kernel of a map by the image of the preceeding map. We would like to find a similar definition for exact categories. For this discussion, let $(Ch(\mathcal{A}),Ch(\mathcal{E}))$ be an exact category. In order for a complex to have kernels and images we need it to consist fully of normal morphisms. This will allow us to find analyses of every differential. Let $\chain{A}$ be such a complex. The question is when does the homology exists?

            \begin{center}
                \begin{tikzcd}
                    \dots \ar{r}{d_{\chain{A}}^{-2}}[marking]{\circ} & A_{-1} \ar[two heads]{d}{p}[marking]{\circ} \ar{r}{d_{\chain{A}}^{-1}}[marking]{\circ} & A_0 \ar{r}{d_{\chain{A}}^0}[marking]{\circ} & A_1 \ar{r}{d_{\chain{A}}^1}[marking]{\circ} & \dots \\
                    & Im(d_{\chain{A}}^{-1}) \ar[tail]{ru}{\iota}[marking]{\circ} \ar[dashed]{r}{h} & Ker(d_{\chain{A}}^0) \ar{u}{\kappa}[marking]{\circ} \ar[dotted]{r}{?} & H^0(\chain{A})
                \end{tikzcd}
            \end{center}

            By looking at the 0-th homology we can find one condition  for when the homology exists. Using the fact that $d^0_{\chain{A}}\iota = 0$ we get that there is an unique morphism $h$, such that $\iota = \kappa h$. The 0-th homology exists whenever the morphism $h$ has a cokernel, and then $h$ satisfies the assumption of the Obscure axiom, making $h$ and inflation. One way to circumvent this assumption is to assume that $\mathcal{A}$ is weakly idempotent complete. Then by Heller's cancellation axiom we know that $h$ is an inflation, which then proves the existence of the cokernel. 

            
            \subsection{The Derived Category/Derived Functors}  
            
            In homological algebra, we want to localize the homotopy category over quasi-isomorphisms. A quasi-isomorphism is a chain map $\chain{f}:\chain{A}\rightarrow\chain{B}$ such that $H^*(\chain{f}):H^*(\chain{A})\rightarrow H^*(\chain{B})$ is an isomorphism in homology. One may observe by the long exact sequence in homology of short exact sequence that a chain map like $\chain{f}$ is a quasi-isomorphism if and only if $cone(\chain{f})\simeq 0$. Moreover, the cone is isomorphic to $0$ in homology if it is an exact sequence. This motivates the definition of the subcategory of acyclic complexes.

            \begin{definition}
                Let $(\mathcal{A},\mathcal{E})$ be an exact category. We define the category $Ac(\mathcal{A})\subset K(\mathcal{A})$ to be the full category whose objects are exact sequences.
            \end{definition}

            The exact complexes are also referred to as acyclic complexes. Note that this subcategory is not in general closed under isomorphisms. To be able to show that it is a triangulated subcategory, it suffices to show that the mapping cone of two acyclic complexes is again acyclic.

            \begin{lemma}
                Let $\chain{f}:\chain{A}\rightarrow\chain{B}$ be a chain map between acyclic chain complexes. Then the mapping cone $cone(\chain{f})$ is acyclic as well.
            \end{lemma}

            \begin{proof}
                Write diagrams here
            \end{proof}

            Since $Ac(\mathcal{A})$ is triangulated, it makes sense to talk about the class of morphisms $Mor_{Ac(\mathcal{A})}$. Observe that in the case of the category $\mathcal{A}$ being abelian, we see that this is the class of quasi-isomorphic chain maps. Thus we may regard the class of morphisms $Mor_{Ac(\mathcal{A})}$ as quasi-isomorphisms.

            \begin{definition}
                The derived category is the Verdier quotient $D(\mathcal{A})=K(\mathcal{A})/Ac(\mathcal{A})$ whenever it exists. 
            \end{definition}

            Given some conditions on the category $(\mathcal{A},\mathcal{E})$ we are able to get a nice description of the derived category. That is we want the category of acyclic chain complexes to be closed under isomorphisms. The following proposition tells us whener this is true.

            \begin{lemma}
                The following are equivalent:
                \begin{enumerate}
                    \item Every null-homotopic chain complex is acyclic
                    \item The category $\mathcal{A}$ is idempotent complete
                    \item The subcategory $Ac(\mathcal{A})$ is closed under isomorphisms
                \end{enumerate}
            \end{lemma}

            \begin{proof}
                
            \end{proof}

            \begin{corollary}
                The subcategory $Ac(\mathcal{A})$ is thick if and only if $\mathcal{A}$ is idempotent complete.
            \end{corollary}

            We may also set boundedness conditions on the chain complex to get weaker assumptions to get nice acyclic chain complexes. We say that a chain complex is called left bounded if there is some $m$ such that for any $n$ $n\leq m<0$ we have that $A^n = 0$. Likewise, right bounded complexes are the defined for $n\geq m>0$ we have that $A^n=0$. A chain complex is called bounded if it is both left bounded and right bounded. 

            \begin{definition}
                The category $K(\mathcal{A})^+$,$K(\mathcal{A})^-$ and $K(\mathcal{A})^{\flat}$ are the homotopy categories of left bounded, right bounded and bounded respectively. $Ac(\mathcal{A})^* \subset K(\mathcal{A})^*$ for $*:\{+,-,\flat\}$ will be the subcategory of acyclic chain complexes satisfying the correct boundedness condition.
            \end{definition}

            \begin{lemma}
                The following are equivalent:
                \begin{enumerate}
                    \item The subcategories $Ac(\mathcal{A})^* \subset K(\mathcal{A})^*$ for $*:\{+,-\}$ are thick
                    \item The subcategory $Ac(\mathcal{A})^{\flat}$
                    \item The category $\mathcal{A}$ is weakly idempotent complete
                \end{enumerate}
            \end{lemma}

        \subsection{Auslander-Reiten triangles}

            Define AR-triangles and prove the basic properties about them, and how they are acting in the derived category.
            
        \subsection{Description of Derived Categories}

            Add something to the introduction about derived categories. The goal is to describe the derived category of a fairly simple algebra.

    \clearpage

    \nocite{*}
    \bibliographystyle{unsrt}
    \bibliography{bib}

    \clearpage

    %    \section*{Appendix A: Category theory}
    %    \subsection{Quotient Categories}
    %    \subsection{Additive and Abelian Categories}
    %    \subsection{Freyd-Mitchell Embedding???}

    %\clearpage
\end{document}