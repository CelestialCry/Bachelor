\documentclass[12pt]{article}
\usepackage[utf8]{inputenc}
\usepackage[english]{babel}
\usepackage{graphicx}
\graphicspath{{../images/}{../../images/}}

\usepackage{tikz}
\usetikzlibrary{cd}
\usepackage{amsthm}

\newtheorem{theorem}{Theorem}[section]
\newtheorem{corollary}{Corollary}[theorem]
\newtheorem{lemma}[theorem]{Lemma}

\theoremstyle{definition}
\newtheorem{definition}{Definition}[section]

\theoremstyle{remark}
\newtheorem*{remark}{Remark}

\usepackage{subfiles} % Best loaded last in the preamble

\title{Thesis}
\author{Thomas Wilskow Thorbjørnsen}
\date{\today}

\begin{document}
    \maketitle
    \section{Introduction}
    
    Write about reasons for writing this text, who is it meant for etc?

    Maybe write some history of triangulated and derived categories and where they find their uses, etc?

    Introduce notation which will be used in text.
    When an 
    \section{Triangulated Categories}
        Probably introduce this section, what is happening and what will be done etc.
        \subsection{Definition and first properties}
        In this section $\mathcal{T}$ denotes an additive category and $T:\mathcal{T}\rightarrow\mathcal{T}$ is an additive autoequivalence of $\mathcal{T}$.
        \begin{definition}
            A sextuple is a collection $(A,B,C,a,b,c)$ of objects \\ $A,B,C\in T$ and morphisms $a:A\rightarrow B$, $b:B\rightarrow C$, $c:C\rightarrow TA$. These sextuples can be drawn as diagrams in the following way:

            \begin{center}
                \begin{tikzcd}
                    A \arrow{r}{a} & B \arrow{r}{b} & C \arrow{r}{c} & TA
                \end{tikzcd}
            \end{center}
        \end{definition}

        When these sextuples fulfill some extra conditions they are referred to as triangles. This name stems from an alternate description of these diagrams:

        \begin{center}
            \begin{tikzcd}[row sep=tiny]
                A \arrow{rd}{a} & \\
                & B \arrow{dl}{b} \\
                C \arrow[very near end, "|" marking]{uu}[near start]{c}[near end]{T}
            \end{tikzcd}
        \end{center}

        Describe a morphism of sextuples here.

        \begin{definition}
            A \emph{triangulated category} is a triple ($\mathcal{T}$, T, $\Delta$), where $\mathcal{T}$ is an additive category, T is an autoequivalence on $\mathcal{T}$ and a triangulation $\Delta$ which is a collection of sextuples satisfying the following axioms:

            \begin{enumerate}
                \item Formation axiom
                
                \item Rotation axiom
                \item Morphism axiom
                \item Octahedron axiom
            \end{enumerate}
        \end{definition}
        %Maybe use later if this file gets way too crammed %\subfile{sections/triangles}
    \section{Exact Categories (and the Frobenius category)}
    \section{The Derived Categories (of Exact Categories)}
    \section{Auslander-Reiten Triangles (in the Derived category)}
\end{document}