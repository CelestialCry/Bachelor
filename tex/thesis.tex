\documentclass[12pt]{article}
\usepackage[utf8]{inputenc}
\usepackage[english]{babel}
\usepackage[margin=1in]{geometry}
\usepackage{graphicx}
\graphicspath{{../images/}{../../images/}}


\setlength{\parindent}{0em}
\setlength{\parskip}{1em}

\usepackage{tikz}
\usetikzlibrary{cd}
\usepackage{amsmath,amsthm,amssymb,amsfonts}  

\newtheorem{theorem}{Theorem}[section]
\newtheorem{corollary}{Corollary}[theorem]
\newtheorem{lemma}[theorem]{Lemma}
\newtheorem{prop}[theorem]{Proposition}

\theoremstyle{definition}
\newtheorem{definition}{Definition}[section]

\theoremstyle{remark}
\newtheorem*{remark}{Remark}

\usepackage{subfiles} % Best loaded last in the preamble

\title{Thesis}
\author{Thomas Wilskow Thorbjørnsen}
\date{\today}

\begin{document}
    \maketitle
    \clearpage

    \section{Introduction}
        This is an introduction, welcome! 

        
        Introduce notation which will be used in text. A list of notation and description would be nice, so that the reader might scroll back up if something is unclear.
    \clearpage
    
    \section{Triangulated Categories}
        Probably introduce this section, what is happening and what will be done etc. I can maybe say something about algebraic triangulated categories and topological triangulated categories, and explaining the name cone, fiber and cofiber.
        
        \subsection{Definition and First Properties}
            In this section $\mathcal{T}$ denotes an additive category and $T:\mathcal{T}\rightarrow\mathcal{T}$ is an additive autoequivalence of $\mathcal{T}$, which is often called translation or suspension functor.
            % Definition Sextuple
            \begin{definition}
                A sextuple is a collection $(A,B,C,a,b,c)$ of objects \\ $A,B,C\in T$ and morphisms $a:A\rightarrow B$, $b:B\rightarrow C$, $c:C\rightarrow TA$. These sextuples can be drawn as diagrams in the following way:

                \begin{center}
                    \begin{tikzcd}
                        A \arrow{r}{a} & B \arrow{r}{b} & C \arrow{r}{c} & TA
                    \end{tikzcd}
                \end{center}

                A morphism between sextuples is a triple of morphism $(\alpha, \beta, \gamma)$, where $\alpha : A \rightarrow A'$, $\beta : B \rightarrow B'$ and $\gamma : C \rightarrow C'$ such that the following diagram commutes:

            \begin{center}
                \begin{tikzcd}
                    A \arrow{r}{a} \arrow{d}{\alpha} & B \arrow{r}{b} \arrow{d}{\beta} & C \arrow{r}{c} \arrow{d}{\gamma} & TA \arrow{d}{T\alpha} \\
                    A' \arrow{r}{a'} & B' \arrow{r}{b'} & C \arrow{r}{c'} & TA'
                \end{tikzcd}
            \end{center}

            \end{definition}

            The naming convention of the sextuples isn't standarized, some literatures calls the sextuples for triangles instead [literature here, learn bibtex you lazy fuck]. This name arises from an alternate description of the diagrams given above. To remove confusion about the domain or codomain of the arrows, one arrow of the triangle is decorated with "$_T$|". This decorator means that the functor T has to be applied to the corresponding edge of the arrow. Thus the c arrow points to TA, not A.

            \begin{center}
                \begin{tikzcd}[row sep=tiny]
                    A \arrow{rd}{a} & \\
                    & B \arrow{dl}{b} & & \\
                    C \arrow[very near end, "|" marking]{uu}[near start]{c}[near end]{T}
                \end{tikzcd}
                \begin{tikzcd}[row sep=tiny]
                    A \arrow{rd}[description]{a} \arrow{rrr}{\phi_a} & & & A' \arrow[ld, "a'" description] \\
                    & B \arrow{dl}[description]{b} \arrow{r}{\phi_b} & B' \arrow{rd}[description]{b'}\\
                    C \arrow{uu}[very near end, marking]{|}[near start, description]{c}[near end]{T} \arrow{rrr}{\phi_c} & & & C' \arrow{uu}[very near end, marking]{|}[near start, description]{c'}[near end]{T}
                \end{tikzcd}
            \end{center}

            A triangulated category is an additive category together with a translation functor $T$ and a triangulation $\Delta$ consisting of sextuples. When a sextuple is an element of $\Delta$ it is usually called a distinguished triangle, an exact triangles or just a triangle. Note that if sextuples are referred to as triangles it is common to either call the elements of $\Delta$ for distinguished triangles or exact triangles. As this is not the case for this thesis these objects will be referred to as triangles.

            % Definition of Triangulation
            \begin{definition}
                A triangulation of an additive category $\mathcal{T}$ with translation $T$ is a collection $\Delta$ of triangles consisting of sextuples in $\mathcal{T}$ satisfying the following axioms: 

                \begin{enumerate}
                    \item (TR1) Formation axiom

                        \begin{enumerate}
                            \item A sextuple isomorphic to a triangle is a triangle.
                            \item Every morphism $a : A \rightarrow B$ can be embedded into a triangle $(A,B,C,a,b,c)$:
                            \begin{center}
                                \begin{tikzcd}
                                    A \arrow{r}{a} & B \arrow{r}{b} & C \arrow{r}{c} & TA
                                \end{tikzcd}
                            \end{center}
                            \item For every object A there is a triangle $(A,A,0,id_A,0,0)$:
                            \begin{center}
                                \begin{tikzcd}
                                    A \arrow{r}{id_A} & A \arrow{r}{0} & 0 \arrow{r}{0} & TA
                                \end{tikzcd}
                            \end{center}
                        \end{enumerate}
                    \item (TR2) Rotation axiom

                        For every triangle $(A,B,C,a,b,c)$ there is a triangle $(B,C,TA,b,c,Ta)$
                        \begin{center}
                            \begin{tikzcd}
                                A \arrow{r}{a} & B \arrow{r}{b} & C \arrow{r}{c} & TA
                            \end{tikzcd} $\implies$
                            \begin{tikzcd}
                                B \arrow{r}{b} & C \arrow{r}{c} & TA \arrow{r}{-Ta} & TB
                            \end{tikzcd}
                            
                        \end{center}
                    \item (TR3) Morphism axiom
                    
                        Given the two triangles $(A,B,C,a,b,c)$ (1) and $(A',B',C',a',b',c')$ (2)
                        \begin{center}
                            (1)
                            \begin{tikzcd}[column sep=small]
                                A \ar{r}{a} & B \ar{r}{b} & C \ar{r}{c} & TA
                            \end{tikzcd}
                            (2) 
                            \begin{tikzcd}[column sep=small]
                                A' \ar{r}{a'} & B' \ar{r}{b'} & C' \ar{r}{c'} & TA'
                            \end{tikzcd}
                        \end{center}
                        and morphisms $\phi_A : A \rightarrow A'$ and $\phi_B : B \rightarrow B'$ such that the square (1) commutes, then there is a morphism $\phi_C : C \rightarrow C'$ (not necessarily unique) such that $(\phi_A ,\phi_B ,\phi_C)$ is a morphism of triangles (2).
                        
                        \begin{center}
                            (1)
                            \begin{tikzcd}
                                A \ar{r}{a} \ar{d}{\phi_A} & B \ar{d}{\phi_B} & \\
                                A' \ar{r}{a'} & B'
                            \end{tikzcd}
                            (2)
                            \begin{tikzcd}
                                A \ar{r}{a} \ar{d}{\phi_A} & B \ar{r}{b} \ar{d}{\phi_B} & C \ar{r}{c} \ar[dashed]{d}{\phi_C} & TA \ar{d}{T\phi_A} \\
                                A' \ar{r}{a'} & B' \ar{r}{b'} & C' \ar{r}{c'} & TA'
                            \end{tikzcd}
                        \end{center}
                    \item (TR4) Octahedron axiom
                    
                        Given the triangles $(A,B,C',a,x,x')$ (1), $(B,C,A',b,y,y')$ (2) \\ and $(A,C,B',b\circ a,z,z')$ (3)
                        \begin{center}
                            (1)
                            \begin{tikzcd}[column sep=small]
                                A \ar{r}{a} & B \ar{r}{x} & C' \ar{r}{x'} & TA
                            \end{tikzcd}

                            (2)
                            \begin{tikzcd}[column sep=small]
                                B \ar{r}{b} & C \ar{r}{y} & A' \ar{r}{y'} & TB
                            \end{tikzcd}
        
                            (3)
                            \begin{tikzcd}[column sep=small]
                                A \ar{r}{b\circ a} & C \ar{r}{z} & B' \ar{r}{z'} & TA
                            \end{tikzcd}                         
                        \end{center}
                        then there exist morphisms $f : C' \rightarrow B'$ and $g : B' \rightarrow A'$, the following diagram commutes and the third row is a triangle:

                        \begin{center}
                            \begin{tikzcd}
                                T^{-1}B' \ar{r}{T^{-1}z'} \ar{d}{T^{-1}g} & A \ar[equal]{r}{id_A} \ar{d}{a} & A \ar{d}{b\circ a} \\
                                T^{-1}A' \ar{r}{T^{-1}y'} & B \ar{r}{b} \ar{d}{x} & C \ar{r}{y} \ar{d}{z} & A' \ar{r}{y'} \ar[equal]{d}{id_{A'}} & TB \ar{d}{Tx'} \\
                                & C' \ar{r}{f} \ar{d}{x'} & B' \ar{r}{g} \ar{d}{z'} & A' \ar{r}{Ti \circ y'} & TC' \\
                                & TA \ar[equal]{r}{id_{TA}} & TA
                            \end{tikzcd}
                        \end{center}
                \end{enumerate}
            \end{definition}

            A triangulated category is denoted as $(\mathcal{T}, T, \Delta)$, where $\mathcal{T}$ is the additive category, $T$ is the translation and $\Delta$ is the triangulation. When $\mathcal{T}$ is called a triangulated category, it should be understanded as the triple given above.
            % Rotation axiom dual
            \begin{remark}
                The rotation axiom has a dual, and it can be thought of as a rotation in the opposite direction. This dual can be proved by the other axioms, so it is here omitted as an axiom. The dual roation axiom goes as:

                Given a triangle \begin{tikzcd}[column sep=small]
                    A \ar{r}{a} & B \ar{r}{b} & C \ar{r}{c} & TA
                \end{tikzcd}, there is a triangle \begin{tikzcd}[column sep=small]
                    T^{-1}C \ar{r}{-T^{-1}c} & A \ar{r}{a} & B \ar{r}{b} & C
                \end{tikzcd} \\ %Hvordan fikser jeg linebreaks for at det blir pent??? Mener at denne skal kun brukes innad i en setning, idk man.
                To be able to prove this, some more lemmata are needed.
            \end{remark}
            % Octahedron axiom alternate
            \begin{remark}
                The final axiom is referred to as the octahedron axiom. By using the alternative description of the triangle diagram, it is possible to rewrite the diagram as an octahedron. The axiom can be restated as the following:
                
                Given the triangles $(A,B,C',a,x,x')$ (1), $(B,C,A',b,y,y')$ (2) \\ and $(A,C,B',b\circ a,z,z')$ (3)
                \begin{center}
                    (1)
                    \begin{tikzcd}[row sep=tiny]
                        A \arrow[red]{rd}[black]{a} & \\
                        & B \arrow[red]{dl}[black]{x} & & \\
                        C' \arrow[red, very near end, "|" marking]{uu}[near start, black]{x'}[near end]{T}
                    \end{tikzcd}
                    (2)
                    \begin{tikzcd}[row sep=tiny]
                        B \arrow[orange]{rd}[black]{b} & \\
                        & C \arrow[orange]{dl}[black]{y} & & \\
                        A' \arrow[orange, very near end, "|" marking]{uu}[near start, black]{y'}[near end]{T}
                    \end{tikzcd}
                    (3)
                    \begin{tikzcd}[row sep=tiny]
                        A \arrow[violet]{rd}[black]{b\circ a} & \\
                        & C \arrow[violet]{dl}[black]{z} & & \\
                        B' \arrow[violet, very near end, "|" marking]{uu}[near start, black]{z'}[near end]{T}
                    \end{tikzcd}
                \end{center}
                then there exists morphisms $f: C' \rightarrow B'$ and $g: B' \rightarrow A'$, the following diagram commutes and the teal back face is a triangle.
                \begin{center}
                    \begin{tikzcd}[row sep=tiny]
                        . & & B' \ar[teal, dashed]{ddddr}[black, description]{g} \ar[violet]{dddddl}[black, description]{z'}[pos=0.9, marking]{|}[pos=0.91]{T} & & \\
                        . \\
                        . \\
                        . \\
                        C' \ar[teal]{uuuurr}[black, description]{f} \ar[red]{dr}[black, description]{x'}[very near end, marking]{|}[near end]{T} & & & A' \ar[teal, dashed]{lll}[pos=0.45, black, description]{Tx\circ y'}[pos=0.91, marking]{|}[very near end, above]{T} \ar[orange, dashed]{dddddl}[black, description]{y'}[pos=0.9, marking]{|}[pos=0.89, above]{T} \\
                        . & A \ar[red]{ddddr}[black, description]{a} \ar[violet]{rrr}[black, description]{b\circ a} & & & C \ar[orange, dashed]{ul}[black, description]{y} \ar[violet]{uuuuull}[black, description]{z}\\
                        . \\
                        . \\
                        . \\
                        & & B \ar[orange]{uuuurr}[black, description]{b} \ar[red]{uuuuull}[black, description]{x} & &

                    \end{tikzcd}
                \end{center}
            \end{remark}

            \begin{lemma}
                Let $(A,B,C,a,b,c)$ be a triangle, then $b\circ a=0$
            \end{lemma}

            \begin{proof}
                By TR2 the triangle $(A,B,C,a,b,c)$ can be rotated to $(B,C,TA,b,c,Ta)$.
                \begin{center}
                    \begin{tikzcd}[row sep=tiny]
                        A \arrow{rd}{a} & \\
                        & B \arrow{dl}{b}\\
                        C \arrow[very near end, "|" marking]{uu}[near start]{c}[near end]{T}
                    \end{tikzcd} $\implies$
                    \begin{tikzcd}[row sep=tiny]
                        B \arrow{rd}{b} \\
                        & C \arrow{dl}{c} \\
                        TA \arrow{uu}[near start]{-Ta}[very near end, marking]{|}[near end]{T}
                    \end{tikzcd}
                \end{center}
                The triangle exists $(C,C,0,id_C,0,0)$ by TR1 and TR3 says there exists a morphism from TA to 0 making the diagram below commute.
                \begin{center}
                    \begin{tikzcd}
                        B \ar{r}{b} \ar{d}{b} & C \ar{r}{c} \ar{d}{id_C} & TA \ar{r}{-Ta} \ar[dashed]{d}{0} & TB \ar{d}{Tb} \\
                        C \ar{r}{id_C} & C \ar{r}{0} & 0 \ar{r}{0} & TC
                    \end{tikzcd}
                \end{center}
                Thus $0 = Tb\circ -Ta = T(-ba) \implies b\circ a = 0$ as T is a translation.
            \end{proof}
            % Triangulatd functor
            \begin{definition}
                An additive functor between triangulated categories $F: (\mathcal{T}, T, \Delta) \rightarrow (\mathcal{R}, R, \Gamma)$ is called exact or triangulated if there exist a natural isomorphisms $\alpha : FT \rightarrow RF$ such that $F(\Delta) \subseteq \Gamma$.

                A functor $F : \mathcal{T} \rightarrow \mathcal{R}$ is called a triangle-equivalence if it is triangulated and an equivalence of categories. In this case $\mathcal{T}$ and $\mathcal{R}$ are called triangle-equivalent.
            \end{definition}
            % Homological functor
            \begin{definition}
                Let $\mathcal{T}$ be a triangulated category and $\mathcal{A}$ be an abelian category. A covariant functor $H:\mathcal{T} \rightarrow \mathcal{A}$ is called a homological functor if $\forall (A,B,C,a,b,c):\Delta$ there is a long exact sequence in $\mathcal{A}$.
                \begin{center}
                    \begin{tikzcd}[row sep=tiny]
                        A \arrow{rd}{a} & \\
                        & B \arrow{dl}{b}\\
                        C \arrow[very near end, "|" marking]{uu}[near start]{c}[near end]{T}
                    \end{tikzcd} $\implies$
                    \begin{tikzcd}[column sep=small]
                        ... \ar{r} & H(T^{i}A) \arrow{r}{H(T^ia)} & H(T^iB)\arrow{r}{H(T^ib)} \arrow[d,phantom, ""{coordinate, name=Z}]& H(T^iC) \arrow[dll, "H(T^ic)" description, rounded corners,to path={ --([xshift=2ex]\tikztostart.east)|- (Z)[near end]\tikztonodes-| ([xshift=-2ex]\tikztotarget.west)-- (\tikztotarget)}] \\
                        & H(T^{i+1}A) \arrow{r}{H(T^{i+1}a)} & H(T^{i+1}B) \arrow{r}{H(T^{i+1}b)} & H(T^{i+1}C) \ar{r} & ...
                    \end{tikzcd}
                \end{center}

                Dually, a contravariant functor $H:\mathcal{T} \rightarrow \mathcal{A}$ is called cohomological if $\forall (A,B,C,a,b,c):\Delta$ there is a long exact sequence in $\mathcal{A}$.
                \begin{center}
                    \begin{tikzcd}[row sep=tiny]
                        A \arrow{rd}{a} & \\
                        & B \arrow{dl}{b}\\
                        C \arrow[very near end, "|" marking]{uu}[near start]{c}[near end]{T}
                    \end{tikzcd} $\implies$
                    \begin{tikzcd}[column sep=small]
                        ... & H(T^{i-1}A) \arrow{l} & H(T^{i-1}B) \arrow{l}{H(T^{i-1}a)} \arrow[d,phantom, ""{coordinate, name=Z}]& H(T^{i-1}C) \ar{l}{H(T^{i-1}b)} \\
                        & H(T^{i}A) \arrow[urr, "H(T^ic)" description, rounded corners,to path={ --([xshift=-2ex]\tikztostart.west)|- (Z)[near end]\tikztonodes-| ([xshift=2ex]\tikztotarget.east)-- (\tikztotarget)}] & H(T^{i}B) \ar{l}{H(T^ia)} & H(T^{i}C) \ar{l}{H(T^ib)} & ... \ar{l}
                    \end{tikzcd}
                \end{center}
            \end{definition}
            % Long exact sequence of representations
            \begin{lemma}
                Let $M:\mathcal{T}$ be any object of $\mathcal{T}$, then the represented functors $\mathcal{T}(M,\_)$ is homological and $\mathcal{T}(\_,M)$ is cohomological.
            \end{lemma}

            \begin{proof}
                Only the covariant case needs to be proved, as the contravariant case is dual. For $\mathcal{T}(M,\_)$ to be homological, it has to create long exact sequences for every triangle in $\Delta$. Let $(A,B,C,a,b,c):\Delta$ be a triangle, then there can be extracted sequences in Ab for any $i:\mathbb{N}$.

                \begin{center}
                    \begin{tikzcd}[row sep=tiny]
                        A \ar{dr}{a} \\
                        & B \ar{dl}{b} \\
                        C \ar{uu}{c}[near end]{T}[very near end, marking]{|}
                    \end{tikzcd} $\implies$
                    \begin{tikzcd}
                        \mathcal{T}(M,T^iA) \ar{r}{T^ia_*} & \mathcal{T}(M,T^iB) \ar{r}{T^ib_*} & \mathcal{T}(M,T^iC)
                    \end{tikzcd}
                \end{center}
                Observe that it is enough to prove that these types of diagrams are exact, as the other diagrams can be obtained by the rotation axiom, thus reducing it to same case. 
                
                The goal is then to prove that $Im(T^ia_*)=Ker(T^ib_*)$. Since $ba=0$ it follows that $Im(T^ia_*) \subseteq Ker(T^ib_*)$. Assume that $f:Ker(T^ib_*)$, that is $f:M\rightarrow T^iB$ such that $b_*(f)=0$. The current goal is to show that $f$ factors through $T^iA$, as this means that $Ker(T^ib_*)\subseteq Im(T^ia_*)$. Note that since $T$ is a translation, it is necessarily a right adjoint to the inverse translation; thus $\mathcal{T}(M,T^iB) \simeq\mathcal{T}(T^{-i}M,B)$, and by this assertion it suffices to assume that $f:T^{-i}M\rightarrow B$ such that $b\circ f = 0$. By TR1 and TR2 there exists triangles $(T^{-i}M,0,T^{-i+1}M,0,0,-T^{-i+1}id)$ and $(B,C,TA,b,c,-Ta)$. 
                \begin{center}
                    \begin{tikzcd}
                        T^{-i}M \ar{r}{0} \ar{d}{f} & 0 \ar{r}{0} \ar{d}{0} & T^{-i+1}M \ar{r}{-T^{-i+1}id} \ar[dashed]{d}{g} & T^{-i+1}M \ar{d}{Tf} \\
                        B \ar{r}{b} & C \ar{r}{c} & TA \ar{r}{-Ta} & TB
                    \end{tikzcd}
                \end{center}
                The left square commutes by the assumption, thus the morphism g exist by TR3, such that $-Ta\circ h = -Tf\circ T^{-i+1}id = -Tf \implies Ta\circ h = Tf$, thus $f = a\circ T^{-1}h$ asserting that $f$ factors through A.
            \end{proof}
            % 2 out of 3 lemma
            \begin{lemma}
                Let $(\phi_A, \phi_B, \phi_C):(A,B,C,a,b,c) \rightarrow (A',B',C',a',b',c')$ be a morphism of triangles. If 2 of the maps are isomorphisms, then the last one is an isomorphism as well.
                \begin{center}
                    \begin{tikzcd}
                        A \ar{r}{a} \ar{d}{\phi_A}[rotate=90, above]{\simeq} & B \ar{r}{b} \ar{d}{\phi_B}[rotate=90, above]{\simeq} & C \ar{r}{c} \ar[dashed]{d}{\phi_C}[rotate=90, above]{\simeq} & TA \ar{d}{T\phi_A}[rotate=90, above]{\simeq} \\
                        A' \ar{r}{a'} & B' \ar{r}{b'} & C' \ar{r}{c'} & TA'
                    \end{tikzcd}
                \end{center}
            \end{lemma}

            \begin{proof}
                Without loss of generality, assume that $\phi_A$ and $\phi_B$ are the isomorphisms. This can be done as the rotation axiom reduce the other cases to this case. Then we have the following diagram:
                \begin{center}
                    \begin{tikzcd}
                        A \ar{r}{a} \ar{d}{\phi_A}[rotate=90, above]{\simeq} & B \ar{r}{b} \ar{d}{\phi_B}[rotate=90, above]{\simeq} & C \ar{r}{c} \ar{d}{\phi_C} & TA \ar{d}{T\phi_A}[rotate=90, above]{\simeq} \\
                        A' \ar{r}{a'} & B' \ar{r}{b'} & C' \ar{r}{c'} & TA'
                    \end{tikzcd}
                \end{center}
                By applying the functor $\mathcal{T}(C',\_)$ we get the following diagram in Ab:
                \begin{center}
                    \begin{tikzcd}
                        \mathcal{T}(C',A) \ar{r}{a_*} \ar{d}{(\phi_A)_*}[rotate=90, above]{\simeq} & \mathcal{T}(C',B) \ar{r}{b_*} \ar{d}{(phi_B)_*}[rotate=90, above]{\simeq} & \mathcal{T}(C',C) \ar{r}{c_*} \ar{d}{(\phi_C)_*} & \mathcal{T}(C',TA) \ar{r}{Ta_*} \ar{d}{(\phi_TA)_*}[rotate=90, above]{\simeq} & \mathcal{T}(C',TB) \ar{d}{(T\phi_B)_*}[rotate=90, above]{\simeq} \\
                        \mathcal{T}(C',A') \ar{r}{a'_*} & \mathcal{T}(C',B') \ar{r}{b'_*} & \mathcal{T}(C',C') \ar{r}{c'_*} & \mathcal{T}(C',TA') \ar{r}{Ta_*} & \mathcal{T}(C',TB)
                    \end{tikzcd}
                \end{center}
                By the 5-lemma, we get that $(\phi_C)_*$ is an isomorphisms, i.e. $(\phi_C)_*$ is both mono and epi. Thus for some unique $s$ in $\mathcal{T}(C',C)$, ${\phi_C}_*(s)=id_{C'}$. 

                By applying the functor $\mathcal{T}(\_,C)$ we get the diagram:
                \begin{center}
                    \begin{tikzcd}
                        \mathcal{T}(A,C) & \mathcal{T}(B,C) \ar{l}{a^*} & \mathcal{T}(C,C) \ar{l}{b^*} & \mathcal{T}(TA,C) \ar{l}{c^*} & \mathcal{T}(TB,C) \ar{l}{Ta^*} \\
                        \mathcal{T}(A',C) \ar{u}{(\phi_A)^*}[rotate=90, below]{\simeq} & \mathcal{T}(B,C) \ar{l}{a'^*} \ar{u}{(\phi_B)^*}[rotate=90, below]{\simeq} & \mathcal{T}(C',C) \ar{l}{b'^*} \ar{u}{(\phi_C)^*} & \mathcal{T}(TA',C) \ar{l}{c'^*} \ar{u}{(\phi_TA)^*}[rotate=90, below]{\simeq} & \mathcal{T}(TB',C) \ar{l}{Ta'^*} \ar{u}{(\phi_TB)^*}[rotate=90, below]{\simeq}
                    \end{tikzcd}
                \end{center}
                By the 5-lemma, we get that $(\phi_C)^*$ is an isomorphisms. By the same argument $id_{C} = s'\circ\phi_C$ for some unique $s'$. $\phi_C$ is both split mono and split epi, which means it is an isomorphism.
            \end{proof}

            \begin{corollary}
                $(A,B,0,a,0,0)$ is a triangle if and only if a is an isomorphism.
            \end{corollary}

            \begin{proof}
                Assume that a is an isomorphism. Then it is seen that $(a,id_B,0)$ is an isomorphism of triangles.
                \begin{center}
                    \begin{tikzcd}
                        A \ar{r}{a} \ar{d}{a}[rotate=90, above]{\simeq} & B \ar{r}{0} \ar{d}{id_B}[rotate=90, above]{\simeq} & 0 \ar{r}{0} \ar{d}{0}[rotate=90, above]{\simeq} & TA \ar{d}{Ta}[rotate=90, above]{\simeq} \\
                        B \ar{r}{id_B} & B \ar{r}{0} & 0 \ar{r}{0} & TB
                    \end{tikzcd}
                \end{center}
                Converesly, assume that $(A,B,0,a,0,0)$ is a triangle. Then the same diagram as above can be constructed, and by the 2 out of 3 property, a has to be an isomorphism.
            \end{proof}

            \begin{lemma}
                For a triangle $(A,B,C,a,b,c)$ the following are equivalent:

                \begin{center}
                    \begin{minipage}[c]{0.3\textwidth}
                        \begin{tikzcd}[row sep=tiny]
                            A \arrow{rd}{a} & \\
                            & B \arrow{dl}{b} & & \\
                            C \arrow[very near end, "|" marking]{uu}[near start]{c}[near end]{T}
                        \end{tikzcd}
                    \end{minipage}
                    \begin{minipage}[c]{0.3\textwidth}
                        \begin{itemize}
                            \item $a$ is split mono
                            \item $b$ is split epi
                            \item $c = 0$
                        \end{itemize}
                    \end{minipage}
                \end{center}
            \end{lemma}

            \begin{proof}
                The proof has two parts. First assume that $a$ is split mono, and prove that $b$ is split epi, and $c = 0$. By duality, it is then known that $b$ being split epi implies that $a$ is split mono and $c = 0$. The final part is to assume that $c = 0$, and prove either $a$ is split mono or $b$ is split epi.

                Assume that $a$ is split mono, then there exist an $a^{-1}$ such that $id_A = a^{-1}a$. Let $M:\mathcal{T}$ be any object, then we can make a long exact sequence:
                \begin{center}
                    \begin{tikzcd}
                        \mathcal{T}(M,T^{-1}C) \ar{r}{T^{-1}c_*} & \mathcal{T}(M,A) \ar[bend left]{r}{a_*} & \mathcal{T}(M,B) \ar{r}{b_*} \ar[dashed, bend left]{l}{a^{-1}_*} & \mathcal{T}(M,C) \ar{r}{c_*} & \mathcal{T}(M,TA)
                    \end{tikzcd}
                \end{center}
                By assumption $a_*$ is split mono, thus $T^{-1}c_* = 0$ and in particular $c = 0$. This implies that $b_*$ is epi, making a split short exact sequence.
                \begin{center}
                    \begin{tikzcd}
                        0 \ar{r}{0} & \mathcal{T}(M,A) \ar[bend left]{r}{a_*} & \mathcal{T}(M,B) \ar[bend left]{r}{b_*} \ar[dashed, bend left]{l}{a^{-1}_*} & \mathcal{T}(M,C) \ar{r}{0} \ar[dashed, bend left]{l}{b^{-1}_*} & 0
                    \end{tikzcd}
                \end{center}
                This gives that b is split epi, completing the first part of the proof.

                For the next part, assume that $c = 0$; then we can construct the following triangles.
                \begin{center}
                    (1)
                    \begin{tikzcd}[row sep=tiny]
                        A \arrow{rd}{a} \\
                        & B \arrow{dl}{b} \\
                        C \arrow[very near end, "|" marking]{uu}[near start]{0}[near end]{T}
                    \end{tikzcd} $\implies$
                    \begin{tikzcd}[row sep=tiny]
                        C \arrow{rd}{0} \\
                        & TA \arrow{dl}{-Ta} \\
                        TB \arrow[very near end, "|" marking]{uu}[near start]{-Tb}[near end]{T}
                    \end{tikzcd}
                \end{center}
                \begin{center}
                    (2)
                    \begin{tikzcd}[row sep=tiny]
                        A \arrow{rd}{id_A} \\
                        & A \arrow{dl}{0} \\
                        0 \arrow[very near end, "|" marking]{uu}[near start]{0}[near end]{T}
                    \end{tikzcd} $\implies$
                    \begin{tikzcd}[row sep=tiny]
                        0 \arrow{rd}{0} & \\
                        & TA \arrow{dl}{-id_{TA}}\\
                        TA \arrow[very near end, "|" marking]{uu}[near start]{0}[near end]{T}
                    \end{tikzcd}
                \end{center}
                (1) is constructed by applying TR2 twice, while (2) is constructed with TR1 and TR2 twice. Observe that there is a commutative square between the triangles, allowing for TR3 to make a morphism of triangles.
                \begin{center}
                    \begin{tikzcd}
                        C \ar{r}{0} \ar{d}{0} & TA \ar{r}{-Ta} \ar[equal]{d}{id_{TA}} & TB \ar{r}{-Tb} \ar[dashed]{d}{Ta^{-1}} & TC \ar{d}{0} \\
                        0 \ar{r}{0} & TA \ar[equal]{r}{-id_{TA}} & TA \ar{r}{0} & 0
                    \end{tikzcd}
                \end{center}
                Thus $T(a^{-1}a)=id_{TA}=T(id_A) \implies id_A = a^{-1}a$, making a split mono.
            \end{proof}

            \begin{lemma}
                Given two triangles $(A,B,C,a,b,c)$ and $(A',B',C',a',b',c')$ the following are equivalent:
                \begin{center}
                    \begin{minipage}[c]{0.4\textwidth}
                        \begin{tikzcd}
                            A \ar{r}{a} \ar{d}{f} & B \ar{r}{b} \ar{d}{g} & C \ar{r}{c} \ar{d}{h} & TA \ar{d}{Tf} \\
                            A' \ar{r}{a'} & B' \ar{r}{b'} & C' \ar{r}{c'} & TA'
                        \end{tikzcd}
                    \end{minipage}
                    \begin{minipage}[c]{0.5\textwidth}
                        \begin{enumerate}
                            \item $(f,g,h)$ is a morphism of triangles
                            \item $\exists g:B\rightarrow B'$ such that $b'ga = 0$
                        \end{enumerate}
                    \end{minipage}
                \end{center}
                Moreover, if $\mathcal{T}(A,T^{-1}C')\simeq 0$, then f and h are unique.
            \end{lemma}

            \begin{proof}
                $1. \implies 2.$ as the composition $ba = 0 = b'a'$, so assume 2. The existence of $f$ and $h$ is evident from the long exact sequence of the bottom triangle at the functor represented by $A$. 
                \begin{center}
                    \begin{tikzcd}
                        \mathcal{T}(A,T^{-1}C') \ar{r}{T^{-1}c'_*} & \mathcal{T}(A,A') \ar{r}{a'_*} & \mathcal{T}(A,B') \ar{r}{b'_*} & \mathcal{T}(A,C')
                    \end{tikzcd}
                \end{center}
                The morphism $ga:\mathcal{T}(A,B')$ such that $b'ga=b'_*(ga)=0$, thus $ga:Ker(b'_*)$. By exactness $\exists f:\mathcal{T}(A,A')$ such that $a'f = ga$, and by TR3 $\exists h: C \rightarrow C'$ such that $(f,g,h)$ is a morphism of triangles.
                Now assume that $\mathcal{T}(A,T^{-1}C')\simeq 0$. Exactness determines that $a'_*$ is a monomorphism, and $f$ is then unique. Since $T$ is a translation, we have that $\mathcal{T}(A,T^{-1}C')\simeq\mathcal{T}(TA,C')$. By using the functor $\mathcal{T}(\_,C')$ at the top triangle, we get that $b^*$ is a monomorphism, and thus $h$ is chosen uniquely.
            \end{proof}

            \begin{lemma}
                If $(A,B,C,a,b,c)$ is a triangle, then $(T^{-1}C,A,B,T^{-1}c,a,b)$ is a triangle.
            \end{lemma}

            \begin{proof}
                By TR2 we can construct a triangle.
                \begin{center}
                    \begin{tikzcd}[row sep=tiny]
                        A \ar{rd}{a} \\
                        & B \ar{ld}{b} \\
                        C \ar{uu}[near start]{c}[very near end, marking]{|}[near end]{T}
                    \end{tikzcd}
                    $\implies$
                    \begin{tikzcd}[row sep=tiny]
                        C \ar{rd}{c} \\
                        & TA \ar{ld}{Ta} \\
                        TB \ar{uu}[near start]{Tb}[very near end, marking]{|}[near end]{T}
                    \end{tikzcd}
                \end{center}
                By TR1 we can create a triangle $(T^{-1}C,A,B',T^{-1}c,a',b')$, and then use TR3 to find a morphism.
                \begin{center}
                    \begin{tikzcd}
                        C \ar{r}{c} \ar[equal]{d}{id_C} & TA \ar{r}{Ta'} \ar[equal]{d}{id_{TA}} & TB' \ar{r}{Tb'} \ar[dashed]{d}{h} & TC \ar[equal]{d}{id_{TC}} \\
                        C \ar{r}{c} & TA \ar{r}{Ta} & TB \ar{r}{Tb} & TC
                    \end{tikzcd}
                \end{center}
                By the 2 out of 3 property it is seen that h is an isomorphism, so the triple $(id_{T^{-1}C}, id_A, T^{-1}h)$ is an isomorphism of sextuples, and by TR1, is an isomorphism of triangles, asserting that $(T^{-1}C,A,B,T^{-1}c,a,b)$ is in fact a triangle.
            \end{proof}

            \begin{remark}
                The Octahedron axiom have not been used once in this section. This motivates the definition of a pre-triangulated category, that is a triangulation satisfying all axioms except TR4. Note that these properties also holds for pre-triangulated categories. However, the octahedron axiom is still important, since TR3 can be proved from the other 3 axioms. [Need sources here]
            \end{remark}
            %Maybe use later if this file gets way too crammed %\subfile{sections/triangles}

        \subsection{Triangulation of \underline{mod}-$K[x]/(x^n)$}
            In this section a quotient of the finitely genereated module categories of the K-algebras $K[x]/(x^n)$ will be studied and proved to be triangulated. As a first example it would be more instructive with a more canonical triangulated category, such as the homotopy category of chain complexes. For such type of categories the underlying triangulated structure would be more intuitive, and it is easier to see how it interplays with long exact sequences in an abelian category. Never the less, this class of categories will be more instructive in showing how flexible the notion of a triangulated category may be.

            The more skilled reader may notice that the algebra $K[x]/(x^n)$ is self-injective and is of finite length. Therefore the category of finitely generated modules over $K[x]/(x^n)$ is a Frobenius category, and the stabilization admits a triangulation. What all of this means will be clarified in the next section, for now we will focus on the structure of this very specific category, and prove that it's stabilized module category has a triangulation.

            When it comes to stabilization of module categories, there is a projective stabilization and an injective stabilization. They use the same method to construct a new category, but uses different classes of objects to form the quotient. We are however interested in the case where these two notions coincide, so we can talk about a stable category.

            \textbf{I don't really like how this section is written, it feels very clumsy, should I postpone the example? I think that would be a lot cleaner as the Frobenius category is the way we actually construct these categories. I AM POSTPONING THIS CHAPTER}

            \begin{definition}
                The injective stable mod-$K[x]/(x^n)$ category is the quotient category of morphisms factoring over injective objects. This means that if $I_1, I_2$ is injective and $f,g:A\rightarrow B$ are morphisms factoring over $I_1$ and $I_2$ respectively, then $f$ and $g$ are related in the quotient. We will denote the category as \underline{mod}-$K[x]/(x^n)$.

                The projective stable categories are defined in the same way, but with projective objects instead.
            \end{definition}

            \begin{remark}
                Cosyzygies are in fact a functor for injective stable categories, and syzygies are functors for the projective stable categories.
            \end{remark}

            Write about the indecomposable objects in this category. Find the projective and injective modules, i.e. show that $K[x]/(x^n)$ is injective.

        \subsection{Localizations of Triangulated Categories}
            Verdier quotient goes here. And more stuff about these localizations, probably state Gabriel-Zisman and how this is often a good trick too prove that the localization actually exists.

        \subsection{Discussion of Triangulations}
            Maybe do some Yoneda-embedding into sheaves and use that to deduce how the triangulation are in fact a shadow of some other abelian category
    
    \clearpage
    
    \section{Exact Categories}
            
        I can maybe write some of the history of the development of the idea of exact categories. 

        \subsection{Definitions and First Properties}

            In this section we will focus on defining what an exact category is and the first elementary properties. We will prove the axiom dubbed as "the obscure axiom" and motivate that it is not as obscure as its name suggest. Some "short" variants of some homological diagram lemmas will also be proved.

            To start with the exact categories we will first take a look towards the abelian ones first. Short exact sequences are of great interest, and they can be characterized with two morphisms $p:A\rightarrow B$ and $q:B\rightarrow C$ such that p is the kernel of q and q is the cokernel of p. This leads to the first definition.

            \begin{definition}
                A kernel-cokernel pair is a pair of maps $(p,q)$ such that p is the kernel of q and q is the cokernel of p. A morphism of kernel-cokernel pairs $(p,q)$ and $(p',q')$ is a triple $(f,g,h)$ such that the following diagram commutes. 
                \begin{center}
                    \begin{tikzcd}
                        A \ar{r}{p} \ar{d}{f} & B \ar{r}{q} \ar{d}{g} & C \ar{d}{h} \\
                        A' \ar{r}{p'} & B' \ar{r}{q'} & C'
                    \end{tikzcd}
                \end{center}
            \end{definition}

            \begin{lemma}
                Let $(p,q)$ be a kernel-cokernel pair, then the image and coimage of p exists and are isomorphic. I.e. this diagram exists, such that the left square is a pushout and the right square is a pullback:
                \begin{center}
                    \begin{tikzcd}
                        0 \ar{r}{0} \ar{rd}{0} & A \ar{r}{p} \ar{d} & B \ar{r}{q} & C \\
                        & Coim(p) \ar{r}{iso} & Im(p) \ar{u} \ar{ur}{0}
                    \end{tikzcd}
                \end{center}
            \end{lemma}
            
            %Usikker på om dette beviset er riktig, jeg burde ha en måte å konstruere isomorfien på, ettersom at vi ønsker at analyse morfien skal være isomorfien, ikke at vi kan velge identiteten
            \begin{proof}
                Since $(p,q)$ is a kernel-cokernel pair we have that the first square is bicartesian and the second square is a push-out.
                \begin{center}
                    \begin{tikzcd}
                    A \ar{r}{p} \ar{rd}{0} & B \ar{d}{q} \\ & C
                    \end{tikzcd}
                    \begin{tikzcd}
                        0 \ar{r}{0} \ar{rd}{0} & A \ar[equal]{d} \\ & A
                    \end{tikzcd}
                \end{center}
                Thus $Im(p)=Coim(p)=A$, asserting the isomorphism as the identity in the diagram.
                \begin{center}
                    \begin{tikzcd}
                        0 \ar{r}{0} \ar{rd}{0} & A \ar{r}{p} \ar[equal]{d} & B \ar{r}{q} & C \\
                        & A \ar[equal]{r} & A \ar{u}{p} \ar{ur}{0}
                    \end{tikzcd}
                \end{center}
            \end{proof}

            \begin{definition}
                An exact category is an additive category $\mathcal{A}$ together with a class $\mathcal{E}$ of kernel-cokernel pairs. A pair $(p,q):\mathcal{E}$ is called a conflation, here $p$ is called an inflation and $q$ is called a deflation. $(\mathcal{A},\mathcal{E})$ is called exact when the following axioms holds:
                % Kan korte ned på antallet av aksiomer, jeg trenger ikke å ha op for alle, blir nok mer oversiktelig om jeg kobler noen sammen.
                \begin{itemize}
                    \item (QE0) $\forall A:\mathcal{A}$ $id_A$ is both an inflation and a deflation.
                    \item (QE1) Both inflations and deflations are closed under composition.
                    \item (QE2) The pushout of an inflation is an inflation.
                    \item (QE2$^{op}$) The pullback of a deflation is a deflation.

                \end{itemize}
            \end{definition}


            \begin{remark}
                When writing diagrams we use decorated arrows to indicate that a morphism is either an inflation or a deflation. A tail with a circle means inflation: \begin{tikzcd}
                    A \ar[tail]{r}[marking]{\circ} & B
                \end{tikzcd}. Double heads with a circle means deflation: \begin{tikzcd}
                    A \ar[two heads]{r}[marking]{\circ} & B
                \end{tikzcd}. We can now rewrite the (QE2) axioms as:
                \begin{center}
                    \begin{tikzcd}
                        A \ar[tail]{r}[marking]{\circ} \ar{d} \ar[phantom]{dr}[very near end]{\lrcorner} & B \ar{d} \\
                        C \ar[tail]{r}[marking]{\circ} & D
                    \end{tikzcd}
                    \begin{tikzcd}
                        A \ar[two heads]{r}[marking]{\circ} \ar{d} \ar[phantom]{dr}[very near start]{\ulcorner} & B \ar{d} \\
                        C \ar[two heads]{r}[marking]{\circ} & D
                    \end{tikzcd}
                \end{center}
            \end{remark}

            \begin{lemma}
                The map $0:0\rightarrow A$ is an inflation. Dually, the map $0:A\rightarrow 0$ is a deflation.
            \end{lemma}

            \begin{proof}
                Consider the diagram \begin{tikzcd}
                    0 \ar[tail]{r}{0} & A \ar[two heads]{r}{id_A} & A
                \end{tikzcd}. The left morphism is the kernel of the right morphism making a kernel-cokernel pair $(0,id_A)$. The identity $id_A$ is assumed to be a deflation, implying that the pair is a conflation.
            \end{proof}

            \begin{corollary}
                A kernel-cokernel pair $(i,p)$ found as sequence (1) is a conflation. 
                
                \begin{center}
                    (1)
                    \begin{tikzcd}
                        A \ar{r}{i} & A \oplus B \ar{r}{p} & B
                    \end{tikzcd}
                \end{center}
            \end{corollary}

            \begin{proof}
                In a category with an initial object the coproduct can be thought of as the pushout with the initial in the upper right corner. This can be assembled into pushout (1).
                By the lemma the zero morphisms are inflations, asserting that $i$ and $i'$ are inflations by (QE2). Thus there are conflations $(i,p)$ and $(i',p')$.

                \begin{center}
                    (1)
                    \begin{tikzcd}
                        0 \ar[tail]{r}{0} \ar[tail]{d}{0} & A \ar[tail]{d}{i} \\
                        B \ar[tail]{r}{i'} & A \oplus B
                    \end{tikzcd}
                \end{center}
            \end{proof}

            \begin{corollary}
                Split-monomorphisms admitting a cokernel is an inflation, with dual statement as split-epimorphisms admitting a kernel is a deflation. 
            \end{corollary}

            \begin{prop}
                Consider the commutative square
                \begin{tikzcd}
                    A \ar[tail]{r}{i} \ar{d}{f} & B \ar{d}{g} \\
                    C \ar[tail]{r}{j} & D
                \end{tikzcd}. The following statements are equivalent:
                \begin{itemize}
                    \item The square is a pushout.
                    \item The sequence \begin{tikzcd}
                        A \ar[tail]{r}{\begin{pmatrix} 
                            i \\ -f 
                        \end{pmatrix}} & B\oplus C \ar[two heads]{r}{\begin{pmatrix} 
                            g \ j
                        \end{pmatrix}} & D \end{tikzcd} is a conflation.
                    \item The square is bicartesian.
                    \item The square is a part of the commutative diagram \begin{tikzcd}
                        A \ar[tail]{r}{i} \ar{d}{f} & B \ar[two heads]{r}{p} \ar{d}{g} & E \ar[equal]{d} \\
                        C \ar[tail]{r}{j} & D \ar[two heads]{r}{q} & E
                    \end{tikzcd}
                \end{itemize}
            \end{prop}

            \begin{proof}
                
            \end{proof}

            \begin{prop}
                The pullback of an inflation along a deflation is an inflation.
            \end{prop}

            \begin{proof}
                
            \end{proof}

            \begin{theorem}
                The obscure axiom
            \end{theorem}

            %Add the classical homological diagram lemmatas down here
            \begin{lemma}
                Let $(p,q)$ be the conflation \begin{tikzcd}
                    A \ar[tail]{r}{p} & B \ar[two heads]{r}{q} & C
                \end{tikzcd}, $(p',q')$ be the conflation \begin{tikzcd}
                    A' \ar[tail]{r}{p'} & B' \ar[two heads]{r}{q'} & C'
                \end{tikzcd}. A morphism of the conflations $(f,g,h):(p,q)\rightarrow (p',q')$ factors through the conflation \begin{tikzcd}
                    A \ar[tail]{r} & D \ar[two heads]{r} & C'
                \end{tikzcd}
            \end{lemma}

            \begin{proof}
                
            \end{proof}

            \begin{corollary}
                The short five lemma
            \end{corollary}

            \begin{proof}
                
            \end{proof}

            \begin{lemma}
                Noethers isomorphism lemma
            \end{lemma}

            \begin{proof}
                
            \end{proof}

        \subsection{The Frobenis Category}

        \subsection{Self-injective Algebras}

        \subsection{The Homotopy Category}

    \clearpage

    \section{The Derived Category}

        \subsection{Admissable Morphisms}

        \subsection{Homology and Long Exact Sequences}

        \subsection{The Derived Category}

        \subsection{If time, derived functors as well}

    \clearpage
    
    \section{Auslander-Reiten Triangles}

        \subsection{Krull-Schmidt Categories}

        \subsection{Definition and First Properties}

        \subsection{Description of Derived Categories}

    \clearpage

    \section*{Appendix A: Category theory}
        \subsection{Quotient Categories}
        \subsection{Additive and Abelian Categories}
        \subsection{Freyd-Mitchell Embedding???}

    \clearpage
\end{document}